\documentclass{article}
\usepackage[utf8]{inputenc}

\title{Constructive Temporal Logic, Categorically}
\author{Valeria de Paiva \and Harley Eades III}
\date{October 2015}

\usepackage{natbib}
\usepackage{graphicx}
\usepackage{amssymb, amsthm, amsmath, stmaryrd}
\usepackage{mathpartir}
\usepackage[barr]{xy}

%% This renames Barr's \Diamond command so that it doesn't conflict
%% with our modality symbols.
\let\BDiamond\Diamond
\let\Diamond\relax

%% This renames Barr's \to to \mto.  This allows us to use \to for imp
%% and \mto for a inline morphism.
\let\mto\to
\let\to\relax
\newcommand{\to}{\rightarrow}

\usepackage{wasysym}
\usepackage{proof}
\usepackage{enumerate}
\usepackage{todonotes}
\usepackage{hyperref}
\usepackage{graphicx}

\renewcommand{\Box}{\oblong}

\newcommand{\F}{\mathop{\textbf{F}}}
\renewcommand{\P}{\mathop{\textbf{P}}}
\newcommand{\G}{\mathop{\textbf{G}}}
\renewcommand{\H}{\mathop{\textbf{H}}}

\newcommand{\cat}[1]{\mathcal{#1}}
\newcommand{\pd}[0]{\times}
\newcommand{\ihom}[0]{\rightarrow}
\newcommand{\st}[2]{\mathsf{st}_{#1,#2}}
\newcommand{\id}[0]{\mathsf{id}}
\newcommand{\m}[1]{\mathsf{m}_{#1}}
\newcommand{\p}[1]{\mathsf{p}_{#1}}

\newtheorem{theorem}{Theorem}
\newtheorem{lemma}[theorem]{Lemma}
\newtheorem{example}[theorem]{Example}
\newtheorem{fact}[theorem]{Fact}
\newtheorem{corollary}[theorem]{Corollary}
\newtheorem{definition}[theorem]{Definition}
\newtheorem{remark}[theorem]{Remark}
\newtheorem{proposition}[theorem]{Proposition}
\newtheorem{notn}[theorem]{Notation}
\newtheorem{observation}[theorem]{Observation}

\begin{document}

\maketitle

\section{Introduction}
Generally speaking, Temporal Logic is any system of rules and
symbolism for representing, and reasoning about propositions qualified
in terms of time.  Temporal logic is also one of the most traditional
kinds of modal logic, introduced by Arthur Prior in the late 1950s,
%, and for which important results were obtained by Hans Kamp. 
but it is also one of the most controversial kinds of modal logic, as
people have different intuitions about time, how to represent it, and
how to reason about it.

There has been a huge amount of work in Modal Logic in the last sixty
years, but mainly in classical modal logic. We are mostly interested
in constructive systems. In particular we are interested in a
constructive version of temporal logic that satisfies some well-known
and desirable proof-theoretical properties, but that is also
algebraically and category-theoretically well-behaved.

Prior's `Time and Modality' introduced a propositional modal logic
with two temporal connectives (modal operators), $F$ and $P$,
corresponding to ``sometime in the {F}uture" and ``sometime in the
{P}ast". Kamp's thesis introduced the binary temporal operators
``since" and ``until" and proved what came to be known as Kamp's
theorem. Kamp's theorem shows that all temporal operators are
definable in terms of ``since" and ``until" -- provided that the
underlying temporal structure is a continuous linear ordering and
provided that the logical basis is classical.

Ewald \cite{ewald1986} has produced a first version of an
intuitionistically based temporal logic system. Basically, Ewald's
system is a pair of Simpson-style IK operators \cite{simpson1994},
representing past and future over intuituionistic propositional logic.

The intuitive reading of the operators is sensible:
\begin{itemize}
\item $P$ “It has at some time been the case that” 
\item $F$ “It will at some time be the case that” 
\item $H$ “It has always been the case that” 
\item $G$  “It will always be the case that” 
\end{itemize}
Ewald and most of the researchers that followed his path of
constructivization of tense logic, did so assuming a symmetry between
past and future. This symmetry, as well as the symmetry between
universal and existential quantifiers, both in the past and in the
future, are at odds with
%is not very characteristic of 
intuitionistic reasoning.

 Simpson remarks that intuitionistic or constructive modal logic is
 full of interesting questions. As he says:
\begin{quote}
Although much work has been done in the field, there is as yet no
consensus on the correct viewpoint for considering intuitionistic
modal logic.  In particular, there is no single semantic framework
rivalling that of possible world semantics for classical modal logic.
Indeed, there is not even any general agreement on what the
intuitionistic analogue of the basic modal logic, K, is.
\end{quote}
In an intuitionistic logic we do not expect perfect duality between
quantifiers, ($\forall x.P(x)$ is not the same as $\neg \exists x.\neg
P(x)$) or even between conjunction and disjunction (De Morgan laws do
not hold for intuitionistic propositional logic). So one should not
expect a perfect duality between possibility and necessity either.

\section{Constructive Temporal Logics}

Each of the logics presented in this section is an extension of the
single-conclusion formalization of Gentzen's intuitionistic sequent
calculus LJ.  The syntax of formulas for LJ is defined by the
following grammar:
\begin{center}
    \begin{math}
        \begin{array}{lllllllll}
            A & ::= & p \mid \perp \mid A \land A \mid A \lor A \mid A \to B
        \end{array}
    \end{math}
\end{center}
The formula $p$ is taken from a set of countably many propositional
constants.

We begin with an initial set of inference rules, and then each system
presented in this section will be given as an extension of this
initial system.  The set of initial inference rules, which just models
propositional intuitionistic logic, are as follows:
\begin{center}
  \small
  \begin{math}
    \begin{array}{c}
      \begin{array}{cccccccc}
        \infer[\text{Id}]{\Delta,A \vdash A}{
          \,
        }
        & \quad &
        \infer[\text{Cut}]{\Gamma,\Delta \vdash C}{
          \Gamma \vdash B
          &
          B,\Delta \vdash C
        }
        & \quad & 
        \infer[\perp_{\mathcal{L}}]{\Gamma,\perp \; \vdash A}{
          \,
        }\\\\
        \infer[\lor_{\mathcal{L}}]{\Gamma,A \lor B \vdash C}{
          \Gamma,A \vdash C
          &
          \Gamma,B \vdash C
        }
        & \quad &
        \infer[\lor_{\mathcal{R}_1}]{\Gamma \vdash A \lor B}{
          \Gamma \vdash A
        }
        & \quad &
        \infer[\lor_{\mathcal{R}_2}]{\Gamma \vdash A \lor B}{
          \Gamma \vdash B
        }\\\\
        \infer[\land_{\mathcal{L}_1}]{\Gamma,A \land B \vdash C}{
          \Gamma,A \vdash C
        }
        & \quad &
        \infer[\land_{\mathcal{L}_2}]{\Gamma,A \land B \vdash C}{
          \Gamma,B \vdash C
        }
        & \quad &
        \infer[\land_{\mathcal{R}}]{\Gamma \vdash A \land B}{
          \Gamma \vdash A
          &
          \Gamma \vdash B
        }\\\\
        
      \end{array}
      \\
      \begin{array}{cccccccc}
        \infer[\to_{\mathcal{L}}]{\Gamma,A \to B \vdash C}{
          \Gamma \vdash A
          &
          \Gamma,B \vdash C
        }
        & \quad &
        \infer[\to_{\mathcal{R}}]{\Gamma \vdash A \to B}{
          \Gamma, A \vdash B
        }
      \end{array}        
    \end{array}
  \end{math}
\end{center}
Sequents denoted $\Gamma \vdash C$ consist of a multiset of formulas,
denoted by either $\Gamma$, $\Delta$, or a numbered version of the
two, and a formula $C$.

\subsection{The constructive system S4} 

In this section we recall the sequent calculus formalization of system
CS4.  This formalization was initially denoted by IS4 (Intuitionistic
modal logic S4) and was proposed by Bierman and de Paiva \cite{CS4}.
This systems gives rise to a complete family of systems, which are
alternatives to Simpson's intuitionistic systems. Since Simpson's
systems, a framework starting from intuitionistic K, were also called
IK, IS4, IS5, it makes sense to call the family of systems originating
with the S4 above, the family of constructive modal systems CS, and
Bierman and de Paiva's system CS4.  This is for instance how
\cite{arisaka2015} describe these systems.

The main difference between systems IS4 and CS4 concerns the binary
distribution of $\Diamond$ over disjunction
$\Diamond (A \lor B) \to \Diamond A \lor \Diamond B$
and its nullary form $\Diamond \bot\bot$.

The following rules, in addition to
the initial set of inference rules, defines the sequent calculus for CS4:
\begin{center}
  \begin{math}
    \begin{array}{ccccc}              
      \infer[\Box_{\mathcal{L}}]{\Gamma, \Box A \vdash B}{
        \Gamma,A \vdash B
      }
      & \quad &
      \infer[\Box_{\mathcal{R}}]{\Box\Gamma,\Delta \vdash \Box A}{
        \Box \Gamma \vdash A
      }\\\\
      \infer[\Diamond_{\mathcal{L}}]{\Delta,\Box\Gamma,\Diamond A \vdash \Diamond B}{
        \Box\Gamma,A \vdash \Diamond B
      }
      & \quad &
      \infer[\Diamond_{\mathcal{R}}]{\Gamma \vdash \Diamond A}{
        \Gamma \vdash A
      }
    \end{array}        
  \end{math}
\end{center}
This system is indeed constructive, and hence, $\Box A$ is not
logically equivalent to $\lnot \Diamond \lnot A$, and $\Diamond A$ is
not logically equivalent to $\lnot \Box \lnot A$.

This system is very well behaved proof theoretically. Bierman and de
Paiva show that it has a Hilbert-style presentation, and a Natural
Deduction one, as well as a sequent formulation. The sequent calculus
satisfies cut-elimination, an old result from Ohnishi and Matsumoto
\cite{ohnishi1957} as well as the subformula property.

The Natural Deduction formulation has a colourful history, described
in \cite{CS4}. One of its distinct features is that it was described
in Prawitz' seminal book in Natural Deduction \cite{prawitz1965},
hence it is still sometimes called Prawitz modal logic.

Most interestingly the system has both Kripke and categorical
semantics, described respectively in \cite{alechinaetal} and
\cite{CS4}. Since we can prove a Curry-Howard correspondence for this
system, it has been used in several applications within programming
language theory \todo{Add references for these applications}.
%Examples include \cite{hmm}.

%\subsection{Presenting CS4 as LNL}
\subsection{An adjunction CS4}
It is not so well-known, but this system can also be given a
presentation in terms of a categorical adjunction, between two
cartesian closed categories.

This is described in both \cite{CS4} and \cite{icalp1998}. In
\cite{CS4} this is called the multicontext formulation of CS4 and the
rules are given (page 17) as follows:

\begin{center}
  \begin{math}
    \begin{array}{ccccc}              
      \infer[\Box_{\mathcal{I}}]{\Gamma; \Delta \vdash \Box A}{
        \Gamma;\emptyset \vdash  A
      }
      & \quad &
      \infer[\Box_{\mathcal{E}}]{\Gamma;\Delta \vdash B}{\Gamma; \Delta \vdash \Box A \hspace{.1in}
        \Gamma, A;\Delta \vdash B
      }\\\\
      \infer[\Diamond_{\mathcal{I}}]{\Gamma;\Delta \vdash \Diamond A}{
        \Gamma;\Delta \vdash A
      }
      & \quad &
      \infer[\Diamond_{\mathcal{E}}]{\Gamma;\Delta \vdash \Diamond B}{
        \Gamma ;\Delta \vdash \Diamond A \hspace{.1in}\Gamma; A\vdash \Diamond B
      }
    \end{array}        
  \end{math}
\end{center}
(Note that the rules are Natural Deduction rules, as it should be clear for the fact that they are introduction and elimination rules.)

These rules have been shown by Benton \cite{benton1995} and Barber to correspond to an adjunction of the categories, as we show in the next section.  \todo{add description
  from Benton here}

\subsection{Models of CS4}
\label{subsec:single_adjoint_model_of_cs4}
%\todo{Threw this in a subsection for now, we can move it around, and reword as needed.} sounds good!

This section depends on the definitions of the
categorical concepts of monoidal category, monoidal functor, and
monoidal adjunction, as well as the concepts of monoidal comonad and monad. 

A monoidal category is simply a category $\cat{C}$  equipped with a tensor product structure $(\otimes, I, \alpha, \rho, \lambda)$, a generalization of traditional categorical products. A monoidal functor between two monoidal categories $\cat{C}$ and $\cat{D}$ is simply a functor $F\colon \cat{C} \mto \cat{D}$ that relates the monoidal structures, that is, there are natural transformations  $F(A)\otimes F(B)\mto F(A \otimes B)$, for each $A,B$ in $\cat{C}$ (as well as a map $m_{I}\colon I \mto FI$) that interact well with the monoidal structures  in $\cat{C}$ and  $\cat{D}$ (older books insist on an isomorphism between the tensor structures, we only require coherent natural transformations). 

A monoidal adjunction $J \dashv H, $ is an adjunction  where both functors are monoidal and the unit, and co-unit are monoidal natural trasnformations too. 

A monoidal comonad is a comonad $(\Box,\varepsilon, \delta)$ on a monoidal category $\cat{C}$, with extra structure preserving the monoidal structure. Thus $\Box\colon \cat{C}\to \cat{C}$ satisfies the following diagrams.


Similarly for a monoidal monad.
%Due to space limitations we omit their full definitions see \cite{benton1995} for their details.
%Similarly, for the definitions of monoidal comonad and monad see\cite{CS4}.

We do need to recall the notion of a functor-strong monad. In particular we are interested in $\Box$-strong monads.

\begin{definition}[$\Box$-strong monad]
  \label{def:comonad-strong-monad}
  Suppose $(\Box, \varepsilon, \delta)$ is a (monoidal) comonad on a
  cartesian closed category $\cat{C}$.  Then a \emph{$\Box$-strong
    monad} is a monad $(\Diamond,\eta,\mu)$ on $\cat{C}$ such that
  there exists a natural transformation:
  \[
  \st{A}{B} : \Box A \pd \Diamond B \mto \Diamond(\Box A \pd B)
  \]
  subject to the following coherence conditions:
  \begin{center}
    \begin{math}      
      \begin{array}{lll}
        \bfig
      \vSquares|ammmmma|/>``>```>`>/[\Box 1 \times \Diamond A`\Diamond (\Box 1 \times A)``\Diamond(1 \times A)`1 \times \Diamond A`\Diamond A;\st{1}{A}``\Diamond(\varepsilon_1 \times \id_A)```\Diamond\lambda`\lambda]
      \morphism(0,1000)|m|/->/<0,-950>[`;\varepsilon_1 \times \id_{\Diamond A}]
      \efig
      &
      \bfig
      \btriangle<800,500>[\Box A \pd B`\Box A \pd \Diamond B`\Diamond(\Box A \pd B);\id_{\Box A} \pd \eta_{B}`\eta_{\Box A \times B}`\st_{A,B}]
      \efig
      \end{array}      
    \end{math}
    \\
    %% \begin{math}
    %%   \bfig
    %%   \hSquares|aamaaaa|/->`->`->``=`->`->/[
    %%     \Box A \pd \Diamond 1`
    %%     \Diamond (\Box A \pd 1)`
    %%     \Diamond\Box A`
    %%     \Box A \pd 1`
    %%     \Box A`
    %%     \Diamond\Box A;\st{A}{1}`\Diamond(\rho_{\Box A})`\id_A \pd \varepsilon```\rho`\eta_{\Box A}]
    %%   \efig
    %% \end{math}
    \begin{math}
      \bfig
      \hSquares|aamaaaa|/->`->`->```->`/[
        \Box A \times 1`
        \Box A \times \Diamond 1`
        \Diamond (\Box A \times 1)`
        \Box A`
        A`
        ;
        \id_{\Box A} \times \eta`
        \st{A}{1}`
        \rho```
        \varepsilon`]
      \qtriangle(1944,0)/->``->/<800,500>[\Diamond (\Box A \times 1)`\Diamond\Box A`\Diamond A;\Diamond \rho``\Diamond\varepsilon]
      \morphism(1040,0)/->/<1630,0>[`;\eta_A]
      \efig
    \end{math}
    \\
    \begin{math}
      \bfig
        \vSquares|ammmmmm|/->`->`->```->`/[
          \Box A \times (\Box B \times \Diamond C)`
          \Box A \times \Diamond(\Box B \times C)`
          (\Box A \times \Box B) \times \Diamond C`
          \Diamond(\Box A \times (\Box B \times C))``
          \Diamond((\Box A \times \Box B) \times C);
          \id_{\Box A} \pd \st{B}{C}`
          \alpha^{-1}`
          \st{A}{\Box B \times C}```
          \Diamond\alpha^{-1}`]
        \morphism(1554,0)|m|/->/<0,-500>[`\Diamond(\Box(A \times B) \times C);\Diamond(\m{A,B} \times \id_C)]
        
        \morphism(0,500)|m|/->/<0,-1000>[`\Box(A \times B) \times \Diamond C;\m{A,B} \times \id_{\Diamond C}]

        \morphism(350,-500)|a|/->/<800,0>[`;\st{A \times B}{C}]
        \efig
    \end{math}
    \\
    \begin{math}
      \bfig
      \vSquares|ammmmma|/->`->```->``->/[
        \Box A \times \Diamond\Diamond B`
        \Box A \times \Diamond B`
        \Diamond(\Box A \times \Diamond B)``
        \Diamond\Diamond(\Box A \times B)`
        \Diamond(\Box A \times B);
        \id_{\Box A} \pd \mu_{B}`
        \st{A}{\Diamond B}```
        \Diamond(\st{A}{B})``
        \mu_{\Box A \pd B}]
      \morphism(1150,1000)|m|<0,-920>[`;\st{A}{B}]
      \efig
    \end{math}
    \\
    \begin{math}
      \bfig
      \hSquares|aamamaa|/``->``->`->`->/[
        \Box A \times \Diamond B``
        \Diamond B \times \Box A`
        \Diamond B \times \Box A`
        \Diamond B \times \Diamond \Box A`
        \Diamond (B \times \Box A);``
        \beta``
        \Diamond \beta`
        \id_{\Diamond B} \pd \eta_{\Box A}`
        \mathsf{n}_{B,\Box A}]
      \morphism(200,500)<1883,0>[`;\st{A}{B}]
      \efig
    \end{math}
    \\
    \begin{math}
      \bfig
      \hSquares|aamamaa|/``->``->`->`->/[
        \Box A \times \Diamond B``
        \Diamond(\Box A \times B)`
        A \times \Diamond B`
        \Diamond A \times \Diamond B`
        \Diamond (A \times B);``
        \varepsilon_A \times \id_{\Diamond B}``
        \Diamond(\varepsilon_A \times \id_{B})`
        \eta_A \times \id_{\Diamond B}`
        \m{A,B}]
      \morphism(200,500)<1605,0>[`;\st{A}{B}]
      \efig
    \end{math}        
  \end{center}
\end{definition}

\begin{definition}[CS4 model]
  \label{def:CS4-model}
  Suppose $\cat{C}$ is a cartesian category equipped with  a monoidal comonad $(\Box, \varepsilon, \delta)$ and a
  \emph{$\Box$-strong monad} with the strength natural transformation
  \[
  \st{A}{B} : \Box A \pd \Diamond B \mto \Diamond(\Box A \pd B)
  \]
  This is a model of CS4 as defined and showed sound and complete in \cite{CS4}.
\end{definition}

Now we want to see how to transform a model of CS4 using a pair comonad-monad as above into a model of the same system using adjunctions.  Benton showed a similar result for linear type theory in \cite{benton1995}.


\begin{definition}[adjoint CS4 model]
  \label{def:CS4-single-adjoint-cat-model}
  An adjoint categorical model of CS4 consists of the following data:
  \begin{enumerate}
  \item A cartesian-closed category with coproducts $(\cat{C},1,0,\pd,+,\ihom)$;
  \item 
    A monoidal adjunction  $J \dashv H$, where $(H,m)$ and  $(J,n)\colon \cat{C} \mto \cat{C}$ are monoidal functors such that their composition HJ is a monoidal comonad, written as $\Box$;
 \item The  monad $(\Diamond, \eta, \mu, \st{A}{B})$, induced by the adjunction $J \dashv H$,   is $\Box$-strong, subject to some conditions.
  \end{enumerate}
\end{definition}

Now the existence of any (monoidal or not) adjunction provides us with a (monoidal) monad $JH$ in $\cat{C}$, as well as a (monoidal) comonad $\Box=HJ$ on $\cat{C}$. Which conditions do we need in the adjunction to make this induced monad $\Box$-strong, if any?

Following Benton's description in page 5 of \cite{benton1995}, we recall (his lemma 2) that given a monoidal adjunction $J \dashv H$  we have, not only a comonad $(\Box, \varepsilon, \delta)$ but also a natural transformation, $q$, whose components $q_{A,B}\colon\Box A\land \Box B \mto \Box (A\land B)$ together with a map $m_{1}\colon 1 \mto \Box 1$ make $(\Box, q)$ a symmetric monoidal functor and $\varepsilon, \delta$ monoidal natural transformations.

Now we need to look at the monad functor in $\cat{C}$ arising from the adjunction $J \dashv H$, which we will call $\Diamond = JH$ for obvious reasons. We need to show that $\Diamond$ is a monoidal monad, with unit given by $\eta\colon 1\mto JH$ and multiplication given by $\mu\colon  JHJH\mto JH$, where $\eta$ is the counit of the adjunction and $\mu$ has components given by $...JH..$

\textit{In the best possible scenario we don't need any conditions,the conditions on the monoidal adjunction do everything for us. in the worst scenario we have to add the morphisms that make a monad strong with respect to its associated comonad. I'm betting that we don't have to add anything.}

\begin{theorem}
  An adjoint CS4 categorical model is a CS4 categorical model.
\end{theorem}
\begin{proof}
  We must show that given the definition of an adjoint CS4 categorical
  model (Definition~\ref{def:CS4-single-adjoint-cat-model}) we can
  define an appropriate monad and comonad on a CCC with coproducts
  where the monad is strong with respect to the comonad.

  Suppose $(H,m)$ and $(J,n)$ are the adjoint monoidal functors given
  in Definition~\ref{def:CS4-single-adjoint-cat-model}, and define
  $\Box = HJ$ and $\Diamond = JH$.  By definition we assumed that
  $(\Box, q)$, where $q_{A,B} : \Box A \times \Box B \to \Box (A
  \times B)$, is monoidal, but we must show that $\Diamond$ is also
  monoidal.  We know that both $(H,n)$ and $(J,m)$ are monoidal
  endofunctors on $\cat{C}$ which implies that their composition
  $\Diamond$ is monoidal where
  \[
  \begin{array}{lll}
    \mathsf{p}_{1} = \eta_{1} : 1 \to \Diamond 1\\
    \mathsf{p}_{A,B} = \m{HA,HB};J(\mathsf{n}_{A,B})
    \colon \Diamond A \pd \Diamond B \mto \Diamond(A \pd B)
  \end{array}
  \]
  and the following diagrams commute (proofs omitted):
  \begin{mathpar}
    \scriptsize
    \bfig
    \vSquares|ammmmma|/->`->`->``->`->`->/[
      (\Diamond A \times \Diamond B) \times \Diamond C`
      \Diamond A \times (\Diamond B \times \Diamond C)`
      \Diamond(A \times B) \times \Diamond C`
      \Diamond A \times \Diamond(B \times C)`
      \Diamond ((A \times B) \times C)`
      \Diamond (A \times (B \times C));
      \alpha`
      \mathsf{p}_{A,B} \times \id_{\Diamond C}`
      \id_{\Diamond A} \times \mathsf{p}_{B,C}``
      \mathsf{p}_{A \times B,C}`
      \mathsf{p}_{A,B \times C}`
      \Diamond \alpha]
    \efig
    \and
    \bfig
    \hSquares|ammmmaa|/->``->`<-``->`/[
      1 \times \Diamond A`
      \Diamond A``
      \Diamond 1 \times \Diamond A`
      \Diamond(1 \times A)`;
      \lambda_{\Diamond A}``
      \mathsf{p}_{1} \times \id_{\Diamond A}`
      \Diamond \lambda_A``
      \mathsf{p}_{1,A}`]
    \efig
    \and
    \bfig
    \hSquares|ammmmaa|/->``->`<-``->`/[
      \Diamond A \times 1`
      \Diamond A``
      \Diamond A \times \Diamond 1`
      \Diamond(A \times 1)`;
      \rho_{\Diamond A}``
      \id_{\Diamond A} \times \mathsf{p}_{1}`
      \Diamond \rho_A``
      \mathsf{p}_{A,1}`]
    \efig
    \and
    \bfig
    \hSquares|ammmmaa|/->``->`->``->`/[
      \Diamond A \times \Diamond B`
      \Diamond B \times \Diamond A``
      \Diamond (A \times B)`
      \Diamond (B \times A)`;
      \beta_{\Diamond A,\Diamond B}``
      \mathsf{p}_{A,B}`
      \mathsf{p}_{B,A}``
      \Diamond\beta_{A,B}`]
    \efig
  \end{mathpar}

  Furthermore, suppose $J \dashv H$, where the unit, $\varepsilon :
  \Box A \to A$, and the counit, $\eta : A \to \Diamond A$, are
  monoidal natural transformations.  This implies that the following
  diagrams commute:
  \begin{mathpar}
    \bfig
    \btriangle<800,500>[A \times B`\Diamond A \times \Diamond B`\Diamond (A \times B);\eta_A \times \eta_B`\eta_{A \times B}`\mathsf{p}_{A,B}]
    \efig
    \and
    \bfig
    \qtriangle<800,500>[\Box A \times \Box B`\Box (A \times B)`A \times B;\mathsf{q}_{A,B}`\varepsilon_A \times \varepsilon_B`\varepsilon_{A \times B}]
    \efig
    \and
    \bfig
    \qtriangle<800,500>[1`J 1`\Diamond 1;n_{1}`\eta_1`J m_1]       
    \efig
    \and
    \bfig
    \hSquares|ammmmaa|/->``=`->``<-`/[
      \Box 1`
      1``
      \Box 1`
      H 1`;
      \varepsilon_1```
      m_1``
      H n_1`]
    \efig
    \and
    \bfig
    \qtriangle<800,500>[H A`H\Box A`H A;\eta_{H A}`\id_{H A}`H\varepsilon_A]       
    \efig
    \and
    \bfig
    \qtriangle<800,500>[J A`\Box J A`J A;J\eta_A`\id_{J A}`\varepsilon_{J A}]
    \efig    
  \end{mathpar}
  It is a well-known fact about adjoints that $(\Box, \varepsilon,
  \delta)$, where $\delta : \Box A \to \Box\Box A$ is a comonad, and
  $(\Diamond, \eta, \mu)$, where $\mu : \Diamond\Diamond A \to
  \Diamond A$ is a monad.  Thus, the following diagrams commute:
  \begin{mathpar}
    \bfig
    \hSquares|ammmmaa|/->``->`->``->`/[
      \Diamond^3 A`
      \Diamond^2 A``
      \Diamond^2 A`
      \Diamond A`;
      \Diamond \mu_A``
      \mu_{\Diamond A}`
      \mu_A``
      \mu_A`]
    \efig
    \and
    \bfig
    \qtriangle/->`=`->/<800,500>[\Diamond A`\Diamond^2 A`\Diamond A;\eta_{\Diamond A}``\mu_A]
    \btriangle(0,0)/->`=`->/<800,500>[\Diamond A`\Diamond^2 A`\Diamond A;\Diamond \eta_{A}``\mu_A]
    \efig
    \and
    \bfig
    \hSquares|ammmmaa|/->``->`->``->`/[
      \Box A`
      \Box^2 A``
      \Box^2 A`
      \Box^3 A`;
      \delta_A``
      \delta_A`
      \delta_{\Box A}``
      \Box\delta_A`]
    \efig
    \and
    \bfig
    \qtriangle/->`=`->/<800,500>[\Box A`\Box^2 A`\Box A;\delta_A``\Box \varepsilon]
    \btriangle(0,0)/->`=`->/<800,500>[\Box A`\Box^2 A`\Box A;\delta_A``\varepsilon_{\Box A}]
    \efig
  \end{mathpar}

  We can now define the $\Box$-strength map as follows:
  \[
  \st{A}{B} = (\eta_{\Box A} \pd \id_{\Diamond B});\mathsf{p}_{\Box A,B} : \Box A \pd \Diamond B \mto \Diamond(\Box A \pd B)
  \]
  We can see that $\st{A}{B}$ is a natural transformation, because it
  is defined as a composition of natural transformations.
  
  %% To prove that the appropriate diagrams commute we first note that
  %% the triangle  
  Next we must show that all of the appropriate diagrams given in
  Definition~\ref{def:comonad-strong-monad} commute.
  \begin{itemize}
  \item[Case.] \ \\
  \textit{1. the object 1 behaves as the unit for products}
    $$
    \bfig
    \vSquares|ammmmma|/>``>```>`>/[\Box 1 \times \Diamond A`\Diamond (\Box 1 \times A)``\Diamond(1 \times A)`1 \times \Diamond A`\Diamond A;\st{1}{A}``\Diamond(\varepsilon_1 \times \id_A)```\Diamond\lambda`\lambda]
    \morphism(0,1000)|m|/->/<0,-950>[`;\varepsilon_1 \times \id_{\Diamond A}]
    \efig
    $$
    This diagram commutes by the following equational reasoning:
    \begin{center}
      \begin{math}
        \begin{array}{rllllllll}
          & & \st{1}{A};\Diamond (\varepsilon_1 \times \id_A);\Diamond\lambda_A\\
          \text{(Definition of $\mathsf{st}$)}
          & = & (\eta_{\Box 1} \times \id_{\Diamond A});\p{\Box 1,\Diamond A};\Diamond (\varepsilon_1 \times \id_A);\Diamond\lambda_A\\
          \text{(Naturality of $\mathsf{p}$)}
          & = & (\eta_{\Box 1} \times \id_{\Diamond A});(\Diamond \varepsilon_1 \times \Diamond\id_A);\p{1,A};\Diamond\lambda_A\\
          \text{(Functoriality of $\times$)}
          & = & ((\eta_{\Box 1};\Diamond \varepsilon_1) \times (\id_{\Diamond A};\Diamond\id_A));\p{1,A};\Diamond\lambda_A\\
          & = & ((\eta_{\Box 1};\Diamond \varepsilon_1) \times (\id_{\Diamond A};\id_{\Diamond A}));\p{1,A};\Diamond\lambda_A\\
          \text{(Naturality of $\eta$)}
          & = & ((\varepsilon_1;\eta_{1}) \times (\id_{\Diamond A};\id_{\Diamond A}));\p{1,A};\Diamond\lambda_A\\
          \text{(Definition of $\mathsf{p}$)}
          & = & ((\varepsilon_1;\p{1}) \times (\id_{\Diamond A};\id_{\Diamond A}));\p{1,A};\Diamond\lambda_A\\
          \text{(Functoriality of $\times$)}
          & = & (\varepsilon_1 \times \id_{\Diamond A});(\p{1} \times \id_{\Diamond A});\p{1,A};\Diamond\lambda_A\\
          \text{($\Diamond$ is Symmetric Monoidal)}
          & = & (\varepsilon_1 \times \id_{\Diamond A});\lambda_{\Diamond A}\\
        \end{array}
      \end{math}
    \end{center}
    
  \item[Case.] \ \\
  \textit{2. unit $\eta$ of the monad and strength interact well, $\Box  A $ is a parameter}
    $$
    \bfig
    \btriangle<800,500>[
      \Box A \pd B`
      \Box A \pd \Diamond B`
      \Diamond(\Box A \pd B);
      \id_{\Box A} \pd \eta_{B}`
      \eta_{\Box A \times B}`
      \st{A}{B}]
    \efig
    $$

    The previous diagram commutes, because the following diagram commutes:
    $$
    \bfig
    \btriangle|ama|/->`->`->/<1500,500>[
      \Box A \pd B`
      \Box A \pd \Diamond B`
      \Diamond\Box A \pd \Diamond B;
      \id_{\Box A} \pd \eta_{B}`
      \eta_{\Box A} \pd \eta_B`
      \eta_{\Box A} \pd \id_{\Diamond B}]

    \qtriangle(0,0)/->``<-/<1500,500>[
      \Box A \pd B`
      \Diamond (\Box A \times B)`
      \Diamond\Box A \pd \Diamond B;
      \eta_{\Box A \times B}``
      \p{\Box A,B}]

    \place(250,200)[1]
    \place(1200,300)[2]
    \efig
    $$
    \noindent
    Diagram 1 clearly commutes, and diagram 2 commutes because $\eta$
    is a symmetric monoidal natural transformation.
    
  \item[Case.]\ \\
  \textit{3. co-unit of the comonad $\varepsilon$ and unit of the monad $\eta$ interact well?}
    $$
    \bfig
    \hSquares|aamaaaa|/->`->`->```->`/[
      \Box A \times 1`
      \Box A \times \Diamond 1`
      \Diamond (\Box A \times 1)`
      \Box A`
      A`
      ;
      \id_{\Box A} \times \eta_1`
      \st{A}{1}`
      \rho_{\Box A}```
      \varepsilon_A`]
    \qtriangle(1978,0)/->``->/<800,500>[\Diamond (\Box A \times 1)`\Diamond\Box A`\Diamond A;\Diamond \rho_{\Box A}``\Diamond\varepsilon_A]
    \morphism(1060,0)/->/<1640,0>[`;\eta_A]
    \efig
    $$
    \noindent
    Recall that
    $\st{A}{1} = (\eta_{\Box A} \pd \id_{\Diamond 1});\p{\Box A,1}$.
    Now the previous diagram commutes, because the following diagram commutes:
    $$
    \bfig
    \btriangle|mma|<1444,1000>[\Box A \pd 1`\Diamond\Box A \pd \Diamond 1`\Diamond(\Box A \pd 1);\eta_{\Box A} \pd \eta_1`\eta_{\Box A \pd 1}`\p{\Box A,1}]
    \dtriangle(-1000,0)/->``->/<1000,1000>[\Box A \pd 1`\Box A \pd 1`\Diamond\Box A \pd \Diamond 1;\id_{\Box A} \pd \eta_1``\eta_{\Box A} \pd
      \id_{\Diamond 1}]

    \hSquares(0,0)/->`->```<-``/<1000>[\Box A \pd 1`\Box A`\Diamond\Box A```\Diamond (\Box A \times 1);\rho_{\Box A}`\eta_{\Box A}```\Diamond (\rho_{\Box A})``]

    \square(746,1000)/->`<-`<-`/<698,500>[A`\Diamond A`\Box A`\Diamond\Box A;\eta_A`\varepsilon_A`\Diamond\varepsilon_A`]

    \place(-400,300)[1]
    \place(400,300)[2]
    \place(1100,700)[3]
    \place(1100,1250)[4]
    \efig
    $$
    \noindent
    Diagram 1 commutes by functorality of $\times$, diagram 2 commutes
    because $\eta$ is a monoidal natural transformation, and diagrams
    3 and 4 commute by naturality of $\eta$.

  \item[Case.]\ \\
  \textit{4. associativity $\alpha$ interacts with co-monoidicity of $\Box$}
    $$
    \bfig
    \vSquares|ammmmmm|/->`->`->```->`/[
      \Box A \times (\Box B \times \Diamond C)`
      \Box A \times \Diamond(\Box B \times C)`
      (\Box A \times \Box B) \times \Diamond C`
      \Diamond(\Box A \times (\Box B \times C))``
      \Diamond((\Box A \times \Box B) \times C);
      \id_{\Box A} \pd \st{B}{C}`
      \alpha^{-1}`
      \st{A}{\Box B \times C}```
      \Diamond\alpha^{-1}`]
    \morphism(1554,0)|m|/->/<0,-500>[`\Diamond(\Box(A \times B) \times C);\Diamond(\m{A,B} \times \id_C)]
    
    \morphism(0,500)|m|/->/<0,-1000>[`\Box(A \times B) \times \Diamond C;\m{A,B} \times \id_{\Diamond C}]

    \morphism(350,-500)|a|/->/<800,0>[`;\st{A \times B}{C}]
    \efig
    $$
    \noindent
    Recall that:
    \[
    \begin{array}{rlll}
      \st{B}{C}              & = & (\eta_{\Box B} \pd \id_{\Diamond C});\p{\Box B,C}\\
      \st{A \pd B}{C}        & = & (\eta_{\Box (A \pd B)} \pd \id_{\Diamond C});\p{\Box (A \pd B),C}\\
      \st{A}{\Box B \pd C} & = & (\eta_{\Box A} \pd \id_{\Diamond (\Box B \pd C)});\p{\Box A,(\Box B \pd C)}\\
    \end{array}
    \]
    In addition, we require the following diagram (whose commutativity
    is implied by the fact that $\Diamond$ is a symmetric monoidal
    functor):
    $$
    \bfig
    \vSquares|ammmmma|/->`->`->``->`->`->/[
      \Diamond A \pd (\Diamond B \pd \Diamond C)`
      (\Diamond A \pd \Diamond B) \pd \Diamond C`
      \Diamond A \pd \Diamond (B \pd C)`
      \Diamond (A \pd B) \pd \Diamond C`
      \Diamond (A \pd (B \pd C))`
      \Diamond ((A \pd B) \pd C);
      \alpha^{-1}_{\Diamond A,\Diamond B,\Diamond C}`
      \id_{\Diamond A} \pd \p{B,C}`
      \p{A,B} \pd \id_{\Diamond C}``
      \p{A,B \pd C}`
      \p{A \pd B,C}`
      \Diamond\alpha^{-1}_{A,B,C}]
    \efig
    $$
    \noindent
    Finally, this case follows because the following diagram commutes:
    \begin{center}
      \rotatebox{90}{$
    \bfig
    \btriangle|mmm|<1769,1000>[
      \Box A \pd (\Diamond\Box B \pd \Diamond C)`
      \Diamond\Box A \pd (\Diamond\Box B \pd \Diamond C)`
      \Diamond\Box A \pd (\Diamond\Box B \pd C);
      \eta_{\Box A} \pd \id_{\Diamond \Box B \pd C}`
      \eta_{\Box A} \pd \p{\Box B,C}`
      \id_{\Diamond\Box A} \pd \p{\Box B,C}]

    \qtriangle|mam|/->``->/<1769,1000>[
      \Box A \pd (\Diamond\Box B \pd \Diamond C)`
      \Box A \pd \Diamond (\Box B \pd C)`
      \Diamond\Box A \pd (\Diamond\Box B \pd C);
      \id_{\Box A} \pd \p{\Box B,C}``
      \eta_{\Box A} \pd \id_{\Diamond (\Box B \pd C)}]    

    \qtriangle(-1800,0)|mmm|<1800,1000>[
      \Box A \pd (\Box B \pd \Diamond C)`
      \Box A \pd (\Diamond\Box B \pd \Diamond C)`
      \Diamond\Box A \pd (\Diamond\Box B \pd \Diamond C);
      \id_{\Box A} \pd (\eta_{\Box B} \pd \id_{\Diamond C})`
      \eta_{\Box A} \pd (\eta_{\Box B} \pd \id_{\Diamond C})`
      \eta_{\Box A} \pd \id_{\Diamond \Box B \pd C}]

    \square(0,-500)|mmmm|/`->`->`/<1769,500>[
      \Diamond\Box A \pd (\Diamond\Box B \pd \Diamond C)`
      \Diamond\Box A \pd (\Diamond\Box B \pd C)`
      (\Diamond\Box A \pd \Diamond\Box B) \pd \Diamond C`
      \Diamond (\Box A \pd (\Box B \pd C));`
      \alpha_{\Diamond\Box A,\Diamond\Box B,\Diamond C}`
      \p{\Box A,\Box B \pd C}`]   

    \square(0,-1000)|mmmm|/`->`->`/<1769,500>[
      (\Diamond\Box A \pd \Diamond\Box B) \pd \Diamond C`
      \Diamond (\Box A \pd (\Box B \pd C))`
      \Diamond (\Box A \pd \Box B) \pd \Diamond C`
      \Diamond ((\Box A \pd \Box B) \pd C);`
      \p{\Box A,\Box B} \pd \id_{\Diamond C}`
      \Diamond \alpha_{\Box A,\Box B,C}`]        

    \square(0,-1500)|mmmm|<1769,500>[
      \Diamond (\Box A \pd \Box B) \pd \Diamond C`
      \Diamond ((\Box A \pd \Box B) \pd C)`
      \Diamond\Box(A \pd B) \pd \Diamond C`
      \Diamond (\Box(A \pd B) \pd C);
      \p{\Box A \pd \Box B, C}`
      \Diamond \m{A,B} \pd \id_{\Diamond C}`
      \Diamond (\m{A,B} \pd \id_C)`
      \p{\Box (A \pd B),C}]

    \btriangle(-1800,-500)|mmm|/->``->/<1800,1500>[
      \Box A \pd (\Box B \pd \Diamond C)`
      (\Box A \pd \Box B) \pd \Diamond C`
      (\Diamond\Box A \pd \Diamond\Box B) \pd \Diamond C;
      \alpha_{\Box A,\Box B,\Diamond C}``
      (\eta_{\Box A} \pd \eta_{\Box B}) \pd \id_{\Diamond C}]

    \qtriangle(-1800,-1000)|mmm|/`->`/<1800,500>[
      (\Box A \pd \Box B) \pd \Diamond C`
      (\Diamond\Box A \pd \Diamond\Box B) \pd \Diamond C`
      \Diamond (\Box A \pd \Box B) \pd \Diamond C;`
      \eta_{\Box A \pd \Box B} \pd \id_{\Diamond C}`]

    \btriangle(-1800,-1500)|mmm|/->``->/<1800,1000>[
      (\Box A \pd \Box B) \pd \Diamond C`
      \Box(A \pd B) \pd \Diamond C`
      \Diamond\Box(A \pd B) \pd \Diamond C;
      \m{A,B} \pd \id_{\Diamond C}``
      \eta_{\Box (A \pd B)} \pd \id_{\Diamond C}]

    \place(1200,700)[1]
    \place(500,300)[2]
    \place(900,-500)[3]
    \place(900,-1250)[4]
    \place(-500,700)[5]
    \place(-1000,0)[6]
    \place(-400,-700)[7]
    \place(-1000,-1100)[8]
    \efig
    $}
    \end{center}
    Diagrams 1, 2 and 5 commute by functorality of $\times$, diagram 3
    commutes by the additional diagram from above, diagram 4 commutes
    by naturality of $\mathsf{p}$, diagram 6 commutes by naturality of
    $\alpha$, diagram 7 commutes by the fact that $\eta$ is a monoidal
    natural transformation, and diagram 8 commutes by naturality of
    $\eta$.
    

  \item[Case.]\ \\
  \textit{5.  strength interacts with monoidicity of $\Diamond$}
    $$
    \bfig
    \vSquares|ammmmma|/->`->```->``->/[
      \Box A \times \Diamond\Diamond B`
      \Box A \times \Diamond B`
      \Diamond(\Box A \times \Diamond B)``
      \Diamond\Diamond(\Box A \times B)`
      \Diamond(\Box A \times B);
      \id_{\Box A} \pd \mu_{B}`
      \st{A}{\Diamond B}```
      \Diamond(\st{A}{B})``
      \mu_{\Box A \pd B}]
    \morphism(1150,1000)|m|<0,-920>[`;\st{A}{B}]
    \efig
    $$
    \noindent
    Recall that:
    \[
    \begin{array}{rlll}
      \st{A}{B} & = & (\eta_{\Box A} \pd \id_{\Diamond B});\p{\Box A,B}\\
      \st{A}{\Diamond B} & = & (\eta_{\Box A} \pd \id_{\Diamond \Diamond B});\p{\Box A,\Diamond B}\\
    \end{array}
    \]
    This case follows from the fact that the following diagram
    commutes:
    \begin{center}
      \rotatebox{90}{$\bfig
    \qtriangle|mmm|<1500,1000>[
      \Box A \pd \Diamond\Diamond B`
      \Box A \pd \Diamond B`
      \Diamond\Box A \pd \Diamond B;
      \id_{\Box A} \pd \mu_B`
      \eta_{\Box A} \pd \mu_B`
      \eta_{\Box A} \pd \id_{\Diamond B}]

    \morphism(-1500,1000)|m|/<-/<1500,0>[
      \Diamond\Box A \pd \Diamond\Diamond B`
      \Box A \pd \Diamond\Diamond B;
      \eta_{\Box A} \pd \id_{\Diamond\Diamond B}]

    \btriangle(-1500,0)|mmm|<3000,1000>[
      \Diamond\Box A \pd \Diamond\Diamond B`
      \Diamond\Diamond\Box A \pd \Diamond\Diamond B`
      \Diamond\Box A \pd \Diamond B;
      \Diamond\eta_{\Box A} \pd \id_{\Diamond\Diamond B}`
      \id_{\Diamond\Box A} \pd \mu_B`
      \mu_{\Box A} \pd \mu_B]

    \square(-3000,0)|mmmm|/<-`->``<-/<1500,1000>[
      \Diamond(\Box A \pd \Diamond B)`
      \Diamond\Box A \pd \Diamond\Diamond B`
      \Diamond(\Diamond\Box A \pd \Diamond B)`
      \Diamond\Diamond\Box A \pd \Diamond\Diamond B;
      \p{\Box A,\Diamond B}`
      \Diamond (\eta_{\Box A} \pd \id_{\Diamond B})``
      \p{\Diamond\Box A,\Diamond B}]

    \square(-3000,-500)|mmmm|/`->`->`->/<4500,500>[
      \Diamond (\Diamond\Box A \pd \Diamond B)`
      \Diamond\Box A \pd \Diamond B`
      \Diamond\Diamond (\Box A \pd B)`
      \Diamond (\Box A \pd B);`
      \Diamond (\p{\Box A,B})`
      \p{\Box A,B}`
      \mu_{\Box A \pd B}]

    \place(-2300,500)[1]
    \place(-800,-250)[2]
    \place(-800,400)[3]
    \place(0,700)[4]
    \place(1050,700)[5]
    \efig$}
    \end{center}       
    Diagram commutes by naturality of $\mathsf{p}$, diagram 2 commutes
    because $\mu$ is a monoidal natural transformation, diagram 3
    commutes because $\mu$ is the monadic multiplication and by
    functorality of $\times$, and diagrams 4 and 5 commute by
    functoriality of $\times$,
    
  \item[Case.]\ \\
  \textit{6. commuting $\beta$ interacts with $\Diamond$}
    $$
    \bfig
    \hSquares|aamamaa|/``->``->`->`->/[
      \Box A \times \Diamond B``
      \Diamond B \times \Box A`
      \Diamond B \times \Box A`
      \Diamond B \times \Diamond \Box A`
      \Diamond (B \times \Box A);``
      \beta``
      st_{B,\Box A}`
      \id_{\Diamond B} \pd \eta_{\Box A}`
      \mathsf{n}_{B,\Box A}]
    \morphism(200,500)<1883,0>[`;\beta]
    \efig
    $$

  \item[Case.]\ \\
 \textit{7.  $\varepsilon$ interacts with $\Diamond$ and its monoidicity} 
    $$
    \bfig
    \hSquares|aamamaa|/``->``->`->`->/[
      \Box A \times \Diamond B``
      \Diamond(\Box A \times B)`
      A \times \Diamond B`
      \Diamond A \times \Diamond B`
      \Diamond (A \times B);``
      \varepsilon_A \times \id_{\Diamond B}``
      \Diamond(\varepsilon_A \times \id_{B})`
      \eta_A \times \id_{\Diamond B}`
      \m{A,B}]
    \morphism(200,500)<1605,0>[`;\st{A}{B}]
    \efig
    $$
  \end{itemize}

\end{proof}

%\textit{If we're lucky, I'm hoping we also have}
Now we would like to, following Benton's lead, 
prove the converse result. That is
\begin{theorem} A CS4 categorical model can be extended to a CS4 adjoint categorical model.
\end{theorem}

% subsection single_adjoint_model_of_cs4 (end)

\section{Twice as nice?}
Previous work on modalities in Linear Logic made it clear that modalities, unlike canonical logical connectives such as conjunctions and disjunctions, are not unique up to iso in a given model. Thus we can have two sets of modalities, say $(\Box_1, \Diamond_1)$ and $(\Box_2, \Diamond_2)$ co-existing in a cartesian closed category $\cat{C}$ without problems. But this is not what we want, as we want holding for all times and for a (single) time, both in the future and in the past, so our two sets of modalities need to be related in  a sensible way.

Work of Dzik et al \cite{dziketal2012,dziketal2014} and Menni and Smith \cite{Menni:2014}  indicates that the two adjunctions have to be twisted, that is, the  existential modality in the past corresponds to the universal modality in the future and vice-versa. Past and future are not completely symmetric, but the adjunctions keep track of the symmetry there is there for quantifiers (in first-order logic) and for modalities (existential and universal ones).

%\subsection{Are past and future symmetric?}

%(half of) the rules below in Ewald's system.

\subsection{Algebras with Adjunctions}
%\todo{Need to  see if we can show a reasonable duality theorem similar to Menni and Smith's corollary 3.5}
Menni and Smith \cite{Menni:2014} (in page 4) define a \textit{Boolean algebra with an adjunction} BAA as a triple $(B, \Diamond, H)$ where $B$ is a Boolean algebra and $\Diamond, H$ are monotone functions such that $\Diamond$ is left adjoint to $H$, $\Diamond \dashv H$. A morphism is a Boolean algebra morphism $f\colon B\mto B'$  preserving both $\Diamond, H$. BAAs and their moprhisms form a category $\sf BAA$.
Clearly nothing in the definition makes reference to the classical (or otherwise) character of the underlying logic and hence we can define \textit{Heyting algebras with adjunctions} HAA exactly in the same manner, as   triples $(H, \Diamond, H)$ where we have the adjunction $\Diamond \dashv H$.  \textit{Mutatits mutandis} we have the category $\sf HAA$. 



Menni and Smith also consider the category $\sf BAO$ of Boolean algebras with operators. Objects of $\sf BAO$ are pairs of $(B,\Diamond)$ with $B$ a Boolean algebra and $\Diamond \colon B\mto B$ the operator, which is a function preserving finite suprema. Dualizing this we can consider the category $\sf HAO$ of Heyting algebras with operators, where $\Box\colon H\mto H$ is a function preserving finite infima.

Menni and Smith then go on to consider `double' BAAs of the form $(B, \Diamond \dashv H, P\dashv \Box)$, where $B$ is a Boolean algebra, $\Diamond$, $\Box$, $H$ and $P$ are  monotone functions  such that $\Diamond \dashv H$ and $ P\dashv\Box$ are adjunctions and a strict duality is asserted between $\Box$ and $\Diamond$ (namely $\Box=\neg \Diamond\neg$) and between $H$ and $P$ (namely $P=\neg H\neg$). Constructivizing these can be done easily by considering the Heyting algebra $H$ with similar adjunctions $(H, \Diamond \dashv G, P\dashv \Box)$ and not taking the very strict duality at face value.

Menni and Smith then state that \begin{quote} Altogether the category $\sf BAA$ provides an algebraic semantics for normal temporal logic, just as $\sf BAO$ provides an algebraic semantics for normal modal logic (Theorem 5.27 loc.cit.).\end{quote} We would like to assert and prove the theorem that ``the category $\sf HAA$ provides an algebraic semantics for a version of constructive temporal logic, just as $\sf HAO$ provides an algebraic semantics for a constructive modal logic".

%They also define \textit{lattice with  adjunction} as a triple $(A, L, R)$ where $A$ is a lattice, and $L,R$ are monotone functions on $A$ that form a preorder $L\dashv R$ adjunction.


\subsection{Ewald's Temporal Logic}

Ewald proposed an intuitionistic temporal logic where the usual modal
operators, $\Box$ and $\Diamond$, have been decomposed into more
traditional temporal operators $\F$, $\H$, $\G$, and
$\P$~\cite{ewald1986}.

The sequent calculus formalization of Ewald's temporal logic is
defined by the following inference rules in addition to the initial
intuitionistic set of inference rules given above:
\begin{center}
  \small
  \begin{math}
    \begin{array}{cccccc}
      \infer[\text{G}]{\G \Gamma \vdash \G C}{
        \Gamma \vdash C
      }
      & \quad &
      \infer[\text{H}]{\H\Gamma \vdash \H C}{
        \Gamma \vdash C
      }\\\\
      \infer[\text{F}]{\G\Gamma,\F A \vdash \F B}{
        \Gamma,A \vdash B
      }
      & \quad &
      \infer[\text{P}]{\H\Gamma,\P A \vdash \P B}{
        \Gamma,A \vdash B
      }\\\\
      \infer[\text{FH}]{\cdot \vdash \F\H A \to A}{
        \,
      }
      & \quad &
      \infer[\text{HF}]{\cdot \vdash A \to \H\F A}{
        \,
      }\\\\
      \infer[\text{PG}]{\cdot \vdash \P\G A \to A}{
        \,
      }
      & \quad &
      \infer[\text{GP}]{\cdot \vdash A \to \G\P A}{
        \,
      }\\\\
      \infer[\text{FG}_1]{\cdot \vdash (\F A \to \G B) \to \G(A \to B)}{
        \,
      }
      & \quad &
      \infer[\text{FG}_2]{\cdot \vdash \F(A \to B) \to (\G A \to \F B)}{
        \,
      }\\\\
      \infer[\text{PH}_1]{\cdot \vdash (\P A \to \H B) \to \H(A \to B)}{
        \,
      }
      & \quad &
      \infer[\text{PH}_2]{\cdot \vdash \P(A \to B) \to (\H A \to \P B)}{
        \,
      }      
    \end{array}
  \end{math}
\end{center}
Looking at Ewald's system with a present day categorical perspective
it is easy to see that these new operators can be seen as adjoint
functors~\cite{Menni:2014}.

The rules $\text{G}$ and $\text{H}$ show that $\G$ and $\H$ are
functors, similarly, the rules $\text{F}$ and $\text{P}$ do the same
for the operators $\F$ and $\P$ when $\Gamma = \cdot$.  Then the rules
$\text{FH}$ and $\text{HF}$ give the co-unit and unit of the
adjunction $\F \dashv \H$, and the rules $\text{PG}$ and $\text{GP}$
give the co-unit and unit of the adjunction $\P \dashv \G$.

The following describes the intuitive meaning of each of Ewald's
operators:
\begin{itemize}
\item[] $\F$: ``it will once be the case in the Future``  
\item[] $\H$: ``it Has always been the case in the past``
\item[] $\P$: ``it was once the case in the Past``
\item[] $\G$: ``it's Going to  always be the case in the future``  
\end{itemize}


Using these operators it is possible to define the  modal
operators by choosing one of the following definitions
\cite{ewald1986}:
\begin{enumerate}[i.]
\item $\Box A := A \land \G A \land \H A$, $\Diamond A := A \lor \F A \lor \P A$; or
\item $\Box A := A \land \G A$, $\Diamond A := A \lor \F A$.
\end{enumerate}

This system has a Kripke model, and an axiomatization (page 171 of
\cite{ewald1986}), and is decidable.  Lastly, Ewald shows that by
adopting various axioms for the modal operators different notions of
time can be studied within this system.

\section{Conclusion}
%``I always thought something was fundamentally wrong with the universe'' 

\bibliographystyle{plain}
\bibliography{references}
\end{document}

References:
Ewald's paper:Intuitionistic tense and modal logic
http://journals.cambridge.org/action/displayAbstract?fromPage=online&aid=9081918&fileId=S0022481200031650

Bierman and de Paiva:On an Intuitionistic Modal Logic (2001)
(I'm now calling CS4 what was in this paper IS4)
http://citeseerx.ist.psu.edu/viewdoc/summary?doi=10.1.1.30.5279

Modalities in Constructive Logics and Type Theories Preface to the special issue on Intuitionistic Modal Logic and Application of the Journal of Logic and Computation, volume 14, number 4, August 2004. Guest Editors: Valeria de Paiva, Rajeev Gore' and Michael Mendler.
http://www.cs.bham.ac.uk/~vdp/publications/final-preface.pdf

and

http://logcom.oxfordjournals.org/content/early/2015/06/12/logcom.exv042.extract
