\documentclass{article}
\usepackage[utf8]{inputenc}

\title{Constructive Modal Logic Revisited}
\author{Valeria de Paiva \and Harley Eades III}
\date{October 2015}

\usepackage{natbib}
\usepackage{graphicx}
\usepackage{amssymb, amsthm, amsmath, stmaryrd}
\usepackage{mathpartir}
\usepackage{mdframed}           % For the boxes around the systems.
\usepackage{cmll}
\usepackage[barr]{xy}

%% This renames Barr's \Diamond command so that it doesn't conflict
%% with our modality symbols.
\let\BDiamond\Diamond
\let\Diamond\relax
%% This renames Barr's \to to \mto.  This allows us to use \to for imp
%% and \mto for a inline morphism.
\let\mto\to
\let\to\relax
\newcommand{\to}{\rightarrow}

\usepackage{wasysym}
\usepackage{proof}
\usepackage{enumerate}
\usepackage{todonotes}
\usepackage{hyperref}
\usepackage{graphicx}

\let\b\relax
\let\d\relax
\let\t\relax

\renewcommand{\Box}{\oblong}

\newcommand{\F}{\mathop{\textbf{F}}}
\renewcommand{\P}{\mathop{\textbf{P}}}
\newcommand{\G}{\mathop{\textbf{G}}}
\renewcommand{\H}{\mathop{\textbf{H}}}

\newcommand{\cat}[1]{\mathcal{#1}}
\newcommand{\pd}[0]{\times}
\newcommand{\ihom}[0]{\rightarrow}
\newcommand{\st}[2]{\mathsf{st}_{#1,#2}}
\newcommand{\id}[0]{\mathsf{id}}
\newcommand{\b}[1]{\mathsf{b}_{#1}}
\newcommand{\d}[1]{\mathsf{d}_{#1}}
\newcommand{\m}[1]{\mathsf{m}_{#1}}
\newcommand{\n}[1]{\mathsf{n}_{#1}}
\newcommand{\p}[1]{\mathsf{p}_{#1}}
\newcommand{\q}[1]{\mathsf{q}_{#1}}
\newcommand{\t}[0]{\mathsf{t}}
\newcommand{\limp}[0]{\multimap}
\newcommand{\Hom}[3]{\mathsf{Hom}_{#1}(#2,#3)}

\newtheorem{theorem}{Theorem}
\newtheorem{lemma}[theorem]{Lemma}
\newtheorem{example}[theorem]{Example}
\newtheorem{fact}[theorem]{Fact}
\newtheorem{corollary}[theorem]{Corollary}
\newtheorem{definition}[theorem]{Definition}
\newtheorem{remark}[theorem]{Remark}
\newtheorem{proposition}[theorem]{Proposition}
\newtheorem{notn}[theorem]{Notation}
\newtheorem{observation}[theorem]{Observation}

\begin{document}

\maketitle

\section{Introduction}
{\tt to be added}

\section{Constructive Modal Logic}

We present two versions of  the intuionistic or constructive modal logic S4, to be written as CS4. 

We start by recalling the basic sequent calculus for intuitonistic propositional logic. Each of the logics presented in this section is an extension of the
single-conclusion formalization of Gentzen's intuitionistic sequent
calculus LJ.  The syntax of formulas for LJ is defined by the
following grammar:
\begin{center}
    \begin{math}
        \begin{array}{lllllllll}
            A & ::= & p \mid \perp \mid A \land A \mid A \lor A \mid A \to B
        \end{array}
    \end{math}
\end{center}
The formula $p$ is taken from a set of countably many propositional atoms. The constant $\top$ could be added, but it is the negation of the the falsum constant $\bot$.

We begin with an initial set of inference rules, and then each system
presented will be given as an extension of this initial system.  The  initial inference rules, which just model
propositional intuitionistic logic, are as follows:

\begin{center}
  \small
  \begin{math}
    \begin{array}{c}
      \begin{array}{cccccccc}
        \infer[\text{Id}]{\Delta,A \vdash A}{
          \,
        }
        & \quad &
        \infer[\text{Cut}]{\Gamma,\Delta \vdash C}{
          \Gamma \vdash B
          &
          B,\Delta \vdash C
        }
        & \quad & 
        \infer[\perp_{\mathcal{L}}]{\Gamma,\perp \; \vdash A}{
          \,
        }\\\\
        \infer[\lor_{\mathcal{L}}]{\Gamma,A \lor B \vdash C}{
          \Gamma,A \vdash C
          &
          \Gamma,B \vdash C
        }
        & \quad &
        \infer[\lor_{\mathcal{R}_1}]{\Gamma \vdash A \lor B}{
          \Gamma \vdash A
        }
        & \quad &
        \infer[\lor_{\mathcal{R}_2}]{\Gamma \vdash A \lor B}{
          \Gamma \vdash B
        }\\\\
        \infer[\land_{\mathcal{L}_1}]{\Gamma,A \land B \vdash C}{
          \Gamma,A \vdash C
        }
        & \quad &
        \infer[\land_{\mathcal{L}_2}]{\Gamma,A \land B \vdash C}{
          \Gamma,B \vdash C
        }
        & \quad &
        \infer[\land_{\mathcal{R}}]{\Gamma \vdash A \land B}{
          \Gamma \vdash A
          &
          \Gamma \vdash B
        }\\\\
        
      \end{array}
      \\
      \begin{array}{cccccccc}
        \infer[\to_{\mathcal{L}}]{\Gamma,A \to B \vdash C}{
          \Gamma \vdash A
          &
          \Gamma,B \vdash C
        }
        & \quad &
        \infer[\to_{\mathcal{R}}]{\Gamma \vdash A \to B}{
          \Gamma, A \vdash B
        }
      \end{array}        
    \end{array}
  \end{math}
\end{center}
Sequents denoted $\Gamma \vdash C$ consist of a multiset of formulas,
(written as either $\Gamma$, $\Delta$, or a numbered version of either), and a formula $C$. The intuitive meaning is that the conjunction of the formulas in $\Gamma$ entails the formula $C$.

\subsection{The constructive system S4} 

We recall the sequent calculus formalization of system
CS4.  This formalization  was proposed by Bierman and de Paiva \cite{CS4} and initially  was called  IS4 (for Intuitionistic system S4).
This system gives rise to a complete family of systems, which are
alternative to Simpson's intuitionistic systems. Since Simpson's
systems form a framework starting from intuitionistic K, that were also called
IK, IS4, IS5,   it made sense to call the family of systems originating
with the S4 above, the family of constructive modal systems CS, and
Bierman and de Paiva's system CS4.  This is for instance how
\cite{arisaka2015} describe these systems.

The main difference between systems IS4 and CS4 concerns the binary
distribution of possibility $\Diamond$ over disjunction (in binary and
nullary forms):
$$\Diamond (A \lor B) \to \Diamond A \lor \Diamond B$$
$$\Diamond \bot \to \bot$$ The system CS4 does not satisfy these
traditional distributions that are basic to classical modal
logic. Some philosophical significance can be attached to this
distribution, or lack thereof, and we discuss it later. First we
define the system.

The rules in Figure~\ref{fig:CS4}, in addition to the initial set of
inference rules, define the sequent calculus for CS4.
\begin{figure}
  \begin{mdframed}
    \begin{center}
      \begin{math}
        \begin{array}{ccccc}              
          \infer[\Box_{\mathcal{L}}]{\Gamma, \Box A \vdash B}{
            \Gamma,A \vdash B
          }
          & \quad &
          \infer[\Box_{\mathcal{R}}]{\Box\Gamma,\Delta \vdash \Box A}{
            \Box \Gamma \vdash A
          }\\\\
          \infer[\Diamond_{\mathcal{L}}]{\Delta,\Box\Gamma,\Diamond A \vdash \Diamond B}{
            \Box\Gamma,A \vdash \Diamond B
          }
          & \quad &
          \infer[\Diamond_{\mathcal{R}}]{\Gamma \vdash \Diamond A}{
            \Gamma \vdash A
          }
        \end{array}        
      \end{math}
    \end{center}
  \end{mdframed}
  \caption{CS4 modal rules}
  \label{fig:CS4}
\end{figure}
Note that we do have right rules and left  rules for introducing the new modal operators $\Box$ (necessity) and $\Diamond$ (posssibility), but these rules are not as well-behaved as the propositional ones. This system is indeed constructive, and  $\Box A$ is not logically equivalent to $\lnot \Diamond \lnot A$, and $\Diamond A$ is
not logically equivalent to $\lnot \Box \lnot A$.

This system has a nice proof theory, as far as modal logics are concerned. Bierman and de
Paiva \cite{CS4} show that it has a Hilbert-style presentation,  a Natural
Deduction presentation, as well as a sequent calculus one. The sequent calculus
satisfies cut-elimination, an old result from Ohnishi and Matsumoto
\cite{ohnishi1957}, as well as the subformula property.

The Natural Deduction formulation has a colourful history, described
in \cite{CS4}. One of its distinct features is that it was described
in Prawitz' seminal book in Natural Deduction \cite{prawitz1965},
hence it is still sometimes called Prawitz S4 modal logic.

Most interestingly the system has both Kripke and categorical
semantics, described respectively in \cite{alechinaetal} and
\cite{CS4}. Since we can prove a Curry-Howard correspondence for this
system, it has been used in several applications within programming
language theory \todo{Add references for these applications}.
%Examples include \cite{hmm}.


\subsection{The adjunction modal calculus}
This is not so well-known, but this system can  be given a
presentation in terms of a categorical adjunction, between two
cartesian closed categories.

This presentation is described  both in \cite{CS4} and in \cite{icalp1998}. In
\cite{CS4} this is called the multicontext formulation of CS4 and the
rules are given (page 17) in Figure~\ref{fig:ADJCS4}.
\begin{figure}
  \begin{mdframed}
    \begin{center}
      \begin{math}
        \begin{array}{ccccc}              
          \infer[\Box_{\mathcal{I}}]{\Gamma; \Delta \vdash \Box A}{
            \Gamma;\emptyset \vdash  A
          }
          & \quad &
          \infer[\Box_{\mathcal{E}}]{\Gamma;\Delta \vdash B}{\Gamma; \Delta \vdash \Box A \hspace{.1in}
            \Gamma, A;\Delta \vdash B
          }\\\\
          \infer[\Diamond_{\mathcal{I}}]{\Gamma;\Delta \vdash \Diamond A}{
            \Gamma;\Delta \vdash A
          }
          & \quad &
          \infer[\Diamond_{\mathcal{E}}]{\Gamma;\Delta \vdash \Diamond B}{
            \Gamma ;\Delta \vdash \Diamond A \hspace{.1in}\Gamma; A\vdash \Diamond B
          }
        \end{array}        
      \end{math}
    \end{center}
  \end{mdframed}
  \caption{The adjunction modal calculus}
  \label{fig:ADJCS4}
\end{figure}
(Note that the rules are Natural Deduction rules, as it should be clear from the fact that they are introduction and elimination rules.)

These rules have been shown by Benton \cite{benton1995} and Barber to correspond to an adjunction of the categories in the case of Linear Logic.  

\section{Categorical Models}
\label{sec:single_adjoint_model_of_cs4}
The main motivation for considering both systems of constructive modal logic described above is that we can provide  categorical semantics for them.

This will require  some categorical definitions that are not the most traditional ones, hence  these are provided in the appendix A, but discussed briefly here.  We will need the categorical concepts of 
\begin{enumerate}
    \item {(symmetric) monoidal category},
    \item {(symmetric) monoidal functor},
    \item {(symmetric) monoidal natural transformation},
    \item {(symmetric) monoidal adjunction},
    \item {monoidal comonad}, 
    \item {monoidal monad.}
\end{enumerate}  
%which we describe in the appendix A.

\subsection{Basic notions}
A  monoidal category (see definition 6 %\label{def:monoidal-category}
in the appendix A) is simply a category $\cat{C}$  equipped with a tensor product structure $(\otimes, I, \alpha, \rho, \lambda)$,  a generalization of traditional categorical products, satisfying two coherence conditions: MacLane's pentagon condition relating associativity of tensor to its bifunctorial structure, and the triangle condition relating the unit of the tensor $I$ to the natural transformations $\rho$ and $\lambda$,  plus the unit equation $\rho_I=\lambda_I$.
    
  A monoidal category can be symmetric, meaning that the tensor product $A\otimes B$ is isomorphic to $B\otimes A$, or not. If the monoidal category is symmetric we call $\beta$ (for  \textit{braid}) the isomorphism going from $ A\otimes B$ to $B\otimes A$. 
  Commutativity of the tensor product brings along  three new diagrams, see definition ?.
  
  While these are traditional diagrams that can be simplified, having them in full generality helps us to see how the strength (the new component we add) has to interact with the established structure.

A (symmetric) monoidal functor between two (symmetric) monoidal categories $\cat{C}$ and $\cat{D}$ is simply a functor $F\colon \cat{C} \mto \cat{D}$ that relates the monoidal structures, that is, there are natural transformations  $F(A)\otimes F(B)\mto F(A \otimes B)$, for each $A,B$ in $\cat{C}$ (as well as a map $m_{I}\colon I \mto FI$) that interact well with the monoidal structures  in $\cat{C}$ and  $\cat{D}$ (older books insist on an isomorphism between the tensor structures, we only require coherent natural transformations). This means another three commuting diagrams, one for associativity, one for left, one for right isos. Since the tensor product we consider is symmetric, one more commuting diagram is neccessary, preserving $\beta$, see definition 8 %\label{def:SMCFUN} 
in the appendix.

Then given monoidal functors $(F,m)$ and $(G,n)$ a monoidal natural transformation from $F$ to $G$ is a natural transformation which is compatible with the comparison maps $m$ and $n$, which means the two extra diagrams in definition 9 in the appendix commute too.

A (symmetric) monoidal adjunction $F \dashv G$ is an adjunction  where both functors $F,G$  are (symmetric) monoidal functors and the unit, and the co-unit are (symmetric) monoidal natural transformations too. 

Lastly, a (symmetric) monoidal comonad is a comonad $(\Box,\varepsilon, \delta, m)$ on a monoidal category $\cat{C}$, with extra structure preserving the monoidal structure. Thus $\Box\colon \cat{C}\to \cat{C}$ satisfies the  diagrams in definition 11 in appendix A.{\todo{add diagrams there}}
Similarly for a monoidal monad.

If, instead of being (symmetric) monoidal closed categories, they are cartesian closed categories, what happens? Which simplifications of the structures described do we get? Surely the results corresponding to Benton's  in \cite{benton1995} remains true, after all a cartesian closed category is also a special kind of symmetric monoidal closed category. But do we get anything else? We discuss this in appendix B.

\subsection{Modelling Modal sstems}
Now to model a constructive modal system like CS4 we first recall that to model simply intuitionistic propositions we need a category  $\cat{C}$ with products, coproducts and internal-homs, to model, respectively conjunctions, disjunctions and logic implications. 

In appendix B we discuss how to adapt the monoidal-like conditions to cartesian products.
Thus we recap that to model a necessity-like S4 operator we will need a (symmetric monoidal) comonad $(\Box,\varepsilon, \delta, m)$ on the cartesian closed category $\cat{C}$.

\begin{definition}[monoidal comonad]
  \label{def:monoidal comonad}
  Suppose $\cat{C}$ is a cartesian closed category. Then $(\Box, \varepsilon, \delta,m)$ is a \emph{(symmetric monoidal) comonad} on $\cat{C}$ if $(\Box, \varepsilon, \delta)$ is a (symmetric) comonad in $\cat{C}$ thais is monoidal, that is such that
  there exists a natural transformation:
  \[
  m_{A, B} : \Box A \pd \Box B \mto \Box( A \pd B)
  \]
  subject to the following  conditions:
  
   \begin{mathpar}
    \bfig
    \square|amma|<600,600>[
      \Box A`
      \Box ^2A`
      \Box ^2A`
      \Box ^3A;
      \delta_A`
      \delta_A`
      \Box\delta_A`
      \delta_{\Box A}]
    \efig
    \and
    \bfig
    \Atrianglepair/=`->`=`<-`->/<600,600>[
      \Box A`
      \Box A`
      \Box^2 A`
      \Box A;`
      \delta_A``
      \varepsilon_{\Box A}`
      \Box\varepsilon_A]
    \efig
  \end{mathpar}
  The assumption that $\varepsilon$ and $\delta$ are symmetric
  monoidal natural transformations amount to the following diagrams
  commuting:
  \begin{mathpar}
    \bfig
    \qtriangle|amm|/->`->`->/<1000,600>[
      \Box A \times \Box B`
      \Box (A \times B)`
      A \times B;
      \m{A,B}`
      \varepsilon_A \times \varepsilon_B`
    \varepsilon_{A \times B}]
    \efig
    \and
    \bfig
    \Vtriangle|amm|/->`<-`=/<600,600>[
      \Box 1`
      1`
      1;
      \m{I}`
      \varepsilon_1`]
    \efig    
  \end{mathpar}
  \begin{mathpar}
    \bfig
    \square|amab|/`->``->/<1050,600>[
      \Box A \times \Box B``
      \Box^2A \times \Box ^2B`
      \Box (\Box A \times \Box B);`
      \delta_A \times \delta_B``
      \m{\Box A,\Box B}]
    \square(1050,0)|mmmb|/``->`->/<1050,600>[`
      \Box (A \otimes B)`
      \Box (\Box A \times \Box B)`
      \Box ^2(A \times B);``
      \delta_{A \times B}`
      \Box \m{A,B}]
    \morphism(0,600)<2100,0>[\Box A \times \Box B`\Box (A \times B);\m{A,B}]
    \efig
    \and
    \bfig
    \square<600,600>[
      1`
      \Box 1`
      \Box 1`
      \Box^21;
      \m{1}`
      \m{1}`
      \delta_1`
      \Box \m{1}]
    \efig
  \end{mathpar}
  \end{definition}
  Note that the two first diagrams are the definition of a comonad. The two second diagrams are the interaction of the natural transformation $\varepsilon$ with the monoidal structure $m$ (in particular they show that $\Box 1$ and $1$ are equiprovable). The third last two diagrams are the interactions of the comultiplication $\delta$ with the monoidal structure $m$.
  
We also need to recall the notion of a functor-strong monad. In particular we are interested in $\Box$-strong monads. The notion was introduced in \cite{CS4}, but not much used since then. This structure is interesting, as it is much less symmetric than the ones we have dealt with so far.The strength

\begin{definition}[$\Box$-strong monad]
  \label{def:comonad-strong-monad}
  Suppose $(\Box, \varepsilon, \delta)$ is a (monoidal) comonad on a
  cartesian closed category $\cat{C}$.  Then a \emph{$\Box$-strong
    monad} is a monad $(\Diamond,\eta,\mu)$ on $\cat{C}$ such that
  there exists a natural transformation:
  \[
  \st{A}{B} : \Box A \pd \Diamond B \mto \Diamond(\Box A \pd B)
  \]
  subject to the following coherence conditions:
  (still need the diagrams for a monad here)
  \begin{center}
    \begin{math}      
      \begin{array}{lllll}
        \bfig
      \vSquares|ammmmma|/>``>```>`>/[\Box 1 \times \Diamond A`\Diamond (\Box 1 \times A)``\Diamond(1 \times A)`1 \times \Diamond A`\Diamond A;\st{1}{A}``\Diamond(\varepsilon_1 \times \id_A)```\Diamond\lambda`\lambda]
      \morphism(0,1000)|m|/->/<0,-950>[`;\varepsilon_1 \times \id_{\Diamond A}]
      \place(500,500)[(1)]
      \efig
      & \quad & 
      \bfig
      \btriangle<800,500>[\Box A \pd B`\Box A \pd \Diamond B`\Diamond(\Box A \pd B);\id_{\Box A} \pd \eta_{B}`\eta_{\Box A \times B}`\st_{A,B}]
      \place(200,200)[2]
      \efig
      \end{array}      
    \end{math}
    \\
    \vspace{30px}
    %% \begin{math}
    %%   \bfig
    %%   \hSquares|aamaaaa|/->`->`->``=`->`->/[
    %%     \Box A \pd \Diamond 1`
    %%     \Diamond (\Box A \pd 1)`
    %%     \Diamond\Box A`
    %%     \Box A \pd 1`
    %%     \Box A`
    %%     \Diamond\Box A;\st{A}{1}`\Diamond(\rho_{\Box A})`\id_A \pd \varepsilon```\rho`\eta_{\Box A}]
    %%   \efig
    %% \end{math}
    \begin{math}
      \bfig
      \hSquares|aamaaaa|/->`->`->```->`/[
        \Box A \times 1`
        \Box A \times \Diamond 1`
        \Diamond (\Box A \times 1)`
        \Box A`
        A`
        ;
        \id_{\Box A} \times \eta`
        \st{A}{1}`
        \rho```
        \varepsilon`]
      \qtriangle(1944,0)/->``->/<800,500>[\Diamond (\Box A \times 1)`\Diamond\Box A`\Diamond A;\Diamond \rho``\Diamond\varepsilon]
      \morphism(1040,0)/->/<1630,0>[`;\eta_A]
      \place(1350,250)[3]
      \efig
    \end{math}
    \\
    \vspace{30px}
    \begin{math}
      \bfig
        \vSquares|ammmmmm|/->`->`->```->`/[
          \Box A \times (\Box B \times \Diamond C)`
          \Box A \times \Diamond(\Box B \times C)`
          (\Box A \times \Box B) \times \Diamond C`
          \Diamond(\Box A \times (\Box B \times C))``
          \Diamond((\Box A \times \Box B) \times C);
          \id_{\Box A} \pd \st{B}{C}`
          \alpha^{-1}`
          \st{A}{\Box B \times C}```
          \Diamond\alpha^{-1}`]
        \morphism(1554,0)|m|/->/<0,-500>[`\Diamond(\Box(A \times B) \times C);\Diamond(\m{A,B} \times \id_C)]
        
        \morphism(0,500)|m|/->/<0,-1000>[`\Box(A \times B) \times \Diamond C;\m{A,B} \times \id_{\Diamond C}]

        \morphism(350,-500)|a|/->/<800,0>[`;\st{A \times B}{C}]
        \place(800,350)[4]
        \efig
    \end{math}
    \\
    \vspace{30px}
    \begin{math}
      \bfig
      \vSquares|ammmmma|/->`->```->``->/[
        \Box A \times \Diamond\Diamond B`
        \Box A \times \Diamond B`
        \Diamond(\Box A \times \Diamond B)``
        \Diamond\Diamond(\Box A \times B)`
        \Diamond(\Box A \times B);
        \id_{\Box A} \pd \mu_{B}`
        \st{A}{\Diamond B}```
        \Diamond(\st{A}{B})``
        \mu_{\Box A \pd B}]
      \morphism(1150,1000)|m|<0,-920>[`;\st{A}{B}]
      \place(600,500)[5]
      \efig
    \end{math}
    \\
    \vspace{30px}
    \begin{math}
      \bfig
      \hSquares|aamamaa|/``->``->`->`->/[
        \Box A \times \Diamond B``
        \Diamond B \times \Box A`
        \Diamond B \times \Box A`
        \Diamond B \times \Diamond \Box A`
        \Diamond (B \times \Box A);``
        \beta``
        \Diamond \beta`
        \id_{\Diamond B} \pd \eta_{\Box A}`
        \mathsf{n}_{B,\Box A}]
      \morphism(200,500)<1883,0>[`;\st{A}{B}]
      \place(1100,250)[6]
      \efig
    \end{math}
    \\
    \vspace{30px}
    \begin{math}
      \bfig
      \hSquares|aamamaa|/``->``->`->`->/[
        \Box A \times \Diamond B``
        \Diamond(\Box A \times B)`
        A \times \Diamond B`
        \Diamond A \times \Diamond B`
        \Diamond (A \times B);``
        \varepsilon_A \times \id_{\Diamond B}``
        \Diamond(\varepsilon_A \times \id_{B})`
        \eta_A \times \id_{\Diamond B}`
        \m{A,B}]
      \morphism(200,500)<1605,0>[`;\st{A}{B}]
      \place(1000,250)[7]
      \efig
    \end{math}        
  \end{center}
\end{definition}

Seven commuting diagrams is quite a lot of structure to require. We share the reader's feeling that we should be able to provide something simpler   and shall proceed to do so. But first we discuss why these seven commuting diagrams seem sensible.

\begin{definition}[modal category]
  \label{def:CS4-model}
  Suppose $\cat{C}$ is a cartesian category equipped with  a monoidal comonad $(\Box, \varepsilon, \delta)$ and a
  \emph{$\Box$-strong monad} with the strength natural transformation
  \[
  \st{A}{B} : \Box A \pd \Diamond B \mto \Diamond(\Box A \pd B)
  \]
  This is a model of CS4 as defined and showed sound and complete in \cite{CS4}. We call this structure a \textit{modal category}, as CS4 is the only kind of modality we consider in this note.
\end{definition}



We want to see how to transform a model of CS4 using a pair comonad-monad as above (i.e a modal category) into a model of the system using adjunctions.  Benton showed a similar result for linear type theory in \cite{benton1995}.


\begin{definition}[adjoint model]
  \label{def:CS4-single-adjoint-cat-model}
  An adjoint categorical model of CS4 consists of the following data:
  \begin{enumerate}
  \item A cartesian-closed category with coproducts $(\cat{C},1,0,\pd,+,\ihom)$;
  \item 
    A monoidal adjunction  $F \dashv G$, where $(F,m)$ and  $(G,n)\colon \cat{C} \mto \cat{C}$ are monoidal functors such that their composition $GF$ is a monoidal comonad, written as $\Box$;
 \item The  monad $(\Diamond, \eta, \mu, \st{A}{B})$, induced by the adjunction $F \dashv G$,   is $\Box$-strong.
  \end{enumerate}
\end{definition}

Now the existence of any (monoidal or not) adjunction provides us with a (monoidal) monad $FG$ in $\cat{C}$, as well as a (monoidal) comonad $\Box=GF$ on $\cat{C}$. Which conditions do we need in the adjunction to make this induced monad $\Box$-strong, if any?

Following Benton's description in page 5 of \cite{benton1995}, we recall (his lemma 2) that given a monoidal adjunction $F \dashv G$  we have, not only a comonad $(\Box, \varepsilon, \delta)$ but also a natural transformation, $q$, whose components $q_{A,B}\colon\Box A\land \Box B \mto \Box (A\land B)$ together with a map $m_{1}\colon 1 \mto \Box 1$ make $(\Box, q)$ a symmetric monoidal functor and $\varepsilon, \delta$ monoidal natural transformations.

Now we need to look at the monad functor in $\cat{C}$ arising from the adjunction $J \dashv H$, which we will call $\Diamond = HJ$ for obvious reasons. We need to show that $\Diamond$ is a monoidal monad, with unit given by $\eta\colon 1\mto JH$ and multiplication given by $\mu\colon  HJHJ\mto HJ$, where $\eta$ is the counit of the adjunction and $\mu$ has components given by $...HJ..$

%\textit{In the best possible scenario we don't need any conditions,the conditions on the monoidal adjunction do everything for us. in the worst scenario we have to add the morphisms that make a monad strong with respect to its associated comonad. I'm betting that we don't have to add anything.}

\begin{theorem}
  An adjoint CS4 categorical model is a modal category.
\end{theorem}

\iffalse
\begin{proof}
  We must show that given the definition of an adjoint CS4 categorical
  model (Definition~\ref{def:CS4-single-adjoint-cat-model}) we can
  define an appropriate monad and comonad on a CCC with coproducts
  where the monad is strong with respect to the comonad.

  Suppose $(H,m)$ and $(J,n)$ are the adjoint monoidal functors given
  in Definition~\ref{def:CS4-single-adjoint-cat-model}, and define
  $\Box = JH$ and $\Diamond = HJ$.  By definition we assumed that
  $(\Box, q)$, where $q_{A,B} : \Box A \times \Box B \to \Box (A
  \times B)$, is monoidal, but we must show that $\Diamond$ is also
  monoidal.  We know that both $(H,n)$ and $(J,m)$ are monoidal
  endofunctors on $\cat{C}$ which implies that their composition
  $\Diamond$ is monoidal where
  \[
  \begin{array}{lll}
    \mathsf{p}_{1} = \eta_{1} : 1 \to \Diamond 1\\
    \mathsf{p}_{A,B} = \m{HA,HB};J(\mathsf{n}_{A,B})
    \colon \Diamond A \pd \Diamond B \mto \Diamond(A \pd B)
  \end{array}
  \]
  and the following diagrams commute (proofs omitted):
  \begin{mathpar}
    \scriptsize
    \bfig
    \vSquares|ammmmma|/->`->`->``->`->`->/[
      (\Diamond A \times \Diamond B) \times \Diamond C`
      \Diamond A \times (\Diamond B \times \Diamond C)`
      \Diamond(A \times B) \times \Diamond C`
      \Diamond A \times \Diamond(B \times C)`
      \Diamond ((A \times B) \times C)`
      \Diamond (A \times (B \times C));
      \alpha`
      \mathsf{p}_{A,B} \times \id_{\Diamond C}`
      \id_{\Diamond A} \times \mathsf{p}_{B,C}``
      \mathsf{p}_{A \times B,C}`
      \mathsf{p}_{A,B \times C}`
      \Diamond \alpha]
    \efig
    \and
    \bfig
    \hSquares|ammmmaa|/->``->`<-``->`/[
      1 \times \Diamond A`
      \Diamond A``
      \Diamond 1 \times \Diamond A`
      \Diamond(1 \times A)`;
      \lambda_{\Diamond A}``
      \mathsf{p}_{1} \times \id_{\Diamond A}`
      \Diamond \lambda_A``
      \mathsf{p}_{1,A}`]
    \efig
    \and
    \bfig
    \hSquares|ammmmaa|/->``->`<-``->`/[
      \Diamond A \times 1`
      \Diamond A``
      \Diamond A \times \Diamond 1`
      \Diamond(A \times 1)`;
      \rho_{\Diamond A}``
      \id_{\Diamond A} \times \mathsf{p}_{1}`
      \Diamond \rho_A``
      \mathsf{p}_{A,1}`]
    \efig
    \and
    \bfig
    \hSquares|ammmmaa|/->``->`->``->`/[
      \Diamond A \times \Diamond B`
      \Diamond B \times \Diamond A``
      \Diamond (A \times B)`
      \Diamond (B \times A)`;
      \beta_{\Diamond A,\Diamond B}``
      \mathsf{p}_{A,B}`
      \mathsf{p}_{B,A}``
      \Diamond\beta_{A,B}`]
    \efig
  \end{mathpar}

  Furthermore, suppose $J \dashv H$, where the unit, $\varepsilon :
  \Box A \to A$, and the counit, $\eta : A \to \Diamond A$, are
  monoidal natural transformations.  This implies that the following
  diagrams commute:
  \begin{mathpar}
    \bfig
    \btriangle<800,500>[A \times B`\Diamond A \times \Diamond B`\Diamond (A \times B);\eta_A \times \eta_B`\eta_{A \times B}`\mathsf{p}_{A,B}]
    \efig
    \and
    \bfig
    \qtriangle<800,500>[\Box A \times \Box B`\Box (A \times B)`A \times B;\mathsf{q}_{A,B}`\varepsilon_A \times \varepsilon_B`\varepsilon_{A \times B}]
    \efig
    \and
    \bfig
    \qtriangle<800,500>[1`J 1`\Diamond 1;n_{1}`\eta_1`J m_1]       
    \efig
    \and
    \bfig
    \hSquares|ammmmaa|/->``=`->``<-`/[
      \Box 1`
      1``
      \Box 1`
      H 1`;
      \varepsilon_1```
      m_1``
      H n_1`]
    \efig
    \and
    \bfig
    \qtriangle<800,500>[H A`H\Box A`H A;\eta_{H A}`\id_{H A}`H\varepsilon_A]       
    \efig
    \and
    \bfig
    \qtriangle<800,500>[J A`\Box J A`J A;J\eta_A`\id_{J A}`\varepsilon_{J A}]
    \efig    
  \end{mathpar}
  It is a well-known fact about adjoints that $(\Box, \varepsilon,
  \delta)$, where $\delta : \Box A \to \Box\Box A$ is a comonad, and
  $(\Diamond, \eta, \mu)$, where $\mu : \Diamond\Diamond A \to
  \Diamond A$ is a monad.  In addition, $\mu$ and $\delta$ are monoidal
  natural transformations where we have the following:
  \[
  \begin{array}{lll}
    \d{1} = \p{1};\Diamond\p{1} : 1 \mto \Diamond^2 1\\
    \d{A,B} =  \p{\Diamond A,\Diamond B};\Diamond\p{A,B} : \Diamond^2 A \times \Diamond^2 B \mto \Diamond^2 (A \times B)\\
    \\
    \b{1} = \q{1};\Box\q{1} : 1 \mto \Box^2 1\\
    \b{A,B} = \q{\Box A,\Box B};\Box\q{A,B} : \Box^2 A \times \Box^2 B \mto \Box^2 (A \times B)\\
  \end{array}
  \]
  Thus, the following diagrams commute:
  \begin{mathpar}
    \bfig
    \hSquares|ammmmaa|/->``->`->``->`/[
      \Diamond^3 A`
      \Diamond^2 A``
      \Diamond^2 A`
      \Diamond A`;
      \Diamond \mu_A``
      \mu_{\Diamond A}`
      \mu_A``
      \mu_A`]
    \efig
    \and
    \bfig
    \qtriangle/->`=`->/<800,500>[\Diamond A`\Diamond^2 A`\Diamond A;\eta_{\Diamond A}``\mu_A]
    \btriangle(0,0)/->`=`->/<800,500>[\Diamond A`\Diamond^2 A`\Diamond A;\Diamond \eta_{A}``\mu_A]
    \efig
    \and
    \bfig
    \hSquares|ammmmaa|/->``->`->``->`/[
      \Box A`
      \Box^2 A``
      \Box^2 A`
      \Box^3 A`;
      \delta_A``
      \delta_A`
      \delta_{\Box A}``
      \Box\delta_A`]
    \efig
    \and
    \bfig
    \qtriangle/->`=`->/<800,500>[\Box A`\Box^2 A`\Box A;\delta_A``\Box \varepsilon]
    \btriangle(0,0)/->`=`->/<800,500>[\Box A`\Box^2 A`\Box A;\delta_A``\varepsilon_{\Box A}]
    \efig
    \and
    %% \bfig   
    %% \vSquares|ammmmma|/->`->```->``->/[
    %%   \Diamond^2 A \times \Diamond^2 B`
    %%   \Diamond A \times \Diamond B`
    %%   \Diamond(\Diamond A \times \Diamond B)``
    %%   \Diamond^2(A \times B)`
    %%   \Diamond(A \times B);
    %%   \mu_A \times \mu_B`
    %%   \p{\Diamond A,\Diamond B}```
    %%   \Diamond\p{A,B}``
    %%   \mu_{A \times B}]
    %% \morphism(1108,0)/<-/<0,1000>[\Diamond(A \times B)`\Diamond A \times \Diamond B;\p{A,B}]
    %% \efig
    \and
    \bfig
    \square<1000,1000>[
      \Diamond^2 A \times \Diamond^2 B`
      \Diamond A \times \Diamond B`
      \Diamond^2(A \times B)`
      \Diamond(A \times B);
      \mu_A \times \mu_B`
      \d{A,B}`
      \p{A,B}`
      \mu_{A \times B}]
    \efig
    \and
    \bfig
    \Vtriangle/->`<-`<-/[
      \Diamond^2 1`
      \Diamond 1`
      1;
      \mu_1`
      \d{1}`
      \p{1}]
    \efig
    \and        
    \bfig
    \square<1000,1000>[
      \Box A \times \Box B`
      \Box^2 A \times \Box^2 B`
      \Box(A \times B)`
      \Box^2(A \times B);
      \delta_A \times \delta_B`
      \q{A,B}`
      \b{A,B}`
      \delta_{A \times B}]
    %% \vSquares|ammmmma|/->``->```->`->/[
    %%   \Box A \times \Box B`
    %%   \Box^2 A \times \Box^2 B``
    %%   \Box(\Box A \times \Box B)`
    %%   \Box(A \times B)`
    %%   \Box^2(A \times B);
    %%   \varepsilon_A \times \varepsilon_B``
    %%   \q{\Box A,\Box B}```
    %%   \Box\q{A,B}`
    %%   \varepsilon_{A \times B}]
    %% \morphism(0,0)/<-/<0,1000>[\Box(A \times B)`\Box A \times \Box B;\q{A,B}]
    \efig
    \and
    \bfig
    \Vtriangle/->`<-`<-/[
      \Box 1`
      \Box^2 1`
      1;
      \delta_1`
      \q{1}`
      \d{1}]
    \efig
  \end{mathpar}

  We can now define the $\Box$-strength map as follows:
  \[
  \st{A}{B} = (\eta_{\Box A} \pd \id_{\Diamond B});\mathsf{p}_{\Box A,B} : \Box A \pd \Diamond B \mto \Diamond(\Box A \pd B)
  \]
  We can see that $\st{A}{B}$ is a natural transformation, because it
  is defined as a composition of natural transformations.
  
  %% To prove that the appropriate diagrams commute we first note that
  %% the triangle  
  Next we must show that all of the appropriate diagrams given in
  Definition~\ref{def:comonad-strong-monad} commute.
  \begin{itemize}
  \item[] \textit{Case 1. the object 1 behaves as the unit for products}
    $$
    \bfig
    \vSquares|ammmmma|/>``>```>`>/[\Box 1 \times \Diamond A`\Diamond (\Box 1 \times A)``\Diamond(1 \times A)`1 \times \Diamond A`\Diamond A;\st{1}{A}``\Diamond(\varepsilon_1 \times \id_A)```\Diamond\lambda`\lambda]
    \morphism(0,1000)|m|/->/<0,-950>[`;\varepsilon_1 \times \id_{\Diamond A}]
    \efig
    $$
    This diagram commutes by commutativity of the following diagram:
    %% Equational version:
    %% \begin{center}
    %%   \begin{math}
    %%     \begin{array}{rllllllll}
    %%       & & \st{1}{A};\Diamond (\varepsilon_1 \times \id_A);\Diamond\lambda_A\\
    %%       \text{(Definition of $\mathsf{st}$)}
    %%       & = & (\eta_{\Box 1} \times \id_{\Diamond A});\p{\Box 1,\Diamond A};\Diamond (\varepsilon_1 \times \id_A);\Diamond\lambda_A\\
    %%       \text{(Naturality of $\mathsf{p}$)}
    %%       & = & (\eta_{\Box 1} \times \id_{\Diamond A});(\Diamond \varepsilon_1 \times \Diamond\id_A);\p{1,A};\Diamond\lambda_A\\
    %%       \text{(Functoriality of $\times$)}
    %%       & = & ((\eta_{\Box 1};\Diamond \varepsilon_1) \times (\id_{\Diamond A};\Diamond\id_A));\p{1,A};\Diamond\lambda_A\\
    %%       & = & ((\eta_{\Box 1};\Diamond \varepsilon_1) \times (\id_{\Diamond A};\id_{\Diamond A}));\p{1,A};\Diamond\lambda_A\\
    %%       \text{(Naturality of $\eta$)}
    %%       & = & ((\varepsilon_1;\eta_{1}) \times (\id_{\Diamond A};\id_{\Diamond A}));\p{1,A};\Diamond\lambda_A\\
    %%       \text{(Definition of $\mathsf{p}$)}
    %%       & = & ((\varepsilon_1;\p{1}) \times (\id_{\Diamond A};\id_{\Diamond A}));\p{1,A};\Diamond\lambda_A\\
    %%       \text{(Functoriality of $\times$)}
    %%       & = & (\varepsilon_1 \times \id_{\Diamond A});(\p{1} \times \id_{\Diamond A});\p{1,A};\Diamond\lambda_A\\
    %%       \text{($\Diamond$ is Symmetric Monoidal)}
    %%       & = & (\varepsilon_1 \times \id_{\Diamond A});\lambda_{\Diamond A}\\
    %%     \end{array}
    %%   \end{math}
    %% \end{center}
    $$
    \bfig
    \square|amma|<1000,500>[
      \Diamond\Box 1 \times \Diamond A`
      \Diamond (\Box 1 \times A)`
      \Diamond 1 \times \Diamond A`
      \Diamond (1 \times A);
      \p{\Box 1,A}`
      \Diamond \varepsilon_1 \times \id_{\Diamond A}`
      \Diamond (\varepsilon_1 \times \id_A)`
      \p{1,A}]

    \square(-1000,0)|amma|<1000,500>[
      \Box 1 \times \Diamond A`
      \Diamond\Box 1 \times \Diamond A`
      1 \times \Diamond A`
      \Diamond 1 \times \Diamond A;
      \eta_{\Box 1} \times \id_{\Diamond A}`
      \varepsilon_1 \times \id_{\Diamond A}`
      \Diamond \varepsilon_1 \times \id_{\Diamond A}`
      (\p{1} = \eta_1) \times \id_{\Diamond A}]

    \qtriangle(-1000,-500)|mmm|/`->`->/<2000,500>[
      1 \times \Diamond A`
      \Diamond (1 \times A)`
      \Diamond A;`
      \lambda_{\Diamond A}`
      \Diamond\lambda_A]

    \place(500,250)[1]
    \place(-500,250)[2]
    \place(500,-200)[3]
    \efig
    $$
    \noindent
    Diagram 1 commutes by naturality of $\mathsf{p}$, diagram 2
    commutes by naturality of $\eta$, and diagram 3 commutes because
    $\Diamond$ is a symmetric monoidal functor.

  \item[] \textit{Case 2. unit $\eta$ of the monad and strength interact well, $\Box  A $ is a parameter}
    $$
    \bfig
    \btriangle<800,500>[
      \Box A \pd B`
      \Box A \pd \Diamond B`
      \Diamond(\Box A \pd B);
      \id_{\Box A} \pd \eta_{B}`
      \eta_{\Box A \times B}`
      \st{A}{B}]
    \efig
    $$

    The previous diagram commutes, because the following diagram commutes:
    $$
    \bfig
    \btriangle|ama|/->`->`->/<1500,500>[
      \Box A \pd B`
      \Box A \pd \Diamond B`
      \Diamond\Box A \pd \Diamond B;
      \id_{\Box A} \pd \eta_{B}`
      \eta_{\Box A} \pd \eta_B`
      \eta_{\Box A} \pd \id_{\Diamond B}]

    \qtriangle(0,0)/->``<-/<1500,500>[
      \Box A \pd B`
      \Diamond (\Box A \times B)`
      \Diamond\Box A \pd \Diamond B;
      \eta_{\Box A \times B}``
      \p{\Box A,B}]

    \place(250,200)[1]
    \place(1200,300)[2]
    \efig
    $$
    \noindent
    Diagram 1 clearly commutes, and diagram 2 commutes because $\eta$
    is a symmetric monoidal natural transformation.
    
  \item[] \textit{Case 3. co-unit of the comonad $\varepsilon$ and unit of the monad $\eta$ interact well?}
    $$
    \bfig
    \hSquares|aamaaaa|/->`->`->```->`/[
      \Box A \times 1`
      \Box A \times \Diamond 1`
      \Diamond (\Box A \times 1)`
      \Box A`
      A`
      ;
      \id_{\Box A} \times \eta_1`
      \st{A}{1}`
      \rho_{\Box A}```
      \varepsilon_A`]
    \qtriangle(1978,0)/->``->/<800,500>[\Diamond (\Box A \times 1)`\Diamond\Box A`\Diamond A;\Diamond \rho_{\Box A}``\Diamond\varepsilon_A]
    \morphism(1060,0)/->/<1640,0>[`;\eta_A]
    \efig
    $$
    \noindent
    Recall that
    $\st{A}{1} = (\eta_{\Box A} \pd \id_{\Diamond 1});\p{\Box A,1}$.
    Now the previous diagram commutes, because the following diagram commutes:
    $$
    \bfig
    \btriangle|mma|<1444,1000>[\Box A \pd 1`\Diamond\Box A \pd \Diamond 1`\Diamond(\Box A \pd 1);\eta_{\Box A} \pd \eta_1`\eta_{\Box A \pd 1}`\p{\Box A,1}]
    \dtriangle(-1000,0)/->``->/<1000,1000>[\Box A \pd 1`\Box A \pd 1`\Diamond\Box A \pd \Diamond 1;\id_{\Box A} \pd \eta_1``\eta_{\Box A} \pd
      \id_{\Diamond 1}]

    \hSquares(0,0)/->`->```<-``/<1000>[\Box A \pd 1`\Box A`\Diamond\Box A```\Diamond (\Box A \times 1);\rho_{\Box A}`\eta_{\Box A}```\Diamond (\rho_{\Box A})``]

    \square(746,1000)/->`<-`<-`/<698,500>[A`\Diamond A`\Box A`\Diamond\Box A;\eta_A`\varepsilon_A`\Diamond\varepsilon_A`]

    \place(-400,300)[1]
    \place(400,300)[2]
    \place(1100,700)[3]
    \place(1100,1250)[4]
    \efig
    $$
    \noindent
    Diagram 1 commutes by functorality of $\times$, diagram 2 commutes
    because $\eta$ is a monoidal natural transformation, and diagrams
    3 and 4 commute by naturality of $\eta$.

  \item[] \textit{Case 4. associativity $\alpha$ interacts with co-monoidicity of $\Box$}
    $$
    \bfig
    \vSquares|ammmmmm|/->`->`->```->`/[
      \Box A \times (\Box B \times \Diamond C)`
      \Box A \times \Diamond(\Box B \times C)`
      (\Box A \times \Box B) \times \Diamond C`
      \Diamond(\Box A \times (\Box B \times C))``
      \Diamond((\Box A \times \Box B) \times C);
      \id_{\Box A} \pd \st{B}{C}`
      \alpha^{-1}`
      \st{A}{\Box B \times C}```
      \Diamond\alpha^{-1}`]
    \morphism(1554,0)|m|/->/<0,-500>[`\Diamond(\Box(A \times B) \times C);\Diamond(\m{A,B} \times \id_C)]
    
    \morphism(0,500)|m|/->/<0,-1000>[`\Box(A \times B) \times \Diamond C;\m{A,B} \times \id_{\Diamond C}]

    \morphism(350,-500)|a|/->/<800,0>[`;\st{A \times B}{C}]
    \efig
    $$
    \noindent
    Recall that:
    \[
    \begin{array}{rlll}
      \st{B}{C}              & = & (\eta_{\Box B} \pd \id_{\Diamond C});\p{\Box B,C}\\
      \st{A \pd B}{C}        & = & (\eta_{\Box (A \pd B)} \pd \id_{\Diamond C});\p{\Box (A \pd B),C}\\
      \st{A}{\Box B \pd C} & = & (\eta_{\Box A} \pd \id_{\Diamond (\Box B \pd C)});\p{\Box A,(\Box B \pd C)}\\
    \end{array}
    \]
    In addition, we require the following diagram (whose commutativity
    is implied by the fact that $\Diamond$ is a symmetric monoidal
    functor):
    $$
    \bfig
    \vSquares|ammmmma|/->`->`->``->`->`->/[
      \Diamond A \pd (\Diamond B \pd \Diamond C)`
      (\Diamond A \pd \Diamond B) \pd \Diamond C`
      \Diamond A \pd \Diamond (B \pd C)`
      \Diamond (A \pd B) \pd \Diamond C`
      \Diamond (A \pd (B \pd C))`
      \Diamond ((A \pd B) \pd C);
      \alpha^{-1}_{\Diamond A,\Diamond B,\Diamond C}`
      \id_{\Diamond A} \pd \p{B,C}`
      \p{A,B} \pd \id_{\Diamond C}``
      \p{A,B \pd C}`
      \p{A \pd B,C}`
      \Diamond\alpha^{-1}_{A,B,C}]
    \efig
    $$
    \noindent
    Finally, this case follows because the following diagram commutes:
    \begin{center}
      \rotatebox{90}{$
    \bfig
    \btriangle|mmm|<1769,1000>[
      \Box A \pd (\Diamond\Box B \pd \Diamond C)`
      \Diamond\Box A \pd (\Diamond\Box B \pd \Diamond C)`
      \Diamond\Box A \pd (\Diamond\Box B \pd C);
      \eta_{\Box A} \pd \id_{\Diamond \Box B \pd C}`
      \eta_{\Box A} \pd \p{\Box B,C}`
      \id_{\Diamond\Box A} \pd \p{\Box B,C}]

    \qtriangle|mam|/->``->/<1769,1000>[
      \Box A \pd (\Diamond\Box B \pd \Diamond C)`
      \Box A \pd \Diamond (\Box B \pd C)`
      \Diamond\Box A \pd (\Diamond\Box B \pd C);
      \id_{\Box A} \pd \p{\Box B,C}``
      \eta_{\Box A} \pd \id_{\Diamond (\Box B \pd C)}]    

    \qtriangle(-1800,0)|mmm|<1800,1000>[
      \Box A \pd (\Box B \pd \Diamond C)`
      \Box A \pd (\Diamond\Box B \pd \Diamond C)`
      \Diamond\Box A \pd (\Diamond\Box B \pd \Diamond C);
      \id_{\Box A} \pd (\eta_{\Box B} \pd \id_{\Diamond C})`
      \eta_{\Box A} \pd (\eta_{\Box B} \pd \id_{\Diamond C})`
      \eta_{\Box A} \pd \id_{\Diamond \Box B \pd C}]

    \square(0,-500)|mmmm|/`->`->`/<1769,500>[
      \Diamond\Box A \pd (\Diamond\Box B \pd \Diamond C)`
      \Diamond\Box A \pd (\Diamond\Box B \pd C)`
      (\Diamond\Box A \pd \Diamond\Box B) \pd \Diamond C`
      \Diamond (\Box A \pd (\Box B \pd C));`
      \alpha_{\Diamond\Box A,\Diamond\Box B,\Diamond C}`
      \p{\Box A,\Box B \pd C}`]   

    \square(0,-1000)|mmmm|/`->`->`/<1769,500>[
      (\Diamond\Box A \pd \Diamond\Box B) \pd \Diamond C`
      \Diamond (\Box A \pd (\Box B \pd C))`
      \Diamond (\Box A \pd \Box B) \pd \Diamond C`
      \Diamond ((\Box A \pd \Box B) \pd C);`
      \p{\Box A,\Box B} \pd \id_{\Diamond C}`
      \Diamond \alpha_{\Box A,\Box B,C}`]        

    \square(0,-1500)|mmmm|<1769,500>[
      \Diamond (\Box A \pd \Box B) \pd \Diamond C`
      \Diamond ((\Box A \pd \Box B) \pd C)`
      \Diamond\Box(A \pd B) \pd \Diamond C`
      \Diamond (\Box(A \pd B) \pd C);
      \p{\Box A \pd \Box B, C}`
      \Diamond \m{A,B} \pd \id_{\Diamond C}`
      \Diamond (\m{A,B} \pd \id_C)`
      \p{\Box (A \pd B),C}]

    \btriangle(-1800,-500)|mmm|/->``->/<1800,1500>[
      \Box A \pd (\Box B \pd \Diamond C)`
      (\Box A \pd \Box B) \pd \Diamond C`
      (\Diamond\Box A \pd \Diamond\Box B) \pd \Diamond C;
      \alpha_{\Box A,\Box B,\Diamond C}``
      (\eta_{\Box A} \pd \eta_{\Box B}) \pd \id_{\Diamond C}]

    \qtriangle(-1800,-1000)|mmm|/`->`/<1800,500>[
      (\Box A \pd \Box B) \pd \Diamond C`
      (\Diamond\Box A \pd \Diamond\Box B) \pd \Diamond C`
      \Diamond (\Box A \pd \Box B) \pd \Diamond C;`
      \eta_{\Box A \pd \Box B} \pd \id_{\Diamond C}`]

    \btriangle(-1800,-1500)|mmm|/->``->/<1800,1000>[
      (\Box A \pd \Box B) \pd \Diamond C`
      \Box(A \pd B) \pd \Diamond C`
      \Diamond\Box(A \pd B) \pd \Diamond C;
      \m{A,B} \pd \id_{\Diamond C}``
      \eta_{\Box (A \pd B)} \pd \id_{\Diamond C}]

    \place(1200,700)[1]
    \place(500,300)[2]
    \place(900,-500)[3]
    \place(900,-1250)[4]
    \place(-500,700)[5]
    \place(-1000,0)[6]
    \place(-400,-700)[7]
    \place(-1000,-1100)[8]
    \efig
    $}
    \end{center}
    Diagrams 1, 2 and 5 commute by functorality of $\times$, diagram 3
    commutes by the additional diagram from above, diagram 4 commutes
    by naturality of $\mathsf{p}$, diagram 6 commutes by naturality of
    $\alpha$, diagram 7 commutes by the fact that $\eta$ is a monoidal
    natural transformation, and diagram 8 commutes by naturality of
    $\eta$.
    

  \item[] \textit{Case. 5.  strength interacts with monoidicity of $\Diamond$}
    $$
    \bfig
    \vSquares|ammmmma|/->`->```->``->/[
      \Box A \times \Diamond\Diamond B`
      \Box A \times \Diamond B`
      \Diamond(\Box A \times \Diamond B)``
      \Diamond\Diamond(\Box A \times B)`
      \Diamond(\Box A \times B);
      \id_{\Box A} \pd \mu_{B}`
      \st{A}{\Diamond B}```
      \Diamond(\st{A}{B})``
      \mu_{\Box A \pd B}]
    \morphism(1150,1000)|m|<0,-920>[`;\st{A}{B}]
    \efig
    $$
    \noindent
    Recall that:
    \[
    \begin{array}{rlll}
      \st{A}{B} & = & (\eta_{\Box A} \pd \id_{\Diamond B});\p{\Box A,B}\\
      \st{A}{\Diamond B} & = & (\eta_{\Box A} \pd \id_{\Diamond \Diamond B});\p{\Box A,\Diamond B}\\
    \end{array}
    \]
    This case follows from the fact that the following diagram
    commutes:
    \begin{center}
      \rotatebox{90}{$\bfig
    \qtriangle|mmm|<1500,1000>[
      \Box A \pd \Diamond\Diamond B`
      \Box A \pd \Diamond B`
      \Diamond\Box A \pd \Diamond B;
      \id_{\Box A} \pd \mu_B`
      \eta_{\Box A} \pd \mu_B`
      \eta_{\Box A} \pd \id_{\Diamond B}]

    \morphism(-1500,1000)|m|/<-/<1500,0>[
      \Diamond\Box A \pd \Diamond\Diamond B`
      \Box A \pd \Diamond\Diamond B;
      \eta_{\Box A} \pd \id_{\Diamond\Diamond B}]

    \btriangle(-1500,0)|mmm|<3000,1000>[
      \Diamond\Box A \pd \Diamond\Diamond B`
      \Diamond\Diamond\Box A \pd \Diamond\Diamond B`
      \Diamond\Box A \pd \Diamond B;
      \Diamond\eta_{\Box A} \pd \id_{\Diamond\Diamond B}`
      \id_{\Diamond\Box A} \pd \mu_B`
      \mu_{\Box A} \pd \mu_B]

    \square(-3000,0)|mmmm|/<-`->``<-/<1500,1000>[
      \Diamond(\Box A \pd \Diamond B)`
      \Diamond\Box A \pd \Diamond\Diamond B`
      \Diamond(\Diamond\Box A \pd \Diamond B)`
      \Diamond\Diamond\Box A \pd \Diamond\Diamond B;
      \p{\Box A,\Diamond B}`
      \Diamond (\eta_{\Box A} \pd \id_{\Diamond B})``
      \p{\Diamond\Box A,\Diamond B}]

    \square(-3000,-500)|mmmm|/`->`->`->/<4500,500>[
      \Diamond (\Diamond\Box A \pd \Diamond B)`
      \Diamond\Box A \pd \Diamond B`
      \Diamond\Diamond (\Box A \pd B)`
      \Diamond (\Box A \pd B);`
      \Diamond (\p{\Box A,B})`
      \p{\Box A,B}`
      \mu_{\Box A \pd B}]

    \place(-2300,500)[1]
    \place(-800,-250)[2]
    \place(-800,400)[3]
    \place(0,700)[4]
    \place(1050,700)[5]
    \efig$}
    \end{center}       
    Diagram commutes by naturality of $\mathsf{p}$, diagram 2 commutes
    because $\mu$ is a monoidal natural transformation, diagram 3
    commutes because $\mu$ is the monadic multiplication and by
    functorality of $\times$, and diagrams 4 and 5 commute by
    functoriality of $\times$.
    
  \item[] \textit{Case 6. commuting $\beta$ interacts with $\Diamond$}
    $$
    \bfig
    \hSquares|aamamaa|/``->``->`->`->/[
      \Diamond B \times \Box A``
      \Box A \times \Diamond B`
      \Diamond B \times \Diamond \Box A`
      \Diamond (B \times \Box A)`
      \Diamond (\Box A \times B);``
      \id_{\Diamond B} \pd \eta_{\Box A}``
      st_{A,B}`
      \p{B,\Box A}`
      \Diamond\beta_{B,\Box A}]
    \morphism(200,500)<1817,0>[`;\beta_{\Diamond B,\Box A}]
    \efig
    $$
    \noindent
    The previous diagram commutes by commutativity of the following
    diagram:
    $$
    \bfig
    \vSquares|ammmmma|[
      \Diamond B \times \Box A`
      \Box A \times \Diamond B`
      \Diamond B \times \Diamond\Box A`
      \Diamond\Box A \times \Diamond B`
      \Diamond (B \times \Box A)`
      \Diamond (\Box A \times B);
      \beta_{\Diamond B,\Box A}`
      \id_{\Diamond B} \times \eta_{\Box A}`
      \eta_{\Box A} \times \id_{\Diamond B}`
      \beta_{\Diamond B,\Diamond\Box A}`
      \p{B,\Box A}`
      \p{\Box A,B}`
      \Diamond\beta_{B,\Box A}]

    \place(600,750)[1]
    \place(600,250)[2]
    \efig
    $$
    \noindent
    Diagram 1 commutes because $\beta$ is a symmetric monoidal
    functor, and diagram 2 commutes by naturality of $\beta$.

  \item[] \textit{Case 7.  $\varepsilon$ interacts with $\Diamond$ and its monoidicity} 
    $$
    \bfig
    \hSquares|aamamaa|/``->``->`->`->/[
      \Box A \times \Diamond B``
      \Diamond(\Box A \times B)`
      A \times \Diamond B`
      \Diamond A \times \Diamond B`
      \Diamond (A \times B);``
      \varepsilon_A \times \id_{\Diamond B}``
      \Diamond(\varepsilon_A \times \id_{B})`
      \eta_A \times \id_{\Diamond B}`
      \p{A,B}]
    \morphism(200,500)<1605,0>[`;\st{A}{B}]
    \efig
    $$
    \noindent
    The previous diagram commutes by commutativity of the following
    diagram:
    $$
    \bfig
    \hSquares|aammmaa|[
      \Box A \times \Diamond B`
      \Diamond\Box A \times \Diamond B`
      \Diamond (\Box A \times B)`
      A \times \Diamond B`
      \Diamond A \times \Diamond B`
      \Diamond (A \times B);
      \eta_{\Box A} \times \id_{\Diamond B}`
      \p{\Box A,B}`
      \varepsilon_A \times \id_{\Diamond B}`
      \Diamond\varepsilon_A \times \id_{\Diamond B}`
      \Diamond (\varepsilon_A \times \id_B)`
      \eta_A \times \id_{\Diamond B}`
      \p{A,B}]
    \place(1750,250)[1]
    \place(600,250)[2]
    \efig
    $$
    \noindent
    Diagram 1 commutes by naturality of $\mathsf{p}$, and diagram 2
    commutes by naturality of $\eta$.
  \end{itemize}

\end{proof}
\fi 


%\textit{If we're lucky, I'm hoping we also have}
Now we would like to, following Benton's lead, 
prove the converse result. That is
\begin{theorem}
  A CS4 categorical model can be extended to a CS4 adjoint categorical model.
\end{theorem}
\begin{proof}
  Suppose $\cat{C}$ is a cartesian category equipped with a monoidal
  comonad $(\Box, \varepsilon, \delta)$ and a \emph{$\Box$-strong
    monad} (Definition~\ref{def:comonad-strong-monad}) with the
  strength natural transformation
  \[
  \st{A}{B} : \Box A \pd \Diamond B \mto \Diamond(\Box A \pd B)
  \]
  
\end{proof}
This is similar to section 2.2.2 in Benton's work \cite{benton1995},
which shows that a \textit{linear category implies a LNL model}.

We know that both the comonad and the monad we have in the definition of a modal category can each provides us with an infinite collection of adjunctions, between the Co-Keisli and the Eilenberg-Moore categories. Thus we need to choose the appropriate adjunction and prove that is monoidal. This is the hard proof in the linear logic case.

% subsection single_adjoint_model_of_cs4 (end)

\section{Conclusion}
%``I always thought something was fundamentally wrong with the universe'' 

\bibliographystyle{plain}
\bibliography{references}

\appendix
\section{Monoidal  Categories}
\label{sec:symmetric_monoidal_closed_categories}

\begin{definition}
  \label{def:monoidal-category}
  A \textbf{symmetric monoidal category (SMC)} is a category, $\cat{M}$,
  with the following data:
  \begin{itemize}
  \item An object $I$ of $\cat{M}$,
  \item A bi-functor $\otimes : \cat{M} \times \cat{M} \mto \cat{M}$,
  \item The following natural isomorphisms:
    \[
    \begin{array}{lll}
      \lambda_A : I \otimes A \mto A\\
      \rho_A : A \otimes I \mto A\\      
      \alpha_{A,B,C} : (A \otimes B) \otimes C \mto A \otimes (B \otimes C)\\
    \end{array}
    \]
  \item A symmetry natural transformation:
    \[
    \beta_{A,B} : A \otimes B \mto B \otimes A
    \]
  \item Subject to the following coherence diagrams:
    \begin{mathpar}
      \bfig
      \vSquares|ammmmma|/->`->```->``<-/[
        ((A \otimes B) \otimes C) \otimes D`
        (A \otimes (B \otimes C)) \otimes D`
        (A \otimes B) \otimes (C \otimes D)``
        A \otimes (B \otimes (C \otimes D))`
        A \otimes ((B \otimes C) \otimes D);
        \alpha_{A,B,C} \otimes \id_D`
        \alpha_{A \otimes B,C,D}```
        \alpha_{A,B,C \otimes D}``
        \id_A \otimes \alpha_{B,C,D}]      
      
      \morphism(1433,1000)|m|<0,-1000>[
        (A \otimes (B \otimes C)) \otimes D`
        A \otimes ((B \otimes C) \otimes D);
        \alpha_{A,B \otimes C,D}]
      \efig
      \and
      \bfig
      \hSquares|aammmaa|/->`->`->``->`->`->/[
        (A \otimes B) \otimes C`
        A \otimes (B \otimes C)`
        (B \otimes C) \otimes A`
        (B \otimes A) \otimes C`
        B \otimes (A \otimes C)`
        B \otimes (C \otimes A);
        \alpha_{A,B,C}`
        \beta_{A,B \otimes C}`
        \beta_{A,B} \otimes \id_C``
        \alpha_{B,C,A}`
        \alpha_{B,A,C}`
        \id_B \otimes \beta_{A,C}]
      \efig      
    \end{mathpar}
    \begin{mathpar}
      \bfig
      \Vtriangle[
        (A \otimes I) \otimes B`
        A \otimes (I \otimes B)`
        A \otimes B;
        \alpha_{A,I,B}`
        \rho_{A}`
        \lambda_{B}]
      \efig
      \and
      \bfig
      \btriangle[
        A \otimes B`
        B \otimes A`
        A \otimes B;
        \beta_{A,B}`
        \id_{A \otimes B}`
        \beta_{B,A}]
      \efig
      \and
      \bfig
      \Vtriangle[
        I \otimes A`
        A \otimes I`
        A;
        \beta_{I,A}`
        \lambda_A`
        \rho_A]
      \efig
    \end{mathpar}    
  \end{itemize}
\end{definition}


\begin{definition}
  \label{def:SMCC}
  A \textbf{symmetric monoidal closed category (SMCC)} is a symmetric
  monoidal category, $(\cat{M},I,\otimes)$, such that, for any object
  $B$ of $\cat{M}$, the functor $- \otimes B : \cat{M} \mto \cat{M}$
  has a specified right adjoint.  Hence, for any objects $A$ and $C$
  of $\cat{M}$ there is an object $A \limp B$ of $\cat{M}$ and a
  natural bijection:
  \[
  \Hom{\cat{M}}{A \otimes B}{C} \cong \Hom{\cat{M}}{A}{B \limp C}
  \]
\end{definition}

A \textit{cartesian closed category} is a symmetric monoidal closed category whose tensor product is a cartesian product and its unit $I$ is a real terminal object 1.

\begin{definition}
  \label{def:SMCFUN}
  Suppose we are given two symmetric monoidal closed categories $(\cat{M}_1,I_1,\otimes_1,\alpha_1,\lambda_1,\rho_1,\beta_1)$ and
  $(\cat{M}_2,I_2,\otimes_2,\alpha_2,\lambda_2,\rho_2,\beta_2)$.  Then a
  \textbf{symmetric monoidal functor} is a functor $F : \cat{M}_1 \mto
  \cat{M}_2$, a map $m_I : I_2 \mto FI_1$ and a natural transformation
  $m_{A,B} : FA \otimes_2 FB \mto F(A \otimes_1 B)$ subject to the
  following coherence conditions:
  \begin{mathpar}
    \bfig
    \vSquares|ammmmma|/->`->`->``->`->`->/[
      (FA \otimes_2 FB) \otimes_2 FC`
      FA \otimes_2 (FB \otimes_2 FC)`
      F(A \otimes_1 B) \otimes_2 FC`
      FA \otimes_2 F(B \otimes_1 C)`
      F((A \otimes_1 B) \otimes_1 C)`
      F(A \otimes_1 (B \otimes_1 C));
      {\alpha_2}_{FA,FB,FC}`
      m_{A,B} \otimes \id_{FC}`
      \id_{FA} \otimes m_{B,C}``
      m_{A \otimes_1 B,C}`
      m_{A,B \otimes_1 C}`
      F{\alpha_1}_{A,B,C}]
    \efig
    \end{mathpar}
%    \and
\begin{mathpar}
    \bfig
    \square|amma|/->`->`<-`->/<1000,500>[
      I_2 \otimes_2 FA`
      FA`
      FI_1 \otimes_2 FA`
      F(I_1 \otimes_1 A);
      {\lambda_2}_{FA}`
      m_{I} \otimes \id_{FA}`
      F{\lambda_1}_{A}`
      m_{I_1,A}]
    \efig
    \and
    \bfig
    \square|amma|/->`->`<-`->/<1000,500>[
      FA \otimes_2 I_2`
      FA`
      FA \otimes_2 FI_1`
      F(A \otimes_1 I_1);
      {\rho_2}_{FA}`
      \id_{FA} \otimes m_{I}`
      F{\rho_1}_{A}`
      m_{A,I_1}]
    \efig
     \end{mathpar}
     
      \begin{mathpar}
    \bfig
    \square|amma|/->`->`->`->/<1000,500>[
      FA \otimes_2 FB`
      FB \otimes_2 FA`
      F(A \otimes_1 B)`
      F(B \otimes_1 A);
      {\beta_2}_{FA,FB}`
      m_{A,B}`
      m_{B,A}`
      F{\beta_1}_{A,B}]
    \efig
  \end{mathpar}
\end{definition}
A \textit{product  functor} is a symmetric monoidal closed functor between (symmetric) cartesian closed categories. The map $m_1\colon 1\to F1$ and the natural transformations $m_{A,B} : FA \times FB \mto F(A \times B)$ are subject to the adapted coherence conditions:
  \begin{mathpar}
    \bfig
    \vSquares|ammmmma|/->`->`->``->`->`->/[
      (FA \times FB) \times FC`
      FA \times (FB \times FC)`
      F(A \times B) \times FC`
      FA \times F(B \times C)`
      F((A \times B) \times C)`
      F(A \times (B \times C));
      {\alpha_2}_{FA,FB,FC}`
      m_{A,B} \times \id_{FC}`
      \id_{FA} \times m_{B,C}``
      m_{A \times B,C}`
      m_{A,B \times C}`
      F{\alpha_1}_{A,B,C}]
    \efig
    \end{mathpar}
%    \and
\begin{mathpar}
    \bfig
    \square|amma|/->`->`<-`->/<1000,500>[
      1 \times FA`
      FA`
      F1 \times FA`
      F(1 \times A);
      {\lambda_2}_{FA}`
      m_{1} \times \id_{FA}`
      F{\lambda_1}_{A}`
      m_{1,A}]
    \efig
    \and
    \bfig
    \square|amma|/->`->`<-`->/<1000,500>[
      FA \times 1`
      FA`
      FA \times F1`
      F(A \times 1);
      {\rho_2}_{FA}`
      \id_{FA} \times m_{1}`
      F{\rho_1}_{A}`
      m_{A,1}]
    \efig
     \end{mathpar}
     
      \begin{mathpar}
    \bfig
    \square|amma|/->`->`->`->/<1000,500>[
      FA \times FB`
      FB \times FA`
      F(A \times B)`
      F(B \times A);
      {\beta_2}_{FA,FB}`
      m_{A,B}`
      m_{B,A}`
      F{\beta_1}_{A,B}]
    \efig
  \end{mathpar}
Because every cartesian closed category is a symmetric monoidal category where the tensor is a real product and because   cartesian products are unique up to isomorphism, we know we can rewrite these diagrams somewhat.

{\tt how to get to your product functors from here?}

\begin{definition}
  \label{def:SMCNAT}
  Suppose $(\cat{M}_1,I_1,\otimes_1)$ and $(\cat{M}_2,I_2,\otimes_2)$
  are SMCs, and $(F,m)$ and $(G,n)$ are a symmetric monoidal functors
  between $\cat{M}_1$ and $\cat{M}_2$.  Then a \textbf{symmetric
    monoidal natural transformation} is a natural transformation,
  $f : F \mto G$, subject to the following coherence diagrams:
  \begin{mathpar}
    \bfig
    \square<1000,500>[
      FA \otimes_2 FB`
      F(A \otimes_1 B)`
      GA \otimes_2 GB`
      G(A \otimes_1 B);
      m_{A,B}`
      f_A \otimes_2 f_B`
      f_{A \otimes_1 B}`
      n_{A,B}]
    \efig
    \and
    \bfig
    \Vtriangle/->`<-`<-/[
      FI_1`
      GI_1`
      I_2;
      f_{I_1}`
      m_{I_1}`
      n_{I_1}]
    \efig
  \end{mathpar}  
\end{definition}
A \textit{product natural transformation} does not simplify things much
 \begin{mathpar}
    \bfig
    \square<1000,500>[
      FA \times FB`
      F(A \times B)`
      GA \times GB`
      G(A \times B);
      m_{A,B}`
      f_A \times f_B`
      f_{A \times B}`
      n_{A,B}]
    \efig
    \and
    \bfig
    \Vtriangle/->`<-`<-/[
      F1`
      G1`
      1;
      f_{1}`
      m_{1}`
      n_{1}]
    \efig
  \end{mathpar}  
  
\begin{definition}
  \label{def:SMCADJ}
  Suppose $(\cat{M}_1,I_1,\otimes_1)$ and $(\cat{M}_2,I_2,\otimes_2)$
  are SMCs, and $(F,m)$ is a symmetric monoidal functor between
  $\cat{M}_1$ and $\cat{M}_2$ and $(G,n)$ is a symmetric monoidal
  functor between $\cat{M}_2$ and $\cat{M}_1$.  Then a
  \textbf{symmetric monoidal adjunction} is an ordinary adjunction
  $\cat{M}_1 : F \dashv G : \cat{M}_2$ such that the unit,
  $\varepsilon : A \to GFA$, and the counit, $\eta_A : FGA \to A$, are
  symmetric monoidal natural transformations.  Thus, the following
  diagrams must commute:
  \begin{mathpar}
    \bfig
    \qtriangle|amm|<1000,500>[
      FGA \otimes_1 FGB`
      FG(A \otimes_1 B)`
      A \otimes_1 B;
      \q{A,B}`
      \eta_A \otimes_1 \eta_B`
      \eta_{A \otimes_1 B}]
    \efig
    \and
    \bfig
    \Vtriangle|amm|/->`<-`=/[
      FGI_1`
      I_1`
      I_1;
      \eta_{I_1}`
      \q{I_1}`]
    \efig
    \and
    \bfig
    \dtriangle|mmb|<1000,500>[
      A \otimes_2 B`
      GFA \otimes_2 GFB`
      GF(A \otimes_2 B);
      \varepsilon_A \otimes_2 \varepsilon_B`
      \varepsilon_{A \otimes_2 B}`
      \p{A,B}]
    \efig
    \and
    \bfig
    \Vtriangle|amm|/->`=`<-/[
      I_2`
      GFI_2`
      I_2;
      \varepsilon_{I_2}``
      p_{I_2}]
    \efig
  \end{mathpar}
  Note that $\p{}$ and $\q{}$ exist because $(FG,\q{})$ and
  $(GF,\p{})$ are symmetric monoidal functors.
\end{definition}

Also a \textit{product preserving adjunction} is not much simpler. The diagrams become:
\begin{mathpar}
    \bfig
    \qtriangle|amm|<1000,500>[
      FGA \times FGB`
      FG(A \times B)`
      A \times B;
      \q{A,B}`
      \eta_A \times \eta_B`
      \eta_{A \times B}]
    \efig
    \and
    \bfig
    \Vtriangle|amm|/->`<-`=/[
      FG1`
      1`
      1;
      \eta_{1}`
      \q{1}`]
    \efig
    \end{mathpar}
 \begin{mathpar}   
    \bfig
    \dtriangle|mmb|<1000,500>[
      A \times B`
      GFA \times GFB`
      GF(A \times B);
      \varepsilon_A \times \varepsilon_B`
      \varepsilon_{A \times B}`
      \p{A,B}]
    \efig
    \and
    \bfig
    \Vtriangle|amm|/->`=`<-/[
      1`
      GF1`
      1;
      \varepsilon_{1}``
      p_{1}]
    \efig
  \end{mathpar}
\begin{definition}
  \label{def:symm-monoidal-monad}
  A \textbf{symmetric monoidal monad} on a symmetric monoidal
  category $\cat{C}$ is a triple $(T,\eta, \mu)$, where
  $(T,\n{})$ is a symmetric monoidal endofunctor on $\cat{C}$,
  $\eta_A : A \mto TA$ and $\mu_A : T^2A \to TA$ are
  symmetric monoidal natural transformations, which make the following
  diagrams commute:
  \begin{mathpar}
    \bfig
    \square|ammb|<600,600>[
      T^3 A`
      T^2A`
      T^2A`
      TA;
      \mu_{TA}`
      T\mu_A`
      \mu_A`
      \mu_A]
    \efig
    \and
    \bfig
    \Atrianglepair/=`<-`=`->`<-/<600,600>[
      TA`
      TA`
      T^2 A`
      TA;`
      \mu_A``
      \eta_{TA}`
      T\eta_A]
    \efig
  \end{mathpar}
  The assumption that $\eta$ and $\mu$ are symmetric
  monoidal natural transformations amount to the following diagrams
  commuting:
  \begin{mathpar}
    \bfig
    \dtriangle|mmb|<1000,600>[
      A \otimes B`
      TA \otimes TB`
      T(A \otimes B);
      \eta_A \otimes \eta_B`
      \eta_A`
      \n{A,B}]    
    \efig
    \and
    \bfig
    \Vtriangle/->`=`<-/<600,600>[
      I`
      TI`
      I;
      \eta_I``
      \n{I}]
    \efig
  \end{mathpar}
  \begin{mathpar}
    \bfig
    \square|ammm|/->`->``/<1050,600>[
      T^2 A \otimes T^2 B`
      T(TA \otimes TB)`
      TA \otimes TB`;
      \n{TA,TB}`
      \mu_A \otimes \mu_B``]

    \square(850,0)|ammm|/->``->`/<1050,600>[
      T(TA \otimes TB)`
      T^2(A \otimes B)``
      T(A \otimes B);
      T\n{A,B}``
      \mu_{A \otimes B}`]
    \morphism(-200,0)<2100,0>[TA \otimes TB`T(A \otimes B);\n{A,B}]
    \efig
    \and
    \bfig
    \square|ammb|/->`->`->`<-/<600,600>[
      I`
      TI`
      TI`
      T^2I;
      \n{I}`
      \n{T}`
      T\n{I}`
      \mu_I]
    \efig
  \end{mathpar}
\end{definition}

Note that the first two diagrams for a \textit{product preserving monad} on a ccc, are exactly the same. The next four diagrams are slightly simplified
\begin{mathpar}
    \bfig
    \dtriangle|mmb|<1000,600>[
      A \times B`
      TA \times TB`
      T(A \times B);
      \eta_A \times \eta_B`
      \eta_A`
      \n{A,B}]    
    \efig
    \and
    \bfig
    \Vtriangle/->`=`<-/<600,600>[
      1`
      T1`
      1;
      \eta_1``
      \n{1}]
    \efig
  \end{mathpar}
  \begin{mathpar}
    \bfig
    \square|ammm|/->`->``/<1050,600>[
      T^2 A \times T^2 B`
      T(TA \times TB)`
      TA \times TB`;
      \n{TA,TB}`
      \mu_A \times \mu_B``]

    \square(850,0)|ammm|/->``->`/<1050,600>[
      T(TA \times TB)`
      T^2(A \times B)``
      T(A \times B);
      T\n{A,B}``
      \mu_{A \times B}`]
    \morphism(-200,0)<2100,0>[TA \times TB`T(A \times B);\n{A,B}]
    \efig
    \and
    \bfig
    \square|ammb|/->`->`->`<-/<600,600>[
      1`
      T1`
      T1`
      T^21;
      \n{1}`
      \n{T}`
      T\n{1}`
      \mu_1]
    \efig
  \end{mathpar}
Finally the dual concept, of a symmetric monoidal comonad, just for completeness.
\begin{definition}
  \label{def:symm-monoidal-comonad}
  A \textbf{symmetric monoidal comonad} on a symmetric monoidal
  category $\cat{C}$ is a triple $(T,\varepsilon, \delta)$, where
  $(T,\m{})$ is a symmetric monoidal endofunctor on $\cat{C}$,
  $\varepsilon_A : TA \mto A$ and $\delta_A : TA \to T^2 A$ are
  symmetric monoidal natural transformations, which make the following
  diagrams commute:
  \begin{mathpar}
    \bfig
    \square|amma|<600,600>[
      TA`
      T^2A`
      T^2A`
      T^3A;
      \delta_A`
      \delta_A`
      T\delta_A`
      \delta_{TA}]
    \efig
    \and
    \bfig
    \Atrianglepair/=`->`=`<-`->/<600,600>[
      TA`
      TA`
      T^2 A`
      TA;`
      \delta_A``
      \varepsilon_{TA}`
      T\varepsilon_A]
    \efig
  \end{mathpar}
  The assumption that $\varepsilon$ and $\delta$ are symmetric
  monoidal natural transformations amount to the following diagrams
  commuting:
  \begin{mathpar}
    \bfig
    \qtriangle|amm|/->`->`->/<1000,600>[
      TA \otimes TB`
      T(A \otimes B)`
      A \otimes B;
      \m{A,B}`
      \varepsilon_A \otimes \varepsilon_B`
    \varepsilon_{A \otimes B}]
    \efig
    \and
    \bfig
    \Vtriangle|amm|/->`<-`=/<600,600>[
      TI`
      I`
      I;
      \m{I}`
      \varepsilon_I`]
    \efig    
  \end{mathpar}
  \begin{mathpar}
    \bfig
    \square|amab|/`->``->/<1050,600>[
      TA \otimes TB``
      T^2A \otimes T^2B`
      T(TA \otimes TB);`
      \delta_A \otimes \delta_B``
      \m{TA,TB}]
    \square(1050,0)|mmmb|/``->`->/<1050,600>[`
      T(A \otimes B)`
      T(TA \otimes TB)`
      T^2(A \otimes B);``
      \delta_{A \otimes B}`
      T\m{A,B}]
    \morphism(0,600)<2100,0>[TA \otimes TB`T(A \otimes B);\m{A,B}]
    \efig
    \and
    \bfig
    \square<600,600>[
      I`
      TI`
      TI`
      T^2I;
      \m{I}`
      \m{I}`
      \delta_I`
      T\m{I}]
    \efig
  \end{mathpar}
\end{definition}
% section symmetric_monoidal_categories (end)

\section{Cartesian Closed Categories and their Adjunctions}
\label{sec:cartesian_closed_categories_and_symmetric_monoidal_adjunctions}
In this appendix we consider simplifications that arise when the categories
in a symmetric monoidal adjunction are cartesian closed.  The first
big simplification is that when a SMCC is cartesian we no longer have
to specify all of the SMC structure, because some of the structure becomes provable.
The following lemma explains this.

\begin{lemma}[cccs are smccs]
  \label{lemma:CCC-is-SMC}
  Suppose $(\cat{C}, 1, \times, \Rightarrow)$ is a CCC.  Then the
  following defines the structure of an SMC:
  \[
  \begin{array}{rll}
    \lambda_A & = & \pi_2 : 1 \times A \mto A\\
    \lambda^{-1}_A & = & \langle \t_A , \id_A  \rangle : A \mto 1 \times A\\
    \\
    \rho_A & = & \pi_1 : A \times 1 \mto A\\
    \rho^{-1}_A & = & \langle \id_A , \t_A  \rangle : A \mto A \times 1\\
    \\
    \alpha_{A,B,C} & = & \langle \pi_1;\pi_1, \pi_2 \times \id_C \rangle : (A \times B) \times C \mto A \times (B \times C)\\
    \alpha^{-1}_{A,B,C} & = & \langle \id_A \times \pi_1, \pi_2;\pi_2 \rangle : A \times (B \times C) \mto (A \times B) \times C\\
    \\
    \beta_{A,B} & = & \langle \pi_2, \pi_1 \rangle : A \times B \mto B \times A\\
  \end{array}
  \]
  Each of the above morphisms satisfy the appropriate diagrams.
\end{lemma}

It turns out that we can also simplify the definition of a symmetric
monoidal functor between cartesian categories.

\begin{definition}
  \label{def:prod-functor}
  A \textbf{product functor}, $(F,m) : \cat{C}_1 \to \cat{C}_2$,
  between two cartesian categories consists of an ordinary functor $F
  : \cat{C}_1 \to \cat{C}_2$, a natural transformation $m_{A,B} : FA
  \times FB \mto F(A \times B)$, and a map $m_1 : 1 \to F1$ subject to the
  following coherence diagrams:
  \begin{mathpar}
    \bfig
    \btriangle|mmm|<1000,500>[
      FA \times FB`
      F(A \times B)`
      FA;
      m_{A.B}`
      \pi_1`
      F\pi_1]
    \efig
    \and
    \bfig    
    \btriangle|mmm|<1000,500>[
      FA \times FB`
      F(A \times B)`
      FB;
      m_{A.B}`
      \pi_2`
      F\pi_2]
    \efig
    \and
    \bfig
    \btriangle|mmm|<1000,500>[
      FC`
      FA \times FB`
      F(A \times B);
      \langle Ff, Fg \rangle`
      F(\langle f, g \rangle)`
      m_{A,B}]
    \efig
    \and
    \bfig
    \btriangle|mmm|<1000,500>[
      FA`
      1`
      F1;
      \t_{FA}`
      F\t_{A}`
      m_1]
    \efig
  \end{mathpar} 
\end{definition}

\begin{lemma}[Product Functors are Symmetric Monoidal]
  \label{lemma:product_functors_are_symmetric_monoidal}
  If $(F,m) : \cat{C}_1 \to \cat{C}_2$ is a product functor betwen cartesian categories, then
  $(F,m)$ is a symmetric monoidal functor.
\end{lemma}

{\tt Harley, surely the idea is that any symmetric monoidal functor between cccs becomes a product functor? I think it's clear that a product functor is a symm monoidal functor, no? the same way it's clear that a ccc is an smcc, lemma 12 only shows how to see it.}

\iffalse
\begin{proof}
  We simply show that each diagram in the definition of symmetric
  monoidal functor commutes.  This proof requires the following facts
  about cartesian categories:
  \[
  \begin{array}{lll}
    f;\langle g , h \rangle = \langle f;g, f;g \rangle & \text{(P1)}\\
    \\
    f \times g = \langle \pi_1;f , \pi_2;g \rangle & \text{(P2)}\\
    \\
    \langle f , g \rangle;(h \times i) = \langle f;h, g;i \rangle & \text{(P3)}\\
    \\
    (f \times g);\pi_1 = \pi_1;f & \text{(P4)}\\
    \\
    (f \times g);\pi_2 = \pi_2;g & \text{(P5)}\\
  \end{array}
  \]
  \begin{itemize}
  \item[] Associativity:\ \\
    \[
    \bfig
    \vSquares|ammmmma|/->`->`->``->`->`->/[
      (FA \times FB) \times FC`
      FA \times (FB \times FC)`
      F(A \times B) \times FC`
      FA \times F(B \times C)`
      F((A \times B) \times C)`
      F(A \times (B \times C));
      {\alpha}_{FA,FB,FC}`
      m_{A,B} \times \id_{FC}`
      \id_{FA} \times m_{B,C}``
      m_{A \times B,C}`
      m_{A,B \times C}`
      F{\alpha}_{A,B,C}]
    \efig
    \]
    In this case it is easiest to show that each side of the diagram
    reduces to the same morphism.

    First, the right side reduces in the following way:
    \begin{center}
      \begin{math}
        \begin{array}{lll}
          \alpha_{FA,FB,FC};(\id_{FA} \times m_{B,C});m_{A,B \times C} \\
          \,\,\,\,\,\,\,\,= \langle \pi_1;\pi_1, \pi_2 \times \id_{FC} \rangle;(\id_{FA} \times m_{B,C});m_{A,B \times C} & \text{(Definition)}\\
          \,\,\,\,\,\,\,\,= \langle \pi_1;\pi_1, (\pi_2 \times \id_{FC});m_{B,C} \rangle;m_{A,B \times C} & \text{(P3)}\\
        \end{array}
      \end{math}
    \end{center}

    Now the left side reduces to the same morphism:
    \[
    \scriptsize
    \begin{array}{lll}
      (m_{A,B} \times \id_{FC});m_{A \times B,C};\underline{F\alpha_{A,B,C}} \\
    \,\,\,\,\,\,\,\,= (m_{A,B} \times \id_{FC});m_{A \times B,C};\underline{F(\langle \pi_1;\pi_1, \pi_2 \times \id_C \rangle)} & \text{(Definition)}\\
   \,\,\,\,\,\,\,\,= (m_{A,B} \times \id_{FC});\underline{m_{A \times B,C};\langle F\pi_1;F\pi_1, F(\pi_2 \times \id_C) \rangle;m_{A,B \times C}} & \text{(Product Functor)}\\
   \,\,\,\,\,\,\,\,= (m_{A,B} \times \id_{FC});\langle \underline{m_{A \times B,C};F\pi_1};F\pi_1, m_{A \times B,C};F(\pi_2 \times \id_C) \rangle;m_{A,B \times C} & \text{(P1)}\\
   \,\,\,\,\,\,\,\,= (m_{A,B} \times \id_{FC});\langle \pi_1;F\pi_1, m_{A \times B,C};\underline{F(\pi_2 \times \id_C)} \rangle;m_{A,B \times C} & \text{(Product Functor)}\\
   \,\,\,\,\,\,\,\,= (m_{A,B} \times \id_{FC});\langle \pi_1;F\pi_1, m_{A \times B,C};\underline{F(\langle \pi_1;\pi_2, \pi_2 \rangle)} \rangle;m_{A,B \times C} & \text{(P2)}\\
   \,\,\,\,\,\,\,\,= (m_{A,B} \times \id_{FC});\langle \pi_1;F\pi_1, \underline{m_{A \times B,C};\langle F\pi_1;F\pi_2, F\pi_2 \rangle};m_{B,C} \rangle;m_{A,B \times C} & \text{(Product Functor)}\\
   \,\,\,\,\,\,\,\,= (m_{A,B} \times \id_{FC});\langle \pi_1;F\pi_1, \langle \underline{m_{A \times B,C};F\pi_1};F\pi_2, \underline{m_{A \times B,C};F\pi_2} \rangle;m_{B,C} \rangle;m_{A,B \times C} & \text{(P1)}\\
   \,\,\,\,\,\,\,\,= (m_{A,B} \times \id_{FC});\langle \pi_1;F\pi_1, \underline{\langle \pi_1;F\pi_2, \pi_2 \rangle};m_{B,C} \rangle;m_{A,B \times C} & \text{(Product Functor)}\\
   \,\,\,\,\,\,\,\,= \underline{(m_{A,B} \times \id_{FC});\langle \pi_1;F\pi_1, (F\pi_2 \times \id_{FC});m_{B,C} \rangle};m_{A,B \times C} & \text{(P2)}\\
   \,\,\,\,\,\,\,\,= \langle \underline{(m_{A,B} \times \id_{FC});\pi_1};F\pi_1, (m_{A,B} \times \id_{FC});(F\pi_2 \times \id_{FC});m_{B,C} \rangle;m_{A,B \times C} & \text{(P1)}\\
   \,\,\,\,\,\,\,\,= \langle \pi_1;\underline{m_{A,B};F\pi_1}, (m_{A,B} \times \id_{FC});(F\pi_2 \times \id_{FC});m_{B,C} \rangle;m_{A,B \times C} & \text{(P4)}\\
   \,\,\,\,\,\,\,\,= \langle \pi_1;\pi_1, \underline{(m_{A,B} \times \id_{FC});(F\pi_2 \times \id_{FC})};m_{B,C} \rangle;m_{A,B \times C} & \text{(Product Functor)}\\
   \,\,\,\,\,\,\,\,= \langle \pi_1;\pi_1, (\underline{(m_{A,B};F\pi_2)} \times \id_{FC});m_{B,C} \rangle;m_{A,B \times C} & \text{(Functor)}\\
   \,\,\,\,\,\,\,\,= \langle \pi_1;\pi_1, (\pi_2 \times \id_{FC});m_{B,C} \rangle;m_{A,B \times C} & \text{(Product Functor)}\\
    \end{array}
    \]

  \item[] Left Unitor:\\
    \[
    \bfig
    \square|amma|/->`->`<-`->/<1000,500>[
      1 \times FA`
      FA`
      F1 \times FA`
      F(1 \times A);
      {\lambda}_{FA}`
      m_{1} \times \id_{FA}`
      F{\lambda}_{A}`
      m_{1,A}]
    \efig
    \]
  
  First, notice that $\lambda_{FA} = \pi_2$. We can show the following:
  \[
  \begin{array}{lll}
    (m_1 \times \id_{FA});m_{1,A};F\lambda_A\\
    \,\,\,\,\,\,\,\,= (m_1 \times \id_{FA});m_{1,A};F\pi_2 & \text{(Definition)}\\
    \,\,\,\,\,\,\,\,= (m_1 \times \id_{FA});\pi_2 & \text{(Product Functor)}\\
    \,\,\,\,\,\,\,\,= \pi_2 & \text{(P5)}\\
  \end{array}
  \]

\item[] Right Unitor:\\
  \[
  \bfig
  \square|amma|/->`->`<-`->/<1000,500>[
    FA \times 1`
    FA`
    FA \times F1`
    F(A \times 1);
    {\rho}_{FA}`
    \id_{FA} \times m_{1}`
    F{\rho}_{A}`
    m_{A,1}]
  \efig
  \]
  \end{itemize}
  This case is similar to the previous one:
  \[
  \begin{array}{lll}
    (\id_{FA} \times m_1);m_{A,1};F\rho_A\\
    \,\,\,\,\,\,\,\,=  (\id_{FA} \times m_1);m_{A,1};F\pi_1 & \text{(Definition)}\\
    \,\,\,\,\,\,\,\,=  (\id_{FA} \times m_1);\pi_1 & \text{(Product Functor)}\\
    \,\,\,\,\,\,\,\,=  \pi_1 & \text{(P4)}\\
    \,\,\,\,\,\,\,\,=  \rho_{FA} & \text{(Definition)}\\
  \end{array}
  \]

\item[] Symmetry:\\
  \[
  \bfig
  \square|amma|/->`->`->`->/<1000,500>[
    FA \times FB`
    FB \times FA`
    F(A \times B)`
    F(B \times A);
    {\beta}_{FA,FB}`
    m_{A,B}`
    m_{B,A}`
    F{\beta}_{A,B}]
  \efig
  \]
  First, notice that $\beta_{A,B} = \langle \pi_2, \pi_1 \rangle$.  Thus, we have the following:
  \[
  \begin{array}{lll}
    \beta_{FA,FB};m_{B,A}\\
    \,\,\,\,\,\,\,\,=  \langle \pi_2, \pi_1 \rangle;m_{B,A} & \text{(Definition)}\\
    \,\,\,\,\,\,\,\,=  \langle m_{A,B};F\pi_2, m_{A,B};F\pi_1 \rangle;m_{B,A} & \text{(Product Functor)}\\
    \,\,\,\,\,\,\,\,=  m_{A,B};\langle F\pi_2, F\pi_1 \rangle;m_{B,A} & \text{(P1)}\\
    \,\,\,\,\,\,\,\,=  m_{A,B};F(\langle \pi_2, \pi_1 \rangle) & \text{(Product Functor)}\\
    \,\,\,\,\,\,\,\,=  m_{A,B};F\beta_{A,B} & \text{(Definition)}\\
  \end{array}
  \]
\end{proof}
\fi 

{\tt I am starting out as strict as possible, but I don't see how to
  prove the first case of the proof below without $m_1$ being an iso.}

\begin{lemma}[Semi-Strong Symmetric Monoidal Functors are Product Functors]
  \label{lemma:semi-strong-symmetric_monoidal_functors_are_product_functors}
  If $(F,m) : \cat{C}_1 \mto \cat{C}_2$ is a symmetric monoidal
  functor between cccs and $m_1 : 1 \to F1$ is an isomorphism, then
  $(F,m)$ is a product functor.
\end{lemma}
\begin{proof}
  \ \\
  \begin{itemize}
  \item[] Case 1.\\
    \[
      \bfig
      \btriangle|mmm|<1000,500>[
        FA`
        1`
        F1;
        \t_{FA}`
        F\t_{A}`
        m_1]
      \efig
    \]
    This case follows because it must be the case that
    \[F\t_{A};\m{1}^{-1} = \t_{FA} : FA \mto 1\]
    by the uniqueness of the terminal arrow.
    
  \item[] Case 2.\\
    \[
      \bfig
      \btriangle|mmm|<1000,500>[
        FA \times FB`
        F(A \times B)`
        FA;
        m_{A,B}`
        \pi_1`
        F\pi_1]
      \efig
      \]
      First, $\rho_A = \pi_1 : A \times 1 \mto A$, and hence, we have the following:
      \[
      \begin{array}{lll}
        (\id_A \times \t_A);\rho_1 & = & \langle \pi_1;\id_A, \pi_2;\t_A \rangle;\rho_1\\
        & = & \langle \pi_1;\id_A, \pi_2;\t_A \rangle;\pi_1\\
        & = & \pi_1 : A \times B \mto A
      \end{array}
      \]
      This case follows by the following equational reasoning:
      \[
      \begin{array}{llllllll}
        \m{A,B};F\pi_1
        & = & \m{A,B};F(\id_A \times \t_B);F\rho_A & \text{(Above Equation)}\\
        & = & (\id_{FA} \times F(\t_{B}));\m{A,1};F\rho_A & \text{(Naturality of $\m{}$)}\\
        & = & (\id_{FA} \times \t_{FB};\m{1});\m{A,1};F\rho_A & \text{(Case 1)}\\
        & = & \langle \id_{FA} \times \t_{FB} \rangle;(\id_{FA} \times \m{1});\m{A,1};F\rho_A & \text{(Cartesian)}\\
        & = & \langle \id_{FA} \times \t_{FB} \rangle;\rho_{FA} & \text{($F$ is Symmetric Monoidal)}\\
        & = & \langle \id_{FA} \times \t_{FB} \rangle;\pi_1 & \text{(Cartesian)}\\
        & = & \pi_1 & \text{(Cartesian)}\\ 
      \end{array}
      \]
    
  \item[] Case 3.\\    
    \[
      \bfig    
      \btriangle|mmm|<1000,500>[
        FA \times FB`
        F(A \times B)`
        FB;
        m_{A,B}`
        \pi_2`
        F\pi_2]
      \efig
    \]
    First, $\lambda_A = \pi_2 : 1 \times A \mto A$, and hence, we have the following:
      \[
      \begin{array}{lll}
        (\t_A \times \id_A);\lambda_A & = & \langle \pi_1;\t_A, \pi_2;\id_A \rangle;\lambda_A\\
        & = & \langle \pi_1;\t_A, \pi_2;\id_A \rangle;\pi_2\\
        & = & \pi_2 : A \times B \mto B
      \end{array}
      \]
      This case follows from similar equational reasoning
      as the previous case.      
      
  \item[] Case 4.\\    
    \[
    \bfig
      \btriangle|mmm|<1000,500>[
        FC`
        FA \times FB`
        F(A \times B);
        \langle Ff, Fg \rangle`
        F(\langle f, g \rangle)`
        m_{A,B}]
      \efig
      \]
  \end{itemize}  
\end{proof}

The definition of a natural transformation between product functors is
the same as the definition of a symmetric monoidal natural
transformation. And the same goes for symmetric monoidal adjunctions.

\begin{lemma}[Product Functors Isomorphisms]
  \label{lemma:product_functors_iso}
  If $(F,m) : \cat{C}_1 \mto \cat{C}_2$ is a product functor between cccs, then
  $m_{A,B} : FA \times FB \mto F(A \times B)$ and $m_1 : 1 \mto F1$ are
  isomorphisms.
\end{lemma}
\begin{proof}
  Define $m^{-1}_1 = \t_{F 1} : F1 \mto 1$.  Then we have the following:
  \[
  \begin{array}{lll}
    m_1;t_{F1} = \id_{1} & \text{(Uniqueness)}\\    
  \end{array}
  \]
  and
  \[
  \begin{array}{llll}
    t_{F1};m_1
    & = & F\t_t & \text{(Product Functor)}\\
    & = & F\id_t & \text{(Uniqueness)}\\
    & = & \id_{Ft}     
  \end{array}
  \]

  Now define $m^{-1}_{A,B} = \langle F\pi_1, F\pi_2 \rangle$.  Then we have the following:
  \[
  \begin{array}{llll}
    m_{A,B};m^{-1}_{A,B}
    & = & m_{A,B};\langle F\pi_1, F\pi_2 \rangle & \text{(Definition)}\\
    & = & \langle m_{A,B};F\pi_1, m_{A,B};F\pi_2 \rangle & \text{(P1)}\\
    & = & \langle \pi_1, \pi_2 \rangle & \text{(Product Functor)}\\
    & = & \id_{FA \times FB} & \text{(Cartesian)}\\
  \end{array}
  \]
  and
  \[
  \begin{array}{llll}
    m^{-1}_{A,B};m_{A,B}
    & = & \langle F\pi_1, F\pi_2 \rangle;m_{A,B} & \text{(Definition)}\\
    & = & F(\langle \pi_1, \pi_2 \rangle) & \text{(Product Functor)}\\
    & = & F(\id_{A \times B}) & \text{(Cartesian)}\\
    & = & \id_{F(A \times B)}\\    
  \end{array}
  \]
\end{proof}
% section cartesian_closed_categories_and_symmetric_monoidal_adjunctions (end)

\begin{lemma}[Symmetric Monoidal Adjunctions have Isomorphisms as Left-adjoints]
  \label{lemma:product_functors_iso}
  If $(F,m)\dashv (G,n) : \cat{C}_1 \mto \cat{C}_2$ is any symmetric monoidal adjunction, then
  $F$  is an 
  isomorphism. 
\end{lemma}
Proof in Benton's LNL-tech report.

Now if we have a symmetric monoidal adjunction between product functors, connecting cccs $\cat{C}_1, \cat{C}_2$, can we say anything else about the unit and co-unit? what about the other monoidal natural transformation $n$? what happens if $\cat{C}_1$ is isomorphic to $\cat{C}_2$?



\end{document}



Modalities in Constructive Logics and Type Theories Preface to the special issue on Intuitionistic Modal Logic and Application of the Journal of Logic and Computation, volume 14, number 4, August 2004. Guest Editors: Valeria de Paiva, Rajeev Gore' and Michael Mendler.
http://www.cs.bham.ac.uk/~vdp/publications/final-preface.pdf

and

http://logcom.oxfordjournals.org/content/early/2015/06/12/logcom.exv042.extract



