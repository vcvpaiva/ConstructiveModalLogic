\subsubsection{Proof of Substitution for Typing}
\label{sec:proof_of_lemma:substitution_for_typing}

\begin{lemma*}[Substitution for Typing]
  If $\Gamma_{{\mathrm{1}}}  \vdash  \TLLnt{t_{{\mathrm{1}}}}  \TLLsym{:}  \TLLnt{A}$, and $\Gamma_{{\mathrm{1}}}  \TLLsym{,}  \mathit{x}  \TLLsym{:}  \TLLnt{A}  \TLLsym{,}  \Gamma_{{\mathrm{2}}}  \vdash  \TLLnt{t_{{\mathrm{2}}}}  \TLLsym{:}  \TLLnt{B}$, then $\Gamma_{{\mathrm{1}}}  \TLLsym{,}  \Gamma_{{\mathrm{2}}}  \vdash  \TLLsym{[}  \TLLnt{t_{{\mathrm{1}}}}  \TLLsym{/}  \mathit{x}  \TLLsym{]}  \TLLnt{t_{{\mathrm{2}}}}  \TLLsym{:}  \TLLnt{B}$.
\end{lemma*}

\begin{proof}
  Suppose $\Gamma_{{\mathrm{1}}}  \vdash  \TLLnt{t_{{\mathrm{1}}}}  \TLLsym{:}  \TLLnt{A}$ and $\Gamma_{{\mathrm{1}}}  \TLLsym{,}  \mathit{x}  \TLLsym{:}  \TLLnt{A}  \TLLsym{,}  \Gamma_{{\mathrm{2}}}  \vdash  \TLLnt{t_{{\mathrm{2}}}}  \TLLsym{:}  \TLLnt{B}$.  We
  case split on the structure of the latter, but only show the
  non-trivial cases.  All other cases are similar.
\begin{itemize}
\item[] Case Identity.\\ 
  \begin{center}
    \begin{math}
      $$\mprset{flushleft}
      \inferrule* [right=$\text{Id}$] {
        \ 
      }{\Gamma_{{\mathrm{1}}}  \TLLsym{,}  \mathit{x}  \TLLsym{:}  \TLLnt{A}  \TLLsym{,}  \Gamma_{{\mathrm{2}}}  \vdash  \mathit{y}  \TLLsym{:}  \TLLnt{C}}
    \end{math}
  \end{center}
  In this case $\TLLnt{t_{{\mathrm{2}}}} = \mathit{y}$ and $\TLLnt{B} = \TLLnt{C}$.  We are not
  sure if $\mathit{x} = \mathit{y}$, thus, we must consider the case when they
  are and are not equal.

  Suppose $\mathit{x} \neq \mathit{y}$.  Then $\TLLsym{[}  \TLLnt{t_{{\mathrm{1}}}}  \TLLsym{/}  \mathit{x}  \TLLsym{]}  \TLLnt{t_{{\mathrm{2}}}} = \TLLsym{[}  \TLLnt{t_{{\mathrm{1}}}}  \TLLsym{/}  \mathit{x}  \TLLsym{]}  \mathit{y} =
  \mathit{y}$ by the definition of substitution.  In addition, it must be
  the case that either $\mathit{y}  \TLLsym{:}  \TLLnt{C} \in \Gamma_{{\mathrm{1}}}$ or $\mathit{y}  \TLLsym{:}  \TLLnt{C} \in
  \Gamma_{{\mathrm{2}}}$.  This implies that $\Gamma_{{\mathrm{1}}}  \TLLsym{,}  \Gamma_{{\mathrm{2}}}  \vdash  \mathit{y}  \TLLsym{:}  \TLLnt{C}$ or $\Gamma_{{\mathrm{1}}}  \TLLsym{,}  \Gamma_{{\mathrm{2}}}  \vdash  \TLLsym{[}  \TLLnt{t_{{\mathrm{1}}}}  \TLLsym{/}  \mathit{x}  \TLLsym{]}  \TLLnt{t_{{\mathrm{2}}}}  \TLLsym{:}  \TLLnt{B}$ hold.

  Now suppose $\mathit{x} = \mathit{y}$.  Then $\TLLnt{A} = \TLLnt{B}$, and $\TLLsym{[}  \TLLnt{t_{{\mathrm{1}}}}  \TLLsym{/}  \mathit{x}  \TLLsym{]}  \TLLnt{t_{{\mathrm{2}}}} = \TLLsym{[}  \TLLnt{t_{{\mathrm{1}}}}  \TLLsym{/}  \mathit{x}  \TLLsym{]}  \mathit{x} = \TLLnt{t_{{\mathrm{1}}}}$ by the definition of
  substitution.  Thus, $\Gamma_{{\mathrm{1}}}  \TLLsym{,}  \Gamma_{{\mathrm{2}}}  \vdash  \TLLsym{[}  \TLLnt{t_{{\mathrm{1}}}}  \TLLsym{/}  \mathit{x}  \TLLsym{]}  \TLLnt{t_{{\mathrm{2}}}}  \TLLsym{:}  \TLLnt{B}$ holds, because we
  know $\Gamma_{{\mathrm{1}}}  \TLLsym{,}  \Gamma_{{\mathrm{2}}}  \vdash  \TLLnt{t_{{\mathrm{1}}}}  \TLLsym{:}  \TLLnt{A}$.

\item[] Case Implication Introduction.\\ 
  \begin{center}
    \begin{math}
      $$\mprset{flushleft}
      \inferrule* [right=$\TLLdruletyXXimpIName$] {
        \Gamma_{{\mathrm{1}}}  \TLLsym{,}  \mathit{x}  \TLLsym{:}  \TLLnt{A}  \TLLsym{,}  \Gamma_{{\mathrm{2}}}  \TLLsym{,}  \mathit{y}  \TLLsym{:}  \TLLnt{C_{{\mathrm{1}}}}  \vdash  \TLLnt{t}  \TLLsym{:}  \TLLnt{C_{{\mathrm{2}}}}
      }{\Gamma_{{\mathrm{1}}}  \TLLsym{,}  \mathit{x}  \TLLsym{:}  \TLLnt{A}  \TLLsym{,}  \Gamma_{{\mathrm{2}}}  \vdash   \lambda  \mathit{y}  :  \TLLnt{C_{{\mathrm{1}}}} . \TLLnt{t}   \TLLsym{:}  \TLLnt{C_{{\mathrm{1}}}}  \to  \TLLnt{C_{{\mathrm{2}}}}}
    \end{math}
  \end{center}
  In this case $\TLLnt{B} = \TLLnt{C_{{\mathrm{1}}}}  \to  \TLLnt{C_{{\mathrm{2}}}}$ and $\TLLnt{t_{{\mathrm{2}}}} =  \lambda  \mathit{y}  :  \TLLnt{C} . \TLLnt{t} $.  By the induction hypothesis
  we know $\Gamma_{{\mathrm{1}}}  \TLLsym{,}  \Gamma_{{\mathrm{2}}}  \TLLsym{,}  \mathit{y}  \TLLsym{:}  \TLLnt{C_{{\mathrm{1}}}}  \vdash  \TLLsym{[}  \TLLnt{t_{{\mathrm{1}}}}  \TLLsym{/}  \mathit{x}  \TLLsym{]}  \TLLnt{t}  \TLLsym{:}  \TLLnt{C_{{\mathrm{2}}}}$, and then by reapplying the rule we know
  $\Gamma_{{\mathrm{1}}}  \TLLsym{,}  \Gamma_{{\mathrm{2}}}  \vdash   \lambda  \mathit{y}  :  \TLLnt{C_{{\mathrm{1}}}} . \TLLsym{[}  \TLLnt{t_{{\mathrm{1}}}}  \TLLsym{/}  \mathit{x}  \TLLsym{]}  \TLLnt{t}   \TLLsym{:}  \TLLnt{C_{{\mathrm{2}}}}$ holds.  However, by the definition of substitution
  we know $ \lambda  \mathit{y}  :  \TLLnt{C_{{\mathrm{1}}}} . \TLLsym{[}  \TLLnt{t_{{\mathrm{1}}}}  \TLLsym{/}  \mathit{x}  \TLLsym{]}  \TLLnt{t}  = \TLLsym{[}  \TLLnt{t_{{\mathrm{1}}}}  \TLLsym{/}  \mathit{x}  \TLLsym{]}  \TLLsym{(}   \lambda  \mathit{y}  :  \TLLnt{C_{{\mathrm{1}}}} . \TLLnt{t}   \TLLsym{)}$, and thus, we obtain our result.

\item[] Case Implication Elimination.\\ 
  \begin{center}
    \begin{math}
      $$\mprset{flushleft}
      \inferrule* [right=$\TLLdruletyXXimpEName$] {
        \Gamma_{{\mathrm{1}}}  \TLLsym{,}  \mathit{x}  \TLLsym{:}  \TLLnt{A}  \TLLsym{,}  \Gamma_{{\mathrm{2}}}  \vdash  \TLLnt{t'_{{\mathrm{1}}}}  \TLLsym{:}  \TLLnt{C_{{\mathrm{1}}}}  \to  \TLLnt{C_{{\mathrm{2}}}}  \quad  \Gamma_{{\mathrm{1}}}  \TLLsym{,}  \mathit{x}  \TLLsym{:}  \TLLnt{A}  \TLLsym{,}  \Gamma_{{\mathrm{2}}}  \vdash  \TLLnt{t'_{{\mathrm{2}}}}  \TLLsym{:}  \TLLnt{C_{{\mathrm{1}}}}
      }{\Gamma_{{\mathrm{1}}}  \TLLsym{,}  \mathit{x}  \TLLsym{:}  \TLLnt{A}  \TLLsym{,}  \Gamma_{{\mathrm{2}}}  \vdash  \TLLnt{t'_{{\mathrm{1}}}} \, \TLLnt{t'_{{\mathrm{2}}}}  \TLLsym{:}  \TLLnt{C_{{\mathrm{2}}}}}
    \end{math}
  \end{center}
  We now have that $\TLLnt{B} = \TLLnt{C_{{\mathrm{2}}}}$ and $\TLLnt{t_{{\mathrm{2}}}} = \TLLnt{t'_{{\mathrm{1}}}} \, \TLLnt{t'_{{\mathrm{2}}}}$.  By the induction hypothesis
  we know that $\Gamma_{{\mathrm{1}}}  \TLLsym{,}  \Gamma_{{\mathrm{2}}}  \vdash  \TLLsym{[}  \TLLnt{t_{{\mathrm{1}}}}  \TLLsym{/}  \mathit{x}  \TLLsym{]}  \TLLnt{t'_{{\mathrm{1}}}}  \TLLsym{:}  \TLLnt{C_{{\mathrm{1}}}}  \to  \TLLnt{C_{{\mathrm{2}}}}$ and $\Gamma_{{\mathrm{1}}}  \TLLsym{,}  \Gamma_{{\mathrm{2}}}  \vdash  \TLLsym{[}  \TLLnt{t_{{\mathrm{1}}}}  \TLLsym{/}  \mathit{x}  \TLLsym{]}  \TLLnt{t'_{{\mathrm{2}}}}  \TLLsym{:}  \TLLnt{C_{{\mathrm{1}}}}$ both hold.
  Then by reapplying the rule we obtain that $\Gamma_{{\mathrm{1}}}  \TLLsym{,}  \Gamma_{{\mathrm{2}}}  \vdash  \TLLsym{(}  \TLLsym{[}  \TLLnt{t_{{\mathrm{1}}}}  \TLLsym{/}  \mathit{x}  \TLLsym{]}  \TLLnt{t'_{{\mathrm{1}}}}  \TLLsym{)} \, \TLLsym{(}  \TLLsym{[}  \TLLnt{t_{{\mathrm{1}}}}  \TLLsym{/}  \mathit{x}  \TLLsym{]}  \TLLnt{t'_{{\mathrm{2}}}}  \TLLsym{)}  \TLLsym{:}  \TLLnt{C_{{\mathrm{2}}}}$, and thus,
  by the definition of substitution $\Gamma_{{\mathrm{1}}}  \TLLsym{,}  \Gamma_{{\mathrm{2}}}  \vdash  \TLLsym{[}  \TLLnt{t_{{\mathrm{1}}}}  \TLLsym{/}  \mathit{x}  \TLLsym{]}  \TLLsym{(}  \TLLnt{t'_{{\mathrm{1}}}} \, \TLLnt{t'_{{\mathrm{2}}}}  \TLLsym{)}  \TLLsym{:}  \TLLnt{C_{{\mathrm{2}}}}$ holds.
  
  
\item[] Case $\Box$ Introduction.\\ 
  \begin{center}
    \scriptsize
    \begin{math}
      $$\mprset{flushleft}
      \inferrule* [right=$\TLLdruletyXXboxIName$] {
         \Gamma_{{\mathrm{1}}}  \TLLsym{,}  \mathit{x}  \TLLsym{:}  \TLLnt{A}  \TLLsym{,}  \Gamma_{{\mathrm{2}}}  \vdash  \TLLnt{t'_{{\mathrm{1}}}}  \TLLsym{:}  \Box \, \TLLnt{C_{{\mathrm{1}}}}  \TLLsym{,} \, ... \, \TLLsym{,}  \Gamma_{{\mathrm{1}}}  \TLLsym{,}  \mathit{x}  \TLLsym{:}  \TLLnt{A}  \TLLsym{,}  \Gamma_{{\mathrm{2}}}  \vdash  \TLLnt{t'_{\TLLmv{k}}}  \TLLsym{:}  \Box \, \TLLnt{C_{\TLLmv{k}}}   \quad  \mathit{x_{{\mathrm{1}}}}  \TLLsym{:}  \Box \, \TLLnt{C_{{\mathrm{1}}}}  \TLLsym{,} \, ... \, \TLLsym{,}  \mathit{x_{\TLLmv{k}}}  \TLLsym{:}  \Box \, \TLLnt{C_{\TLLmv{k}}}  \vdash  \TLLnt{t}  \TLLsym{:}  \TLLnt{C}
      }{\Gamma_{{\mathrm{1}}}  \TLLsym{,}  \mathit{x}  \TLLsym{:}  \TLLnt{A}  \TLLsym{,}  \Gamma_{{\mathrm{2}}}  \vdash   \mathsf{let}_\Box\, \mathit{x_{{\mathrm{1}}}}  \TLLsym{:}  \Box \, \TLLnt{C_{{\mathrm{1}}}}  \TLLsym{,} \, ... \, \TLLsym{,}  \mathit{x_{\TLLmv{k}}}  \TLLsym{:}  \Box \, \TLLnt{C_{\TLLmv{k}}} \,\mathsf{be}\, \TLLnt{t'_{{\mathrm{1}}}}  \TLLsym{,} \, ... \, \TLLsym{,}  \TLLnt{t'_{\TLLmv{k}}} \,\mathsf{in}\, \TLLnt{t}   \TLLsym{:}  \Box \, \TLLnt{C}}
    \end{math}
  \end{center}
  In this case $\TLLnt{B} = \Box \, \TLLnt{C}$ and
  $\TLLnt{t_{{\mathrm{2}}}} =  \mathsf{let}_\Box\, \mathit{x_{{\mathrm{1}}}}  \TLLsym{:}  \Box \, \TLLnt{C_{{\mathrm{1}}}}  \TLLsym{,} \, ... \, \TLLsym{,}  \mathit{x_{\TLLmv{k}}}  \TLLsym{:}  \Box \, \TLLnt{C_{\TLLmv{k}}} \,\mathsf{be}\, \TLLnt{t'_{{\mathrm{1}}}}  \TLLsym{,} \, ... \, \TLLsym{,}  \TLLnt{t'_{\TLLmv{k}}} \,\mathsf{in}\, \TLLnt{t} $.  By the induction hypothesis
  we know that 
  \[ \Gamma_{{\mathrm{1}}}  \TLLsym{,}  \Gamma_{{\mathrm{2}}}  \vdash  \TLLsym{[}  \TLLnt{t_{{\mathrm{1}}}}  \TLLsym{/}  \mathit{x}  \TLLsym{]}  \TLLnt{t'_{{\mathrm{1}}}}  \TLLsym{:}  \Box \, \TLLnt{C_{{\mathrm{1}}}} , \ldots , \Gamma_{{\mathrm{1}}}  \TLLsym{,}  \mathit{x}  \TLLsym{:}  \TLLnt{A}  \TLLsym{,}  \Gamma_{{\mathrm{2}}}  \vdash  \TLLsym{[}  \TLLnt{t_{{\mathrm{1}}}}  \TLLsym{/}  \mathit{x}  \TLLsym{]}  \TLLnt{t'_{\TLLmv{k}}}  \TLLsym{:}  \Box \, \TLLnt{C_{\TLLmv{k}}} \] all hold. Then by reapplying
  the rule we know that
  \[ \Gamma_{{\mathrm{1}}}  \TLLsym{,}  \Gamma_{{\mathrm{2}}}  \vdash   \mathsf{let}_\Box\, \mathit{x_{{\mathrm{1}}}}  \TLLsym{:}  \Box \, \TLLnt{C_{{\mathrm{1}}}}  \TLLsym{,} \, ... \, \TLLsym{,}  \mathit{x_{\TLLmv{k}}}  \TLLsym{:}  \Box \, \TLLnt{C_{\TLLmv{k}}} \,\mathsf{be}\, \TLLsym{[}  \TLLnt{t_{{\mathrm{1}}}}  \TLLsym{/}  \mathit{x}  \TLLsym{]}  \TLLnt{t'_{{\mathrm{1}}}}  \TLLsym{,} \, ... \, \TLLsym{,}  \TLLsym{[}  \TLLnt{t_{{\mathrm{1}}}}  \TLLsym{/}  \mathit{x}  \TLLsym{]}  \TLLnt{t'_{\TLLmv{k}}} \,\mathsf{in}\, \TLLnt{t}   \TLLsym{:}  \Box \, \TLLnt{C}, \] but by the definition
  of substitution and the fact that $\TLLsym{[}  \TLLnt{t_{{\mathrm{1}}}}  \TLLsym{/}  \mathit{x}  \TLLsym{]}  \TLLnt{t} = \TLLnt{t}$ because $\TLLnt{t}$ does not depend on $\mathit{x}$ we know that
  \[ \Gamma_{{\mathrm{1}}}  \TLLsym{,}  \Gamma_{{\mathrm{2}}}  \vdash  \TLLsym{[}  \TLLnt{t_{{\mathrm{1}}}}  \TLLsym{/}  \mathit{x}  \TLLsym{]}  \TLLsym{(}   \mathsf{let}_\Box\, \mathit{x_{{\mathrm{1}}}}  \TLLsym{:}  \Box \, \TLLnt{C_{{\mathrm{1}}}}  \TLLsym{,} \, ... \, \TLLsym{,}  \mathit{x_{\TLLmv{k}}}  \TLLsym{:}  \Box \, \TLLnt{C_{\TLLmv{k}}} \,\mathsf{be}\, \TLLnt{t'_{{\mathrm{1}}}}  \TLLsym{,} \, ... \, \TLLsym{,}  \TLLnt{t'_{\TLLmv{k}}} \,\mathsf{in}\, \TLLnt{t}   \TLLsym{)}  \TLLsym{:}  \Box \, \TLLnt{C} \] holds.
\end{itemize}
\end{proof}
% subsubsection proof_of_lemma_substitution_for_typing (end)

\subsubsection{Proof of Weakening}
\label{subsubsec:proof_of_lemma:weakening}

\begin{lemma*}[Weakening]
  If $\Gamma_{{\mathrm{1}}}  \TLLsym{,}  \Gamma_{{\mathrm{2}}}  \vdash  \TLLnt{t}  \TLLsym{:}  \TLLnt{B}$, then $\Gamma_{{\mathrm{1}}}  \TLLsym{,}  \mathit{x}  \TLLsym{:}  \TLLnt{A}  \TLLsym{,}  \Gamma_{{\mathrm{2}}}  \vdash  \TLLnt{t}  \TLLsym{:}  \TLLnt{B}$.
\end{lemma*}

\begin{proof}
  This proof is by induction on the form of $\Gamma_{{\mathrm{1}}}  \TLLsym{,}  \Gamma_{{\mathrm{2}}}  \vdash  \TLLnt{t}  \TLLsym{:}  \TLLnt{B}$.  We
  only show a few cases, because the others are similar.
\begin{itemize}
\item[] Case Identity.\\ 
  \begin{center}
    \begin{math}
      $$\mprset{flushleft}
      \inferrule* [right=$\TLLdruletyXXaxName$] {
        \ 
      }{\Gamma_{{\mathrm{1}}}  \TLLsym{,}  \Gamma_{{\mathrm{2}}}  \TLLsym{,}  \mathit{y}  \TLLsym{:}  \TLLnt{C}  \vdash  \mathit{y}  \TLLsym{:}  \TLLnt{C}}
    \end{math}
  \end{center}
  In this case we have that $\TLLnt{B} = \TLLnt{C}$ and $\TLLnt{t} = \mathit{y}$.  We
  must show that $\Gamma_{{\mathrm{1}}}  \TLLsym{,}  \mathit{x}  \TLLsym{:}  \TLLnt{A}  \TLLsym{,}  \Gamma_{{\mathrm{2}}}  \TLLsym{,}  \mathit{y}  \TLLsym{:}  \TLLnt{C}  \vdash  \mathit{y}  \TLLsym{:}  \TLLnt{C}$ holds, but this
  clearly holds by reapplying the rule.

\item[] Case $\Box$ Introduction.\\ 
  \begin{center}
    \scriptsize
    \begin{math}
      $$\mprset{flushleft}
      \inferrule* [right=$\TLLdruletyXXboxIName$] {
         \Gamma_{{\mathrm{1}}}  \TLLsym{,}  \Gamma_{{\mathrm{2}}}  \vdash  \TLLnt{t_{{\mathrm{1}}}}  \TLLsym{:}  \Box \, \TLLnt{C_{{\mathrm{1}}}}  \TLLsym{,} \, ... \, \TLLsym{,}  \Gamma_{{\mathrm{1}}}  \TLLsym{,}  \Gamma_{{\mathrm{2}}}  \vdash  \TLLnt{t_{\TLLmv{k}}}  \TLLsym{:}  \Box \, \TLLnt{C_{\TLLmv{k}}}   \quad  \mathit{x_{{\mathrm{1}}}}  \TLLsym{:}  \Box \, \TLLnt{C_{{\mathrm{1}}}}  \TLLsym{,} \, ... \, \TLLsym{,}  \mathit{x_{\TLLmv{k}}}  \TLLsym{:}  \Box \, \TLLnt{C_{\TLLmv{k}}}  \vdash  \TLLnt{t'}  \TLLsym{:}  \TLLnt{C}
      }{\Gamma_{{\mathrm{1}}}  \TLLsym{,}  \Gamma_{{\mathrm{2}}}  \vdash   \mathsf{let}_\Box\, \mathit{x_{{\mathrm{1}}}}  \TLLsym{:}  \Box \, \TLLnt{C_{{\mathrm{1}}}}  \TLLsym{,} \, ... \, \TLLsym{,}  \mathit{x_{\TLLmv{k}}}  \TLLsym{:}  \Box \, \TLLnt{C_{\TLLmv{k}}} \,\mathsf{be}\, \TLLnt{t_{{\mathrm{1}}}}  \TLLsym{,} \, ... \, \TLLsym{,}  \TLLnt{t_{\TLLmv{k}}} \,\mathsf{in}\, \TLLnt{t'}   \TLLsym{:}  \Box \, \TLLnt{C}}
    \end{math}
  \end{center}
  This case is similar to the previous case.  First, apply the
  induction hypothesis to the left-most premise, and then reapply
  the rule.
\end{itemize}
\end{proof}
% subsubsection proof_of_lemma:weakening (end)

\subsubsection{Proof of Soundness of $\sf{TCS4}$}
\label{subsec:proof_of_soundness_of_tcs4}

\begin{theorem*}
The type theory $\sf{TCS4}$ has \textit{sound} models provided by the
structures $\cal C$ defined above.  In other words, given a tense
adjoint modal category $\cal C$, using the above interpretation, the
following hold:
\begin{itemize}
\item Assume $\Gamma \vdash t : A$ in $\sf{TCS4}$. Then $\sem{\Gamma
  \vdash t \colon A}$ is a morphism with domain $\sem{\Gamma}$ and
  codomain $\sem{A}$;
\item Assume $\Gamma \vdash t = s \colon A$. Then $\sem{\Gamma
  \vdash t \colon A} = \sem{\Gamma \vdash s \colon A}$.
\end{itemize}
\end{theorem*}
\begin{proof}
  The first part holds by induction on $\Gamma  \vdash  \TLLnt{t}  \TLLsym{:}  \TLLnt{A}$, and the
  second by induction on $ \Gamma  \vdash  \TLLnt{t}  =  \TLLnt{s}  :  \TLLnt{A} $.  We give a few cases of each part, as the others are similar.  Throughout the proof we
  drop semantic brackets on objects, and we assume, without loss of
  generality, that the interpretation of contexts are left associated.
  We begin with the first part.

  \begin{itemize}
  \item[] Case Identity.\\
    \[
    \TLLdruletyXXax{}
    \]
    We need to provide a morphism $\Gamma \times \TLLnt{A} \mto^{f}
    \TLLnt{A}$ and we choose $f = \pi_2$ (the $2$nd projection), as usual.

    \item[] Case Implication Introduction.\\
    \[
    \TLLdruletyXXimpI{}
    \]
    By the induction hypothesis we know that there is a morphism
    $\Gamma \times \TLLnt{A} \mto^f \TLLnt{B}$.  Then we need  to find a
    morphism $\Gamma \mto^{g} (\TLLnt{A}  \to  \TLLnt{B})$.  Choose $g =
    \mathsf{curry}(f)$ where $\mathsf{curry} :
    \mathsf{Hom}_{\cat{C}}( \TLLnt{A}  \times  \TLLnt{B} ,\TLLnt{C}) \mto
    \mathsf{Hom}_{\cat{C}}(\TLLnt{A},\TLLnt{B}  \to  \TLLnt{C})$ is a natural isomorphism
    that exists because $\cat{C}$ is closed.

  \item[] Case $\Box$ Introduction.\\
    \[
    \TLLdruletyXXboxI{}
    \]

    By the induction hypothesis we have the family of morphisms $\Gamma
    \mto^{f_1} \Box \, \TLLnt{A_{{\mathrm{1}}}}, \ldots, \Gamma \mto^{f_k} \Box \, \TLLnt{A_{\TLLmv{k}}}$, and a given morphism
    $\Box \, \TLLnt{A_{{\mathrm{1}}}} \times \cdots \times \Box \, \TLLnt{A_{\TLLmv{k}}} \mto^f \TLLnt{B}$.  We need 
     to find a morphism $\Gamma \mto^{g} \TLLnt{B}$.  As in previous work, we choose $g =
    \langle f_1;\delta_{\TLLnt{A_{{\mathrm{1}}}}},\ldots,f_k;\delta_{\TLLnt{A_{\TLLmv{k}}}} \rangle;\mathsf{m};\Box f$, where $\langle -,-\rangle :
    \mathsf{Hom}_{\cat{C}}(\Gamma,\Box \, \TLLnt{A_{{\mathrm{1}}}}) \times \cdots \times
    \mathsf{Hom}_{\cat{C}}(\Gamma,\Box \, \TLLnt{A_{\TLLmv{k}}}) \mto \mathsf{Hom}_{\cat{C}}(\Gamma,\Box \, \TLLnt{A_{{\mathrm{1}}}}
    \times \cdots \times \Box \, \TLLnt{A_{\TLLmv{k}}})$ exists because $\cat{C}$ is cartesian and we make the simplifying assumption that $\Box$ is an endofunctor.

  \item[] Case $\Box$ Elimination.\\
    \[
    \TLLdruletyXXboxE{}
    \]
    By the induction hypothesis there is a morphism $\Gamma \mto^f
    \Box \, \TLLnt{B}$.  It suffices to find a morphism $\Gamma \mto^{g}
    \TLLnt{B}$.  Choose $g = f;\eta_B$ where $\eta_B : \Box \, \TLLnt{B} \mto \TLLnt{B}$
    is the unit of the adjunction.
  \end{itemize}

  We now turn to the second part:
  \begin{itemize}
  \item[] Case Unboxing $\Box$.\\
    {\scriptsize
      \begin{mathpar}
        \TLLdruleeqXXunbox{}
      \end{mathpar}
    }
    Using the interpretations given above we must show that:
    \[
    \langle f_1;\delta_{\TLLnt{A_{{\mathrm{1}}}}},\ldots,f_k;\delta_{\TLLnt{A_{\TLLmv{k}}}} \rangle;\mathsf{m};\Box f;\eta_B = \langle f_1,\ldots,f_k \rangle;f : \Gamma \mto \TLLnt{B}.
    \]
    This holds by the following equational reasoning:
    \[\small
    \begin{array}{lll}
      \langle f_1;\delta_{\TLLnt{A_{{\mathrm{1}}}}},\ldots,f_k;\delta_{\TLLnt{A_{\TLLmv{k}}}} \rangle;\mathsf{m};\Box f;\eta_B
      & = & \langle f_1;\delta_{\TLLnt{A_{{\mathrm{1}}}}},\ldots,f_k;\delta_{\TLLnt{A_{\TLLmv{k}}}} \rangle;\mathsf{m};\eta;f\\
      & = & \langle f_1;\delta_{\TLLnt{A_{{\mathrm{1}}}}},\ldots,f_k;\delta_{\TLLnt{A_{\TLLmv{k}}}} \rangle;(\eta_{\TLLnt{A_{{\mathrm{1}}}}} \times \cdots \times \eta_{\TLLnt{A_{\TLLmv{k}}}});f\\
      & = & \langle f_1;\delta_{\TLLnt{A_{{\mathrm{1}}}}};\eta_{\TLLnt{A_{{\mathrm{1}}}}},\ldots,f_k;\delta_{\TLLnt{A_{\TLLmv{k}}}};\eta_{\TLLnt{A_{\TLLmv{k}}}} \rangle;f\\
      & = & \langle f_1,\ldots,f_k \rangle;f\\
    \end{array}
    \]
  \end{itemize}
\end{proof}
% subsubsection proof_of_soundness_of_tcs4 (end)

\subsubsection{Proof of Completeness for $\sf{TCS4}$}
\label{subsec:proof_of_completeness_for_tcs4}

\begin{theorem*}
\label{thm:tcs4-completeness}
The adjoint modal models are \textit{complete} in the appropriate
sense for the type theory $\sf{TCS4}$. This is to say, if we have
equality of the interpretations $\sem{\Gamma \vdash t \colon A} =
\sem{\Gamma \vdash s \colon A}$ (where \mbox{$\sem{\ } $} is the
interpretation defined above) in the tense modal category $\cal C$ for
any derived sequents $\Gamma \vdash t \colon A$ and $\Gamma \vdash s
\colon A$ then we can derive the equation in the type theory
$\sf{TCS4}$ $\;$ $\Gamma \vdash t = s \colon A$.
\end{theorem*}

\begin{proof}
  This result can be shown by constructing a cartesian closed category
  with two comonads, one for $\Box$ and one for $\BBox$, internal to the type theory
  $\sf{TCS4}$ where the objects are types and the morphisms are
  $\alpha$-equivalence classes of terms in context $\Gamma  \vdash  \TLLnt{t}  \TLLsym{:}  \TLLnt{A}$.
  This category is called the syntactic category for the $\sf{TCS4}$
  type theory.

  Showing that this syntactic category is cartesian closed is well
  known, but we illustrate the proof by describing the case of the
  $\Box$ comonad.

  We denote a morphism by the $\alpha$-equivalence class:
  \[
  [\vec{x},\TLLnt{t} ]^{\vec{\TLLnt{A}},\TLLnt{B}} = [ \vec{x}:\vec{\TLLnt{A}} \vdash \TLLnt{t} : \TLLnt{B} ]
  \]
  We then have the following definitions:
  \begin{itemize}
  \item (Identity) $\id = [x,x]^{\TLLnt{A},\TLLnt{A}}$
  \item (Composition) Given morphisms $[\vec{x},\TLLnt{t} ]^{\vec{A},\TLLnt{B_{\TLLmv{i}}}}$ and
    $[\vec{y},\TLLnt{t'} ]^{\vec{B},C}$, their composition
    $[\vec{x},\TLLnt{t} ]^{\vec{A},\TLLnt{B_{\TLLmv{i}}}};[\vec{y},\TLLnt{t'} ]^{\vec{B},C} =
    [\vec{x_{i-1}},\vec{y},\vec{x_{i+1}},[ \TLLnt{t} / x_i ] \TLLnt{t'} ]^{\vec{A},C}$.
  \item (Equality) Two parallel morphisms $[\vec{x},\TLLnt{t} ]^{\vec{A},B}$ and $[\vec{x},\TLLnt{t'} ]^{\vec{A},B}$ are
    equal if and only if
    $\vec{x} : \vec{A} \vdash \TLLnt{t} = \TLLnt{t'} : B$.
  \end{itemize}
  Using basic facts about substitution one can show that composition
  preserves identity and is associative.

  We first must show that $\Box$ is an endofunctor on the syntactic
  category.  Suppose we have the morphism
  $[\vec{x},t]^{\vec{A},B}$.  Then we must construct a morphism
  $[\vec{y},t']^{\overrightarrow{\Box \, \TLLnt{A}},\Box \, \TLLnt{B}}$.  The latter morphism can be defined in two steps.
  The first is to change the $\vec{A}$ to $\overrightarrow{\Box \, \TLLnt{A}}$:
  \[
    [ \vec{y},\wedge \mathsf{unbox}_\Box\, \, \mathit{y_{\TLLmv{i}}} ]^{\overrightarrow{\Box \, \TLLnt{A}},\vec{A}};[\vec{x},t]^{\vec{A},\vec{B}}
    = [\vec{y}, [ \mathsf{unbox}_\Box\, \, \mathit{y_{\TLLmv{i}}}/\mathit{x_{\TLLmv{i}}} ] \TLLnt{t} ]^{\overrightarrow{\Box \, \TLLnt{A}},\vec{B}}
    \]
    The second step is to change $B$ into $\Box \, \TLLnt{B}$:
    \[
      [\vec{y},\mathsf{let}_\Box\,\vec{y}:\overrightarrow{\Box \, \TLLnt{A}}\,\mathsf{be}\,\vec{y}\,\mathsf{in}\,[ \mathsf{unbox}_\Box\, \, \mathit{y_{\TLLmv{i}}}/\mathit{x_{\TLLmv{i}}} ]\TLLnt{t})]^{\overrightarrow{\Box \, \TLLnt{A}},\Box \, \TLLnt{B}}
      \]
  Straightforward calculations  show that
  this construction preserves identities and composition.

  The unit of the comonad is defined as $[ y,\mathsf{unbox}_\Box\, \, \mathit{y} ]^{\Box \, \TLLnt{A},A}$.  Next we need to define a morphism between $\Box \, \TLLnt{A}$ and $\Box \, \Box \, \TLLnt{A}$:
  \[
  [y, \mathsf{let}_\Box\, \mathit{y}  \TLLsym{:}  \Box \, \TLLnt{A} \,\mathsf{be}\, \mathit{y} \,\mathsf{in}\, \mathit{y}  ]^{\Box \, \TLLnt{A},\Box \, \Box \, \TLLnt{A}}
  \]
  Finally, using these constructions it is possible to show the usual
  diagrams defining the comonad $\Box$ hold.  The definition for
  $\BBox$ is similar.
\end{proof}
% subsubsection proof_of_completeness_for_tcs4 (end)
