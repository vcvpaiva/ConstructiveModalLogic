\subsubsection{Proof of Substitution for Typing}
\label{sec:proof_of_lemma:substitution_for_typing}

\begin{lemma*}[Substitution for Typing]
  If $\Gamma_{{\mathrm{1}}}  \vdash  \TLLnt{t_{{\mathrm{1}}}}  \TLLsym{:}  \TLLnt{A}$, and $\Gamma_{{\mathrm{1}}}  \TLLsym{,}  \mathit{x}  \TLLsym{:}  \TLLnt{A}  \TLLsym{,}  \Gamma_{{\mathrm{2}}}  \vdash  \TLLnt{t_{{\mathrm{2}}}}  \TLLsym{:}  \TLLnt{B}$, then $\Gamma_{{\mathrm{1}}}  \TLLsym{,}  \Gamma_{{\mathrm{2}}}  \vdash  \TLLsym{[}  \TLLnt{t_{{\mathrm{1}}}}  \TLLsym{/}  \mathit{x}  \TLLsym{]}  \TLLnt{t_{{\mathrm{2}}}}  \TLLsym{:}  \TLLnt{B}$.
\end{lemma*}

\begin{proof}
  Suppose $\Gamma_{{\mathrm{1}}}  \vdash  \TLLnt{t_{{\mathrm{1}}}}  \TLLsym{:}  \TLLnt{A}$ and $\Gamma_{{\mathrm{1}}}  \TLLsym{,}  \mathit{x}  \TLLsym{:}  \TLLnt{A}  \TLLsym{,}  \Gamma_{{\mathrm{2}}}  \vdash  \TLLnt{t_{{\mathrm{2}}}}  \TLLsym{:}  \TLLnt{B}$.  We case
split on the structure of the latter.
\begin{itemize}
\item[] Case.\\ 
  \begin{center}
    \begin{math}
      $$\mprset{flushleft}
      \inferrule* [right=$\text{Id}$] {
        \ 
      }{\Gamma_{{\mathrm{1}}}  \TLLsym{,}  \mathit{x}  \TLLsym{:}  \TLLnt{A}  \TLLsym{,}  \Gamma_{{\mathrm{2}}}  \vdash  \mathit{y}  \TLLsym{:}  \TLLnt{C}}
    \end{math}
  \end{center}
  In this case $\TLLnt{t_{{\mathrm{2}}}} = \mathit{y}$ and $\TLLnt{B} = \TLLnt{C}$.  We are not
  sure if $\mathit{x} = \mathit{y}$, thus, we must consider the case when they
  are and are not equal.

  Suppose $\mathit{x} \neq \mathit{y}$.  Then $\TLLsym{[}  \TLLnt{t_{{\mathrm{1}}}}  \TLLsym{/}  \mathit{x}  \TLLsym{]}  \TLLnt{t_{{\mathrm{2}}}} = \TLLsym{[}  \TLLnt{t_{{\mathrm{1}}}}  \TLLsym{/}  \mathit{x}  \TLLsym{]}  \mathit{y} =
  \mathit{y}$ by the definition of substitution.  In addition, it must be
  the case that either $\mathit{y}  \TLLsym{:}  \TLLnt{C} \in \Gamma_{{\mathrm{1}}}$ or $\mathit{y}  \TLLsym{:}  \TLLnt{C} \in
  \Gamma_{{\mathrm{2}}}$.  This implies that $\Gamma_{{\mathrm{1}}}  \TLLsym{,}  \Gamma_{{\mathrm{2}}}  \vdash  \mathit{y}  \TLLsym{:}  \TLLnt{C}$ or $\Gamma_{{\mathrm{1}}}  \TLLsym{,}  \Gamma_{{\mathrm{2}}}  \vdash  \TLLsym{[}  \TLLnt{t_{{\mathrm{1}}}}  \TLLsym{/}  \mathit{x}  \TLLsym{]}  \TLLnt{t_{{\mathrm{2}}}}  \TLLsym{:}  \TLLnt{B}$ hold.

  Now suppose $\mathit{x} = \mathit{y}$.  Then $\TLLnt{A} = \TLLnt{B}$, and $\TLLsym{[}  \TLLnt{t_{{\mathrm{1}}}}  \TLLsym{/}  \mathit{x}  \TLLsym{]}  \TLLnt{t_{{\mathrm{2}}}} = \TLLsym{[}  \TLLnt{t_{{\mathrm{1}}}}  \TLLsym{/}  \mathit{x}  \TLLsym{]}  \mathit{x} = \TLLnt{t_{{\mathrm{1}}}}$ by the definition of
  substitution.  Thus, $\Gamma_{{\mathrm{1}}}  \TLLsym{,}  \Gamma_{{\mathrm{2}}}  \vdash  \TLLsym{[}  \TLLnt{t_{{\mathrm{1}}}}  \TLLsym{/}  \mathit{x}  \TLLsym{]}  \TLLnt{t_{{\mathrm{2}}}}  \TLLsym{:}  \TLLnt{B}$ holds, because we
  know $\Gamma_{{\mathrm{1}}}  \TLLsym{,}  \Gamma_{{\mathrm{2}}}  \vdash  \TLLnt{t_{{\mathrm{1}}}}  \TLLsym{:}  \TLLnt{A}$.

\item[] Case.\\ 
  \begin{center}
    \begin{math}
      $$\mprset{flushleft}
      \inferrule* [right=$\perp_\mathcal{E}$] {
        \ 
      }{\Gamma_{{\mathrm{1}}}  \TLLsym{,}  \mathit{x}  \TLLsym{:}  \TLLnt{A}  \TLLsym{,}  \Gamma_{{\mathrm{2}}}  \TLLsym{,}  \mathit{y}  \TLLsym{:}  \perp  \vdash   \mathsf{contra}   \TLLsym{:}  \TLLnt{C}}
    \end{math}
  \end{center}
  We must show that $\Gamma_{{\mathrm{1}}}  \TLLsym{,}  \Gamma_{{\mathrm{2}}}  \TLLsym{,}  \mathit{y}  \TLLsym{:}  \perp  \vdash  \TLLsym{[}  \TLLnt{t_{{\mathrm{1}}}}  \TLLsym{/}  \mathit{x}  \TLLsym{]}   \mathsf{contra}   \TLLsym{:}  \TLLnt{C}$ holds,
  because $\TLLnt{B} = \TLLnt{C}$ and $\TLLnt{t_{{\mathrm{2}}}} =  \mathsf{contra} $, but this holds
  by reapplying the rule.


\item[] Case.\\ 
  \begin{center}
    \begin{math}
      $$\mprset{flushleft}
      \inferrule* [right=$\TLLdruletyXXimpIName$] {
        \Gamma_{{\mathrm{1}}}  \TLLsym{,}  \mathit{x}  \TLLsym{:}  \TLLnt{A}  \TLLsym{,}  \Gamma_{{\mathrm{2}}}  \TLLsym{,}  \mathit{y}  \TLLsym{:}  \TLLnt{C_{{\mathrm{1}}}}  \vdash  \TLLnt{t}  \TLLsym{:}  \TLLnt{C_{{\mathrm{2}}}}
      }{\Gamma_{{\mathrm{1}}}  \TLLsym{,}  \mathit{x}  \TLLsym{:}  \TLLnt{A}  \TLLsym{,}  \Gamma_{{\mathrm{2}}}  \vdash   \lambda  \mathit{y}  :  \TLLnt{C_{{\mathrm{1}}}} . \TLLnt{t}   \TLLsym{:}  \TLLnt{C_{{\mathrm{1}}}}  \to  \TLLnt{C_{{\mathrm{2}}}}}
    \end{math}
  \end{center}
  In this case $\TLLnt{B} = \TLLnt{C_{{\mathrm{1}}}}  \to  \TLLnt{C_{{\mathrm{2}}}}$ and $\TLLnt{t_{{\mathrm{2}}}} =  \lambda  \mathit{y}  :  \TLLnt{C} . \TLLnt{t} $.  By the induction hypothesis
  we know $\Gamma_{{\mathrm{1}}}  \TLLsym{,}  \Gamma_{{\mathrm{2}}}  \TLLsym{,}  \mathit{y}  \TLLsym{:}  \TLLnt{C_{{\mathrm{1}}}}  \vdash  \TLLsym{[}  \TLLnt{t_{{\mathrm{1}}}}  \TLLsym{/}  \mathit{x}  \TLLsym{]}  \TLLnt{t}  \TLLsym{:}  \TLLnt{C_{{\mathrm{2}}}}$, and then by reapplying the rule we know
  $\Gamma_{{\mathrm{1}}}  \TLLsym{,}  \Gamma_{{\mathrm{2}}}  \vdash   \lambda  \mathit{y}  :  \TLLnt{C_{{\mathrm{1}}}} . \TLLsym{[}  \TLLnt{t_{{\mathrm{1}}}}  \TLLsym{/}  \mathit{x}  \TLLsym{]}  \TLLnt{t}   \TLLsym{:}  \TLLnt{C_{{\mathrm{2}}}}$ holds.  However, by the definition of substitution
  we know $ \lambda  \mathit{y}  :  \TLLnt{C_{{\mathrm{1}}}} . \TLLsym{[}  \TLLnt{t_{{\mathrm{1}}}}  \TLLsym{/}  \mathit{x}  \TLLsym{]}  \TLLnt{t}  = \TLLsym{[}  \TLLnt{t_{{\mathrm{1}}}}  \TLLsym{/}  \mathit{x}  \TLLsym{]}  \TLLsym{(}   \lambda  \mathit{y}  :  \TLLnt{C_{{\mathrm{1}}}} . \TLLnt{t}   \TLLsym{)}$, and thus, we obtain our result.

\item[] Case.\\ 
  \begin{center}
    \begin{math}
      $$\mprset{flushleft}
      \inferrule* [right=$\TLLdruletyXXimpEName$] {
        \Gamma_{{\mathrm{1}}}  \TLLsym{,}  \mathit{x}  \TLLsym{:}  \TLLnt{A}  \TLLsym{,}  \Gamma_{{\mathrm{2}}}  \vdash  \TLLnt{t'_{{\mathrm{1}}}}  \TLLsym{:}  \TLLnt{C_{{\mathrm{1}}}}  \to  \TLLnt{C_{{\mathrm{2}}}}  \quad  \Gamma_{{\mathrm{1}}}  \TLLsym{,}  \mathit{x}  \TLLsym{:}  \TLLnt{A}  \TLLsym{,}  \Gamma_{{\mathrm{2}}}  \vdash  \TLLnt{t'_{{\mathrm{2}}}}  \TLLsym{:}  \TLLnt{C_{{\mathrm{1}}}}
      }{\Gamma_{{\mathrm{1}}}  \TLLsym{,}  \mathit{x}  \TLLsym{:}  \TLLnt{A}  \TLLsym{,}  \Gamma_{{\mathrm{2}}}  \vdash  \TLLnt{t'_{{\mathrm{1}}}} \, \TLLnt{t'_{{\mathrm{2}}}}  \TLLsym{:}  \TLLnt{C_{{\mathrm{2}}}}}
    \end{math}
  \end{center}
  We now have that $\TLLnt{B} = \TLLnt{C_{{\mathrm{2}}}}$ and $\TLLnt{t_{{\mathrm{2}}}} = \TLLnt{t'_{{\mathrm{1}}}} \, \TLLnt{t'_{{\mathrm{2}}}}$.  By the induction hypothesis
  we know that $\Gamma_{{\mathrm{1}}}  \TLLsym{,}  \Gamma_{{\mathrm{2}}}  \vdash  \TLLsym{[}  \TLLnt{t_{{\mathrm{1}}}}  \TLLsym{/}  \mathit{x}  \TLLsym{]}  \TLLnt{t'_{{\mathrm{1}}}}  \TLLsym{:}  \TLLnt{C_{{\mathrm{1}}}}  \to  \TLLnt{C_{{\mathrm{2}}}}$ and $\Gamma_{{\mathrm{1}}}  \TLLsym{,}  \Gamma_{{\mathrm{2}}}  \vdash  \TLLsym{[}  \TLLnt{t_{{\mathrm{1}}}}  \TLLsym{/}  \mathit{x}  \TLLsym{]}  \TLLnt{t'_{{\mathrm{2}}}}  \TLLsym{:}  \TLLnt{C_{{\mathrm{1}}}}$ both hold.
  Then by reapplying the rule we obtain that $\Gamma_{{\mathrm{1}}}  \TLLsym{,}  \Gamma_{{\mathrm{2}}}  \vdash  \TLLsym{(}  \TLLsym{[}  \TLLnt{t_{{\mathrm{1}}}}  \TLLsym{/}  \mathit{x}  \TLLsym{]}  \TLLnt{t'_{{\mathrm{1}}}}  \TLLsym{)} \, \TLLsym{(}  \TLLsym{[}  \TLLnt{t_{{\mathrm{1}}}}  \TLLsym{/}  \mathit{x}  \TLLsym{]}  \TLLnt{t'_{{\mathrm{2}}}}  \TLLsym{)}  \TLLsym{:}  \TLLnt{C_{{\mathrm{2}}}}$, and thus,
  by the definition of substitution $\Gamma_{{\mathrm{1}}}  \TLLsym{,}  \Gamma_{{\mathrm{2}}}  \vdash  \TLLsym{[}  \TLLnt{t_{{\mathrm{1}}}}  \TLLsym{/}  \mathit{x}  \TLLsym{]}  \TLLsym{(}  \TLLnt{t'_{{\mathrm{1}}}} \, \TLLnt{t'_{{\mathrm{2}}}}  \TLLsym{)}  \TLLsym{:}  \TLLnt{C_{{\mathrm{2}}}}$ holds.
  
  
\item[] Case.\\ 
  \begin{center}
    \scriptsize
    \begin{math}
      $$\mprset{flushleft}
      \inferrule* [right=$\TLLdruletyXXboxIName$] {
         \Gamma_{{\mathrm{1}}}  \TLLsym{,}  \mathit{x}  \TLLsym{:}  \TLLnt{A}  \TLLsym{,}  \Gamma_{{\mathrm{2}}}  \vdash  \TLLnt{t'_{{\mathrm{1}}}}  \TLLsym{:}  \Box \, \TLLnt{C_{{\mathrm{1}}}}  \TLLsym{,} \, ... \, \TLLsym{,}  \Gamma_{{\mathrm{1}}}  \TLLsym{,}  \mathit{x}  \TLLsym{:}  \TLLnt{A}  \TLLsym{,}  \Gamma_{{\mathrm{2}}}  \vdash  \TLLnt{t'_{\TLLmv{k}}}  \TLLsym{:}  \Box \, \TLLnt{C_{\TLLmv{k}}}   \quad  \mathit{x_{{\mathrm{1}}}}  \TLLsym{:}  \Box \, \TLLnt{C_{{\mathrm{1}}}}  \TLLsym{,} \, ... \, \TLLsym{,}  \mathit{x_{\TLLmv{k}}}  \TLLsym{:}  \Box \, \TLLnt{C_{\TLLmv{k}}}  \vdash  \TLLnt{t}  \TLLsym{:}  \TLLnt{C}
      }{\Gamma_{{\mathrm{1}}}  \TLLsym{,}  \mathit{x}  \TLLsym{:}  \TLLnt{A}  \TLLsym{,}  \Gamma_{{\mathrm{2}}}  \vdash   \mathsf{let}_\Box\, \mathit{x_{{\mathrm{1}}}}  \TLLsym{:}  \Box \, \TLLnt{C_{{\mathrm{1}}}}  \TLLsym{,} \, ... \, \TLLsym{,}  \mathit{x_{\TLLmv{k}}}  \TLLsym{:}  \Box \, \TLLnt{C_{\TLLmv{k}}} \,\mathsf{be}\, \TLLnt{t'_{{\mathrm{1}}}}  \TLLsym{,} \, ... \, \TLLsym{,}  \TLLnt{t'_{\TLLmv{k}}} \,\mathsf{in}\, \TLLnt{t}   \TLLsym{:}  \Box \, \TLLnt{C}}
    \end{math}
  \end{center}
  In this case $\TLLnt{B} = \Box \, \TLLnt{C}$ and
  $\TLLnt{t_{{\mathrm{2}}}} =  \mathsf{let}_\Box\, \mathit{x_{{\mathrm{1}}}}  \TLLsym{:}  \Box \, \TLLnt{C_{{\mathrm{1}}}}  \TLLsym{,} \, ... \, \TLLsym{,}  \mathit{x_{\TLLmv{k}}}  \TLLsym{:}  \Box \, \TLLnt{C_{\TLLmv{k}}} \,\mathsf{be}\, \TLLnt{t'_{{\mathrm{1}}}}  \TLLsym{,} \, ... \, \TLLsym{,}  \TLLnt{t'_{\TLLmv{k}}} \,\mathsf{in}\, \TLLnt{t} $.  By the induction hypothesis
  we know that 
  \[ \Gamma_{{\mathrm{1}}}  \TLLsym{,}  \Gamma_{{\mathrm{2}}}  \vdash  \TLLsym{[}  \TLLnt{t_{{\mathrm{1}}}}  \TLLsym{/}  \mathit{x}  \TLLsym{]}  \TLLnt{t'_{{\mathrm{1}}}}  \TLLsym{:}  \Box \, \TLLnt{C_{{\mathrm{1}}}} , \ldots , \Gamma_{{\mathrm{1}}}  \TLLsym{,}  \mathit{x}  \TLLsym{:}  \TLLnt{A}  \TLLsym{,}  \Gamma_{{\mathrm{2}}}  \vdash  \TLLsym{[}  \TLLnt{t_{{\mathrm{1}}}}  \TLLsym{/}  \mathit{x}  \TLLsym{]}  \TLLnt{t'_{\TLLmv{k}}}  \TLLsym{:}  \Box \, \TLLnt{C_{\TLLmv{k}}} \] all hold. Then by reapplying
  the rule we know that
  \[ \Gamma_{{\mathrm{1}}}  \TLLsym{,}  \Gamma_{{\mathrm{2}}}  \vdash   \mathsf{let}_\Box\, \mathit{x_{{\mathrm{1}}}}  \TLLsym{:}  \Box \, \TLLnt{C_{{\mathrm{1}}}}  \TLLsym{,} \, ... \, \TLLsym{,}  \mathit{x_{\TLLmv{k}}}  \TLLsym{:}  \Box \, \TLLnt{C_{\TLLmv{k}}} \,\mathsf{be}\, \TLLsym{[}  \TLLnt{t_{{\mathrm{1}}}}  \TLLsym{/}  \mathit{x}  \TLLsym{]}  \TLLnt{t'_{{\mathrm{1}}}}  \TLLsym{,} \, ... \, \TLLsym{,}  \TLLsym{[}  \TLLnt{t_{{\mathrm{1}}}}  \TLLsym{/}  \mathit{x}  \TLLsym{]}  \TLLnt{t'_{\TLLmv{k}}} \,\mathsf{in}\, \TLLnt{t}   \TLLsym{:}  \Box \, \TLLnt{C}, \] but by the definition
  of substitution and the fact that $\TLLsym{[}  \TLLnt{t_{{\mathrm{1}}}}  \TLLsym{/}  \mathit{x}  \TLLsym{]}  \TLLnt{t} = \TLLnt{t}$ because $\TLLnt{t}$ does not depend on $\mathit{x}$ we know that
  \[ \Gamma_{{\mathrm{1}}}  \TLLsym{,}  \Gamma_{{\mathrm{2}}}  \vdash  \TLLsym{[}  \TLLnt{t_{{\mathrm{1}}}}  \TLLsym{/}  \mathit{x}  \TLLsym{]}  \TLLsym{(}   \mathsf{let}_\Box\, \mathit{x_{{\mathrm{1}}}}  \TLLsym{:}  \Box \, \TLLnt{C_{{\mathrm{1}}}}  \TLLsym{,} \, ... \, \TLLsym{,}  \mathit{x_{\TLLmv{k}}}  \TLLsym{:}  \Box \, \TLLnt{C_{\TLLmv{k}}} \,\mathsf{be}\, \TLLnt{t'_{{\mathrm{1}}}}  \TLLsym{,} \, ... \, \TLLsym{,}  \TLLnt{t'_{\TLLmv{k}}} \,\mathsf{in}\, \TLLnt{t}   \TLLsym{)}  \TLLsym{:}  \Box \, \TLLnt{C} \] holds.
  
\item[] Case.\\ 
  \begin{center}
    \begin{math}
      $$\mprset{flushleft}
      \inferrule* [right=$\TLLdruletyXXboxEName$] {
        \Gamma_{{\mathrm{1}}}  \TLLsym{,}  \mathit{x}  \TLLsym{:}  \TLLnt{A}  \TLLsym{,}  \Gamma_{{\mathrm{2}}}  \vdash  \TLLnt{t}  \TLLsym{:}  \Box \, \TLLnt{C}
      }{\Gamma_{{\mathrm{1}}}  \TLLsym{,}  \mathit{x}  \TLLsym{:}  \TLLnt{A}  \TLLsym{,}  \Gamma_{{\mathrm{2}}}  \vdash  \mathsf{unbox}_\Box\, \, \TLLnt{t}  \TLLsym{:}  \TLLnt{C}}
    \end{math}
  \end{center}
  This case easily follows by first applying the induction hypothesis and reapplying the rule.

\item[] Case.\\ 
  \begin{center}
    \scriptsize
    \begin{math}
      $$\mprset{flushleft}
      \inferrule* [right=$\TLLdruletyXXbboxIName$] {
         \Gamma_{{\mathrm{1}}}  \TLLsym{,}  \mathit{x}  \TLLsym{:}  \TLLnt{A}  \TLLsym{,}  \Gamma_{{\mathrm{2}}}  \vdash  \TLLnt{t_{{\mathrm{1}}}}  \TLLsym{:}  \blacksquare \, \TLLnt{C_{{\mathrm{1}}}}  \TLLsym{,} \, ... \, \TLLsym{,}  \Gamma_{{\mathrm{1}}}  \TLLsym{,}  \mathit{x}  \TLLsym{:}  \TLLnt{A}  \TLLsym{,}  \Gamma_{{\mathrm{2}}}  \vdash  \TLLnt{t_{\TLLmv{k}}}  \TLLsym{:}  \blacksquare \, \TLLnt{C_{\TLLmv{k}}}   \quad  \mathit{x_{{\mathrm{1}}}}  \TLLsym{:}  \blacksquare \, \TLLnt{C_{{\mathrm{1}}}}  \TLLsym{,} \, ... \, \TLLsym{,}  \mathit{x_{\TLLmv{k}}}  \TLLsym{:}  \blacksquare \, \TLLnt{C_{\TLLmv{k}}}  \vdash  \TLLnt{t}  \TLLsym{:}  \TLLnt{C}
      }{\Gamma_{{\mathrm{1}}}  \TLLsym{,}  \mathit{x}  \TLLsym{:}  \TLLnt{A}  \TLLsym{,}  \Gamma_{{\mathrm{2}}}  \vdash   \mathsf{let}_\blacksquare\, \mathit{x_{{\mathrm{1}}}}  \TLLsym{:}  \blacksquare \, \TLLnt{C_{{\mathrm{1}}}}  \TLLsym{,} \, ... \, \TLLsym{,}  \mathit{x_{\TLLmv{k}}}  \TLLsym{:}  \blacksquare \, \TLLnt{C_{\TLLmv{k}}} \,\mathsf{be}\, \TLLnt{t_{{\mathrm{1}}}}  \TLLsym{,} \, ... \, \TLLsym{,}  \TLLnt{t_{\TLLmv{k}}} \,\mathsf{in}\, \TLLnt{t}   \TLLsym{:}  \blacksquare \, \TLLnt{C}}
    \end{math}
  \end{center}
  This case is similar to the case for $\TLLdruletyXXboxIName$.

\item[] Case.\\ 
  \begin{center}
    \begin{math}
      $$\mprset{flushleft}
      \inferrule* [right=$\TLLdruletyXXbboxEName$] {
        \Gamma_{{\mathrm{1}}}  \TLLsym{,}  \mathit{x}  \TLLsym{:}  \TLLnt{A}  \TLLsym{,}  \Gamma_{{\mathrm{2}}}  \vdash  \TLLnt{t}  \TLLsym{:}  \blacksquare \, \TLLnt{C}
      }{\Gamma_{{\mathrm{1}}}  \TLLsym{,}  \mathit{x}  \TLLsym{:}  \TLLnt{A}  \TLLsym{,}  \Gamma_{{\mathrm{2}}}  \vdash  \mathsf{unbox}_\blacksquare\, \, \TLLnt{t}  \TLLsym{:}  \TLLnt{C}}
    \end{math}
  \end{center}
  This case is similar to the case for $\TLLdruletyXXboxEName$.
\end{itemize}
\end{proof}
% subsubsection proof_of_lemma_substitution_for_typing (end)

\subsubsection{Proof of Weakening}
\label{subsubsec:proof_of_lemma:weakening}

\begin{lemma*}[Weakening]
  If $\Gamma_{{\mathrm{1}}}  \TLLsym{,}  \Gamma_{{\mathrm{2}}}  \vdash  \TLLnt{t}  \TLLsym{:}  \TLLnt{B}$, then $\Gamma_{{\mathrm{1}}}  \TLLsym{,}  \mathit{x}  \TLLsym{:}  \TLLnt{A}  \TLLsym{,}  \Gamma_{{\mathrm{2}}}  \vdash  \TLLnt{t}  \TLLsym{:}  \TLLnt{B}$.
\end{lemma*}

\begin{proof}
  This proof is by induction on the form of $\Gamma_{{\mathrm{1}}}  \TLLsym{,}  \Gamma_{{\mathrm{2}}}  \vdash  \TLLnt{t}  \TLLsym{:}  \TLLnt{B}$.
\begin{itemize}
\item[] Case.\\ 
  \begin{center}
    \begin{math}
      $$\mprset{flushleft}
      \inferrule* [right=$\TLLdruletyXXaxName$] {
        \ 
      }{\Gamma_{{\mathrm{1}}}  \TLLsym{,}  \Gamma_{{\mathrm{2}}}  \TLLsym{,}  \mathit{y}  \TLLsym{:}  \TLLnt{C}  \vdash  \mathit{y}  \TLLsym{:}  \TLLnt{C}}
    \end{math}
  \end{center}
  In this case we have that $\TLLnt{B} = \TLLnt{C}$ and $\TLLnt{t} = \mathit{y}$.  We
  must show that $\Gamma_{{\mathrm{1}}}  \TLLsym{,}  \mathit{x}  \TLLsym{:}  \TLLnt{A}  \TLLsym{,}  \Gamma_{{\mathrm{2}}}  \TLLsym{,}  \mathit{y}  \TLLsym{:}  \TLLnt{C}  \vdash  \mathit{y}  \TLLsym{:}  \TLLnt{C}$ holds, but this
  clearly holds by reapplying the rule.

\item[] Case.\\ 
  \begin{center}
    \begin{math}
      $$\mprset{flushleft}
      \inferrule* [right=$\TLLdruletyXXfalseName$] {
        \ 
      }{\Gamma_{{\mathrm{1}}}  \TLLsym{,}  \Gamma_{{\mathrm{2}}}  \TLLsym{,}  \mathit{y}  \TLLsym{:}  \perp  \vdash   \mathsf{contra}   \TLLsym{:}  \TLLnt{C}}
    \end{math}
  \end{center}
  Similar to the previous case.

\item[] Case.\\ 
  \begin{center}
    \begin{math}
      $$\mprset{flushleft}
      \inferrule* [right=$\TLLdruletyXXimpIName$] {
        \Gamma_{{\mathrm{1}}}  \TLLsym{,}  \Gamma_{{\mathrm{2}}}  \TLLsym{,}  \mathit{y}  \TLLsym{:}  \TLLnt{C_{{\mathrm{1}}}}  \vdash  \TLLnt{t'}  \TLLsym{:}  \TLLnt{C_{{\mathrm{2}}}}
      }{\Gamma_{{\mathrm{1}}}  \TLLsym{,}  \Gamma_{{\mathrm{2}}}  \vdash   \lambda  \mathit{y}  :  \TLLnt{C_{{\mathrm{1}}}} . \TLLnt{t'}   \TLLsym{:}  \TLLnt{C_{{\mathrm{1}}}}  \to  \TLLnt{C_{{\mathrm{2}}}}}
    \end{math}
  \end{center}
  In this case $\TLLnt{B} = \TLLnt{C_{{\mathrm{1}}}}  \to  \TLLnt{C_{{\mathrm{2}}}}$ and $\TLLnt{t} =  \lambda  \mathit{y}  :  \TLLnt{C_{{\mathrm{1}}}} . \TLLnt{t'} $.  By the induction hypothesis
  $\Gamma_{{\mathrm{1}}}  \TLLsym{,}  \mathit{x}  \TLLsym{:}  \TLLnt{A}  \TLLsym{,}  \Gamma_{{\mathrm{2}}}  \TLLsym{,}  \mathit{y}  \TLLsym{:}  \TLLnt{C_{{\mathrm{1}}}}  \vdash  \TLLnt{t'}  \TLLsym{:}  \TLLnt{C_{{\mathrm{2}}}}$ holds, and then by reapplying the rule
  $\Gamma_{{\mathrm{1}}}  \TLLsym{,}  \mathit{x}  \TLLsym{:}  \TLLnt{A}  \TLLsym{,}  \Gamma_{{\mathrm{2}}}  \vdash   \lambda  \mathit{y}  :  \TLLnt{C_{{\mathrm{1}}}} . \TLLnt{t'}   \TLLsym{:}  \TLLnt{C_{{\mathrm{2}}}}$ holds.

\item[] Case.\\ 
  \begin{center}
    \begin{math}
      $$\mprset{flushleft}
      \inferrule* [right=$\TLLdruletyXXimpEName$] {
        \Gamma_{{\mathrm{1}}}  \TLLsym{,}  \Gamma_{{\mathrm{2}}}  \vdash  \TLLnt{t_{{\mathrm{1}}}}  \TLLsym{:}  \TLLnt{C_{{\mathrm{1}}}}  \to  \TLLnt{C_{{\mathrm{2}}}}  \quad  \Gamma_{{\mathrm{1}}}  \TLLsym{,}  \Gamma_{{\mathrm{2}}}  \vdash  \TLLnt{t_{{\mathrm{2}}}}  \TLLsym{:}  \TLLnt{C_{{\mathrm{1}}}}
      }{\Gamma_{{\mathrm{1}}}  \TLLsym{,}  \Gamma_{{\mathrm{2}}}  \vdash  \TLLnt{t_{{\mathrm{1}}}} \, \TLLnt{t_{{\mathrm{2}}}}  \TLLsym{:}  \TLLnt{C_{{\mathrm{2}}}}}
    \end{math}
  \end{center}
  By the induction hypothesis $\Gamma_{{\mathrm{1}}}  \TLLsym{,}  \mathit{x}  \TLLsym{:}  \TLLnt{A}  \TLLsym{,}  \Gamma_{{\mathrm{2}}}  \vdash  \TLLnt{t_{{\mathrm{1}}}}  \TLLsym{:}  \TLLnt{C_{{\mathrm{1}}}}  \to  \TLLnt{C_{{\mathrm{2}}}}$ and
  $\Gamma_{{\mathrm{1}}}  \TLLsym{,}  \mathit{x}  \TLLsym{:}  \TLLnt{A}  \TLLsym{,}  \Gamma_{{\mathrm{2}}}  \vdash  \TLLnt{t_{{\mathrm{2}}}}  \TLLsym{:}  \TLLnt{C_{{\mathrm{1}}}}$, and thus by reapplying the rule
  $\Gamma_{{\mathrm{1}}}  \TLLsym{,}  \mathit{x}  \TLLsym{:}  \TLLnt{A}  \TLLsym{,}  \Gamma_{{\mathrm{2}}}  \vdash  \TLLnt{t_{{\mathrm{1}}}} \, \TLLnt{t_{{\mathrm{2}}}}  \TLLsym{:}  \TLLnt{C_{{\mathrm{2}}}}$.


\item[] Case.\\ 
  \begin{center}
    \scriptsize
    \begin{math}
      $$\mprset{flushleft}
      \inferrule* [right=$\TLLdruletyXXboxIName$] {
         \Gamma_{{\mathrm{1}}}  \TLLsym{,}  \Gamma_{{\mathrm{2}}}  \vdash  \TLLnt{t_{{\mathrm{1}}}}  \TLLsym{:}  \Box \, \TLLnt{C_{{\mathrm{1}}}}  \TLLsym{,} \, ... \, \TLLsym{,}  \Gamma_{{\mathrm{1}}}  \TLLsym{,}  \Gamma_{{\mathrm{2}}}  \vdash  \TLLnt{t_{\TLLmv{k}}}  \TLLsym{:}  \Box \, \TLLnt{C_{\TLLmv{k}}}   \quad  \mathit{x_{{\mathrm{1}}}}  \TLLsym{:}  \Box \, \TLLnt{C_{{\mathrm{1}}}}  \TLLsym{,} \, ... \, \TLLsym{,}  \mathit{x_{\TLLmv{k}}}  \TLLsym{:}  \Box \, \TLLnt{C_{\TLLmv{k}}}  \vdash  \TLLnt{t'}  \TLLsym{:}  \TLLnt{C}
      }{\Gamma_{{\mathrm{1}}}  \TLLsym{,}  \Gamma_{{\mathrm{2}}}  \vdash   \mathsf{let}_\Box\, \mathit{x_{{\mathrm{1}}}}  \TLLsym{:}  \Box \, \TLLnt{C_{{\mathrm{1}}}}  \TLLsym{,} \, ... \, \TLLsym{,}  \mathit{x_{\TLLmv{k}}}  \TLLsym{:}  \Box \, \TLLnt{C_{\TLLmv{k}}} \,\mathsf{be}\, \TLLnt{t_{{\mathrm{1}}}}  \TLLsym{,} \, ... \, \TLLsym{,}  \TLLnt{t_{\TLLmv{k}}} \,\mathsf{in}\, \TLLnt{t'}   \TLLsym{:}  \Box \, \TLLnt{C}}
    \end{math}
  \end{center}
  This case is similar to the previous case.  First, apply the
  induction hypothesis to the left-most premise, and then reapply
  the rule.

\item[] Case.\\ 
  \begin{center}
    \begin{math}
      $$\mprset{flushleft}
      \inferrule* [right=$\TLLdruletyXXboxEName$] {
        \Gamma_{{\mathrm{1}}}  \TLLsym{,}  \Gamma_{{\mathrm{2}}}  \vdash  \TLLnt{t'}  \TLLsym{:}  \Box \, \TLLnt{C}
      }{\Gamma_{{\mathrm{1}}}  \TLLsym{,}  \Gamma_{{\mathrm{2}}}  \vdash  \mathsf{unbox}_\Box\, \, \TLLnt{t'}  \TLLsym{:}  \TLLnt{C}}
    \end{math}
  \end{center}      
  Similar to the previous cases.
  
\item[] Case.\\ 
  \begin{center}
    \scriptsize
    \begin{math}
      $$\mprset{flushleft}
      \inferrule* [right=$\TLLdruletyXXbboxIName$] {
         \Gamma_{{\mathrm{1}}}  \TLLsym{,}  \Gamma_{{\mathrm{2}}}  \vdash  \TLLnt{t_{{\mathrm{1}}}}  \TLLsym{:}  \blacksquare \, \TLLnt{C_{{\mathrm{1}}}}  \TLLsym{,} \, ... \, \TLLsym{,}  \Gamma_{{\mathrm{1}}}  \TLLsym{,}  \Gamma_{{\mathrm{2}}}  \vdash  \TLLnt{t_{\TLLmv{k}}}  \TLLsym{:}  \blacksquare \, \TLLnt{C_{\TLLmv{k}}}   \quad  \mathit{x_{{\mathrm{1}}}}  \TLLsym{:}  \blacksquare \, \TLLnt{C_{{\mathrm{1}}}}  \TLLsym{,} \, ... \, \TLLsym{,}  \mathit{x_{\TLLmv{k}}}  \TLLsym{:}  \blacksquare \, \TLLnt{C_{\TLLmv{k}}}  \vdash  \TLLnt{t'}  \TLLsym{:}  \TLLnt{C}
      }{\Gamma_{{\mathrm{1}}}  \TLLsym{,}  \Gamma_{{\mathrm{2}}}  \vdash   \mathsf{let}_\blacksquare\, \mathit{x_{{\mathrm{1}}}}  \TLLsym{:}  \blacksquare \, \TLLnt{C_{{\mathrm{1}}}}  \TLLsym{,} \, ... \, \TLLsym{,}  \mathit{x_{\TLLmv{k}}}  \TLLsym{:}  \blacksquare \, \TLLnt{C_{\TLLmv{k}}} \,\mathsf{be}\, \TLLnt{t_{{\mathrm{1}}}}  \TLLsym{,} \, ... \, \TLLsym{,}  \TLLnt{t_{\TLLmv{k}}} \,\mathsf{in}\, \TLLnt{t'}   \TLLsym{:}  \blacksquare \, \TLLnt{C}}
    \end{math}
  \end{center}
  Similar to the case for $\TLLdruletyXXboxIName$.

\item[] Case.\\ 
  \begin{center}
    \begin{math}
      $$\mprset{flushleft}
      \inferrule* [right=$\TLLdruletyXXbboxEName$] {
        \Gamma_{{\mathrm{1}}}  \TLLsym{,}  \Gamma_{{\mathrm{2}}}  \vdash  \TLLnt{t'}  \TLLsym{:}  \blacksquare \, \TLLnt{C}
      }{\Gamma_{{\mathrm{1}}}  \TLLsym{,}  \Gamma_{{\mathrm{2}}}  \vdash  \mathsf{unbox}_\blacksquare\, \, \TLLnt{t'}  \TLLsym{:}  \TLLnt{C}}
    \end{math}
  \end{center}
  Similar to the case for $\TLLdruletyXXboxEName$.

\end{itemize}
\end{proof}
% subsubsection proof_of_lemma:weakening (end)
