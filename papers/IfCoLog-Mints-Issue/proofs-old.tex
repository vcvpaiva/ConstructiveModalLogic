\subsubsection{Proof of Subject Reduction}
\label{subsec:proof_of_subject_reduction}
This is a proof by induction on the form of $[[G;D |- t : A]]$.  We
only show non-trivial cases, the absent cases either hold vacuously,
by simply using the induction hypothesis and reapplying the rule, or
are similar to the cases we give here.

\begin{description}        
\item[\cW]
  \[
  \TLLdruletyXXconjEOne{}
  \leqno{\raise 8 pt\hbox{\textbf{Case}}}
  \]
  The only non-trivial case is when $[[t]] = [[(t1, t2)]]$ and
  $[[t']] = [[t1]]$ for some terms $[[t1]]$ and $[[t2]]$.  Thus, we
  know $[[G;D |- (t1, t2) : A x B]]$, and we must show that
  $[[G;D |- t1 : A]]$, but this holds by inversion.
  
\item[\cW]
  \[
  \TLLdruletyXXdisjE{}
  \leqno{\raise 8 pt\hbox{\textbf{Case}}}
  \]     
  The only non-trivial cases arise when either $[[t]] = [[inj1 t3]]$ and $[[t']] = [[ [t3/x]t1]]$, or
  $[[t]] = [[inj2 t3]]$ and $[[t']] = [[ [t3/x]t2]]$.  We prove the former, because the latter is similar.
  It suffices to show that $[[G;D |- [t3/x]t1 : C]]$.  We know by assumption that $[[G;D, x : A |- t1 : C]]$,
  and since we know $[[G;D |- inj2 t3 : A + B]]$ by assumption, then we know $[[G;D |- t3 : A]]$ by inversion.
  Finally, by substitution for typing we obtain our desired result.

\item[\cW]
  \[
  \TLLdruletyXXimpE{}
  \leqno{\raise 8 pt\hbox{\textbf{Case}}}
  \]  
  The only non-trivial case arises when $[[t1]] = [[\x:A.t3]]$ and $[[t']] = [[ [t2/x]t3]]$.
  We know by assumption that $[[G;D |- \x:A.t3 : A -> B]]$, and by inversion we know
  $[[G;D,x : A |- t3 : B]]$.  Finally, we also know by assumption
  that $[[G;D |- t2 : A]]$, and so by substitution for typing we
  know $[[G;D |- [t2/x]t3 : B]]$ which is our desired result.

\item[\cW]
  \[
  \TLLdruletyXXboxE{}
  \leqno{\raise 8 pt\hbox{\textbf{Case}}}
  \]
  The only non-trivial cases arise when either $[[t1]] = [[Box t3]]$
  and $[[t']] = [[ [t3/x]t2]]$, or $[[t1]] = [[letBox y = t3 in t4]]$
  and $[[t']] = [[ letBox y = t3 in letBox x = t4 in t2]]$.  We show
  the latter, because the former is similar to the previous case.

  We must show $[[ G;D |- letBox y = t3 in letBox x = t4 in t2 : B]]$.
  That is, the following must hold:
  \[
  \small
  \mprset{flushleft}
  \inferrule* [right=$\Box_{\mathcal{E}}$] {
    [[G;D |- t3 : Box C]]
    \\
    $$\mprset{flushleft}
    \inferrule* [right=$\Box_{\mathcal{E}}$] {
      [[G, y : C;D |- t4 : Box A]]
      \\
        [[G, y : C, x : A;D |- t2 : B]]
    }{[[G, y : C;D |- letBox x = t4 in t2 : B]]}
  }{[[ G;D |- letBox y = t3 in letBox x = t4 in t2 : B]]}
  \]    
  The $[[t3]]$ and $[[t4]]$ premises hold by inversion, and the
  $[[t2]]$ premise holds by weakening using our assumption that
  $[[G, x : A;D |- t2 : B]]$ holds.  Thus, we obtain our result.
\end{description}  
% subsubsection proof_of_subject_reduction (end)
