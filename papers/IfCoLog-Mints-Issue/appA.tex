\section{Monoidal  Categories}
\label{sec:symmetric_monoidal_closed_categories}

\begin{definition}
  \label{def:monoidal-category}
  A \textbf{symmetric monoidal category (SMC)} is a category, $\cat{M}$,
  with the following data:
  \begin{itemize}
  \item An object $I$ of $\cat{M}$,
  \item A bi-functor $\otimes : \cat{M} \times \cat{M} \mto \cat{M}$,
  \item The following natural isomorphisms:
    \[
    \begin{array}{lll}
      \lambda_A : I \otimes A \mto A\\
      \rho_A : A \otimes I \mto A\\      
      \alpha_{A,B,C} : (A \otimes B) \otimes C \mto A \otimes (B \otimes C)\\
    \end{array}
    \]
  \item A symmetry natural transformation:
    \[
    \beta_{A,B} : A \otimes B \mto B \otimes A
    \]
  \item Subject to the following coherence diagrams:
    \begin{mathpar}
      \bfig
      \vSquares|ammmmma|/->`->```->``<-/[
        ((A \otimes B) \otimes C) \otimes D`
        (A \otimes (B \otimes C)) \otimes D`
        (A \otimes B) \otimes (C \otimes D)``
        A \otimes (B \otimes (C \otimes D))`
        A \otimes ((B \otimes C) \otimes D);
        \alpha_{A,B,C} \otimes \id_D`
        \alpha_{A \otimes B,C,D}```
        \alpha_{A,B,C \otimes D}``
        \id_A \otimes \alpha_{B,C,D}]      
      
      \morphism(1433,1000)|m|<0,-1000>[
        (A \otimes (B \otimes C)) \otimes D`
        A \otimes ((B \otimes C) \otimes D);
        \alpha_{A,B \otimes C,D}]
      \efig
      \and
      \bfig
      \hSquares|aammmaa|/->`->`->``->`->`->/[
        (A \otimes B) \otimes C`
        A \otimes (B \otimes C)`
        (B \otimes C) \otimes A`
        (B \otimes A) \otimes C`
        B \otimes (A \otimes C)`
        B \otimes (C \otimes A);
        \alpha_{A,B,C}`
        \beta_{A,B \otimes C}`
        \beta_{A,B} \otimes \id_C``
        \alpha_{B,C,A}`
        \alpha_{B,A,C}`
        \id_B \otimes \beta_{A,C}]
      \efig      
    \end{mathpar}
    \begin{mathpar}
      \bfig
      \Vtriangle[
        (A \otimes I) \otimes B`
        A \otimes (I \otimes B)`
        A \otimes B;
        \alpha_{A,I,B}`
        \rho_{A}`
        \lambda_{B}]
      \efig
      \and
      \bfig
      \btriangle[
        A \otimes B`
        B \otimes A`
        A \otimes B;
        \beta_{A,B}`
        \id_{A \otimes B}`
        \beta_{B,A}]
      \efig
      \and
      \bfig
      \Vtriangle[
        I \otimes A`
        A \otimes I`
        A;
        \beta_{I,A}`
        \lambda_A`
        \rho_A]
      \efig
    \end{mathpar}    
  \end{itemize}
\end{definition}


\begin{definition}
  \label{def:SMCC}
  A \textbf{symmetric monoidal closed category (SMCC)} is a symmetric
  monoidal category, $(\cat{M},I,\otimes)$, such that, for any object
  $B$ of $\cat{M}$, the functor $- \otimes B : \cat{M} \mto \cat{M}$
  has a specified right adjoint.  Hence, for any objects $A$ and $C$
  of $\cat{M}$ there is an object $A \limp B$ of $\cat{M}$ and a
  natural bijection:
  \[
  \Hom{\cat{M}}{A \otimes B}{C} \cong \Hom{\cat{M}}{A}{B \limp C}
  \]
\end{definition}

A \textit{cartesian closed category} is a symmetric monoidal closed category whose tensor product is a cartesian product and its unit $I$ is a real terminal object 1.

\begin{definition}
  \label{def:SMCFUN}
  Suppose we are given two symmetric monoidal closed categories $(\cat{M}_1,I_1,\otimes_1,\alpha_1,\lambda_1,\rho_1,\beta_1)$ and
  $(\cat{M}_2,I_2,\otimes_2,\alpha_2,\lambda_2,\rho_2,\beta_2)$.  Then a
  \textbf{symmetric monoidal functor} is a functor $F : \cat{M}_1 \mto
  \cat{M}_2$, a map $m_I : I_2 \mto FI_1$ and a natural transformation
  $m_{A,B} : FA \otimes_2 FB \mto F(A \otimes_1 B)$ subject to the
  following coherence conditions:
  \begin{mathpar}
    \bfig
    \vSquares|ammmmma|/->`->`->``->`->`->/[
      (FA \otimes_2 FB) \otimes_2 FC`
      FA \otimes_2 (FB \otimes_2 FC)`
      F(A \otimes_1 B) \otimes_2 FC`
      FA \otimes_2 F(B \otimes_1 C)`
      F((A \otimes_1 B) \otimes_1 C)`
      F(A \otimes_1 (B \otimes_1 C));
      {\alpha_2}_{FA,FB,FC}`
      m_{A,B} \otimes \id_{FC}`
      \id_{FA} \otimes m_{B,C}``
      m_{A \otimes_1 B,C}`
      m_{A,B \otimes_1 C}`
      F{\alpha_1}_{A,B,C}]
    \efig
    \end{mathpar}
%    \and
\begin{mathpar}
    \bfig
    \square|amma|/->`->`<-`->/<1000,500>[
      I_2 \otimes_2 FA`
      FA`
      FI_1 \otimes_2 FA`
      F(I_1 \otimes_1 A);
      {\lambda_2}_{FA}`
      m_{I} \otimes \id_{FA}`
      F{\lambda_1}_{A}`
      m_{I_1,A}]
    \efig
    \and
    \bfig
    \square|amma|/->`->`<-`->/<1000,500>[
      FA \otimes_2 I_2`
      FA`
      FA \otimes_2 FI_1`
      F(A \otimes_1 I_1);
      {\rho_2}_{FA}`
      \id_{FA} \otimes m_{I}`
      F{\rho_1}_{A}`
      m_{A,I_1}]
    \efig
     \end{mathpar}
     
      \begin{mathpar}
    \bfig
    \square|amma|/->`->`->`->/<1000,500>[
      FA \otimes_2 FB`
      FB \otimes_2 FA`
      F(A \otimes_1 B)`
      F(B \otimes_1 A);
      {\beta_2}_{FA,FB}`
      m_{A,B}`
      m_{B,A}`
      F{\beta_1}_{A,B}]
    \efig
  \end{mathpar}
\end{definition}
A \textit{product  functor} is a symmetric monoidal closed functor between (symmetric) cartesian closed categories. The map $m_1\colon 1\to F1$ and the natural transformations $m_{A,B} : FA \times FB \mto F(A \times B)$ are subject to the adapted coherence conditions:
  \begin{mathpar}
    \bfig
    \vSquares|ammmmma|/->`->`->``->`->`->/[
      (FA \times FB) \times FC`
      FA \times (FB \times FC)`
      F(A \times B) \times FC`
      FA \times F(B \times C)`
      F((A \times B) \times C)`
      F(A \times (B \times C));
      {\alpha_2}_{FA,FB,FC}`
      m_{A,B} \times \id_{FC}`
      \id_{FA} \times m_{B,C}``
      m_{A \times B,C}`
      m_{A,B \times C}`
      F{\alpha_1}_{A,B,C}]
    \efig
    \end{mathpar}
%    \and
\begin{mathpar}
    \bfig
    \square|amma|/->`->`<-`->/<1000,500>[
      1 \times FA`
      FA`
      F1 \times FA`
      F(1 \times A);
      {\lambda_2}_{FA}`
      m_{1} \times \id_{FA}`
      F{\lambda_1}_{A}`
      m_{1,A}]
    \efig
    \and
    \bfig
    \square|amma|/->`->`<-`->/<1000,500>[
      FA \times 1`
      FA`
      FA \times F1`
      F(A \times 1);
      {\rho_2}_{FA}`
      \id_{FA} \times m_{1}`
      F{\rho_1}_{A}`
      m_{A,1}]
    \efig
     \end{mathpar}
     
      \begin{mathpar}
    \bfig
    \square|amma|/->`->`->`->/<1000,500>[
      FA \times FB`
      FB \times FA`
      F(A \times B)`
      F(B \times A);
      {\beta_2}_{FA,FB}`
      m_{A,B}`
      m_{B,A}`
      F{\beta_1}_{A,B}]
    \efig
  \end{mathpar}
Because every cartesian closed category is a symmetric monoidal category where the tensor is a real product and because   cartesian products are unique up to isomorphism, we know we can rewrite these diagrams somewhat.

{\tt how to get to your product functors from here?}

\begin{definition}
  \label{def:SMCNAT}
  Suppose $(\cat{M}_1,I_1,\otimes_1)$ and $(\cat{M}_2,I_2,\otimes_2)$
  are SMCs, and $(F,m)$ and $(G,n)$ are a symmetric monoidal functors
  between $\cat{M}_1$ and $\cat{M}_2$.  Then a \textbf{symmetric
    monoidal natural transformation} is a natural transformation,
  $f : F \mto G$, subject to the following coherence diagrams:
  \begin{mathpar}
    \bfig
    \square<1000,500>[
      FA \otimes_2 FB`
      F(A \otimes_1 B)`
      GA \otimes_2 GB`
      G(A \otimes_1 B);
      m_{A,B}`
      f_A \otimes_2 f_B`
      f_{A \otimes_1 B}`
      n_{A,B}]
    \efig
    \and
    \bfig
    \Vtriangle/->`<-`<-/[
      FI_1`
      GI_1`
      I_2;
      f_{I_1}`
      m_{I_1}`
      n_{I_1}]
    \efig
  \end{mathpar}  
\end{definition}
A \textit{product natural transformation} does not simplify things much
 \begin{mathpar}
    \bfig
    \square<1000,500>[
      FA \times FB`
      F(A \times B)`
      GA \times GB`
      G(A \times B);
      m_{A,B}`
      f_A \times f_B`
      f_{A \times B}`
      n_{A,B}]
    \efig
    \and
    \bfig
    \Vtriangle/->`<-`<-/[
      F1`
      G1`
      1;
      f_{1}`
      m_{1}`
      n_{1}]
    \efig
  \end{mathpar}  
  
\begin{definition}
  \label{def:SMCADJ}
  Suppose $(\cat{M}_1,I_1,\otimes_1)$ and $(\cat{M}_2,I_2,\otimes_2)$
  are SMCs, and $(F,m)$ is a symmetric monoidal functor between
  $\cat{M}_1$ and $\cat{M}_2$ and $(G,n)$ is a symmetric monoidal
  functor between $\cat{M}_2$ and $\cat{M}_1$.  Then a
  \textbf{symmetric monoidal adjunction} is an ordinary adjunction
  $\cat{M}_1 : F \dashv G : \cat{M}_2$ such that the unit,
  $\varepsilon : A \to GFA$, and the counit, $\eta_A : FGA \to A$, are
  symmetric monoidal natural transformations.  Thus, the following
  diagrams must commute:
  \begin{mathpar}
    \bfig
    \qtriangle|amm|<1000,500>[
      FGA \otimes_1 FGB`
      FG(A \otimes_1 B)`
      A \otimes_1 B;
      \q{A,B}`
      \eta_A \otimes_1 \eta_B`
      \eta_{A \otimes_1 B}]
    \efig
    \and
    \bfig
    \Vtriangle|amm|/->`<-`=/[
      FGI_1`
      I_1`
      I_1;
      \eta_{I_1}`
      \q{I_1}`]
    \efig
    \and
    \bfig
    \dtriangle|mmb|<1000,500>[
      A \otimes_2 B`
      GFA \otimes_2 GFB`
      GF(A \otimes_2 B);
      \varepsilon_A \otimes_2 \varepsilon_B`
      \varepsilon_{A \otimes_2 B}`
      \p{A,B}]
    \efig
    \and
    \bfig
    \Vtriangle|amm|/->`=`<-/[
      I_2`
      GFI_2`
      I_2;
      \varepsilon_{I_2}``
      p_{I_2}]
    \efig
  \end{mathpar}
  Note that $\p{}$ and $\q{}$ exist because $(FG,\q{})$ and
  $(GF,\p{})$ are symmetric monoidal functors.
\end{definition}

Also a \textit{product preserving adjunction} is not much simpler. The diagrams become:
\begin{mathpar}
    \bfig
    \qtriangle|amm|<1000,500>[
      FGA \times FGB`
      FG(A \times B)`
      A \times B;
      \q{A,B}`
      \eta_A \times \eta_B`
      \eta_{A \times B}]
    \efig
    \and
    \bfig
    \Vtriangle|amm|/->`<-`=/[
      FG1`
      1`
      1;
      \eta_{1}`
      \q{1}`]
    \efig
    \end{mathpar}
 \begin{mathpar}   
    \bfig
    \dtriangle|mmb|<1000,500>[
      A \times B`
      GFA \times GFB`
      GF(A \times B);
      \varepsilon_A \times \varepsilon_B`
      \varepsilon_{A \times B}`
      \p{A,B}]
    \efig
    \and
    \bfig
    \Vtriangle|amm|/->`=`<-/[
      1`
      GF1`
      1;
      \varepsilon_{1}``
      p_{1}]
    \efig
  \end{mathpar}
\begin{definition}
  \label{def:symm-monoidal-monad}
  A \textbf{symmetric monoidal monad} on a symmetric monoidal
  category $\cat{C}$ is a triple $(T,\eta, \mu)$, where
  $(T,\n{})$ is a symmetric monoidal endofunctor on $\cat{C}$,
  $\eta_A : A \mto TA$ and $\mu_A : T^2A \to TA$ are
  symmetric monoidal natural transformations, which make the following
  diagrams commute:
  \begin{mathpar}
    \bfig
    \square|ammb|<600,600>[
      T^3 A`
      T^2A`
      T^2A`
      TA;
      \mu_{TA}`
      T\mu_A`
      \mu_A`
      \mu_A]
    \efig
    \and
    \bfig
    \Atrianglepair/=`<-`=`->`<-/<600,600>[
      TA`
      TA`
      T^2 A`
      TA;`
      \mu_A``
      \eta_{TA}`
      T\eta_A]
    \efig
  \end{mathpar}
  The assumption that $\eta$ and $\mu$ are symmetric
  monoidal natural transformations amount to the following diagrams
  commuting:
  \begin{mathpar}
    \bfig
    \dtriangle|mmb|<1000,600>[
      A \otimes B`
      TA \otimes TB`
      T(A \otimes B);
      \eta_A \otimes \eta_B`
      \eta_A`
      \n{A,B}]    
    \efig
    \and
    \bfig
    \Vtriangle/->`=`<-/<600,600>[
      I`
      TI`
      I;
      \eta_I``
      \n{I}]
    \efig
  \end{mathpar}
  \begin{mathpar}
    \bfig
    \square|ammm|/->`->``/<1050,600>[
      T^2 A \otimes T^2 B`
      T(TA \otimes TB)`
      TA \otimes TB`;
      \n{TA,TB}`
      \mu_A \otimes \mu_B``]

    \square(850,0)|ammm|/->``->`/<1050,600>[
      T(TA \otimes TB)`
      T^2(A \otimes B)``
      T(A \otimes B);
      T\n{A,B}``
      \mu_{A \otimes B}`]
    \morphism(-200,0)<2100,0>[TA \otimes TB`T(A \otimes B);\n{A,B}]
    \efig
    \and
    \bfig
    \square|ammb|/->`->`->`<-/<600,600>[
      I`
      TI`
      TI`
      T^2I;
      \n{I}`
      \n{T}`
      T\n{I}`
      \mu_I]
    \efig
  \end{mathpar}
\end{definition}

Note that the first two diagrams for a \textit{product preserving monad} on a ccc, are exactly the same. The next four diagrams are slightly simplified
\begin{mathpar}
    \bfig
    \dtriangle|mmb|<1000,600>[
      A \times B`
      TA \times TB`
      T(A \times B);
      \eta_A \times \eta_B`
      \eta_A`
      \n{A,B}]    
    \efig
    \and
    \bfig
    \Vtriangle/->`=`<-/<600,600>[
      1`
      T1`
      1;
      \eta_1``
      \n{1}]
    \efig
  \end{mathpar}
  \begin{mathpar}
    \bfig
    \square|ammm|/->`->``/<1050,600>[
      T^2 A \times T^2 B`
      T(TA \times TB)`
      TA \times TB`;
      \n{TA,TB}`
      \mu_A \times \mu_B``]

    \square(850,0)|ammm|/->``->`/<1050,600>[
      T(TA \times TB)`
      T^2(A \times B)``
      T(A \times B);
      T\n{A,B}``
      \mu_{A \times B}`]
    \morphism(-200,0)<2100,0>[TA \times TB`T(A \times B);\n{A,B}]
    \efig
    \and
    \bfig
    \square|ammb|/->`->`->`<-/<600,600>[
      1`
      T1`
      T1`
      T^21;
      \n{1}`
      \n{T}`
      T\n{1}`
      \mu_1]
    \efig
  \end{mathpar}
Finally the dual concept, of a symmetric monoidal comonad, just for completeness.
\begin{definition}
  \label{def:symm-monoidal-comonad}
  A \textbf{symmetric monoidal comonad} on a symmetric monoidal
  category $\cat{C}$ is a triple $(T,\varepsilon, \delta)$, where
  $(T,\m{})$ is a symmetric monoidal endofunctor on $\cat{C}$,
  $\varepsilon_A : TA \mto A$ and $\delta_A : TA \to T^2 A$ are
  symmetric monoidal natural transformations, which make the following
  diagrams commute:
  \begin{mathpar}
    \bfig
    \square|amma|<600,600>[
      TA`
      T^2A`
      T^2A`
      T^3A;
      \delta_A`
      \delta_A`
      T\delta_A`
      \delta_{TA}]
    \efig
    \and
    \bfig
    \Atrianglepair/=`->`=`<-`->/<600,600>[
      TA`
      TA`
      T^2 A`
      TA;`
      \delta_A``
      \varepsilon_{TA}`
      T\varepsilon_A]
    \efig
  \end{mathpar}
  The assumption that $\varepsilon$ and $\delta$ are symmetric
  monoidal natural transformations amount to the following diagrams
  commuting:
  \begin{mathpar}
    \bfig
    \qtriangle|amm|/->`->`->/<1000,600>[
      TA \otimes TB`
      T(A \otimes B)`
      A \otimes B;
      \m{A,B}`
      \varepsilon_A \otimes \varepsilon_B`
    \varepsilon_{A \otimes B}]
    \efig
    \and
    \bfig
    \Vtriangle|amm|/->`<-`=/<600,600>[
      TI`
      I`
      I;
      \m{I}`
      \varepsilon_I`]
    \efig    
  \end{mathpar}
  \begin{mathpar}
    \bfig
    \square|amab|/`->``->/<1050,600>[
      TA \otimes TB``
      T^2A \otimes T^2B`
      T(TA \otimes TB);`
      \delta_A \otimes \delta_B``
      \m{TA,TB}]
    \square(1050,0)|mmmb|/``->`->/<1050,600>[`
      T(A \otimes B)`
      T(TA \otimes TB)`
      T^2(A \otimes B);``
      \delta_{A \otimes B}`
      T\m{A,B}]
    \morphism(0,600)<2100,0>[TA \otimes TB`T(A \otimes B);\m{A,B}]
    \efig
    \and
    \bfig
    \square<600,600>[
      I`
      TI`
      TI`
      T^2I;
      \m{I}`
      \m{I}`
      \delta_I`
      T\m{I}]
    \efig
  \end{mathpar}
\end{definition}
% section symmetric_monoidal_categories (end)
