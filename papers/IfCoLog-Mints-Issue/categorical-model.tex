There is not much essentially new in what we discuss here about the
tense logic based on $\sf{CS4}$. Similar ideas were discussed by
Ghilardi and Meloni \cite{ghilardi1988}, Makkai and Reyes
\cite{makkai1995} and more recently in by Dzik et al
\cite{dziketal2012,dziketal2014} and Menni and Smith
\cite{Menni:2014}.

The upshot of our discussion is that the categorical model we advance
is a cartesian closed category endowed with two adjunctions,
corresponding to the (limited) universal and existential
quantifications relative to the past and to the future that correspond
to the two sets of necessity and possibility operators.
%, relative to the past and the future.

This setting is though different enough from the precursors we know
about, to justify this note. First, as discussed elsewhere
\cite{bierman2000}, we see no reason for the monads/comonads emerging
from this setting to be \textbf{idempotent} operators, as they are in \cite{ghilardi1988} or \cite{makkai1995} (the idempotency
simplification does not seen warranted by the proof theory). Secondly we see no reason to take our models as part of toposes, as we are not interested in the extra structure provided by toposes.  However, we also see no reason to confine ourselves to algebraic models such as Heyting algebras with operators, as degenerate posetal categories, as both \cite{dziketal2012} and \cite{Menni:2014} do. Different proofs of
the same theorem are important to us as they correspond to different
morphisms in the category between the same objects. Thus we are
interested in \textbf{proof relevant} semantics, not simply
provability.

We build our main definition in stages. To begin with, a categorical
model of propositional intuitionistic logic is a \textbf{cartesian
  closed} category $\cat{C}$ with coproducts.
%% \iffalse
%% \begin{definition}[modal S4 category]
%%   \label{def:CS4-model}
%%   Suppose $\cat{C}$ is a cartesian closed category equipped with a
%%   monoidal comonad $(\Box, \varepsilon, \delta)$ and a
%%   \emph{$\Box$-strong monad} with a strength natural transformation
%%   \[
%%   \st{A}{B} : \Box A \pd \Diamond B \mto \Diamond(\Box A \pd B)
%%   \]
%%   This is a model of CS4 as defined and showed sound and complete in
%%   \cite{bierman2000}. We call this structure a \textit{$\sf{S4}$ modal
%%     category}.
%%   % as CS4 is the only kind of modality we consider in this note.
%% \end{definition}
%% \fi 
Then we recall from \cite{bierman2000} that to model a pair of
modalities using dual contexts we need a monoidal adjunction.

\begin{definition}[adjoint model]
  \label{def:CS4-single-adjoint-cat-model}
  An adjoint categorical model of dual context modal logic $\sf{DCS4}$
  consists of the following data:
  \begin{enumerate}
  \item A cartesian closed category with coproducts
    $(\cat{C},1,0,\pd,+,\ihom)$;
  \item 
    A monoidal adjunction $F \dashv G$, where $(F,m)$ and $(G,n)\colon
    \cat{C} \mto \cat{C}$ are monoidal functors such that their
    composition $GF$ is a monoidal comonad, written as $\Box$;
 \item The  monad $(\Diamond, \eta, \mu, \st{A}{B})$, induced by the adjunction $F \dashv G$,   is $\Box$-strong.
  \end{enumerate}
\end{definition}
Recall that a {\textit{monoidal}} comonad $\Box$ implies that there is a natural transformation $m\colon \Box A\times \Box B\to \Box (A\times B)$ (and $m_{\true}\colon \true \to \Box \true$) satisfying the coherence conditions described in page 23 of \cite{bierman2000}. We need to recall as well that by a monad being $\Box$-strong, we mean that there is a \textit{strenght} natural transformation $st_{A,B}\colon \Box A\times \Diamond B\to \Diamond(\Box A\times B)$ satisfying the four equations in page 27 of \cite{bierman2000}.This natural transformation is required to model the Fisher-Servi axioms, which are a weakening of the duality between $\Box$ and $\Diamond$ that the classical systems satisfy.

Finally we consider two pairs of modalities (or two adjunctions),
intertwined, as in tense logic.

\begin{definition}[tense calculus model]
A categorical model of tense calculus dual context modal logic
$\sf{TCDS4}$ is a cartesian closed category $\cal{C}$ as above,
together with two intertwined adjunctions $(\bLozenge\dashv \Box,
\Diamond\dashv \blacksquare)$.  The adjunctions $(\bLozenge\dashv
\Box)$ and $(\Diamond\dashv \blacksquare)$ on $\cal{C}$ are connected
by Fisher-Servi axioms, namely $\Diamond (A\to B)\to (\Box A\to
\Diamond B)$ and $(\Diamond A\to \Box B)\to \Box (A\to B)$.
\end{definition}

This model is more general than the system, $\sf{TCS4}$, given above
in that it contains two possibility modalities, which we do not deal
with, in the type theory. These possibility operators could be treated
as syntactic sugar for $\neg \Box \neg A$ (and respectively $\neg
\blacksquare \neg A$), as they usually are in Intuitionistic Linear
Logic, for instance. We refrain from doing so explicitly and prefer to
consider the necessity-only fragment in the type theory, as this
allows us to bypass the discussion of which possibility modality is
more appropriate for each setting.  More importantly it allows us to
dodge the question of how to provide a Curry-Howard categorical
interpretation for what we called Simpson-style modal S4.  Thus the
model should be seen as an over approximation.  We give the more
general model here to set the stage for future work.

Categorical soundness is proved, as usual, checking the natural
deduction rules preserve validity of the constructions used, i.e
function spaces, products, coproducts and the two adjunctions.

Define an interpretation $\sem{\_} : \sf{TCS4} \to {\cal C}$ which
takes the types and sequents of $\sf{TCS4}$ (over a basic set of
types) to a model $\cal C$ as follows:
\begin{align*}
\sem{p} &= I(p) \mbox{  for $p$ a base type}\\
\sem{\top} &= \top\\
\sem{  A \wedge B  } &=  \sem{ A}  \times  \sem{ B }\\
\sem{  A \lor B  } &=  \sem{ A}  +  \sem{ B }\\
\sem{  A \to B  } &=  \sem{ A}  \to  \sem{ B }\\
\sem{\Box A} &= F G(\sem{A})\\
\sem{\blacksquare A} &= F' G'(\sem{A})
\end{align*}
 We extend this interpretation to lists of types by saying that for a
 list $A_1, ..., A_n$ of types, the interpretation is the product of
 the interpretations $\sem{A_1,...A_n} = \sem{A_1} \times \ldots
 \times \sem{A_n}$.  The interpretation will take a sequent $\Gamma
 \vdash t : A$ to an arrow $ \sem {\Gamma \vdash t : A} : \sem{
   \Gamma} \to \sem{A} $ in the tense modal category.

\begin{theorem}
\label{thm:tcs4-csound}
The type theory $\sf{TCS4}$ has \textit{sound} models provided by the
structures $\cal C$ defined above.  In other words, given a tense
adjoint modal category $\cal C$, using the above interpretation, the
following hold:
\begin{itemize}
\item Assume $\Gamma \vdash t : A$ in $\sf{TCS4}$. Then $\sem{\Gamma
  \vdash t \colon A}$ is a morphism with domain $\sem{\Gamma}$ and
  codomain $\sem{A}$;
\item Assume $\Gamma \vdash t = s \colon A$. Then $\sem{\Gamma
  \vdash t \colon A} = \sem{\Gamma \vdash s \colon A}$.
\end{itemize}
\end{theorem}
\begin{proof}
  The first part holds by induction on $[[G |- t : A]]$, and the
  second by induction on $[[G |- t = s : A]]$, but uses the first
  part.  Please see Appendix~\ref{subsec:proof_of_soundness_of_tcs4}
  for the complete proof.  
\end{proof}

We  have completeness of the tense modal categories when the model is restricted to box modalities only.

\begin{theorem}
\label{thm:tcs4-completeness}
The adjoint modal models are \textit{complete} in the appropriate
sense for the type theory $\sf{TCS4}$. This is to say, if we have
equality of the interpretations $\sem{\Gamma \vdash t \colon A} =
\sem{\Gamma \vdash s \colon A}$ (where \mbox{$\sem{\ } $} is the
interpretation defined above) in the tense modal category $\cal C$ for
any derived sequents $\Gamma \vdash t \colon A$ and $\Gamma \vdash s
\colon A$ then we can derive the equation in the type theory
$\sf{TCS4}$ $\;$ $\Gamma \vdash t = s \colon A$.
\end{theorem}
\begin{proof}
  This result can be shown by constructing a cartesian closed category
  with coproducts and two comonads, one for $\Box$ and one for
  $\BBox$, internal to $\sf{TCS4}$ where the objects are types and the
  morphisms are $\alpha$-equivalence classes of terms in context $[[G
      |- t : A]]$.  This category is called the syntactic
  category. Please see
  Appendix~\ref{subsec:proof_of_completeness_for_tcs4} for the
  remainder of the proof.
\end{proof}
Categorical completeness requires providing an equivalence relation in
the Lindenbaum algebra of the formulae, as usual in algebraic
semantics. The basic calculations, for traditional algebraic semantics
in Heyting algebras were provided, for instance, by Figallo et al in
\cite{figallo2014} or Dzik et al in \cite{dzik2010}. \textit{Mutatis mutantis} these calculations will
apply for our version of the system (no distribution of diamonds over
disjunctions, no definibility of diamonds in terms of negated boxes).
