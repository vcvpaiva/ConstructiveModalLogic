old introduction
Generally speaking, Temporal Logic is any system of rules and
symbolism for representing, and reasoning about propositions qualified
in terms of time.  Temporal logic is also one of the most traditional
kinds of modal logic, introduced by Arthur Prior in the late 1950s,
%, and for which important results were obtained by Hans Kamp. 
but it is also one of the most controversial kinds of modal logic, as
people have different intuitions about time, how to represent it, and
how to reason about it.

There has been a huge amount of work in Modal Logic in the last sixty
years, but mainly in classical modal logic. We are mostly interested
in constructive systems. In particular we are interested in a
constructive version of temporal logic that satisfies some well-known
and desirable proof-theoretical properties, but that is also
algebraically and category-theoretically well-behaved.

Prior's `Time and Modality' introduced a propositional modal logic
with two temporal connectives (modal operators), $F$ and $P$,
corresponding to ``sometime in the {F}uture" and ``sometime in the
{P}ast". Kamp's thesis introduced the binary temporal operators
``since" and ``until" and proved what came to be known as Kamp's
theorem. Kamp's theorem shows that all temporal operators are
definable in terms of ``since" and ``until" -- provided that the
underlying temporal structure is a continuous linear ordering and
provided that the logical basis is classical.

Ewald \cite{ewald1986} has produced a first version of an
intuitionistically based temporal logic system. Basically, Ewald's
system is a pair of Simpson-style IK operators \cite{simpson1994},
representing past and future over intuituionistic propositional logic.

The intuitive reading of the operators is sensible:
\begin{itemize}
\item $P$ “It has at some time been the case that” 
\item $F$ “It will at some time be the case that” 
\item $H$ “It has always been the case that” 
\item $G$  “It will always be the case that” 
\end{itemize}
Ewald and most of the researchers that followed his path of
constructivization of tense logic, did so assuming a symmetry between
past and future. This symmetry, as well as the symmetry between
universal and existential quantifiers, both in the past and in the
future, are at odds with
%is not very characteristic of 
intuitionistic reasoning.

 Simpson remarks that intuitionistic or constructive modal logic is
 full of interesting questions. As he says:
\begin{quote}
Although much work has been done in the field, there is as yet no
consensus on the correct viewpoint for considering intuitionistic
modal logic.  In particular, there is no single semantic framework
rivalling that of possible world semantics for classical modal logic.
Indeed, there is not even any general agreement on what the
intuitionistic analogue of the basic modal logic, K, is.
\end{quote}
In an intuitionistic logic we do not expect perfect duality between
quantifiers, ($\forall x.P(x)$ is not the same as $\neg \exists x.\neg
P(x)$) or even between conjunction and disjunction (De Morgan laws do
not hold for intuitionistic propositional logic). So one should not
expect a perfect duality between possibility and necessity either. 

It has been thought for a while that temporal logic should consist of two modal systems intertwinned. Some times the modal systems are K modal systems, some times S5. Here we will consider S4 systems, so we start by recalling some facts about S4 constructive modal logic.

\section{Twice as nice?}
Previous work on modalities in Linear Logic made it clear that modalities, unlike canonical logical connectives such as conjunctions and disjunctions, are not unique up to iso in a given model. Thus we can have two sets of modalities, say $(\Box_1, \Diamond_1)$ and $(\Box_2, \Diamond_2)$ co-existing in a cartesian closed category $\cat{C}$ without problems. But this is not what we want, as we want holding for all times and for a (single) time, both in the future and in the past, so our two sets of modalities need to be related in  a sensible way.

Work of Dzik et al \cite{dziketal2012,dziketal2014} and Menni and Smith \cite{Menni:2014}  indicates that the two adjunctions have to be twisted, that is, the  existential modality in the past corresponds to the universal modality in the future and vice-versa. Past and future are not completely symmetric, but the adjunctions keep track of the symmetry there is there for quantifiers (in first-order logic) and for modalities (existential and universal ones).

%\subsection{Are past and future symmetric?}

%(half of) the rules below in Ewald's system.

\subsection{Algebras with Adjunctions}
%\todo{Need to  see if we can show a reasonable duality theorem similar to Menni and Smith's corollary 3.5}
Menni and Smith \cite{Menni:2014} (in page 4) define a \textit{Boolean algebra with an adjunction} BAA as a triple $(B, \Diamond, H)$ where $B$ is a Boolean algebra and $\Diamond, H$ are monotone functions such that $\Diamond$ is left adjoint to $H$, $\Diamond \dashv H$. A morphism is a Boolean algebra morphism $f\colon B\mto B'$  preserving both $\Diamond, H$. BAAs and their moprhisms form a category $\sf BAA$.
Clearly nothing in the definition makes reference to the classical (or otherwise) character of the underlying logic and hence we can define \textit{Heyting algebras with adjunctions} HAA exactly in the same manner, as   triples $(H, \Diamond, H)$ where we have the adjunction $\Diamond \dashv H$.  \textit{Mutatits mutandis} we have the category $\sf HAA$. 



Menni and Smith also consider the category $\sf BAO$ of Boolean algebras with operators. Objects of $\sf BAO$ are pairs of $(B,\Diamond)$ with $B$ a Boolean algebra and $\Diamond \colon B\mto B$ the operator, which is a function preserving finite suprema. Dualizing this we can consider the category $\sf HAO$ of Heyting algebras with operators, where $\Box\colon H\mto H$ is a function preserving finite infima.

Menni and Smith then go on to consider `double' BAAs of the form $(B, \Diamond \dashv H, P\dashv \Box)$, where $B$ is a Boolean algebra, $\Diamond$, $\Box$, $H$ and $P$ are  monotone functions  such that $\Diamond \dashv H$ and $ P\dashv\Box$ are adjunctions and a strict duality is asserted between $\Box$ and $\Diamond$ (namely $\Box=\neg \Diamond\neg$) and between $H$ and $P$ (namely $P=\neg H\neg$). Constructivizing these can be done easily by considering the Heyting algebra $H$ with similar adjunctions $(H, \Diamond \dashv G, P\dashv \Box)$ and not taking the very strict duality at face value.

Menni and Smith then state that \begin{quote} Altogether the category $\sf BAA$ provides an algebraic semantics for normal temporal logic, just as $\sf BAO$ provides an algebraic semantics for normal modal logic (Theorem 5.27 loc.cit.).\end{quote} We would like to assert and prove the theorem that ``the category $\sf HAA$ provides an algebraic semantics for a version of constructive temporal logic, just as $\sf HAO$ provides an algebraic semantics for a constructive modal logic".

%They also define \textit{lattice with  adjunction} as a triple $(A, L, R)$ where $A$ is a lattice, and $L,R$ are monotone functions on $A$ that form a preorder $L\dashv R$ adjunction.


\subsection{Ewald's Temporal Logic}

Ewald proposed an intuitionistic temporal logic where the usual modal
operators, $\Box$ and $\Diamond$, have been decomposed into more
traditional temporal operators $\F$, $\H$, $\G$, and
$\P$~\cite{ewald1986}.

The sequent calculus formalization of Ewald's temporal logic is
defined by the following inference rules in addition to the initial
intuitionistic set of inference rules given above:
\begin{center}
  \small
  \begin{math}
    \begin{array}{cccccc}
      \infer[\text{G}]{\G \Gamma \vdash \G C}{
        \Gamma \vdash C
      }
      & \quad &
      \infer[\text{H}]{\H\Gamma \vdash \H C}{
        \Gamma \vdash C
      }\\\\
      \infer[\text{F}]{\G\Gamma,\F A \vdash \F B}{
        \Gamma,A \vdash B
      }
      & \quad &
      \infer[\text{P}]{\H\Gamma,\P A \vdash \P B}{
        \Gamma,A \vdash B
      }\\\\
      \infer[\text{FH}]{\cdot \vdash \F\H A \to A}{
        \,
      }
      & \quad &
      \infer[\text{HF}]{\cdot \vdash A \to \H\F A}{
        \,
      }\\\\
      \infer[\text{PG}]{\cdot \vdash \P\G A \to A}{
        \,
      }
      & \quad &
      \infer[\text{GP}]{\cdot \vdash A \to \G\P A}{
        \,
      }\\\\
      \infer[\text{FG}_1]{\cdot \vdash (\F A \to \G B) \to \G(A \to B)}{
        \,
      }
      & \quad &
      \infer[\text{FG}_2]{\cdot \vdash \F(A \to B) \to (\G A \to \F B)}{
        \,
      }\\\\
      \infer[\text{PH}_1]{\cdot \vdash (\P A \to \H B) \to \H(A \to B)}{
        \,
      }
      & \quad &
      \infer[\text{PH}_2]{\cdot \vdash \P(A \to B) \to (\H A \to \P B)}{
        \,
      }      
    \end{array}
  \end{math}
\end{center}
Looking at Ewald's system with a present day categorical perspective
it is easy to see that these new operators can be seen as adjoint
functors~\cite{Menni:2014}.

The rules $\text{G}$ and $\text{H}$ show that $\G$ and $\H$ are
functors, similarly, the rules $\text{F}$ and $\text{P}$ do the same
for the operators $\F$ and $\P$ when $\Gamma = \cdot$.  Then the rules
$\text{FH}$ and $\text{HF}$ give the co-unit and unit of the
adjunction $\F \dashv \H$, and the rules $\text{PG}$ and $\text{GP}$
give the co-unit and unit of the adjunction $\P \dashv \G$.

The following describes the intuitive meaning of each of Ewald's
operators:
\begin{itemize}
\item[] $\F$: ``it will once be the case in the Future``  
\item[] $\H$: ``it Has always been the case in the past``
\item[] $\P$: ``it was once the case in the Past``
\item[] $\G$: ``it's Going to  always be the case in the future``  
\end{itemize}


Using these operators it is possible to define the  modal
operators by choosing one of the following definitions
\cite{ewald1986}:
\begin{enumerate}[i.]
\item $\Box A := A \land \G A \land \H A$, $\Diamond A := A \lor \F A \lor \P A$; or
\item $\Box A := A \land \G A$, $\Diamond A := A \lor \F A$.
\end{enumerate}

This system has a Kripke model, and an axiomatization (page 171 of
\cite{ewald1986}), and is decidable.  Lastly, Ewald shows that by
adopting various axioms for the modal operators different notions of
time can be studied within this system.
