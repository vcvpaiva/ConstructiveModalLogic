Given the Natural Deduction system presented, we simply follow the
recipes described in \cite{barber1997} to produce a term assignment
for these four kinds of modal operators.

We start by giving a calculus for annotating natural deduction proofs
in our CS4 tense calculus with typed $\lambda$-calculus terms. The
annotations must be such that types correspond to modal propositions,
terms correspond to proofs and proof normalisation corresponds to term
reduction. The basic typying judgements of our CS4 tense calculus are
of the form $\Gamma ; \Delta\vdash t\colon A$ where $\Gamma$ declares
modal variables, which may occurr arbitrarily in a term, while
$\Delta$ declares intuitionistic variables that may not occurr in a
subterm of the form $\Box t$ or $\blacksquare t$.
