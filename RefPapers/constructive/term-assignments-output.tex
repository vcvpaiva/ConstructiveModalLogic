In this section we provide a term assignment to constructive tense
logic with only $\Box$ and $\BBox$.  We leave term assignments to the
other varieties of tense logic with $\Diamond$ and $\BDia$ for future
work.

The typing rules can be found in Figure~\ref{fig:TCS4-typing-rules} with the typed equality rules in Figure~\ref{fig:TCS4-eq}.  
\begin{figure}
  \begin{mdframed}    
    \begin{mathpar}
      \TLLdruletyXXax{} \and
      \TLLdruletyXXfalse{} \and
      \TLLdruletyXXimpI{} \and
      \TLLdruletyXXimpE{} \and
      \TLLdruletyXXboxE{} \and
      \TLLdruletyXXboxI{} \and
      \TLLdruletyXXbboxE{} \and
      \TLLdruletyXXbboxI{} 
    \end{mathpar}
  \end{mdframed}
  \caption{TCS4 Typing Rules}
  \label{fig:TCS4-typing-rules}
\end{figure}
\begin{figure}
  \begin{mdframed}
    \small
    \begin{mathpar}      
      \TLLdruleeqXXbeta{}   \and
      \TLLdruleeqXXunbox{}  \and
      \TLLdruleeqXXunbbox{} \and
      \TLLdruleeqXXrefl{}   \and
      \TLLdruleeqXXsym{}    \and
      \TLLdruleeqXXtrans{}
    \end{mathpar}
  \end{mdframed}
  \caption{TCS4 Equality Rules}
  \label{fig:TCS4-eq}
\end{figure}
Here we can see that types are tense S4 formulas.  The sequents have
the form $\Gamma  \vdash  \TLLnt{t}  \TLLsym{:}  \TLLnt{A}$ and $ \Gamma  \vdash  \TLLnt{s}  \approx  \TLLnt{t}  :  \TLLnt{A} $ where $\Gamma$ is a
multiset of free variables and their types denoted $\mathit{x}  \TLLsym{:}  \TLLnt{A}$, and
$\TLLnt{s}$ and $\TLLnt{t}$ are terms with the following syntax:
\[
\begin{array}{lll}
  \TLLnt{t} & := & \mathit{x} \mid  \lambda  \mathit{x}  :  \TLLnt{A} . \TLLnt{t}  \mid \TLLnt{s} \, \TLLnt{t} \mid  \mathsf{let}_\Box\, \mathit{x_{{\mathrm{1}}}}  \TLLsym{:}  \Box \, \TLLnt{A_{{\mathrm{1}}}}  \TLLsym{,} \, ... \, \TLLsym{,}  \mathit{x_{\TLLmv{k}}}  \TLLsym{:}  \Box \, \TLLnt{A_{\TLLmv{k}}} \,\mathsf{be}\, \TLLnt{t_{{\mathrm{1}}}}  \TLLsym{,} \, ... \, \TLLsym{,}  \TLLnt{t_{\TLLmv{k}}} \,\mathsf{in}\, \TLLnt{t}  \mid \\
  & &  \mathsf{let}_\blacksquare\, \mathit{x_{{\mathrm{1}}}}  \TLLsym{:}  \blacksquare \, \TLLnt{A_{{\mathrm{1}}}}  \TLLsym{,} \, ... \, \TLLsym{,}  \mathit{x_{\TLLmv{k}}}  \TLLsym{:}  \blacksquare \, \TLLnt{A_{\TLLmv{k}}} \,\mathsf{be}\, \TLLnt{t_{{\mathrm{1}}}}  \TLLsym{,} \, ... \, \TLLsym{,}  \TLLnt{t_{\TLLmv{k}}} \,\mathsf{in}\, \TLLnt{t}  \mid \mathsf{unbox}_\Box\, \, \TLLnt{t} \mid \mathsf{unbox}_\blacksquare\, \, \TLLnt{t}
\end{array}
\]
Equality is straightforward where it is apparent that the
let-expressions model explicit substitutions. These substitutions are
triggered when they are applied to an unbox-expression.

We have the following basic properties of this term assignment.
\begin{lemma}[Substitution for Typing]
  \label{lemma:substitution_for_typing}
  If $\Gamma  \vdash  \TLLnt{t_{{\mathrm{1}}}}  \TLLsym{:}  \TLLnt{A}$, and $\Gamma  \TLLsym{,}  \mathit{x}  \TLLsym{:}  \TLLnt{A}  \vdash  \TLLnt{t_{{\mathrm{2}}}}  \TLLsym{:}  \TLLnt{B}$, then $\Gamma  \vdash  \TLLsym{[}  \TLLnt{t_{{\mathrm{1}}}}  \TLLsym{/}  \mathit{x}  \TLLsym{]}  \TLLnt{t_{{\mathrm{2}}}}  \TLLsym{:}  \TLLnt{B}$.
\end{lemma}
\begin{proof}
  This proof hold by straightforward induction on the
  form of the assumed typing derivation.  
\end{proof}

\begin{lemma}[Weakening]
  \label{lemma:weakening}
  If $\Gamma  \vdash  \TLLnt{t}  \TLLsym{:}  \TLLnt{B}$, then $\Gamma  \TLLsym{,}  \mathit{x}  \TLLsym{:}  \TLLnt{A}  \vdash  \TLLnt{t}  \TLLsym{:}  \TLLnt{B}$.
\end{lemma}
\begin{proof}
  This proof hold by straightforward induction on the
  form of the assumed typing derivation. 
\end{proof}
