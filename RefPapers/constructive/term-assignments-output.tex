Given the Natural Deduction system presented in the previous section,
we simply follow the recipes described in \cite{barber1997} to produce
a term assignment for these four kinds of modal operators.

We start by giving a calculus for annotating natural deduction proofs
in our CS4 tense calculus with typed $\lambda$-calculus terms. The
annotations must be such that types correspond to modal propositions,
terms correspond to proofs and proof normalisation corresponds to term
reduction. The basic typying judgements of our CS4 tense calculus are
of the form $\Gamma  \TLLsym{;}  \Delta  \vdash  \TLLnt{t}  \TLLsym{:}  \TLLnt{A}$ where $\Gamma$ declares modal
variables, which may occurr arbitrarily in a term, while $\Delta$
declares intuitionistic variables that may not occurr in a subterm of
the form $\Box \, \TLLnt{t}$ or $\blacksquare \, \TLLnt{t}$.

The typing rules for basic propositional logic can be found in
Figure~\ref{fig:term-assignment-basic}, and the typing rules for the
TCS4 modalities are in Figure~\ref{fig:term-assignment-TCS4}.
Finally, the reduction relation consists of the $\beta$-reduction
rules defined in Figure~\ref{fig:reduction-TCS4}, and the commuting
conversions defined in Figure~\ref{fig:commuting-conv-TCS4}; we omit
congruence rules in the interest of space.
\begin{figure}
  \begin{mdframed}
    \begin{mathpar}
      \TLLdruletyXXax{} \and
      \TLLdruletyXXbax{} \and
      \TLLdruletyXXtrue{} \and
      \TLLdruletyXXfalse{} \and
      \TLLdruletyXXconjI{} \and
      \TLLdruletyXXconjEOne{} \and
      \TLLdruletyXXconjETwo{} \and
      \TLLdruletyXXdisjIOne{} \and
      \TLLdruletyXXdisjITwo{} \and
      \TLLdruletyXXdisjE{} \and
      \TLLdruletyXXimpI{} \and
      \TLLdruletyXXimpE{}      
    \end{mathpar}
  \end{mdframed}
  \caption{Term assignment for basic propositional logic}
  \label{fig:term-assignment-basic}
\end{figure}
\begin{figure}
  \begin{mdframed}
    \begin{mathpar}
      \TLLdruletyXXboxI{} \and
      \TLLdruletyXXboxE{} \and
      \TLLdruletyXXbdiaI{} \and
      \TLLdruletyXXbdiaE{} \and
      \TLLdruletyXXbboxI{} \and
      \TLLdruletyXXbboxE{} \and
      \TLLdruletyXXdiaI{} \and
      \TLLdruletyXXdiaE{}
    \end{mathpar}
  \end{mdframed}
  \caption{Term assignment for TCS4}
  \label{fig:term-assignment-TCS4}
\end{figure}
\begin{figure}
  \begin{mdframed}
    \begin{mathpar}
      \TLLdrulerXXbeta{} \and
      \TLLdrulerXXfirst{} \and
      \TLLdrulerXXsecond{} \and
      \TLLdrulerXXcaseOne{} \and
      \TLLdrulerXXcaseTwo{} \and
      \TLLdrulerXXbox{} \and
      \TLLdrulerXXbdia{} \and
      \TLLdrulerXXbbox{} \and
      \TLLdrulerXXdia{}      
    \end{mathpar}
  \end{mdframed}
  \caption{The $\beta$-reduction rules for TCS4}
  \label{fig:reduction-TCS4}
\end{figure}
\begin{figure}
  \begin{mdframed}
    \begin{mathpar}
      \TLLdrulerXXboxBox{} \and
      \TLLdrulerXXbdiaBdia{} \and
      \TLLdrulerXXboxBBox{} \and
      \TLLdrulerXXdiadia{}
    \end{mathpar}
  \end{mdframed}
  \caption{The commuting conversions for TCS4}
  \label{fig:commuting-conv-TCS4}
\end{figure}
In the remainder of this section we prove various properties of this
system.  TCS4 has two different types of assumptions, and hence, of
free variables, those that are members of $\Gamma$ and those that are
members of $\Delta$, where the former are implicitly boxed.  This
implies that there are two types of substitution for typing and
weakening results.
\begin{lemma}[Substitution for Typing]
  \label{lemma:substitution_for_typing}
  \begin{enumerate}[i.]
  \item[] 
  \item If $\Gamma  \TLLsym{;}  \Delta  \vdash  \TLLnt{t_{{\mathrm{1}}}}  \TLLsym{:}  \TLLnt{A}$, and $\Gamma  \TLLsym{,}  \mathit{x}  \TLLsym{:}  \TLLnt{A}  \TLLsym{;}  \Delta  \vdash  \TLLnt{t_{{\mathrm{2}}}}  \TLLsym{:}  \TLLnt{B}$, then $\Gamma  \TLLsym{;}  \Delta  \vdash  \TLLsym{[}  \TLLnt{t_{{\mathrm{1}}}}  \TLLsym{/}  \mathit{x}  \TLLsym{]}  \TLLnt{t_{{\mathrm{2}}}}  \TLLsym{:}  \TLLnt{B}$.
  \item If $\Gamma  \TLLsym{;}  \Delta  \vdash  \TLLnt{t_{{\mathrm{1}}}}  \TLLsym{:}  \TLLnt{A}$, and $\Gamma  \TLLsym{;}  \Delta  \TLLsym{,}  \mathit{x}  \TLLsym{:}  \TLLnt{A}  \vdash  \TLLnt{t_{{\mathrm{2}}}}  \TLLsym{:}  \TLLnt{B}$, then $\Gamma  \TLLsym{;}  \Delta  \vdash  \TLLsym{[}  \TLLnt{t_{{\mathrm{1}}}}  \TLLsym{/}  \mathit{x}  \TLLsym{]}  \TLLnt{t_{{\mathrm{2}}}}  \TLLsym{:}  \TLLnt{B}$.
  \end{enumerate}
\end{lemma}
\begin{proof}
  Both parts of this proof hold by straightforward induction on the
  form of the second assumed typing derivation.  Notice that rules
  $\Box_{\mathcal{I}}$ and $\BBox_{\mathcal{I}}$ do not need to be
  considered in in the proof of part two, because we assume the
  propositional context, $\Delta$, is non-empty.
\end{proof}

\begin{lemma}[Weakening]
  \label{lemma:weakening}
  \begin{enumerate}[i.]
  \item[]
  \item If $\Gamma  \TLLsym{;}  \Delta  \vdash  \TLLnt{t}  \TLLsym{:}  \TLLnt{B}$, then $\Gamma  \TLLsym{,}  \mathit{x}  \TLLsym{:}  \TLLnt{A}  \TLLsym{;}  \Delta  \vdash  \TLLnt{t}  \TLLsym{:}  \TLLnt{B}$.
  \item If $\Gamma  \TLLsym{;}  \Delta  \vdash  \TLLnt{t}  \TLLsym{:}  \TLLnt{B}$, then $\Gamma  \TLLsym{;}  \Delta  \TLLsym{,}  \mathit{x}  \TLLsym{:}  \TLLnt{A}  \vdash  \TLLnt{t}  \TLLsym{:}  \TLLnt{B}$.
  \end{enumerate}
\end{lemma}
\begin{proof}
  Both parts of this proof hold by straightforward induction on the
  form of the assumed typing derivation.  In the case of the rules
  $\Box_{\mathcal{I}}$ and $\BBox_{\mathcal{I}}$ we simply reapply the
  rule using the assumed premise.
\end{proof}

Type safety is an important property for any type system.  We show
TCS4 type safe, but showing that both subject reduction and progress
hold.  Subject reduction requires the following inversion principles.
\begin{lemma}[Inversion]
  \label{lemma:inversion}
  \begin{enumerate}[i.]
  \item[]
  \item If $\Gamma  \TLLsym{;}  \Delta  \vdash  \TLLsym{(}  \TLLnt{t_{{\mathrm{1}}}}  \TLLsym{,}  \TLLnt{t_{{\mathrm{2}}}}  \TLLsym{)}  \TLLsym{:}   \TLLnt{A}  \land  \TLLnt{B} $, then $\Gamma  \TLLsym{;}  \Delta  \vdash  \TLLnt{t_{{\mathrm{1}}}}  \TLLsym{:}  \TLLnt{A}$, and $\Gamma  \TLLsym{;}  \Delta  \vdash  \TLLnt{t_{{\mathrm{2}}}}  \TLLsym{:}  \TLLnt{B}$.
  \item If $\Gamma  \TLLsym{;}  \Delta  \vdash  \mathsf{fst}\, \, \TLLnt{t}  \TLLsym{:}  \TLLnt{B}$, then there is a type $\TLLnt{A}$, such that, $\Gamma  \TLLsym{;}  \Delta  \vdash  \TLLnt{t}  \TLLsym{:}   \TLLnt{A}  \land  \TLLnt{B} $.
  \item If $\Gamma  \TLLsym{;}  \Delta  \vdash  \mathsf{snd}\, \, \TLLnt{t}  \TLLsym{:}  \TLLnt{B}$, then there is a type $\TLLnt{A}$, such that, $\Gamma  \TLLsym{;}  \Delta  \vdash  \TLLnt{t}  \TLLsym{:}   \TLLnt{A}  \land  \TLLnt{B} $.
  \item If $\Gamma  \TLLsym{;}  \Delta  \vdash  \mathsf{inj}_1\,  \TLLnt{t}  \TLLsym{:}   \TLLnt{A}  \lor  \TLLnt{B} $, then $\Gamma  \TLLsym{;}  \Delta  \vdash  \TLLnt{t}  \TLLsym{:}  \TLLnt{A}$.
  \item If $\Gamma  \TLLsym{;}  \Delta  \vdash  \mathsf{inj}_2\,  \TLLnt{t}  \TLLsym{:}   \TLLnt{A}  \lor  \TLLnt{B} $, then $\Gamma  \TLLsym{;}  \Delta  \vdash  \TLLnt{t}  \TLLsym{:}  \TLLnt{B}$.
  \item If $\Gamma  \TLLsym{;}  \Delta  \vdash   \mathsf{case}\, \TLLnt{t} \,\mathsf{of}\, \mathit{x} . \TLLnt{t_{{\mathrm{1}}}} , \mathit{x} . \TLLnt{t_{{\mathrm{2}}}}   \TLLsym{:}  \TLLnt{C}$, then there are types $\TLLnt{A}$ and $\TLLnt{B}$,
    such that, $\Gamma  \TLLsym{;}  \Delta  \TLLsym{,}  \mathit{x}  \TLLsym{:}  \TLLnt{A}  \vdash  \TLLnt{t_{{\mathrm{1}}}}  \TLLsym{:}  \TLLnt{C}$, $\Gamma  \TLLsym{;}  \Delta  \TLLsym{,}  \mathit{x}  \TLLsym{:}  \TLLnt{B}  \vdash  \TLLnt{t_{{\mathrm{2}}}}  \TLLsym{:}  \TLLnt{C}$, and $\Gamma  \TLLsym{;}  \Delta  \vdash  \TLLnt{t}  \TLLsym{:}   \TLLnt{A}  \lor  \TLLnt{B} $.
  \item If $\Gamma  \TLLsym{;}  \Delta  \vdash   \lambda  \mathit{x}  :  \TLLnt{A} . \TLLnt{t}   \TLLsym{:}  \TLLnt{B}$, then $\Gamma  \TLLsym{;}  \Delta  \TLLsym{,}  \mathit{x}  \TLLsym{:}  \TLLnt{A}  \vdash  \TLLnt{t}  \TLLsym{:}  \TLLnt{B}$.    
  \item If $\Gamma  \TLLsym{;}  \Delta  \vdash  \TLLnt{t_{{\mathrm{1}}}} \, \TLLnt{t_{{\mathrm{2}}}}  \TLLsym{:}  \TLLnt{B}$, then there exists a type $\TLLnt{A}$, such that, $\Gamma  \TLLsym{;}  \Delta  \vdash  \TLLnt{t_{{\mathrm{1}}}}  \TLLsym{:}  \TLLnt{A}  \to  \TLLnt{B}$ and $\Gamma  \TLLsym{;}  \Delta  \vdash  \TLLnt{t_{{\mathrm{2}}}}  \TLLsym{:}  \TLLnt{A}$.
  \item If $\Gamma  \TLLsym{;}  \Delta  \vdash  \Box \, \TLLnt{t}  \TLLsym{:}  \Box \, \TLLnt{A}$, then $\Gamma  \TLLsym{;}  \emptyset  \vdash  \TLLnt{t}  \TLLsym{:}  \TLLnt{A}$.
  \item If $\Gamma  \TLLsym{;}  \Delta  \vdash   \mathsf{let}\,\Box  \mathit{x}  =  \TLLnt{t_{{\mathrm{1}}}} \,\mathsf{in}\, \TLLnt{t_{{\mathrm{2}}}}   \TLLsym{:}  \TLLnt{B}$, then there is a type $\TLLnt{A}$, such that,
    $\Gamma  \TLLsym{;}  \Delta  \vdash  \TLLnt{t_{{\mathrm{1}}}}  \TLLsym{:}  \Box \, \TLLnt{A}$ and $\Gamma  \TLLsym{,}  \mathit{x}  \TLLsym{:}  \TLLnt{A}  \TLLsym{;}  \Delta  \vdash  \TLLnt{t_{{\mathrm{2}}}}  \TLLsym{:}  \TLLnt{B}$.
  \item If $\Gamma  \TLLsym{;}  \Delta  \vdash  \mathbin{\blacklozenge} \, \TLLnt{t}  \TLLsym{:}  \mathbin{\blacklozenge} \, \TLLnt{A}$, then $\Gamma  \TLLsym{;}  \Delta  \vdash  \TLLnt{t}  \TLLsym{:}  \TLLnt{A}$.
  \item If $\Gamma  \TLLsym{;}  \Delta  \vdash   \mathsf{let}\,\blacklozenge  \mathit{x}  =  \TLLnt{t_{{\mathrm{1}}}} \,\mathsf{in}\, \TLLnt{t_{{\mathrm{2}}}}   \TLLsym{:}  \TLLnt{B}$, then there is a type $\TLLnt{A}$, such that,
    $\Gamma  \TLLsym{;}  \Delta  \vdash  \TLLnt{t_{{\mathrm{1}}}}  \TLLsym{:}  \mathbin{\blacklozenge} \, \TLLnt{A}$ and $\Gamma  \TLLsym{,}  \mathit{x}  \TLLsym{:}  \TLLnt{A}  \TLLsym{;}  \Delta  \vdash  \TLLnt{t_{{\mathrm{2}}}}  \TLLsym{:}  \TLLnt{B}$.
  \item If $\Gamma  \TLLsym{;}  \Delta  \vdash  \blacksquare \, \TLLnt{t}  \TLLsym{:}  \blacksquare \, \TLLnt{A}$, then $\Gamma  \TLLsym{;}  \emptyset  \vdash  \TLLnt{t}  \TLLsym{:}  \TLLnt{A}$.
  \item If $\Gamma  \TLLsym{;}  \Delta  \vdash   \mathsf{let}\,\blacksquare  \mathit{x}  =  \TLLnt{t_{{\mathrm{1}}}} \,\mathsf{in}\, \TLLnt{t_{{\mathrm{2}}}}   \TLLsym{:}  \TLLnt{B}$, then there is a type $\TLLnt{A}$, such that,
    $\Gamma  \TLLsym{;}  \Delta  \vdash  \TLLnt{t_{{\mathrm{1}}}}  \TLLsym{:}  \blacksquare \, \TLLnt{A}$ and $\Gamma  \TLLsym{,}  \mathit{x}  \TLLsym{:}  \TLLnt{A}  \TLLsym{;}  \Delta  \vdash  \TLLnt{t_{{\mathrm{2}}}}  \TLLsym{:}  \TLLnt{B}$.
  \item If $\Gamma  \TLLsym{;}  \Delta  \vdash  \Diamond \, \TLLnt{t}  \TLLsym{:}  \Diamond \, \TLLnt{A}$, then $\Gamma  \TLLsym{;}  \Delta  \vdash  \TLLnt{t}  \TLLsym{:}  \TLLnt{A}$.
  \item If $\Gamma  \TLLsym{;}  \Delta  \vdash   \mathsf{let}\,\Diamond  \mathit{x}  =  \TLLnt{t_{{\mathrm{1}}}} \,\mathsf{in}\, \TLLnt{t_{{\mathrm{2}}}}   \TLLsym{:}  \TLLnt{B}$, then there is a type $\TLLnt{A}$, such that,
    $\Gamma  \TLLsym{;}  \Delta  \vdash  \TLLnt{t_{{\mathrm{1}}}}  \TLLsym{:}  \Diamond \, \TLLnt{A}$ and $\Gamma  \TLLsym{,}  \mathit{x}  \TLLsym{:}  \TLLnt{A}  \TLLsym{;}  \Delta  \vdash  \TLLnt{t_{{\mathrm{2}}}}  \TLLsym{:}  \TLLnt{B}$.
  \end{enumerate}
\end{lemma}
\begin{proof}
  Each of these principles hold by a simple induction over the assumed
  typing derivation.
\end{proof}

\begin{lemma}[Subject Reduction]
  \label{lemma:subject_reduction}
  If $\Gamma  \TLLsym{;}  \Delta  \vdash  \TLLnt{t}  \TLLsym{:}  \TLLnt{A}$ and $\TLLnt{t}  \rightsquigarrow  \TLLnt{t'}$, then $\Gamma  \TLLsym{;}  \Delta  \vdash  \TLLnt{t'}  \TLLsym{:}  \TLLnt{A}$.
\end{lemma}
\begin{proof}
  This is a proof by induction on the form of $\Gamma  \TLLsym{;}  \Delta  \vdash  \TLLnt{t}  \TLLsym{:}  \TLLnt{A}$.  We
  only show non-trivial cases, the absent cases either hold vacuously,
  by simply using the induction hypothesis and reapplying the rule, or
  are similar to the cases we give here.

  \begin{description}        
  \item[\cW]
    \[
    \TLLdruletyXXconjEOne{}
    \leqno{\raise 8 pt\hbox{\textbf{Case}}}
    \]
    The only non-trivial case is when $\TLLnt{t} = \TLLsym{(}  \TLLnt{t_{{\mathrm{1}}}}  \TLLsym{,}  \TLLnt{t_{{\mathrm{2}}}}  \TLLsym{)}$ and
    $\TLLnt{t'} = \TLLnt{t_{{\mathrm{1}}}}$ for some terms $\TLLnt{t_{{\mathrm{1}}}}$ and $\TLLnt{t_{{\mathrm{2}}}}$.  Thus, we
    know $\Gamma  \TLLsym{;}  \Delta  \vdash  \TLLsym{(}  \TLLnt{t_{{\mathrm{1}}}}  \TLLsym{,}  \TLLnt{t_{{\mathrm{2}}}}  \TLLsym{)}  \TLLsym{:}   \TLLnt{A}  \land  \TLLnt{B} $, and we must show that
    $\Gamma  \TLLsym{;}  \Delta  \vdash  \TLLnt{t_{{\mathrm{1}}}}  \TLLsym{:}  \TLLnt{A}$, but this holds by inversion.
        
  \item[\cW]
    \[
    \TLLdruletyXXdisjE{}
    \leqno{\raise 8 pt\hbox{\textbf{Case}}}
    \]     
    The only non-trivial cases arise when either $\TLLnt{t} = \mathsf{inj}_1\,  \TLLnt{t_{{\mathrm{3}}}}$ and $\TLLnt{t'} = \TLLsym{[}  \TLLnt{t_{{\mathrm{3}}}}  \TLLsym{/}  \mathit{x}  \TLLsym{]}  \TLLnt{t_{{\mathrm{1}}}}$, or
    $\TLLnt{t} = \mathsf{inj}_2\,  \TLLnt{t_{{\mathrm{3}}}}$ and $\TLLnt{t'} = \TLLsym{[}  \TLLnt{t_{{\mathrm{3}}}}  \TLLsym{/}  \mathit{x}  \TLLsym{]}  \TLLnt{t_{{\mathrm{2}}}}$.  We prove the former, because the latter is similar.
    It suffices to show that $\Gamma  \TLLsym{;}  \Delta  \vdash  \TLLsym{[}  \TLLnt{t_{{\mathrm{3}}}}  \TLLsym{/}  \mathit{x}  \TLLsym{]}  \TLLnt{t_{{\mathrm{1}}}}  \TLLsym{:}  \TLLnt{C}$.  We know by assumption that $\Gamma  \TLLsym{;}  \Delta  \TLLsym{,}  \mathit{x}  \TLLsym{:}  \TLLnt{A}  \vdash  \TLLnt{t_{{\mathrm{1}}}}  \TLLsym{:}  \TLLnt{C}$,
    and since we know $\Gamma  \TLLsym{;}  \Delta  \vdash  \mathsf{inj}_2\,  \TLLnt{t_{{\mathrm{3}}}}  \TLLsym{:}   \TLLnt{A}  \lor  \TLLnt{B} $ by assumption, then we know $\Gamma  \TLLsym{;}  \Delta  \vdash  \TLLnt{t_{{\mathrm{3}}}}  \TLLsym{:}  \TLLnt{A}$ by inversion.
    Finally, by substitution for typing we obtain our desired result.

  \item[\cW]
    \[
    \TLLdruletyXXimpE{}
    \leqno{\raise 8 pt\hbox{\textbf{Case}}}
    \]  
    The only non-trivial case arises when $\TLLnt{t_{{\mathrm{1}}}} =  \lambda  \mathit{x}  :  \TLLnt{A} . \TLLnt{t_{{\mathrm{3}}}} $ and $\TLLnt{t'} = \TLLsym{[}  \TLLnt{t_{{\mathrm{2}}}}  \TLLsym{/}  \mathit{x}  \TLLsym{]}  \TLLnt{t_{{\mathrm{3}}}}$.
    We know by assumption that $\Gamma  \TLLsym{;}  \Delta  \vdash   \lambda  \mathit{x}  :  \TLLnt{A} . \TLLnt{t_{{\mathrm{3}}}}   \TLLsym{:}  \TLLnt{A}  \to  \TLLnt{B}$, and by inversion we know
    $\Gamma  \TLLsym{;}  \Delta  \TLLsym{,}  \mathit{x}  \TLLsym{:}  \TLLnt{A}  \vdash  \TLLnt{t_{{\mathrm{3}}}}  \TLLsym{:}  \TLLnt{B}$.  Finally, we also know by assumption
    that $\Gamma  \TLLsym{;}  \Delta  \vdash  \TLLnt{t_{{\mathrm{2}}}}  \TLLsym{:}  \TLLnt{A}$, and so by substitution for typing we
    know $\Gamma  \TLLsym{;}  \Delta  \vdash  \TLLsym{[}  \TLLnt{t_{{\mathrm{2}}}}  \TLLsym{/}  \mathit{x}  \TLLsym{]}  \TLLnt{t_{{\mathrm{3}}}}  \TLLsym{:}  \TLLnt{B}$ which is our desired result.

  \item[\cW]
    \[
    \TLLdruletyXXboxE{}
    \leqno{\raise 8 pt\hbox{\textbf{Case}}}
    \]
    The only non-trivial cases arise when either $\TLLnt{t_{{\mathrm{1}}}} = \Box \, \TLLnt{t_{{\mathrm{3}}}}$
    and $\TLLnt{t'} = \TLLsym{[}  \TLLnt{t_{{\mathrm{3}}}}  \TLLsym{/}  \mathit{x}  \TLLsym{]}  \TLLnt{t_{{\mathrm{2}}}}$, or $\TLLnt{t_{{\mathrm{1}}}} =  \mathsf{let}\,\Box  \mathit{y}  =  \TLLnt{t_{{\mathrm{3}}}} \,\mathsf{in}\, \TLLnt{t_{{\mathrm{4}}}} $
    and $\TLLnt{t'} =  \mathsf{let}\,\Box  \mathit{y}  =  \TLLnt{t_{{\mathrm{3}}}} \,\mathsf{in}\,  \mathsf{let}\,\Box  \mathit{x}  =  \TLLnt{t_{{\mathrm{4}}}} \,\mathsf{in}\, \TLLnt{t_{{\mathrm{2}}}}  $.  We show
    the latter, because the former is similar to the previous case.

    We must show $\Gamma  \TLLsym{;}  \Delta  \vdash   \mathsf{let}\,\Box  \mathit{y}  =  \TLLnt{t_{{\mathrm{3}}}} \,\mathsf{in}\,  \mathsf{let}\,\Box  \mathit{x}  =  \TLLnt{t_{{\mathrm{4}}}} \,\mathsf{in}\, \TLLnt{t_{{\mathrm{2}}}}    \TLLsym{:}  \TLLnt{B}$.
    That is, the following must hold:
    \[
    \small
    \mprset{flushleft}
    \inferrule* [right=$\Box_{\mathcal{E}}$] {
      \Gamma  \TLLsym{;}  \Delta  \vdash  \TLLnt{t_{{\mathrm{3}}}}  \TLLsym{:}  \Box \, \TLLnt{C}
      \\
      $$\mprset{flushleft}
      \inferrule* [right=$\Box_{\mathcal{E}}$] {
        \Gamma  \TLLsym{,}  \mathit{y}  \TLLsym{:}  \TLLnt{C}  \TLLsym{;}  \Delta  \vdash  \TLLnt{t_{{\mathrm{4}}}}  \TLLsym{:}  \Box \, \TLLnt{A}
        \\
        \Gamma  \TLLsym{,}  \mathit{y}  \TLLsym{:}  \TLLnt{C}  \TLLsym{,}  \mathit{x}  \TLLsym{:}  \TLLnt{A}  \TLLsym{;}  \Delta  \vdash  \TLLnt{t_{{\mathrm{2}}}}  \TLLsym{:}  \TLLnt{B}
      }{\Gamma  \TLLsym{,}  \mathit{y}  \TLLsym{:}  \TLLnt{C}  \TLLsym{;}  \Delta  \vdash   \mathsf{let}\,\Box  \mathit{x}  =  \TLLnt{t_{{\mathrm{4}}}} \,\mathsf{in}\, \TLLnt{t_{{\mathrm{2}}}}   \TLLsym{:}  \TLLnt{B}}
    }{\Gamma  \TLLsym{;}  \Delta  \vdash   \mathsf{let}\,\Box  \mathit{y}  =  \TLLnt{t_{{\mathrm{3}}}} \,\mathsf{in}\,  \mathsf{let}\,\Box  \mathit{x}  =  \TLLnt{t_{{\mathrm{4}}}} \,\mathsf{in}\, \TLLnt{t_{{\mathrm{2}}}}    \TLLsym{:}  \TLLnt{B}}
    \]    
    The $\TLLnt{t_{{\mathrm{3}}}}$ and $\TLLnt{t_{{\mathrm{4}}}}$ premises hold by inversion, and the
    $\TLLnt{t_{{\mathrm{2}}}}$ premise holds by weakening using our assumption that
    $\Gamma  \TLLsym{,}  \mathit{x}  \TLLsym{:}  \TLLnt{A}  \TLLsym{;}  \Delta  \vdash  \TLLnt{t_{{\mathrm{2}}}}  \TLLsym{:}  \TLLnt{B}$ holds.  Thus, we obtain our result.
  \end{description}
  
\end{proof}
