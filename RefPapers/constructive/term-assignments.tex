Given the Natural Deduction system presented in the previous section,
we simply follow the recipes described in \cite{barber1997} to produce
a term assignment for these four kinds of modal operators.

We start by giving a calculus for annotating natural deduction proofs
in our CS4 tense calculus with typed $\lambda$-calculus terms. The
annotations must be such that types correspond to modal propositions,
terms correspond to proofs and proof normalisation corresponds to term
reduction. The basic typying judgements of our CS4 tense calculus are
of the form $[[G ; D |- t : A]]$ where $[[G]]$ declares modal
variables, which may occurr arbitrarily in a term, while $\Delta$
declares intuitionistic variables that may not occurr in a subterm of
the form $[[Box t]]$ or $[[BBox t]]$.

The typing rules for basic propositional logic can be found in Figure
\begin{figure}
  \begin{mdframed}
    \begin{mathpar}
      \TLLdruletyXXax{} \and
      \TLLdruletyXXbax{} \and
      \TLLdruletyXXtrue{} \and
      \TLLdruletyXXfalse{} \and
      \TLLdruletyXXconjI{} \and
      \TLLdruletyXXconjEOne{} \and
      \TLLdruletyXXconjETwo{} \and
      \TLLdruletyXXdisjIOne{} \and
      \TLLdruletyXXdisjITwo{} \and
      \TLLdruletyXXdisjE{} \and
      \TLLdruletyXXimpI{} \and
      \TLLdruletyXXimpE{}      
    \end{mathpar}
  \end{mdframed}
  \caption{Term assignment for basic propositional logic}
  \label{fig:term-assignment-basic}
\end{figure}

\begin{figure}
  \begin{mdframed}
    \begin{mathpar}
      \TLLdruletyXXboxI{} \and
      \TLLdruletyXXboxE{} \and
      \TLLdruletyXXbdiaI{} \and
      \TLLdruletyXXbdiaE{} \and
      \TLLdruletyXXbboxI{} \and
      \TLLdruletyXXbboxE{} \and
      \TLLdruletyXXdiaI{} \and
      \TLLdruletyXXdiaE{}
    \end{mathpar}
  \end{mdframed}
  \caption{Term assignment for TCS4}
  \label{fig:term-assignment-TCS4}
\end{figure}


