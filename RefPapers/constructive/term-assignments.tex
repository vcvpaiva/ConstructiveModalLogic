Given the Natural Deduction system presented in the previous section,
we simply follow the recipes described in \cite{barber1997} to produce
a term assignment for these four kinds of modal operators.

We start by giving a calculus for annotating natural deduction proofs
in our $\sf{CS4}$ tense calculus with typed $\lambda$-calculus terms. The
annotations must be such that types correspond to modal propositions,
terms correspond to proofs and proof normalisation corresponds to term
reduction. The basic typying judgements of our CS4 tense calculus are
of the form $[[G ; D |- t : A]]$ where $[[G]]$ declares modal
variables, which may occurr arbitrarily in a term, while $\Delta$
declares intuitionistic variables that may not occurr in a subterm of
the form $[[Box t]]$ or $[[BBox t]]$.

The typing rules for basic propositional logic can be found in
Figure~\ref{fig:term-assignment-basic}, and the typing rules for the
TCS4 modalities are in Figure~\ref{fig:term-assignment-TCS4}.
Finally, the reduction relation consists of the $\beta$-reduction
rules defined in Figure~\ref{fig:reduction-TCS4}, and the commuting
conversions defined in Figure~\ref{fig:commuting-conv-TCS4}; we omit
congruence rules in the interest of space.
\begin{figure}
  \begin{mdframed}
    \begin{mathpar}
      \TLLdruletyXXax{} \and
      \TLLdruletyXXbax{} \and
      \TLLdruletyXXtrue{} \and
      \TLLdruletyXXfalse{} \and
      \TLLdruletyXXconjI{} \and
      \TLLdruletyXXconjEOne{} \and
      \TLLdruletyXXconjETwo{} \and
      \TLLdruletyXXdisjIOne{} \and
      \TLLdruletyXXdisjITwo{} \and
      \TLLdruletyXXdisjE{} \and
      \TLLdruletyXXimpI{} \and
      \TLLdruletyXXimpE{}      
    \end{mathpar}
  \end{mdframed}
  \caption{Term assignment for basic propositional logic}
  \label{fig:term-assignment-basic}
\end{figure}
\begin{figure}
  \begin{mdframed}
    \begin{mathpar}
      \TLLdruletyXXboxI{} \and
      \TLLdruletyXXboxE{} \and
      \TLLdruletyXXbdiaI{} \and
      \TLLdruletyXXbdiaE{} \and
      \TLLdruletyXXbboxI{} \and
      \TLLdruletyXXbboxE{} \and
      \TLLdruletyXXdiaI{} \and
      \TLLdruletyXXdiaE{}
    \end{mathpar}
  \end{mdframed}
  \caption{Term assignment for TCS4}
  \label{fig:term-assignment-TCS4}
\end{figure}
\begin{figure}
  \begin{mdframed}
    \begin{mathpar}
      \TLLdrulerXXbeta{} \and
      \TLLdrulerXXfirst{} \and
      \TLLdrulerXXsecond{} \and
      \TLLdrulerXXcaseOne{} \and
      \TLLdrulerXXcaseTwo{} \and
      \TLLdrulerXXbox{} \and
      \TLLdrulerXXbdia{} \and
      \TLLdrulerXXbbox{} \and
      \TLLdrulerXXdia{}      
    \end{mathpar}
  \end{mdframed}
  \caption{The $\beta$-reduction rules for TCS4}
  \label{fig:reduction-TCS4}
\end{figure}
\begin{figure}
  \begin{mdframed}
    \begin{mathpar}
      \TLLdrulerXXboxBox{} \and
      \TLLdrulerXXbdiaBdia{} \and
      \TLLdrulerXXboxBBox{} \and
      \TLLdrulerXXdiadia{}
    \end{mathpar}
  \end{mdframed}
  \caption{The commuting conversions for TCS4}
  \label{fig:commuting-conv-TCS4}
\end{figure}
In the remainder of this section we prove various properties of this
system.


\subsection{Subject Reduction}
\label{subsec:subject_reduction}
The system TCS4 has two different types of assumptions, and hence, of
free variables, those that are members of $[[G]]$ and those that are
members of $[[D]]$, where the former are implicitly boxed.  This
implies that there are two types of substitution for typing and
weakening results.
\begin{lemma}[Substitution for Typing]
  \label{lemma:substitution_for_typing}
  \begin{enumerate}[i.]
  \item[] 
  \item If $[[G;D |- t1 : A]]$, and $[[G, x : A; D |- t2 : B]]$, then $[[G;D |- [t1/x]t2 : B]]$.
  \item If $[[G;D |- t1 : A]]$, and $[[G; D, x : A |- t2 : B]]$, then $[[G;D |- [t1/x]t2 : B]]$.
  \end{enumerate}
\end{lemma}
\begin{proof}
  Both parts of this proof hold by straightforward induction on the
  form of the second assumed typing derivation.  Notice that rules
  $\Box_{\mathcal{I}}$ and $\BBox_{\mathcal{I}}$ do not need to be
  considered in in the proof of part two, because we assume the
  propositional context, $[[D]]$, is non-empty.
\end{proof}

\begin{lemma}[Weakening]
  \label{lemma:weakening}
  \begin{enumerate}[i.]
  \item[]
  \item If $[[G;D |- t : B]]$, then $[[G,x : A; D |- t : B]]$.
  \item If $[[G;D |- t : B]]$, then $[[G; D,x : A |- t : B]]$.
  \end{enumerate}
\end{lemma}
\begin{proof}
  Both parts of this proof hold by straightforward induction on the
  form of the assumed typing derivation.  In the case of the rules
  $\Box_{\mathcal{I}}$ and $\BBox_{\mathcal{I}}$ we simply reapply the
  rule using the assumed premise.
\end{proof}

Next we show subject reduction, but this first requires the following
inversion principles.  Then subject reduction holds by straightforward
induction.
\begin{lemma}[Inversion]
  \label{lemma:inversion}
  \begin{enumerate}[i.]
  \item[]
  \item If $[[G; D |- (t1, t2) : A x B]]$, then $[[G;D |- t1 : A]]$, and $[[G;D |- t2 : B]]$.
  \item If $[[G; D |- fst t : B]]$, then there is a type $[[A]]$, such that, $[[G; D |- t : A x B]]$.
  \item If $[[G; D |- snd t : B]]$, then there is a type $[[A]]$, such that, $[[G; D |- t : A x B]]$.
  \item If $[[G; D |- inj1 t : A + B]]$, then $[[G; D |- t : A]]$.
  \item If $[[G; D |- inj2 t : A + B]]$, then $[[G; D |- t : B]]$.
  \item If $[[G; D |- case t of x.t1,x.t2 : C]]$, then there are types $[[A]]$ and $[[B]]$,
    such that, $[[G;D, x : A |- t1 : C]]$, $[[G;D, x : B |- t2 : C]]$, and $[[G;D |- t : A + B]]$.
  \item If $[[G;D |- \x:A.t : B]]$, then $[[G;D,x : A |- t : B]]$.    
  \item If $[[G;D |- t1 t2 : B]]$, then there exists a type $[[A]]$, such that, $[[G;D |- t1 : A -> B]]$ and $[[G;D |- t2 : A]]$.
  \item If $[[G;D |- Box t : Box A]]$, then $[[G;. |- t : A]]$.
  \item If $[[G;D |- letBox x = t1 in t2 : B]]$, then there is a type $[[A]]$, such that,
    $[[G;D |- t1 : Box A]]$ and $[[G,x : A;D |- t2 : B]]$.
  \item If $[[G;D |- BDia t : BDia A]]$, then $[[G;D |- t : A]]$.
  \item If $[[G;D |- letBDia x = t1 in t2 : B]]$, then there is a type $[[A]]$, such that,
    $[[G;D |- t1 : BDia A]]$ and $[[G,x : A;D |- t2 : B]]$.
  \item If $[[G;D |- BBox t : BBox A]]$, then $[[G;. |- t : A]]$.
  \item If $[[G;D |- letBBox x = t1 in t2 : B]]$, then there is a type $[[A]]$, such that,
    $[[G;D |- t1 : BBox A]]$ and $[[G,x : A;D |- t2 : B]]$.
  \item If $[[G;D |- Dia t : Dia A]]$, then $[[G;D |- t : A]]$.
  \item If $[[G;D |- letDia x = t1 in t2 : B]]$, then there is a type $[[A]]$, such that,
    $[[G;D |- t1 : Dia A]]$ and $[[G,x : A;D |- t2 : B]]$.
  \end{enumerate}
\end{lemma}
\begin{proof}
  Each of these principles hold by a simple induction over the assumed
  typing derivation.
\end{proof}

\begin{lemma}[Subject Reduction]
  \label{lemma:subject_reduction}
  If $[[G;D |- t : A]]$ and $[[t ~> t']]$, then $[[G;D |- t' : A]]$.
\end{lemma}
\begin{proof}
  This proof holds by induction on the assumed typing derivation; see
  the Appendix~\ref{subsec:proof_of_subject_reduction} for the
  complete proof.
\end{proof}
% subsection subject_reduction (end)

\subsection{Strong Normalization}
\label{subsec:strong_normalization}

We show that by a straightforward extension of Joachimski and Matthes'
\cite{Joachimski:2003} proof of strong normalization for the simply
typed $\lambda$-calculus with sum types that TCS4 is strongly
normalizing.  Thus, by their result we know the system described by
Figure~\ref{fig:term-assignment-basic} is strongly normalizing.  In
this proof we do not concern ourselves with the units, because they
are easy to handle and do not offer any insights.  However, we do show
how this technique applies to products which was left out by
Joachimski and Matthes.  Since the proof of strong normalization for
TCS4 is so close to the one provided by Joachimski and Matthes we only
give a sketch here.

We begin by extending their syntax for multiple eliminations.  Suppose
$[[r]]$ is any term. Then we define the following metasyntax:
\begin{description}
\item denote the expression $[[case r of x.t1,x.t2]]$ by $[[r (x.t1, x.t2)]]$, and
\item denote the expression $[[letM x = r in t]]$ by $[[r x.t]]$ where $M \in \{[[Box]], [[BBox]], [[Dia]], [[BDia]]\}$.
\end{description}
We denote an arbitrary elimination by $[[r R]]$ where $[[R]]$ is
either a term in the case of function application, or $[[r R]]$ has
the form just described above.  As an example, the metasyntax for
the redex $[[letBox x = Box t1 in t2]]$ is $[[(Box t1) x.t2]]$
which both step to $[[ [t1/x]t2]]$.  The commuting conversions
\[ [[letM x = (letM y = t1 in t2) in t3 ~> letM y = t1 in letM x = t2 in t3]] \]
are denoted in the metasyntax by $[[(t1 y.t2) x.t3 ~> t1 y.(t2 x.t3)]]$.

Finally, we will make heavy use of vector notation, $[[vec r]]$, to
stand for the sequence $[[r1]]\,[[r2]]\,\cdots\,[[ri]]$.  Thus, the
application $[[t vec r]]$ stands for
$[[t]]\,[[r1]]\,[[r2]]\,\cdots\,[[ri]]$ -- note that the sequence
could be empty, and hence, would simply be equal to $[[t]]$.  This
notation extends to substitution in the usual way and is denoted by
$[[ [vec t / vec x]t']]$.

Now we recast the syntax for terms using the metasyntax as follows:
\[
\begin{array}{rll}
  M & := & [[Box]] \mid [[BBox]] \mid [[Dia]] \mid [[BDia]]\\
  [[r]],[[s]],[[t]] & := & [[x vec n]] \mid [[x h(vec n) (t1 , t2) vec R]] \mid [[x h(vec n) (x.t1 , x.t2) vec R]]
  \mid [[h(x h(vec n) x.t) vec R]] \mid [[\x:A.r]] \mid \\
  & & [[fst r]] \mid [[snd r]] \mid [[inj1 r]] \mid [[inj2 r]] \mid [[(\x:A.r) s vec R]] \mid [[h((M r) x.s) vec R]]\\
  & & [[(inj1 r) (x.s1,x.s2) vec R]] \mid [[(inj2 r) (x.s1,x.s2) vec R]]
\end{array}
\]

We call any term that does not contain a redex as a subterm a normal
form.  That is, a term is normal if none of the rules in
Figure~\ref{fig:reduction-TCS4} nor
Figure~\ref{fig:commuting-conv-TCS4} apply.  These normal terms are
syntactically captured by the following grammar:
\[
\begin{array}{rll}
  [[n]] & := & [[x vec n]] \mid [[x h(vec n) (t1 , t2)]] \mid [[x h(vec n) (x.t1 , x.t2)]]
  \mid [[x h(vec n) x.t]] \mid [[\x:A.r]] \mid \\
  & & [[fst r]] \mid [[snd r]] \mid [[inj1 r]] \mid [[inj2 r]]
\end{array}
\]

These are straightforward extensions of the definitions given by
Joachimski and Matthes.  Notice that there are more normal forms then
then our grammar captures, but all of these are untyped, furthermore,
there are some untyped normal forms generated by this grammar, but
that is okay, because all typed normal forms are generated by the
previous grammar.  Thus, it captures all that we need it to.  The
syntax given here will only be used within this section.

The set of strongly normalizing terms is defined by the following
rules:
\begin{mathpar}
  \small
  \mprset{flushleft}
  \inferrule* [right=] {
    [[h(vec r)]] \in \mathsf{SN}
  }{[[x h(vec r)]] \in \mathsf{SN}}
  \and
  \mprset{flushleft}
  \inferrule* [right=] {
    [[h(vec r) R]] \in \mathsf{SN}
  }{[[x h(vec r) R]] \in \mathsf{SN}}
  \and
  \mprset{flushleft}
  \inferrule* [right=] {
    [[r]] \in \mathsf{SN}
  }{[[\x:A.r]] \in \mathsf{SN}}
  \and
  \mprset{flushleft}
  \inferrule* [right=] {
    [[r]] \in \mathsf{SN}
  }{[[inj1 r]] \in \mathsf{SN}}
  \and
  \mprset{flushleft}
  \inferrule* [right=] {
    [[r]] \in \mathsf{SN}
  }{[[inj2 r]] \in \mathsf{SN}}
  \and
  \mprset{flushleft}
  \inferrule* [right=] {
    [[r]] \in \mathsf{SN}
  }{[[fst r]] \in \mathsf{SN}}
  \and
  \mprset{flushleft}
  \inferrule* [right=] {
    [[r]] \in \mathsf{SN}
  }{[[snd r]] \in \mathsf{SN}}
  \and
  \mprset{flushleft}
  \inferrule* [right=] {
    [[r1]] \in \mathsf{SN}
    \\
    [[r2]] \in \mathsf{SN}
  }{[[(r1 , r2)]] \in \mathsf{SN}}
  \and
  \mprset{flushleft}
  \inferrule* [right=] {
    [[ [r1/x]r2 vec R]] \in \mathsf{SN}
    \\
    [[r1]] \in \mathsf{SN}
  }{[[(\x:A.r2) r1 vec R]] \in \mathsf{SN}}
  \and
  \inferrule* [right=] {
    [[r1 vec R]] \in \mathsf{SN}
    \\
    [[r2]] \in \mathsf{SN}
  }{[[(fst (r1,r2)) vec R]] \in \mathsf{SN}}
  \and
  \inferrule* [right=] {
    [[r2 vec R]] \in \mathsf{SN}
    \\
    [[r1]] \in \mathsf{SN}
  }{[[(snd (r1,r2)) vec R]] \in \mathsf{SN}}
  \and
  \mprset{flushleft}
  \inferrule* [right=] {
    [[ [r/x]r1 vec R]] \in \mathsf{SN}
    \\
    [[r2 vec R]] \in \mathsf{SN}
    \\
    [[r]] \in \mathsf{SN}  
  }{[[(inj1 r) (x.r1,x.r2) vec R]] \in \mathsf{SN}}
  \and
  \mprset{flushleft}
  \inferrule* [right=] {
    [[ [r/x]r2 vec R]] \in \mathsf{SN}
    \\
    [[r1 vec R]] \in \mathsf{SN}
    \\
    [[r]] \in \mathsf{SN}  
  }{[[(inj2 r) (x.r1,x.r2) vec R]] \in \mathsf{SN}}
  \and
  \mprset{flushleft}
  \inferrule* [right=] {
    [[ [r1/x]r2 vec R]] \in \mathsf{SN}
    \\
    [[r1]] \in \mathsf{SN}
    \\
    M \in \{ [[Box]], [[BBox]], [[Dia]], [[BDia]] \}
  }{[[(M r1) (x.r2) vec R]] \in \mathsf{SN}}
  \and
  \mprset{flushleft}
  \inferrule* [right=] {
    [[x h(vec r) (r1 y.(r2 x.r3)) vec R]] \in \mathsf{SN}
  }{[[x h(vec r) ((r1 y.r2) x.r3) vec R]] \in \mathsf{SN}}
\end{mathpar}
All of the rules from the previous definition are from Joachimski and
Matthes' proof, but the last two, which are for all the
$\beta$-redexes and commuting conversions for modalities.

At this point we have everything we need to conclude strong
normalization which follows from the following results.
\begin{lemma}[All strongly normalizing terms are in $\mathsf{SN}$]
  \label{lemma:all_strongly_normalizing_terms_are_in_SN}
  If $[[G;D |- r : A]]$, and $[[r]]$ is strongly normalizing, then $[[r]] \in \mathsf{SN}$.
\end{lemma}
\begin{proof}
  The first half of this proof is essentially the same as the proof by
  Joachimski and Matthes, see page 12 of \cite{Joachimski:2003}, and
  the new parts are similar to their proofs.
\end{proof}

\begin{lemma}[Commuting conversions in $\mathsf{SN}$]
  \label{lemma:commuting_conversions_in_SN}
  If $[[(r y.(r1 x.r2)) vec R]] \in \mathsf{SN}$, then $[[((r y.r1) x.r2) vec R]] \in \mathsf{SN}$.
\end{lemma}
\begin{proof}
  This proof holds by induction on the assumption $[[(r y.(r1 x.r2))
      vec R]] \in \mathsf{SN}$, but this proof is similar to the proof
  for commuting conversions for the sum types; see page 13 of
  \cite{Joachimski:2003}.
\end{proof}

\begin{lemma}[Typable eliminations are in $\mathsf{SN}$]
  \label{lemma:typable_eliminations_are_in_SN}
  Suppose $[[A]]$ is a type, and $[[r]] \in \mathsf{SN}$.  Then:
  \begin{enumerate}[i.]
  \item if $[[G;D |- s : A]]$, then $[[r s]] \in \mathsf{SN}$,
  \item if $[[G;D |- r : A]]$ and $[[A]] \equiv [[B x C]]$, then $[[(fst r)]],[[(snd r)]] \in \mathsf{SN}$,
  \item if $[[G;D |- r : A]]$, $[[A]] \equiv [[B + C]]$ and $[[s1]],[[s2]] \in \mathsf{SN}$,
    then $[[r (x.s1,x.s2)]] \in \mathsf{SN}$,
  \item if $[[G;D |- r : A]]$, $[[A]] \equiv [[M B]]$, and $[[s]] \in \mathsf{SN}$, then $[[r x.s]] \in \mathsf{SN}$,
    where $M \in \{[[Box]], [[BBox]], [[Dia]], [[BDia]] \}$, and
  \item if $[[G;D |- s : A]]$, then $[[ [s/x]r]] \in \mathsf{SN}$.
  \end{enumerate}
\end{lemma}
\begin{proof}
  The proofs of the first three and the last parts are essentially the
  same proofs given by Joachimski and Matthes where they differ is
  similar to the other cases.  Finally, the forth case is similar to
  the third. See Lemma~5.7 on page 14 of \cite{Joachimski:2003}.
\end{proof}

\begin{corollary}[Strong Normalization]
  \label{corollary:strong_normalization}
  If $[[G;D |- r : A]]$, then $[[r]]$ is strongly normalizing.
\end{corollary}
% subsection strong_normalization (end)

\subsection{Confluence}
\label{subsec:confluence}

% subsection confluence (end)
