\subsubsection{Proof of Subject Reduction}
\label{subsec:proof_of_subject_reduction}
This is a proof by induction on the form of $\Gamma  \TLLsym{;}  \Delta  \vdash  \TLLnt{t}  \TLLsym{:}  \TLLnt{A}$.  We
only show non-trivial cases, the absent cases either hold vacuously,
by simply using the induction hypothesis and reapplying the rule, or
are similar to the cases we give here.

\begin{description}        
\item[\cW]
  \[
  \TLLdruletyXXconjEOne{}
  \leqno{\raise 8 pt\hbox{\textbf{Case}}}
  \]
  The only non-trivial case is when $\TLLnt{t} = \TLLsym{(}  \TLLnt{t_{{\mathrm{1}}}}  \TLLsym{,}  \TLLnt{t_{{\mathrm{2}}}}  \TLLsym{)}$ and
  $\TLLnt{t'} = \TLLnt{t_{{\mathrm{1}}}}$ for some terms $\TLLnt{t_{{\mathrm{1}}}}$ and $\TLLnt{t_{{\mathrm{2}}}}$.  Thus, we
  know $\Gamma  \TLLsym{;}  \Delta  \vdash  \TLLsym{(}  \TLLnt{t_{{\mathrm{1}}}}  \TLLsym{,}  \TLLnt{t_{{\mathrm{2}}}}  \TLLsym{)}  \TLLsym{:}   \TLLnt{A}  \land  \TLLnt{B} $, and we must show that
  $\Gamma  \TLLsym{;}  \Delta  \vdash  \TLLnt{t_{{\mathrm{1}}}}  \TLLsym{:}  \TLLnt{A}$, but this holds by inversion.
  
\item[\cW]
  \[
  \TLLdruletyXXdisjE{}
  \leqno{\raise 8 pt\hbox{\textbf{Case}}}
  \]     
  The only non-trivial cases arise when either $\TLLnt{t} = \mathsf{inj}_1\,  \TLLnt{t_{{\mathrm{3}}}}$ and $\TLLnt{t'} = \TLLsym{[}  \TLLnt{t_{{\mathrm{3}}}}  \TLLsym{/}  \mathit{x}  \TLLsym{]}  \TLLnt{t_{{\mathrm{1}}}}$, or
  $\TLLnt{t} = \mathsf{inj}_2\,  \TLLnt{t_{{\mathrm{3}}}}$ and $\TLLnt{t'} = \TLLsym{[}  \TLLnt{t_{{\mathrm{3}}}}  \TLLsym{/}  \mathit{x}  \TLLsym{]}  \TLLnt{t_{{\mathrm{2}}}}$.  We prove the former, because the latter is similar.
  It suffices to show that $\Gamma  \TLLsym{;}  \Delta  \vdash  \TLLsym{[}  \TLLnt{t_{{\mathrm{3}}}}  \TLLsym{/}  \mathit{x}  \TLLsym{]}  \TLLnt{t_{{\mathrm{1}}}}  \TLLsym{:}  \TLLnt{C}$.  We know by assumption that $\Gamma  \TLLsym{;}  \Delta  \TLLsym{,}  \mathit{x}  \TLLsym{:}  \TLLnt{A}  \vdash  \TLLnt{t_{{\mathrm{1}}}}  \TLLsym{:}  \TLLnt{C}$,
  and since we know $\Gamma  \TLLsym{;}  \Delta  \vdash  \mathsf{inj}_2\,  \TLLnt{t_{{\mathrm{3}}}}  \TLLsym{:}   \TLLnt{A}  \lor  \TLLnt{B} $ by assumption, then we know $\Gamma  \TLLsym{;}  \Delta  \vdash  \TLLnt{t_{{\mathrm{3}}}}  \TLLsym{:}  \TLLnt{A}$ by inversion.
  Finally, by substitution for typing we obtain our desired result.

\item[\cW]
  \[
  \TLLdruletyXXimpE{}
  \leqno{\raise 8 pt\hbox{\textbf{Case}}}
  \]  
  The only non-trivial case arises when $\TLLnt{t_{{\mathrm{1}}}} =  \lambda  \mathit{x}  :  \TLLnt{A} . \TLLnt{t_{{\mathrm{3}}}} $ and $\TLLnt{t'} = \TLLsym{[}  \TLLnt{t_{{\mathrm{2}}}}  \TLLsym{/}  \mathit{x}  \TLLsym{]}  \TLLnt{t_{{\mathrm{3}}}}$.
  We know by assumption that $\Gamma  \TLLsym{;}  \Delta  \vdash   \lambda  \mathit{x}  :  \TLLnt{A} . \TLLnt{t_{{\mathrm{3}}}}   \TLLsym{:}  \TLLnt{A}  \to  \TLLnt{B}$, and by inversion we know
  $\Gamma  \TLLsym{;}  \Delta  \TLLsym{,}  \mathit{x}  \TLLsym{:}  \TLLnt{A}  \vdash  \TLLnt{t_{{\mathrm{3}}}}  \TLLsym{:}  \TLLnt{B}$.  Finally, we also know by assumption
  that $\Gamma  \TLLsym{;}  \Delta  \vdash  \TLLnt{t_{{\mathrm{2}}}}  \TLLsym{:}  \TLLnt{A}$, and so by substitution for typing we
  know $\Gamma  \TLLsym{;}  \Delta  \vdash  \TLLsym{[}  \TLLnt{t_{{\mathrm{2}}}}  \TLLsym{/}  \mathit{x}  \TLLsym{]}  \TLLnt{t_{{\mathrm{3}}}}  \TLLsym{:}  \TLLnt{B}$ which is our desired result.

\item[\cW]
  \[
  \TLLdruletyXXboxE{}
  \leqno{\raise 8 pt\hbox{\textbf{Case}}}
  \]
  The only non-trivial cases arise when either $\TLLnt{t_{{\mathrm{1}}}} = \Box \, \TLLnt{t_{{\mathrm{3}}}}$
  and $\TLLnt{t'} = \TLLsym{[}  \TLLnt{t_{{\mathrm{3}}}}  \TLLsym{/}  \mathit{x}  \TLLsym{]}  \TLLnt{t_{{\mathrm{2}}}}$, or $\TLLnt{t_{{\mathrm{1}}}} =  \mathsf{let}\,\Box  \mathit{y}  =  \TLLnt{t_{{\mathrm{3}}}} \,\mathsf{in}\, \TLLnt{t_{{\mathrm{4}}}} $
  and $\TLLnt{t'} =  \mathsf{let}\,\Box  \mathit{y}  =  \TLLnt{t_{{\mathrm{3}}}} \,\mathsf{in}\,  \mathsf{let}\,\Box  \mathit{x}  =  \TLLnt{t_{{\mathrm{4}}}} \,\mathsf{in}\, \TLLnt{t_{{\mathrm{2}}}}  $.  We show
  the latter, because the former is similar to the previous case.

  We must show $\Gamma  \TLLsym{;}  \Delta  \vdash   \mathsf{let}\,\Box  \mathit{y}  =  \TLLnt{t_{{\mathrm{3}}}} \,\mathsf{in}\,  \mathsf{let}\,\Box  \mathit{x}  =  \TLLnt{t_{{\mathrm{4}}}} \,\mathsf{in}\, \TLLnt{t_{{\mathrm{2}}}}    \TLLsym{:}  \TLLnt{B}$.
  That is, the following must hold:
  \[
  \small
  \mprset{flushleft}
  \inferrule* [right=$\Box_{\mathcal{E}}$] {
    \Gamma  \TLLsym{;}  \Delta  \vdash  \TLLnt{t_{{\mathrm{3}}}}  \TLLsym{:}  \Box \, \TLLnt{C}
    \\
    $$\mprset{flushleft}
    \inferrule* [right=$\Box_{\mathcal{E}}$] {
      \Gamma  \TLLsym{,}  \mathit{y}  \TLLsym{:}  \TLLnt{C}  \TLLsym{;}  \Delta  \vdash  \TLLnt{t_{{\mathrm{4}}}}  \TLLsym{:}  \Box \, \TLLnt{A}
      \\
        \Gamma  \TLLsym{,}  \mathit{y}  \TLLsym{:}  \TLLnt{C}  \TLLsym{,}  \mathit{x}  \TLLsym{:}  \TLLnt{A}  \TLLsym{;}  \Delta  \vdash  \TLLnt{t_{{\mathrm{2}}}}  \TLLsym{:}  \TLLnt{B}
    }{\Gamma  \TLLsym{,}  \mathit{y}  \TLLsym{:}  \TLLnt{C}  \TLLsym{;}  \Delta  \vdash   \mathsf{let}\,\Box  \mathit{x}  =  \TLLnt{t_{{\mathrm{4}}}} \,\mathsf{in}\, \TLLnt{t_{{\mathrm{2}}}}   \TLLsym{:}  \TLLnt{B}}
  }{\Gamma  \TLLsym{;}  \Delta  \vdash   \mathsf{let}\,\Box  \mathit{y}  =  \TLLnt{t_{{\mathrm{3}}}} \,\mathsf{in}\,  \mathsf{let}\,\Box  \mathit{x}  =  \TLLnt{t_{{\mathrm{4}}}} \,\mathsf{in}\, \TLLnt{t_{{\mathrm{2}}}}    \TLLsym{:}  \TLLnt{B}}
  \]    
  The $\TLLnt{t_{{\mathrm{3}}}}$ and $\TLLnt{t_{{\mathrm{4}}}}$ premises hold by inversion, and the
  $\TLLnt{t_{{\mathrm{2}}}}$ premise holds by weakening using our assumption that
  $\Gamma  \TLLsym{,}  \mathit{x}  \TLLsym{:}  \TLLnt{A}  \TLLsym{;}  \Delta  \vdash  \TLLnt{t_{{\mathrm{2}}}}  \TLLsym{:}  \TLLnt{B}$ holds.  Thus, we obtain our result.
\end{description}  
% subsubsection proof_of_subject_reduction (end)
