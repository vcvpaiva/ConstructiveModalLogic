\documentclass{article}
\usepackage[utf8]{inputenc}

\title{Constructive Temporal Logic, Categorically}
\author{Valeria de Paiva \and Harley Eades III}
\date{July 2016}

\usepackage{natbib}
\usepackage{graphicx}
\usepackage{amssymb, amsthm, amsmath, stmaryrd}
\usepackage{mathpartir}
\usepackage{mdframed}           % For the boxes around the systems.
\usepackage{cmll}
\usepackage[barr]{xy}

%% This renames Barr's \Diamond command so that it doesn't conflict
%% with our modality symbols.
\let\BDiamond\Diamond
\let\Diamond\relax
\newcommand{\bLozenge}{\mathbin{\blacklozenge}}
%% This renames Barr's \to to \mto.  This allows us to use \to for imp
%% and \mto for a inline morphism.
\let\mto\to
\let\to\relax
\newcommand{\to}{\rightarrow}

\usepackage{wasysym}
\usepackage{proof}
\usepackage{enumerate}
\usepackage{todonotes}
\usepackage{hyperref}
\usepackage{graphicx}

\let\b\relax
\let\d\relax
\let\t\relax

\renewcommand{\Box}{\oblong}
\newcommand{\BBox}{\blacksquare}
\newcommand{\BDia}{\mathbin{\blacklozenge}}

\newcommand{\F}{\mathop{\textbf{F}}}
\renewcommand{\P}{\mathop{\textbf{P}}}
\newcommand{\G}{\mathop{\textbf{G}}}
\renewcommand{\H}{\mathop{\textbf{H}}}

\newcommand{\cat}[1]{\mathcal{#1}}
\newcommand{\pd}[0]{\times}
\newcommand{\ihom}[0]{\rightarrow}
\newcommand{\st}[2]{\mathsf{st}_{#1,#2}}
\newcommand{\id}[0]{\mathsf{id}}
\newcommand{\b}[1]{\mathsf{b}_{#1}}
\newcommand{\d}[1]{\mathsf{d}_{#1}}
\newcommand{\m}[1]{\mathsf{m}_{#1}}
\newcommand{\n}[1]{\mathsf{n}_{#1}}
\newcommand{\p}[1]{\mathsf{p}_{#1}}
\newcommand{\q}[1]{\mathsf{q}_{#1}}
\newcommand{\t}[0]{\mathsf{t}}
\newcommand{\limp}[0]{\multimap}
\newcommand{\Hom}[3]{\mathsf{Hom}_{#1}(#2,#3)}

\def\cW{\color{white}}

%% Begin Ott
% generated by Ott 0.25 from: ott/TCS4.ott
\newcommand{\TLLdrule}[4][]{{\displaystyle\frac{\begin{array}{l}#2\end{array}}{#3}\quad\TLLdrulename{#4}}}
\newcommand{\TLLusedrule}[1]{\[#1\]}
\newcommand{\TLLpremise}[1]{ #1 \\}
\newenvironment{TLLdefnblock}[3][]{ \framebox{\mbox{#2}} \quad #3 \\[0pt]}{}
\newenvironment{TLLfundefnblock}[3][]{ \framebox{\mbox{#2}} \quad #3 \\[0pt]\begin{displaymath}\begin{array}{l}}{\end{array}\end{displaymath}}
\newcommand{\TLLfunclause}[2]{ #1 \equiv #2 \\}
\newcommand{\TLLnt}[1]{\mathit{#1}}
\newcommand{\TLLmv}[1]{\mathit{#1}}
\newcommand{\TLLkw}[1]{\mathbf{#1}}
\newcommand{\TLLsym}[1]{#1}
\newcommand{\TLLcom}[1]{\text{#1}}
\newcommand{\TLLdrulename}[1]{\textsc{#1}}
\newcommand{\TLLcomplu}[5]{\overline{#1}^{\,#2\in #3 #4 #5}}
\newcommand{\TLLcompu}[3]{\overline{#1}^{\,#2<#3}}
\newcommand{\TLLcomp}[2]{\overline{#1}^{\,#2}}
\newcommand{\TLLgrammartabular}[1]{\begin{supertabular}{llcllllll}#1\end{supertabular}}
\newcommand{\TLLmetavartabular}[1]{\begin{supertabular}{ll}#1\end{supertabular}}
\newcommand{\TLLrulehead}[3]{$#1$ & & $#2$ & & & \multicolumn{2}{l}{#3}}
\newcommand{\TLLprodline}[6]{& & $#1$ & $#2$ & $#3 #4$ & $#5$ & $#6$}
\newcommand{\TLLfirstprodline}[6]{\TLLprodline{#1}{#2}{#3}{#4}{#5}{#6}}
\newcommand{\TLLlongprodline}[2]{& & $#1$ & \multicolumn{4}{l}{$#2$}}
\newcommand{\TLLfirstlongprodline}[2]{\TLLlongprodline{#1}{#2}}
\newcommand{\TLLbindspecprodline}[6]{\TLLprodline{#1}{#2}{#3}{#4}{#5}{#6}}
\newcommand{\TLLprodnewline}{\\}
\newcommand{\TLLinterrule}{\\[5.0mm]}
\newcommand{\TLLafterlastrule}{\\}
\newcommand{\TLLmetavars}{
\TLLmetavartabular{
 $ \mathit{termvar} ,\, \mathit{x} ,\, \mathit{y} $ & \TLLcom{term variable} \\
 $ \TLLmv{index} ,\, \TLLmv{i} ,\, \TLLmv{j} ,\, \TLLmv{k} $ &  \\
}}

\newcommand{\TLLterm}{
\TLLrulehead{\TLLnt{term}  ,\ \TLLnt{t}  ,\ \TLLnt{r}  ,\ \TLLnt{s}  ,\ \TLLnt{n}}{::=}{\TLLcom{term}}\TLLprodnewline
\TLLfirstprodline{|}{\mathit{x}}{}{}{}{\TLLcom{variable}}\TLLprodnewline
\TLLprodline{|}{ \mathsf{contra} }{}{}{}{}\TLLprodnewline
\TLLprodline{|}{ \lambda  \mathit{x}  :  \TLLnt{T} . \TLLnt{t} }{}{}{}{\TLLcom{unary functions}}\TLLprodnewline
\TLLprodline{|}{\TLLnt{t_{{\mathrm{1}}}} \, \TLLnt{t_{{\mathrm{2}}}}}{}{}{}{\TLLcom{function application}}\TLLprodnewline
\TLLprodline{|}{\mathsf{unbox}_\Box\, \, \TLLnt{t}}{}{}{}{}\TLLprodnewline
\TLLprodline{|}{\mathsf{unbox}_\blacksquare\, \, \TLLnt{t}}{}{}{}{}\TLLprodnewline
\TLLprodline{|}{\Box \, \TLLnt{t}}{}{}{}{\TLLcom{past necessity functor}}\TLLprodnewline
\TLLprodline{|}{\Diamond \, \TLLnt{t}}{}{}{}{\TLLcom{past possibility functor}}\TLLprodnewline
\TLLprodline{|}{\blacksquare \, \TLLnt{t}}{}{}{}{\TLLcom{necessity functor}}\TLLprodnewline
\TLLprodline{|}{\mathbin{\blacklozenge} \, \TLLnt{t}}{}{}{}{\TLLcom{possibility functor}}\TLLprodnewline
\TLLprodline{|}{ \mathsf{adj_R}\,\blacklozenge  \mathit{y}  =  \mathit{x} \,\mathsf{in}\,\Box  \TLLnt{t} }{}{}{}{}\TLLprodnewline
\TLLprodline{|}{ \mathsf{adj_L}\,\blacklozenge  \mathit{y}  =  \mathit{x} \,\mathsf{in}\, \TLLnt{t} }{}{}{}{}\TLLprodnewline
\TLLprodline{|}{ \mathsf{let}_\Box\, \Gamma \,\mathsf{be}\, \TLLnt{t_{{\mathrm{1}}}} \,\mathsf{in}\, \TLLnt{t_{{\mathrm{2}}}} }{}{}{}{\TLLcom{past necessity elim}}\TLLprodnewline
\TLLprodline{|}{ \mathsf{let}_\blacksquare\, \Gamma \,\mathsf{be}\, \TLLnt{t_{{\mathrm{1}}}} \,\mathsf{in}\, \TLLnt{t_{{\mathrm{2}}}} }{}{}{}{\TLLcom{past necessity elim}}\TLLprodnewline
\TLLprodline{|}{ \mathsf{let}\,\Diamond  \mathit{x} : \TLLnt{A} = \TLLnt{s}  \mid  \Gamma \,\mathsf{be}\, \TLLnt{t_{{\mathrm{1}}}} \,\mathsf{in}\, \TLLnt{t_{{\mathrm{2}}}} }{}{}{}{}\TLLprodnewline
\TLLprodline{|}{ \mathsf{let}\,\blacklozenge  \mathit{x} : \TLLnt{A} = \TLLnt{s}  \mid  \Gamma \,\mathsf{be}\, \TLLnt{t_{{\mathrm{1}}}} \,\mathsf{in}\, \TLLnt{t_{{\mathrm{2}}}} }{}{}{}{}\TLLprodnewline
\TLLprodline{|}{\TLLsym{[}  \TLLnt{t_{{\mathrm{1}}}}  \TLLsym{/}  \TLLnt{t}  \TLLsym{]}  \TLLnt{t_{{\mathrm{2}}}}} {\textsf{M}}{}{}{}\TLLprodnewline
\TLLprodline{|}{\TLLsym{(}  \TLLnt{t}  \TLLsym{)}} {\textsf{S}}{}{}{}\TLLprodnewline
\TLLprodline{|}{ \TLLnt{t} } {\textsf{M}}{}{}{}\TLLprodnewline
\TLLprodline{|}{\TLLnt{r} \, \TLLkw{R}} {\textsf{M}}{}{}{}\TLLprodnewline
\TLLprodline{|}{ \mathsf{let}\,\mathbin{M}  \TLLnt{t_{{\mathrm{1}}}} : \TLLnt{T}  =  \TLLnt{t_{{\mathrm{2}}}} \,\mathsf{in}\, \TLLnt{t_{{\mathrm{3}}}} } {\textsf{M}}{}{}{}\TLLprodnewline
\TLLprodline{|}{ \overrightarrow{ \TLLnt{t} } } {\textsf{M}}{}{}{}\TLLprodnewline
\TLLprodline{|}{ \mathbin{M}  \TLLnt{t} } {\textsf{M}}{}{}{}\TLLprodnewline
\TLLprodline{|}{\TLLnt{t_{{\mathrm{1}}}}  \TLLsym{,} \, ... \, \TLLsym{,}  \TLLnt{t_{\TLLmv{k}}}} {\textsf{M}}{}{}{}\TLLprodnewline
\TLLprodline{|}{\TLLsym{[}  \TLLnt{t_{{\mathrm{1}}}}  \TLLsym{/}  \mathit{x_{{\mathrm{1}}}}  \TLLsym{]} \, ... \, \TLLsym{[}  \TLLnt{t_{\TLLmv{k}}}  \TLLsym{/}  \mathit{x_{\TLLmv{k}}}  \TLLsym{]}  \TLLnt{t}} {\textsf{M}}{}{}{}}

\newcommand{\TLLform}{
\TLLrulehead{\TLLnt{form}  ,\ \TLLnt{type}  ,\ \TLLnt{A}  ,\ \TLLnt{B}  ,\ \TLLnt{C}  ,\ \TLLnt{T}}{::=}{\TLLcom{formula and type}}\TLLprodnewline
\TLLfirstprodline{|}{\perp}{}{}{}{\TLLcom{false or the empty type}}\TLLprodnewline
\TLLprodline{|}{\Box \, \TLLnt{A}}{}{}{}{\TLLcom{past necessity}}\TLLprodnewline
\TLLprodline{|}{\blacksquare \, \TLLnt{A}}{}{}{}{\TLLcom{necessity}}\TLLprodnewline
\TLLprodline{|}{\Diamond \, \TLLnt{A}}{}{}{}{\TLLcom{past possibility}}\TLLprodnewline
\TLLprodline{|}{\mathbin{\blacklozenge} \, \TLLnt{A}}{}{}{}{\TLLcom{possibility}}\TLLprodnewline
\TLLprodline{|}{\TLLnt{A}  \to  \TLLnt{B}}{}{}{}{\TLLcom{implication}}\TLLprodnewline
\TLLprodline{|}{ \mathbin{M}  \TLLnt{A} } {\textsf{M}}{}{}{}}

\newcommand{\TLLG}{
\TLLrulehead{\Gamma  ,\ \Delta}{::=}{\TLLcom{type context}}\TLLprodnewline
\TLLfirstprodline{|}{\emptyset}{}{}{}{\TLLcom{empty context}}\TLLprodnewline
\TLLprodline{|}{\TLLnt{A}}{}{}{}{\TLLcom{formula el}}\TLLprodnewline
\TLLprodline{|}{\mathit{x}  \TLLsym{:}  \TLLnt{T}}{}{}{}{\TLLcom{typed el}}\TLLprodnewline
\TLLprodline{|}{\Gamma  \TLLsym{,}  \Gamma'}{}{}{}{\TLLcom{append}}\TLLprodnewline
\TLLprodline{|}{\Gamma_{{\mathrm{1}}}  \TLLsym{,} \, ... \, \TLLsym{,}  \Gamma_{\TLLmv{k}}} {\textsf{M}}{}{}{}}

\newcommand{\TLLterminals}{
\TLLrulehead{\TLLnt{terminals}}{::=}{}\TLLprodnewline
\TLLfirstprodline{|}{ \mathsf{unbox}_\Box\, }{}{}{}{}\TLLprodnewline
\TLLprodline{|}{ \mathsf{unbox}_\blacksquare\, }{}{}{}{}\TLLprodnewline
\TLLprodline{|}{ \emptyset }{}{}{}{}\TLLprodnewline
\TLLprodline{|}{ \perp }{}{}{}{}\TLLprodnewline
\TLLprodline{|}{ \vdash }{}{}{}{}\TLLprodnewline
\TLLprodline{|}{ \Box }{}{}{}{}\TLLprodnewline
\TLLprodline{|}{ \blacksquare }{}{}{}{}\TLLprodnewline
\TLLprodline{|}{ \Diamond }{}{}{}{}\TLLprodnewline
\TLLprodline{|}{ \mathbin{\blacklozenge} }{}{}{}{}\TLLprodnewline
\TLLprodline{|}{ \to }{}{}{}{}\TLLprodnewline
\TLLprodline{|}{ \quad } {\textsf{M}}{}{}{}}

\newcommand{\TLLformula}{
\TLLrulehead{\TLLnt{formula}}{::=}{}\TLLprodnewline
\TLLfirstprodline{|}{\TLLnt{judgement}}{}{}{}{}\TLLprodnewline
\TLLprodline{|}{\TLLnt{formula_{{\mathrm{1}}}}  \quad  \TLLnt{formula_{{\mathrm{2}}}}} {\textsf{M}}{}{}{}\TLLprodnewline
\TLLprodline{|}{ \TLLnt{formula} } {\textsf{S}}{}{}{}\TLLprodnewline
\TLLprodline{|}{ M \in \{\Diamond, \mathbin{\blacklozenge} \} } {\textsf{M}}{}{}{}\TLLprodnewline
\TLLprodline{|}{\TLLnt{formula_{{\mathrm{1}}}}  \TLLsym{,} \, ... \, \TLLsym{,}  \TLLnt{formula_{\TLLmv{k}}}}{}{}{}{}}

\newcommand{\TLLJtype}{
\TLLrulehead{\TLLnt{Jtype}}{::=}{}\TLLprodnewline
\TLLfirstprodline{|}{\Gamma  \vdash  \TLLnt{t}  \TLLsym{:}  \TLLnt{A}}{}{}{}{}\TLLprodnewline
\TLLprodline{|}{ \Gamma  \vdash  \TLLnt{t_{{\mathrm{1}}}}  \approx  \TLLnt{t_{{\mathrm{2}}}}  :  \TLLnt{A} }{}{}{}{}}

\newcommand{\TLLjudgement}{
\TLLrulehead{\TLLnt{judgement}}{::=}{}\TLLprodnewline
\TLLfirstprodline{|}{\TLLnt{Jtype}}{}{}{}{}}

\newcommand{\TLLuserXXsyntax}{
\TLLrulehead{\TLLnt{user\_syntax}}{::=}{}\TLLprodnewline
\TLLfirstprodline{|}{\mathit{termvar}}{}{}{}{}\TLLprodnewline
\TLLprodline{|}{\TLLmv{index}}{}{}{}{}\TLLprodnewline
\TLLprodline{|}{\TLLnt{term}}{}{}{}{}\TLLprodnewline
\TLLprodline{|}{\TLLnt{form}}{}{}{}{}\TLLprodnewline
\TLLprodline{|}{\Gamma}{}{}{}{}\TLLprodnewline
\TLLprodline{|}{\TLLnt{terminals}}{}{}{}{}\TLLprodnewline
\TLLprodline{|}{\TLLnt{formula}}{}{}{}{}}

\newcommand{\TLLgrammar}{\TLLgrammartabular{
\TLLterm\TLLinterrule
\TLLform\TLLinterrule
\TLLG\TLLinterrule
\TLLterminals\TLLinterrule
\TLLformula\TLLinterrule
\TLLJtype\TLLinterrule
\TLLjudgement\TLLinterrule
\TLLuserXXsyntax\TLLafterlastrule
}}

% defnss
% defns Jtype
%% defn type
\newcommand{\TLLdruletyXXaxName}[0]{\TLLdrulename{ty\_ax}}
\newcommand{\TLLdruletyXXax}[1]{\TLLdrule[#1]{%
}{
\Gamma  \TLLsym{,}  \mathit{x}  \TLLsym{:}  \TLLnt{A}  \vdash  \mathit{x}  \TLLsym{:}  \TLLnt{A}}{%
{\TLLdruletyXXaxName}{}%
}}


\newcommand{\TLLdruletyXXfalseName}[0]{\TLLdrulename{ty\_false}}
\newcommand{\TLLdruletyXXfalse}[1]{\TLLdrule[#1]{%
}{
\Gamma  \TLLsym{,}  \mathit{x}  \TLLsym{:}  \perp  \vdash   \mathsf{contra}   \TLLsym{:}  \TLLnt{A}}{%
{\TLLdruletyXXfalseName}{}%
}}


\newcommand{\TLLdruletyXXimpIName}[0]{\TLLdrulename{ty\_impI}}
\newcommand{\TLLdruletyXXimpI}[1]{\TLLdrule[#1]{%
\TLLpremise{\Gamma  \TLLsym{,}  \mathit{x}  \TLLsym{:}  \TLLnt{A}  \vdash  \TLLnt{t}  \TLLsym{:}  \TLLnt{B}}%
}{
\Gamma  \vdash   \lambda  \mathit{x}  :  \TLLnt{A} . \TLLnt{t}   \TLLsym{:}  \TLLnt{A}  \to  \TLLnt{B}}{%
{\TLLdruletyXXimpIName}{}%
}}


\newcommand{\TLLdruletyXXimpEName}[0]{\TLLdrulename{ty\_impE}}
\newcommand{\TLLdruletyXXimpE}[1]{\TLLdrule[#1]{%
\TLLpremise{\Gamma  \vdash  \TLLnt{t_{{\mathrm{1}}}}  \TLLsym{:}  \TLLnt{A}  \to  \TLLnt{B}  \quad  \Gamma  \vdash  \TLLnt{t_{{\mathrm{2}}}}  \TLLsym{:}  \TLLnt{A}}%
}{
\Gamma  \vdash  \TLLnt{t_{{\mathrm{1}}}} \, \TLLnt{t_{{\mathrm{2}}}}  \TLLsym{:}  \TLLnt{B}}{%
{\TLLdruletyXXimpEName}{}%
}}


\newcommand{\TLLdruletyXXboxEName}[0]{\TLLdrulename{ty\_boxE}}
\newcommand{\TLLdruletyXXboxE}[1]{\TLLdrule[#1]{%
\TLLpremise{\Gamma  \vdash  \TLLnt{t}  \TLLsym{:}  \Box \, \TLLnt{B}}%
}{
\Gamma  \vdash  \mathsf{unbox}_\Box\, \, \TLLnt{t}  \TLLsym{:}  \TLLnt{B}}{%
{\TLLdruletyXXboxEName}{}%
}}


\newcommand{\TLLdruletyXXboxIName}[0]{\TLLdrulename{ty\_boxI}}
\newcommand{\TLLdruletyXXboxI}[1]{\TLLdrule[#1]{%
\TLLpremise{ \Gamma  \vdash  \TLLnt{t_{{\mathrm{1}}}}  \TLLsym{:}  \Box \, \TLLnt{A_{{\mathrm{1}}}}  \TLLsym{,} \, ... \, \TLLsym{,}  \Gamma  \vdash  \TLLnt{t_{\TLLmv{k}}}  \TLLsym{:}  \Box \, \TLLnt{A_{\TLLmv{k}}}   \quad  \mathit{x_{{\mathrm{1}}}}  \TLLsym{:}  \Box \, \TLLnt{A_{{\mathrm{1}}}}  \TLLsym{,} \, ... \, \TLLsym{,}  \mathit{x_{\TLLmv{k}}}  \TLLsym{:}  \Box \, \TLLnt{A_{\TLLmv{k}}}  \vdash  \TLLnt{t}  \TLLsym{:}  \TLLnt{B}}%
}{
\Gamma  \vdash   \mathsf{let}_\Box\, \mathit{x_{{\mathrm{1}}}}  \TLLsym{:}  \Box \, \TLLnt{A_{{\mathrm{1}}}}  \TLLsym{,} \, ... \, \TLLsym{,}  \mathit{x_{\TLLmv{k}}}  \TLLsym{:}  \Box \, \TLLnt{A_{\TLLmv{k}}} \,\mathsf{be}\, \TLLnt{t_{{\mathrm{1}}}}  \TLLsym{,} \, ... \, \TLLsym{,}  \TLLnt{t_{\TLLmv{k}}} \,\mathsf{in}\, \TLLnt{t}   \TLLsym{:}  \Box \, \TLLnt{B}}{%
{\TLLdruletyXXboxIName}{}%
}}


\newcommand{\TLLdruletyXXbboxEName}[0]{\TLLdrulename{ty\_bboxE}}
\newcommand{\TLLdruletyXXbboxE}[1]{\TLLdrule[#1]{%
\TLLpremise{\Gamma  \vdash  \TLLnt{t}  \TLLsym{:}  \blacksquare \, \TLLnt{B}}%
}{
\Gamma  \vdash  \mathsf{unbox}_\blacksquare\, \, \TLLnt{t}  \TLLsym{:}  \TLLnt{B}}{%
{\TLLdruletyXXbboxEName}{}%
}}


\newcommand{\TLLdruletyXXbboxIName}[0]{\TLLdrulename{ty\_bboxI}}
\newcommand{\TLLdruletyXXbboxI}[1]{\TLLdrule[#1]{%
\TLLpremise{ \Gamma  \vdash  \TLLnt{t_{{\mathrm{1}}}}  \TLLsym{:}  \blacksquare \, \TLLnt{A_{{\mathrm{1}}}}  \TLLsym{,} \, ... \, \TLLsym{,}  \Gamma  \vdash  \TLLnt{t_{\TLLmv{k}}}  \TLLsym{:}  \blacksquare \, \TLLnt{A_{\TLLmv{k}}}   \quad  \mathit{x_{{\mathrm{1}}}}  \TLLsym{:}  \blacksquare \, \TLLnt{A_{{\mathrm{1}}}}  \TLLsym{,} \, ... \, \TLLsym{,}  \mathit{x_{\TLLmv{k}}}  \TLLsym{:}  \blacksquare \, \TLLnt{A_{\TLLmv{k}}}  \vdash  \TLLnt{t}  \TLLsym{:}  \TLLnt{B}}%
}{
\Gamma  \vdash   \mathsf{let}_\blacksquare\, \mathit{x_{{\mathrm{1}}}}  \TLLsym{:}  \blacksquare \, \TLLnt{A_{{\mathrm{1}}}}  \TLLsym{,} \, ... \, \TLLsym{,}  \mathit{x_{\TLLmv{k}}}  \TLLsym{:}  \blacksquare \, \TLLnt{A_{\TLLmv{k}}} \,\mathsf{be}\, \TLLnt{t_{{\mathrm{1}}}}  \TLLsym{,} \, ... \, \TLLsym{,}  \TLLnt{t_{\TLLmv{k}}} \,\mathsf{in}\, \TLLnt{t}   \TLLsym{:}  \blacksquare \, \TLLnt{B}}{%
{\TLLdruletyXXbboxIName}{}%
}}

\newcommand{\TLLdefntype}[1]{\begin{TLLdefnblock}[#1]{$\Gamma  \vdash  \TLLnt{t}  \TLLsym{:}  \TLLnt{A}$}{}
\TLLusedrule{\TLLdruletyXXax{}}
\TLLusedrule{\TLLdruletyXXfalse{}}
\TLLusedrule{\TLLdruletyXXimpI{}}
\TLLusedrule{\TLLdruletyXXimpE{}}
\TLLusedrule{\TLLdruletyXXboxE{}}
\TLLusedrule{\TLLdruletyXXboxI{}}
\TLLusedrule{\TLLdruletyXXbboxE{}}
\TLLusedrule{\TLLdruletyXXbboxI{}}
\end{TLLdefnblock}}

%% defn eq
\newcommand{\TLLdruleeqXXbetaName}[0]{\TLLdrulename{eq\_beta}}
\newcommand{\TLLdruleeqXXbeta}[1]{\TLLdrule[#1]{%
\TLLpremise{ \Gamma  \TLLsym{,}  \mathit{x}  \TLLsym{:}  \TLLnt{A}  \vdash  \TLLnt{t_{{\mathrm{2}}}}  \approx  \TLLnt{s_{{\mathrm{2}}}}  :  \TLLnt{B}   \quad   \Gamma  \vdash  \TLLnt{t_{{\mathrm{1}}}}  \approx  \TLLnt{s_{{\mathrm{1}}}}  :  \TLLnt{A} }%
}{
 \Gamma  \vdash  \TLLsym{(}   \lambda  \mathit{x}  :  \TLLnt{A} . \TLLnt{t_{{\mathrm{2}}}}   \TLLsym{)} \, \TLLnt{t_{{\mathrm{1}}}}  \approx  \TLLsym{[}  \TLLnt{s_{{\mathrm{1}}}}  \TLLsym{/}  \mathit{x}  \TLLsym{]}  \TLLnt{s_{{\mathrm{2}}}}  :  \TLLnt{B} }{%
{\TLLdruleeqXXbetaName}{}%
}}


\newcommand{\TLLdruleeqXXunboxName}[0]{\TLLdrulename{eq\_unbox}}
\newcommand{\TLLdruleeqXXunbox}[1]{\TLLdrule[#1]{%
\TLLpremise{  \Gamma  \vdash  \TLLnt{t_{{\mathrm{1}}}}  \approx  \TLLnt{s_{{\mathrm{1}}}}  :  \Box \, \TLLnt{A_{{\mathrm{1}}}}   \TLLsym{,} \, ... \, \TLLsym{,}   \Gamma  \vdash  \TLLnt{t_{\TLLmv{k}}}  \approx  \TLLnt{s_{\TLLmv{k}}}  :  \Box \, \TLLnt{A_{\TLLmv{k}}}    \quad   \mathit{x_{{\mathrm{1}}}}  \TLLsym{:}  \Box \, \TLLnt{A_{{\mathrm{1}}}}  \TLLsym{,} \, ... \, \TLLsym{,}  \mathit{x_{\TLLmv{k}}}  \TLLsym{:}  \Box \, \TLLnt{A_{\TLLmv{k}}}  \vdash  \TLLnt{t}  \approx  \TLLnt{s}  :  \TLLnt{B} }%
}{
 \Gamma  \vdash  \mathsf{unbox}_\Box\, \, \TLLsym{(}   \mathsf{let}_\Box\, \mathit{x_{{\mathrm{1}}}}  \TLLsym{:}  \Box \, \TLLnt{A_{{\mathrm{1}}}}  \TLLsym{,} \, ... \, \TLLsym{,}  \mathit{x_{\TLLmv{k}}}  \TLLsym{:}  \Box \, \TLLnt{A_{\TLLmv{k}}} \,\mathsf{be}\, \TLLnt{t_{{\mathrm{1}}}}  \TLLsym{,} \, ... \, \TLLsym{,}  \TLLnt{t_{\TLLmv{k}}} \,\mathsf{in}\, \TLLnt{t}   \TLLsym{)}  \approx  \TLLsym{[}  \TLLnt{s_{{\mathrm{1}}}}  \TLLsym{/}  \mathit{x_{{\mathrm{1}}}}  \TLLsym{]} \, ... \, \TLLsym{[}  \TLLnt{s_{\TLLmv{k}}}  \TLLsym{/}  \mathit{x_{\TLLmv{k}}}  \TLLsym{]}  \TLLnt{s}  :  \TLLnt{B} }{%
{\TLLdruleeqXXunboxName}{}%
}}


\newcommand{\TLLdruleeqXXunbboxName}[0]{\TLLdrulename{eq\_unbbox}}
\newcommand{\TLLdruleeqXXunbbox}[1]{\TLLdrule[#1]{%
\TLLpremise{  \Gamma  \vdash  \TLLnt{t_{{\mathrm{1}}}}  \approx  \TLLnt{s_{{\mathrm{1}}}}  :  \blacksquare \, \TLLnt{A_{{\mathrm{1}}}}   \TLLsym{,} \, ... \, \TLLsym{,}   \Gamma  \vdash  \TLLnt{t_{\TLLmv{k}}}  \approx  \TLLnt{s_{\TLLmv{k}}}  :  \blacksquare \, \TLLnt{A_{\TLLmv{k}}}    \quad   \mathit{x_{{\mathrm{1}}}}  \TLLsym{:}  \blacksquare \, \TLLnt{A_{{\mathrm{1}}}}  \TLLsym{,} \, ... \, \TLLsym{,}  \mathit{x_{\TLLmv{k}}}  \TLLsym{:}  \blacksquare \, \TLLnt{A_{\TLLmv{k}}}  \vdash  \TLLnt{t}  \approx  \TLLnt{s}  :  \TLLnt{B} }%
}{
 \Gamma  \vdash  \mathsf{unbox}_\blacksquare\, \, \TLLsym{(}   \mathsf{let}_\blacksquare\, \mathit{x_{{\mathrm{1}}}}  \TLLsym{:}  \blacksquare \, \TLLnt{A_{{\mathrm{1}}}}  \TLLsym{,} \, ... \, \TLLsym{,}  \mathit{x_{\TLLmv{k}}}  \TLLsym{:}  \blacksquare \, \TLLnt{A_{\TLLmv{k}}} \,\mathsf{be}\, \TLLnt{t_{{\mathrm{1}}}}  \TLLsym{,} \, ... \, \TLLsym{,}  \TLLnt{t_{\TLLmv{k}}} \,\mathsf{in}\, \TLLnt{t}   \TLLsym{)}  \approx  \TLLsym{[}  \TLLnt{s_{{\mathrm{1}}}}  \TLLsym{/}  \mathit{x_{{\mathrm{1}}}}  \TLLsym{]} \, ... \, \TLLsym{[}  \TLLnt{s_{\TLLmv{k}}}  \TLLsym{/}  \mathit{x_{\TLLmv{k}}}  \TLLsym{]}  \TLLnt{s}  :  \TLLnt{B} }{%
{\TLLdruleeqXXunbboxName}{}%
}}


\newcommand{\TLLdruleeqXXreflName}[0]{\TLLdrulename{eq\_refl}}
\newcommand{\TLLdruleeqXXrefl}[1]{\TLLdrule[#1]{%
\TLLpremise{\Gamma  \vdash  \TLLnt{t}  \TLLsym{:}  \TLLnt{A}}%
}{
 \Gamma  \vdash  \TLLnt{t}  \approx  \TLLnt{t}  :  \TLLnt{A} }{%
{\TLLdruleeqXXreflName}{}%
}}


\newcommand{\TLLdruleeqXXsymName}[0]{\TLLdrulename{eq\_sym}}
\newcommand{\TLLdruleeqXXsym}[1]{\TLLdrule[#1]{%
\TLLpremise{ \Gamma  \vdash  \TLLnt{t_{{\mathrm{2}}}}  \approx  \TLLnt{t_{{\mathrm{1}}}}  :  \TLLnt{A} }%
}{
 \Gamma  \vdash  \TLLnt{t_{{\mathrm{1}}}}  \approx  \TLLnt{t_{{\mathrm{2}}}}  :  \TLLnt{A} }{%
{\TLLdruleeqXXsymName}{}%
}}


\newcommand{\TLLdruleeqXXtransName}[0]{\TLLdrulename{eq\_trans}}
\newcommand{\TLLdruleeqXXtrans}[1]{\TLLdrule[#1]{%
\TLLpremise{ \Gamma  \vdash  \TLLnt{t_{{\mathrm{1}}}}  \approx  \TLLnt{t_{{\mathrm{2}}}}  :  \TLLnt{A}   \quad   \Gamma  \vdash  \TLLnt{t_{{\mathrm{2}}}}  \approx  \TLLnt{t_{{\mathrm{3}}}}  :  \TLLnt{A} }%
}{
 \Gamma  \vdash  \TLLnt{t_{{\mathrm{1}}}}  \approx  \TLLnt{t_{{\mathrm{3}}}}  :  \TLLnt{A} }{%
{\TLLdruleeqXXtransName}{}%
}}

\newcommand{\TLLdefneq}[1]{\begin{TLLdefnblock}[#1]{$ \Gamma  \vdash  \TLLnt{t_{{\mathrm{1}}}}  \approx  \TLLnt{t_{{\mathrm{2}}}}  :  \TLLnt{A} $}{}
\TLLusedrule{\TLLdruleeqXXbeta{}}
\TLLusedrule{\TLLdruleeqXXunbox{}}
\TLLusedrule{\TLLdruleeqXXunbbox{}}
\TLLusedrule{\TLLdruleeqXXrefl{}}
\TLLusedrule{\TLLdruleeqXXsym{}}
\TLLusedrule{\TLLdruleeqXXtrans{}}
\end{TLLdefnblock}}


\newcommand{\TLLdefnsJtype}{
\TLLdefntype{}\TLLdefneq{}}

\newcommand{\TLLdefnss}{
\TLLdefnsJtype
}

\newcommand{\TLLall}{\TLLmetavars\\[0pt]
\TLLgrammar\\[5.0mm]
\TLLdefnss}


\renewcommand{\TLLdrule}[4][]{{\displaystyle\frac{\begin{array}{l}#2\end{array}}{#3}\quad #4}}
\renewcommand{\TLLdruletyXXaxName}{\hspace{-8px}\text{Id}}
\renewcommand{\TLLdruletyXXbaxName}{\hspace{-8px}\Box\text{Id}}
\renewcommand{\TLLdruletyXXtrueName}{\hspace{-8px}\top_{\mathcal{I}}} 
\renewcommand{\TLLdruletyXXfalseName}{\hspace{-8px}\perp_{\mathcal{E}}}
\renewcommand{\TLLdruletyXXconjIName}{\hspace{-8px}\land_{\mathcal{I}}}
\renewcommand{\TLLdruletyXXconjEOneName}{\hspace{-8px}\land_{\mathcal{E}_1}}
\renewcommand{\TLLdruletyXXconjETwoName}{\hspace{-8px}\land_{\mathcal{E}_2}}
\renewcommand{\TLLdruletyXXdisjIOneName}{\hspace{-8px}\lor_{\mathcal{I}_1}}
\renewcommand{\TLLdruletyXXdisjITwoName}{\hspace{-8px}\lor_{\mathcal{I}_2}}
\renewcommand{\TLLdruletyXXdisjEName}{\hspace{-8px}\lor_{\mathcal{E}}}
\renewcommand{\TLLdruletyXXimpIName}{\hspace{-8px}\to_{\mathcal{I}}}
\renewcommand{\TLLdruletyXXimpEName}{\hspace{-8px}\to_{\mathcal{E}}}
\renewcommand{\TLLdruletyXXboxIName}{\hspace{-8px}\Box_{\mathcal{I}}} 
\renewcommand{\TLLdruletyXXboxEName}{\hspace{-8px}\Box_{\mathcal{E}}} 
\renewcommand{\TLLdruletyXXbdiaIName}{\hspace{-8px}\BDia_{\mathcal{I}}} 
\renewcommand{\TLLdruletyXXbdiaEName}{\hspace{-8px}\BDia_{\mathcal{E}}} 
\renewcommand{\TLLdruletyXXbboxIName}{\hspace{-8px}\BBox_{\mathcal{I}}} 
\renewcommand{\TLLdruletyXXbboxEName}{\hspace{-8px}\BBox_{\mathcal{E}}} 
\renewcommand{\TLLdruletyXXdiaIName}{\hspace{-8px}\Diamond_{\mathcal{I}}} 
\renewcommand{\TLLdruletyXXdiaEName}{\hspace{-8px}\Diamond_{\mathcal{E}}}
\renewcommand{\TLLdrulerXXbetaName}{} 
\renewcommand{\TLLdrulerXXfirstName}{} 
\renewcommand{\TLLdrulerXXsecondName}{} 
\renewcommand{\TLLdrulerXXcaseOneName}{} 
\renewcommand{\TLLdrulerXXcaseTwoName}{} 
\renewcommand{\TLLdrulerXXboxName}{} 
\renewcommand{\TLLdrulerXXbdiaName}{} 
\renewcommand{\TLLdrulerXXbboxName}{} 
\renewcommand{\TLLdrulerXXdiaName}{}
\renewcommand{\TLLdrulerXXboxBoxName}{} 
\renewcommand{\TLLdrulerXXbdiaBdiaName}{} 
\renewcommand{\TLLdrulerXXboxBBoxName}{} 
\renewcommand{\TLLdrulerXXdiadiaName}{}
%% End Ott

\newtheorem{theorem}{Theorem}
\newtheorem{lemma}[theorem]{Lemma}
\newtheorem{example}[theorem]{Example}
\newtheorem{fact}[theorem]{Fact}
\newtheorem{corollary}[theorem]{Corollary}
\newtheorem{definition}[theorem]{Definition}
\newtheorem{remark}[theorem]{Remark}
\newtheorem{proposition}[theorem]{Proposition}
\newtheorem{notn}[theorem]{Notation}
\newtheorem{observation}[theorem]{Observation}

\begin{document}

\maketitle

\section*{Preface}
This paper is dedicated to the memory of Professor Grigori Mints, to whom the first author owes a huge debt. Not only an intellectual debt (most people working with proof theory nowadays owe this one), not only a friendship one (there are plenty of us who owe this), but also a mentoring (and a personal help when I needed it)  one. Grisha,  would not be gratuitously conversational, you could say that his style was `tough love':  work hard, then  he would talk to you. But when you needed him, he was there for you. This work might not be, yet, at the stage that he would approve of, especially given all his work on Dynamic Topological Logic with \cite{kremer2005}, which might be connected, but it is the best that we can do in the time we have, so it  will have to do. 

\section{Motivation}
%{\tt to be added}
Generally speaking, Temporal Logic is any system of rules and
symbolism for representing, and reasoning about propositions qualified
in terms of time.  Temporal logic is also one of the most traditional
kinds of modal logic, introduced by Arthur Prior in the late 1950s,
%, and for which important results were obtained by Hans Kamp. 
but it is also one of the most controversial kinds of modal logic, as
people have different intuitions about time, how to represent it, and
how to reason about it.

There has been a huge amount of work in Modal Logic in the last sixty
years, mainly in classical modal logic. We are mostly interested
in constructive systems, not classical ones. In particular we are interested in a
constructive version of temporal logic that satisfies some well-known
and desirable proof-theoretical properties, but that is also
algebraically and category-theoretically {well-behaved}.

Prior's `Time and Modality' \cite{prior1957} introduced a propositional modal logic
with two temporal connectives (modal operators), $F$ and $P$,
corresponding to ``sometime in the {F}uture" and ``sometime in the {P}ast". This has been called \textbf{tense logic} to distinguish it from other temporal systems.
%Kamp's thesis introduced the binary temporal operators ``since" and ``until" and proved what came to be known as Kamp's theorem. Kamp's theorem shows that all temporal operators are definable in terms of ``since" and ``until" -- provided that theunderlying temporal structure is a continuous linear ordering and provided that the logical basis is classical.

Ewald \cite{ewald1986}  produced a first version of an
intuitionistically based temporal logic system. 
The intuitive reading of the operators is very reasonable:
\begin{itemize}
\item $P$ “It has at some time been the case that” 
\item $F$ “It will at some time be the case that” 
\item $H$ “It has always been the case that” 
\item $G$  “It will always be the case that” 
\end{itemize}
Ewald and most of the researchers that followed his path of
constructivization of tense logic, did so assuming a symmetry between
past and future. This symmetry, as well as the symmetry between
universal and existential quantifiers, both in the past and in the
future, are somewhat at odds with
%is not very characteristic of 
intuitionistic reasoning.

Constructivizing a classical system is alwys prone to proliferation of the system, as is evident when considering the several versions on intuitionistic set theory, for example. In particular the basic constructive modal logic S4 (in Lewis original naming convention) has two main variants.

The first version of an intuitionsitic S4, originally presented by Dag Prawitz in his Natural Deduction book \cite{prawitz1965} does not satisfy the distributivity of the possibility operator $\Diamond$ over the logical disjunction. Prawitz's system satifies neither the binary distribution nor   its nullary form, as given in Figure 1. We call this system $\sf CS4$. This system was investigated from a proof theoretical and categorical perspective in \cite{bierman2000}.

\begin{figure}
  \begin{mdframed}
  \begin{center}
      \begin{math}
        \begin{array}{ccc}
        \Diamond (A \lor B) &\to &\Diamond A \lor \Diamond B\\
        \Diamond \bot &\to &\bot
        \end{array}
       \end{math}
\end{center}
 \end{mdframed}
  \caption{Distributivity rules}
  \label{distrib}
\end{figure}


The second main version of an intuitionistic modal S4 does enforce these distributivities and it was thoroughly investigated by A. Simpson in \cite{simpson1994}. This system is part of a framework for constructive modal logics, based on incorporating, as part of the syntax, the intended semantics of modal logics, as possible worlds. We call this system $\sf IS4$. 

Ewald's tense logic system consists of a pair of Simpson-style S4 operators \cite{simpson1994},
representing past and future over intuituionistic propositional logic. This is historically inaccurate, as Simpson based his systems in Ewald's, but it will serve to make some of our main points clearer below. The system we describe in this note is the tense logic system obtained by joining together two pairs of Prawitz-style S4 operators. 

Simpson remarks that intuitionistic or constructive modal logic is full of interesting questions. As he says:
\begin{quote}
Although much work has been done in the field, there is as yet no consensus on the correct viewpoint for considering intuitionistic modal logic.  In particular, there is no single semantic framework rivalling that of possible world semantics for classical modal logic. Indeed, there is not even any general agreement on what the \textbf{intuitionistic analogue} of the basic modal logic, K, is.
\end{quote}
In an intuitionistic logic we do not expect perfect duality between quantifiers, ($\forall x.P(x)$ is not the same as $\neg \exists x.\neg
P(x)$) or even between conjunction and disjunction (De Morgan laws do not hold for intuitionistic propositional logic). So one should not expect a perfect duality between possibility and necessity either. But just considerations from first principles do not seem to clearly indicate whether distributivity rules as the ones in Figure \ref{distrib} should hold or not. Hence it seems sensible to develop different kinds of systems in parallel, proving equivalences, whenever possible. In this paper we develop the idea of tense logic in the Prawitz style. We recall some deductive systems for this tense logic and provide categorical semantics for it.


Much has been done recently in the proof theory of constructive modal logics using more informative sequent systems (e.g. hypersequents, labelled sequents, nested sequents, tree-style sequents, etc..) In particular nested sequents have been used to produce `modal cubes' for the two variants of constructive modal logics described above. See the pictures below from \cite{arisaka2015,strasburger2013}.

\begin{figure}[h!]
\centering
\includegraphics[width=1.5in]{intmodalcube.pdf}
\includegraphics[width=1.6in]{constructivemodalcube.pdf}
%\caption{Modal Cubes}
\label{fig:modalcube}
\end{figure}

Sequent calculi  by themselves are not enough to provide us with Curry-Howard correspondences and/or term assignments for these systems. However, using the Prawitz S4 version of these modal systems we can easily produce a Curry-Howard correspondence and a categorical model for the Prawitz-style intuitionistic tense logic, our goal in this paper. 

We start by recalling the system using axioms, plain sequent calculus and plain natural deduction. In the next section we described a term assignment based on the dual calculus described in \cite{icalp1998} and show some of its syntactic prooperties. The next section  introduces the categorical model (a cartesian closed category with two intertwinned adjunctions) and show the usual soundness and completeness results. Finally we discuss potential applications and  limitations of our constructive tense logic.

\section{Tense Logic CS4-style}
We build up to the constructive tense logic we are interested in $\sf{TCS4}$ in progressive steps. We start with the intuitionistic basis $\sf{LJ}$, add the modalities to get the constructive S4 system, $\sf{CS4}$, provide the dual context modification (to help with the reuse of libraries, amongst other things), obtaining dual $\sf{CS4}$, $\sf{DCS4}$ and then finally consider the two adjunctions to obtain the tense constructive system $\sf{TCS4}$.

%\subsection{Axioms}

\subsection{Intuitionistic Sequent Calculus}
We start by recalling the basic sequent calculus for intuitionistic propositional logic, 
%Each of the logics presented in this section is an extension of the single-conclusion formalization of 
Gentzen's intuitionistic sequent calculus LJ.  
The syntax of formulas for LJ is defined by the
following grammar:
\begin{center}
    \begin{math}
        \begin{array}{lllllllll}
            A & ::= & p \mid \perp \mid A \land A \mid A \lor A \mid A \to B
        \end{array}
    \end{math}
\end{center}
The formula $p$ is taken from a set of countably many  propositional atoms. The constant $\top$ could be added, but it is the negation of the the falsum constant $\bot$.
%We begin with an initial set of inference rules, and then each system presented will be given as an extension of this initial system.  
The  initial inference rules, which just model
propositional intuitionistic logic, are as in 
Figure~\ref{fig:LJ}.

\begin{figure}
  \begin{mdframed}
\begin{center}
  \small
  \begin{math}
    \begin{array}{c}
      \begin{array}{cccccccc}
        \infer[\text{Id}]{\Delta,A \vdash A}{
          \,
        }
        & \quad &
        \infer[\text{Cut}]{\Gamma,\Delta \vdash C}{
          \Gamma \vdash B
          &
          B,\Delta \vdash C
        }
        & \quad & 
        \infer[\perp_{\mathcal{L}}]{\Gamma,\perp \; \vdash A}{
          \,
        }\\\\
        \infer[\lor_{\mathcal{L}}]{\Gamma,A \lor B \vdash C}{
          \Gamma,A \vdash C
          &
          \Gamma,B \vdash C
        }
        & \quad &
        \infer[\lor_{\mathcal{R}_1}]{\Gamma \vdash A \lor B}{
          \Gamma \vdash A
        }
        & \quad &
        \infer[\lor_{\mathcal{R}_2}]{\Gamma \vdash A \lor B}{
          \Gamma \vdash B
        }\\\\
        \infer[\land_{\mathcal{L}_1}]{\Gamma,A \land B \vdash C}{
          \Gamma,A \vdash C
        }
        & \quad &
        \infer[\land_{\mathcal{L}_2}]{\Gamma,A \land B \vdash C}{
          \Gamma,B \vdash C
        }
        & \quad &
        \infer[\land_{\mathcal{R}}]{\Gamma \vdash A \land B}{
          \Gamma \vdash A
          &
          \Gamma \vdash B
        }\\\\
        
      \end{array}
      \\
      \begin{array}{cccccccc}
        \infer[\to_{\mathcal{L}}]{\Gamma,A \to B \vdash C}{
          \Gamma \vdash A
          &
          \Gamma,B \vdash C
        }
        & \quad &
        \infer[\to_{\mathcal{R}}]{\Gamma \vdash A \to B}{
          \Gamma, A \vdash B
        }
      \end{array}        
    \end{array}
  \end{math}
\end{center}
 \end{mdframed}
  \caption{Intuitionistic Propositional  Calculus ({\sf{LJ}})}
  \label{fig:LJ}
\end{figure}

Sequents denoted $\Gamma \vdash C$ consist of a multiset of formulas, (written as either $\Gamma$, $\Delta$, or a numbered version of either), and a formula $C$. The intuitive meaning is that the conjunction of the formulas in $\Gamma$ entails the formula $C$. So far this is our intuitionistic basis.

\subsection{Constructive modal \sf{S4}} 
Next we recall the sequent calculus formalization of system CS4.  
%This formalization  was proposed by Bierman and de Paiva \cite{CS4} and initially  was called  IS4 (for Intuitionistic system S4). This system gives rise to a complete family of systems, which are alternative to Simpson's intuitionistic systems~\cite{simpson1994}.
%Simpson proposed in his thesis ~\cite{simpson1994}systems  of intuitonsitic modal logics, forming a framework starting from intuitionistic K. These  were also calledIK, IKT, IS4, IS5, etc, hence it made sense to call the family of systems originating with the S4 above, the family of constructive modal systems CS, and Bierman and de Paiva's system CS4.  This is for instance how new work, such as \cite{arisaka2015}, describes these systems.
%The system CS4 does not satisfy thesetraditional distributions that are basic to classical modal logic. Some philosophical significance can be attached to thisdistribution, or lack thereof, and we discuss it later. First wedefine the system.


\begin{figure}
  \begin{mdframed}
    \begin{center}
      \begin{math}
        \begin{array}{ccccc}              
          \infer[\Box_{\mathcal{L}}]{\Gamma, \Box A \vdash B}{
            \Gamma,A \vdash B
          }
          & \quad &
          \infer[\Box_{\mathcal{R}}]{\Box\Gamma,\Delta \vdash \Box A}{
            \Box \Gamma \vdash A
          }\\\\
          \infer[\Diamond_{\mathcal{L}}]{\Delta,\Box\Gamma,\Diamond A \vdash \Diamond B}{
            \Box\Gamma,A \vdash \Diamond B
          }
          & \quad &
          \infer[\Diamond_{\mathcal{R}}]{\Gamma \vdash \Diamond A}{
            \Gamma \vdash A
          }
        \end{array}        
      \end{math}
    \end{center}
  \end{mdframed}
  \caption{Constructive S4 modal rules ({\sf{CS4}})}
  \label{fig:CS4}
\end{figure}

  We recap the modality rules in Figure~\ref{fig:CS4}. These, in addition to the initial set of inference rules, define the sequent calculus for CS4. In Figure 3, we write $\Box \Gamma$ for the sequence of boxed formulas $\Box G_1, \Box G_2, \ldots, \Box G_k$ where $\Gamma$ is the set $G_1, G_2, \ldots,  G_k$.

Note that we do have right rules and left  rules for introducing the new modal operators $\Box$ (necessity) and $\Diamond$ (posssibility), but these rules are not as symmetric as the propositional ones. Most importantly, we have a local restriciton on the rule that introduces the $\Box$ operator: We can only introduce $\Box$ in the conclusion, if all the assumptions are already boxed. Also the rules for  the $\Diamond$ operator presuppose that you have already $\Box$ operators.
This system is indeed constructive, $\Box$ and $\Diamond$ are independent logical operators and  $\Box A$ is not logically equivalent to $\lnot \Diamond \lnot A$, nor is $\Diamond A$ logically equivalent to $\lnot \Box \lnot A$.


This system has a nice proof theory.
%, as far as modal logics are concerned. 
Bierman and de Paiva \cite{bierman2000} show that it has a Hilbert-style presentation,  a Natural Deduction presentation, as well as a sequent calculus presentation and these are proved equivalent, in the sense of proving the same theorems. The sequent calculus satisfies cut-elimination, an old result from Ohnishi and Matsumoto~\cite{ohnishi1957}, as well as the subformula property.
The Natural Deduction formulation has a colourful history: one of its distinct features is that it was described in Prawitz' seminal book in Natural Deduction \cite{prawitz1965},
hence it is  sometimes called Prawitz S4 modal logic.
Most interestingly the system has both Kripke and categorical
semantics, described respectively in \cite{alechinaetal} and
\cite{bierman2000} as well as an independent mathematical semantics in terms of simplicial sets, described by Goubault-Larrecq\cite{goubault-larrecq}. 
%We can prove a Curry-Howard correspondence for this system.
%, it has been used in several applications within programming
%language theory \todo{Add references for these applications}.
%Examples include \cite{hmm}.


\subsection{The dual context modal S4 calculus}
An equivalent (in terms of provability) but more type-theoretic system can be produced for the modal logic CS4. This is not so well-known, but this system can  be given a presentation in terms of a categorical adjunction, between two cartesian closed categories, as we will describe in the next section. This categorical presentation has been described  both in \cite{bierman2000} and in \cite{icalp1998}, in the former, this is called the \textbf{dual context} formulation of CS4 and the
rules are given  in Figure~\ref{fig:ADJCS4}. Note that the rules are Natural Deduction rules, as it should be clear from the fact that they are introduction and elimination rules.

\begin{figure}
  \begin{mdframed}
    \begin{center}
      \begin{math}
        \begin{array}{ccccc}              
          \infer[\Box_{\mathcal{I}}]{\Gamma; \Delta \vdash \Box A}{
            \Gamma;\emptyset \vdash  A
          }
          & \quad &
          \infer[\Box_{\mathcal{E}}]{\Gamma;\Delta \vdash B}{\Gamma; \Delta \vdash \Box A \hspace{.1in}
            \Gamma, A;\Delta \vdash B
          }\\\\
          \infer[\Diamond_{\mathcal{I}}]{\Gamma;\Delta \vdash \Diamond A}{
            \Gamma;\Delta \vdash A
          }
          & \quad &
          \infer[\Diamond_{\mathcal{E}}]{\Gamma;\Delta \vdash \Diamond B}{
            \Gamma ;\Delta \vdash \Diamond A \hspace{.1in}\Gamma; A\vdash \Diamond B
          }
        \end{array}        
      \end{math}
    \end{center}
  \end{mdframed}
  \caption{The dual context modal calculus ({\sf{DCS4}})}
  \label{fig:DCS4}
\end{figure}

The main difference in the dual context formulation of CS4 is the fact that the context now has modal formulas and non-necessarily modal ones, separated by a semi-colon as in $\Gamma ; \Delta$. The difficult rule of $\Box$ introduction now insists that to introduce a necessity operator $\Box$ on a conclusion, we need to have  an empty context of non-modal assumptions. This corresponds to the traditional idea that to prove something is necessarily the case, all its assumptions have to be also necessary (or it has no assumptions whatsoever).

These rules have been shown by Benton \cite{benton1995} and Barber \cite{barber1997} to correspond to an adjunction of the categories, in the case where the basis is Linear Logic and the modalities correspond to the exponentials.  

We  describe the simplifications that can be made to models when, instead of Linear Logic, we deal with constructive modal logic and the adjunction is between functors $\Box\vdash \Diamond$.

\subsection{The tense CS4 calculus}
Finally to get to the tense logic that is the main aim of this note, we need two such adjunctions, but intertwined. Thus $\blacksquare$ is left-adjoint to $\Diamond$ and $\Box$ is left-adjoint to $\blacklozenge$, where we are writing $\blacksquare$ for the operator we called past universal $H$ before and $\Box$ for the future necessity $G$.

\begin{figure}
  \begin{mdframed}
    \begin{center}
      \begin{math}
        \begin{array}{l}
          \begin{array}{ccccc}              
            \infer[\Box_{\mathcal{I}}]{\Gamma; \Delta \vdash \Box A}{
              \Gamma;\emptyset \vdash  A
            }
            & \quad &
            \infer[\Box_{\mathcal{E}}]{\Gamma;\Delta \vdash B}{\Gamma; \Delta \vdash \Box A \hspace{.1in}
              \Gamma, A;\Delta \vdash B
            }\\\\
            \infer[\bLozenge_{\mathcal{I}}]{\Gamma;\Delta \vdash \bLozenge A}{
              \Gamma;\Delta \vdash A
            }
            & \quad &
            \infer[\bLozenge_{\mathcal{E}}]{\Gamma;\Delta \vdash \bLozenge B}{
              \Gamma ;\Delta \vdash \bLozenge A \hspace{.1in}\Gamma; A\vdash \bLozenge B
            }
          \end{array}
          \\\\
          \begin{array}{ccccc}              
            \infer[\blacksquare_{\mathcal{I}}]{\Gamma; \Delta \vdash \blacksquare A}{
              \Gamma;\emptyset \vdash  A
            }
            & \quad &
            \infer[\blacksquare_{\mathcal{E}}]{\Gamma;\Delta \vdash B}{\Gamma; \Delta \vdash \blacksquare A \hspace{.1in}
              \Gamma, A;\Delta \vdash B
            }\\\\
            \infer[\Diamond_{\mathcal{I}}]{\Gamma;\Delta \vdash \Diamond A}{
              \Gamma;\Delta \vdash A
            }
            & \quad &
            \infer[\Diamond_{\mathcal{E}}]{\Gamma;\Delta \vdash \Diamond B}{
              \Gamma ;\Delta \vdash \Diamond A \hspace{.1in}\Gamma; A\vdash \Diamond B
            }
          \end{array}    
        \end{array}
      \end{math}
    \end{center}
  \end{mdframed}
  \caption{The tense CS4 calculus ({\sf{TCS4}})}
  \label{fig:ADJCS4}
\end{figure}

\subsection{Axioms}
Axiom sets for the system $\sf{TCS4}$ are easier to provide. We need a set for the basic system intuitionistic logic $\sf{LJ}$,  any traditional set would do, plus the axioms for modalities, as well as the rules \textbf{modus ponens} and \textbf{necessitation} for the two necessity operators, that is:

\begin{figure}
  \begin{mdframed}
  \begin{center}
      \begin{math}
        \begin{array}{ccc}
        \Box (A \to B) &\to &(\Box A \to \Box B)\\
        \Box (A \to B) &\to &(\bLozenge A\to \bLozenge B)\\
        (\Box A\to A) &\land &(A\to \bLozenge A)\\
        (\Box A\to \Box\Box A) &\land &(\bLozenge \bLozenge A \to \bLozenge A)\\
        \blacksquare (A \to B) &\to &(\blacksquare A \to \blacksquare B)\\
        \blacksquare (A \to B) &\to &(\Diamond A\to \Diamond B)\\
        (\blacksquare A\to A) &\land &(A\to \Diamond A)\\
        (\blacksquare A\to \blacksquare\blacksquare A) &\land &(\Diamond \Diamond A \to \Diamond A)
        \end{array}
       \end{math}
\end{center}
 \end{mdframed}
  \caption{Modal Axioms for $\sf{TCS4}$}
  \label{axiomCS4}
\end{figure}

\section{Term Assignments}
\label{sec:term_assignments}

\begin{figure}
  \begin{mdframed}    
    \begin{mathpar}
      \TLLdruletyXXax{} \and
      \TLLdruletyXXfalse{} \and
      \TLLdruletyXXimpI{} \and
      \TLLdruletyXXimpE{} \and
      \TLLdruletyXXboxE{} \and
      \TLLdruletyXXboxI{} \and
      \TLLdruletyXXbboxE{} \and
      \TLLdruletyXXbboxI{} \and
      \TLLdruletyXXdiaI{} \and
      \TLLdruletyXXbdiaI{} \and
      \TLLdruletyXXadjL{}  \and
      \TLLdruletyXXadjR{} \and      
      \TLLdruletyXXdiaE{} \and      
      \TLLdruletyXXbdiaE{}       
    \end{mathpar}
  \end{mdframed}
  \caption{TCS4 Typing Rules}
  \label{fig:TCS4-typing-rules}
\end{figure}


% section term_assignments (end)

\section{The Categorical Model}
There is not much essentially new in what we discuss here about the tense logic based on CS4. Similar ideas were discussed by  Ghilardi and Meloni \cite{ghilardi1988}, Makkai and Reyes \cite{makkai1995} and more recently in by Dzik et al \cite{dziketal2012} and Menni and Smith \cite{Menni:2014}.

The upshot of our discussion is that the categorical model we advance is a cartesian closed category endowed with two adjunctions, corresponding to the (limited) universal and existential quantifications relative to the past and to the future that correspond to the necessity and possibility operators, relative to the past and the future.

This setting is though  different enough from the precursors we know about, to justify this note. First, as discussed elsewhere \cite{bierman2000}, we see no reason for the monads/comonads emerging from this setting to be \textbf{idempotent} operators, as they are in \cite{ghilardi1988}. Secondly we see no reason to take our models as part of toposes, as we are not  interested in extra structure provided by toposes. 
However, we also see no reason to confine ourselves to algebraic models such as Heyting algebras with operators, as degenerate posetal categories, as both \cite{dziketal2012} and  \cite{Menni:2014} do. Different proofs of the same theorem are important to us as they correspond to different morphisms in the categories between the same objects.

On the other hand, we are interested in the term assignment systems and its properties, as our aims are to use these as type systems for innovative programming languages. So we need to provide the systematic work that shows basic properties of the type system $\sf{TCS4}$ we are interested in.

%\section{Applications}

\section{Conclusions}


\bibliographystyle{plain}
\bibliography{references}

\appendix

\section{Appendix}
\label{sec:appendix}

\subsection{Proofs}
\label{subsec:proofs}
\subsubsection{Proof of Substitution for Typing}
\label{sec:proof_of_lemma:substitution_for_typing}

\begin{lemma*}[Substitution for Typing]
  If $\Gamma_{{\mathrm{1}}}  \vdash  \TLLnt{t_{{\mathrm{1}}}}  \TLLsym{:}  \TLLnt{A}$, and $\Gamma_{{\mathrm{1}}}  \TLLsym{,}  \mathit{x}  \TLLsym{:}  \TLLnt{A}  \TLLsym{,}  \Gamma_{{\mathrm{2}}}  \vdash  \TLLnt{t_{{\mathrm{2}}}}  \TLLsym{:}  \TLLnt{B}$, then $\Gamma_{{\mathrm{1}}}  \TLLsym{,}  \Gamma_{{\mathrm{2}}}  \vdash  \TLLsym{[}  \TLLnt{t_{{\mathrm{1}}}}  \TLLsym{/}  \mathit{x}  \TLLsym{]}  \TLLnt{t_{{\mathrm{2}}}}  \TLLsym{:}  \TLLnt{B}$.
\end{lemma*}

\begin{proof}
  Suppose $\Gamma_{{\mathrm{1}}}  \vdash  \TLLnt{t_{{\mathrm{1}}}}  \TLLsym{:}  \TLLnt{A}$ and $\Gamma_{{\mathrm{1}}}  \TLLsym{,}  \mathit{x}  \TLLsym{:}  \TLLnt{A}  \TLLsym{,}  \Gamma_{{\mathrm{2}}}  \vdash  \TLLnt{t_{{\mathrm{2}}}}  \TLLsym{:}  \TLLnt{B}$.  We
  case split on the structure of the latter, but only show the
  non-trivial cases.  All other cases are similar.
\begin{itemize}
\item[] Case Identity.\\ 
  \begin{center}
    \begin{math}
      $$\mprset{flushleft}
      \inferrule* [right=$\text{Id}$] {
        \ 
      }{\Gamma_{{\mathrm{1}}}  \TLLsym{,}  \mathit{x}  \TLLsym{:}  \TLLnt{A}  \TLLsym{,}  \Gamma_{{\mathrm{2}}}  \vdash  \mathit{y}  \TLLsym{:}  \TLLnt{C}}
    \end{math}
  \end{center}
  In this case $\TLLnt{t_{{\mathrm{2}}}} = \mathit{y}$ and $\TLLnt{B} = \TLLnt{C}$.  We are not
  sure if $\mathit{x} = \mathit{y}$, thus, we must consider the case when they
  are and are not equal.

  Suppose $\mathit{x} \neq \mathit{y}$.  Then $\TLLsym{[}  \TLLnt{t_{{\mathrm{1}}}}  \TLLsym{/}  \mathit{x}  \TLLsym{]}  \TLLnt{t_{{\mathrm{2}}}} = \TLLsym{[}  \TLLnt{t_{{\mathrm{1}}}}  \TLLsym{/}  \mathit{x}  \TLLsym{]}  \mathit{y} =
  \mathit{y}$ by the definition of substitution.  In addition, it must be
  the case that either $\mathit{y}  \TLLsym{:}  \TLLnt{C} \in \Gamma_{{\mathrm{1}}}$ or $\mathit{y}  \TLLsym{:}  \TLLnt{C} \in
  \Gamma_{{\mathrm{2}}}$.  This implies that $\Gamma_{{\mathrm{1}}}  \TLLsym{,}  \Gamma_{{\mathrm{2}}}  \vdash  \mathit{y}  \TLLsym{:}  \TLLnt{C}$ or $\Gamma_{{\mathrm{1}}}  \TLLsym{,}  \Gamma_{{\mathrm{2}}}  \vdash  \TLLsym{[}  \TLLnt{t_{{\mathrm{1}}}}  \TLLsym{/}  \mathit{x}  \TLLsym{]}  \TLLnt{t_{{\mathrm{2}}}}  \TLLsym{:}  \TLLnt{B}$ hold.

  Now suppose $\mathit{x} = \mathit{y}$.  Then $\TLLnt{A} = \TLLnt{B}$, and $\TLLsym{[}  \TLLnt{t_{{\mathrm{1}}}}  \TLLsym{/}  \mathit{x}  \TLLsym{]}  \TLLnt{t_{{\mathrm{2}}}} = \TLLsym{[}  \TLLnt{t_{{\mathrm{1}}}}  \TLLsym{/}  \mathit{x}  \TLLsym{]}  \mathit{x} = \TLLnt{t_{{\mathrm{1}}}}$ by the definition of
  substitution.  Thus, $\Gamma_{{\mathrm{1}}}  \TLLsym{,}  \Gamma_{{\mathrm{2}}}  \vdash  \TLLsym{[}  \TLLnt{t_{{\mathrm{1}}}}  \TLLsym{/}  \mathit{x}  \TLLsym{]}  \TLLnt{t_{{\mathrm{2}}}}  \TLLsym{:}  \TLLnt{B}$ holds, because we
  know $\Gamma_{{\mathrm{1}}}  \TLLsym{,}  \Gamma_{{\mathrm{2}}}  \vdash  \TLLnt{t_{{\mathrm{1}}}}  \TLLsym{:}  \TLLnt{A}$.

\item[] Case Implication Introduction.\\ 
  \begin{center}
    \begin{math}
      $$\mprset{flushleft}
      \inferrule* [right=$\TLLdruletyXXimpIName$] {
        \Gamma_{{\mathrm{1}}}  \TLLsym{,}  \mathit{x}  \TLLsym{:}  \TLLnt{A}  \TLLsym{,}  \Gamma_{{\mathrm{2}}}  \TLLsym{,}  \mathit{y}  \TLLsym{:}  \TLLnt{C_{{\mathrm{1}}}}  \vdash  \TLLnt{t}  \TLLsym{:}  \TLLnt{C_{{\mathrm{2}}}}
      }{\Gamma_{{\mathrm{1}}}  \TLLsym{,}  \mathit{x}  \TLLsym{:}  \TLLnt{A}  \TLLsym{,}  \Gamma_{{\mathrm{2}}}  \vdash   \lambda  \mathit{y}  :  \TLLnt{C_{{\mathrm{1}}}} . \TLLnt{t}   \TLLsym{:}  \TLLnt{C_{{\mathrm{1}}}}  \to  \TLLnt{C_{{\mathrm{2}}}}}
    \end{math}
  \end{center}
  In this case $\TLLnt{B} = \TLLnt{C_{{\mathrm{1}}}}  \to  \TLLnt{C_{{\mathrm{2}}}}$ and $\TLLnt{t_{{\mathrm{2}}}} =  \lambda  \mathit{y}  :  \TLLnt{C} . \TLLnt{t} $.  By the induction hypothesis
  we know $\Gamma_{{\mathrm{1}}}  \TLLsym{,}  \Gamma_{{\mathrm{2}}}  \TLLsym{,}  \mathit{y}  \TLLsym{:}  \TLLnt{C_{{\mathrm{1}}}}  \vdash  \TLLsym{[}  \TLLnt{t_{{\mathrm{1}}}}  \TLLsym{/}  \mathit{x}  \TLLsym{]}  \TLLnt{t}  \TLLsym{:}  \TLLnt{C_{{\mathrm{2}}}}$, and then by reapplying the rule we know
  $\Gamma_{{\mathrm{1}}}  \TLLsym{,}  \Gamma_{{\mathrm{2}}}  \vdash   \lambda  \mathit{y}  :  \TLLnt{C_{{\mathrm{1}}}} . \TLLsym{[}  \TLLnt{t_{{\mathrm{1}}}}  \TLLsym{/}  \mathit{x}  \TLLsym{]}  \TLLnt{t}   \TLLsym{:}  \TLLnt{C_{{\mathrm{2}}}}$ holds.  However, by the definition of substitution
  we know $ \lambda  \mathit{y}  :  \TLLnt{C_{{\mathrm{1}}}} . \TLLsym{[}  \TLLnt{t_{{\mathrm{1}}}}  \TLLsym{/}  \mathit{x}  \TLLsym{]}  \TLLnt{t}  = \TLLsym{[}  \TLLnt{t_{{\mathrm{1}}}}  \TLLsym{/}  \mathit{x}  \TLLsym{]}  \TLLsym{(}   \lambda  \mathit{y}  :  \TLLnt{C_{{\mathrm{1}}}} . \TLLnt{t}   \TLLsym{)}$, and thus, we obtain our result.

\item[] Case Implication Elimination.\\ 
  \begin{center}
    \begin{math}
      $$\mprset{flushleft}
      \inferrule* [right=$\TLLdruletyXXimpEName$] {
        \Gamma_{{\mathrm{1}}}  \TLLsym{,}  \mathit{x}  \TLLsym{:}  \TLLnt{A}  \TLLsym{,}  \Gamma_{{\mathrm{2}}}  \vdash  \TLLnt{t'_{{\mathrm{1}}}}  \TLLsym{:}  \TLLnt{C_{{\mathrm{1}}}}  \to  \TLLnt{C_{{\mathrm{2}}}}  \quad  \Gamma_{{\mathrm{1}}}  \TLLsym{,}  \mathit{x}  \TLLsym{:}  \TLLnt{A}  \TLLsym{,}  \Gamma_{{\mathrm{2}}}  \vdash  \TLLnt{t'_{{\mathrm{2}}}}  \TLLsym{:}  \TLLnt{C_{{\mathrm{1}}}}
      }{\Gamma_{{\mathrm{1}}}  \TLLsym{,}  \mathit{x}  \TLLsym{:}  \TLLnt{A}  \TLLsym{,}  \Gamma_{{\mathrm{2}}}  \vdash  \TLLnt{t'_{{\mathrm{1}}}} \, \TLLnt{t'_{{\mathrm{2}}}}  \TLLsym{:}  \TLLnt{C_{{\mathrm{2}}}}}
    \end{math}
  \end{center}
  We now have that $\TLLnt{B} = \TLLnt{C_{{\mathrm{2}}}}$ and $\TLLnt{t_{{\mathrm{2}}}} = \TLLnt{t'_{{\mathrm{1}}}} \, \TLLnt{t'_{{\mathrm{2}}}}$.  By the induction hypothesis
  we know that $\Gamma_{{\mathrm{1}}}  \TLLsym{,}  \Gamma_{{\mathrm{2}}}  \vdash  \TLLsym{[}  \TLLnt{t_{{\mathrm{1}}}}  \TLLsym{/}  \mathit{x}  \TLLsym{]}  \TLLnt{t'_{{\mathrm{1}}}}  \TLLsym{:}  \TLLnt{C_{{\mathrm{1}}}}  \to  \TLLnt{C_{{\mathrm{2}}}}$ and $\Gamma_{{\mathrm{1}}}  \TLLsym{,}  \Gamma_{{\mathrm{2}}}  \vdash  \TLLsym{[}  \TLLnt{t_{{\mathrm{1}}}}  \TLLsym{/}  \mathit{x}  \TLLsym{]}  \TLLnt{t'_{{\mathrm{2}}}}  \TLLsym{:}  \TLLnt{C_{{\mathrm{1}}}}$ both hold.
  Then by reapplying the rule we obtain that $\Gamma_{{\mathrm{1}}}  \TLLsym{,}  \Gamma_{{\mathrm{2}}}  \vdash  \TLLsym{(}  \TLLsym{[}  \TLLnt{t_{{\mathrm{1}}}}  \TLLsym{/}  \mathit{x}  \TLLsym{]}  \TLLnt{t'_{{\mathrm{1}}}}  \TLLsym{)} \, \TLLsym{(}  \TLLsym{[}  \TLLnt{t_{{\mathrm{1}}}}  \TLLsym{/}  \mathit{x}  \TLLsym{]}  \TLLnt{t'_{{\mathrm{2}}}}  \TLLsym{)}  \TLLsym{:}  \TLLnt{C_{{\mathrm{2}}}}$, and thus,
  by the definition of substitution $\Gamma_{{\mathrm{1}}}  \TLLsym{,}  \Gamma_{{\mathrm{2}}}  \vdash  \TLLsym{[}  \TLLnt{t_{{\mathrm{1}}}}  \TLLsym{/}  \mathit{x}  \TLLsym{]}  \TLLsym{(}  \TLLnt{t'_{{\mathrm{1}}}} \, \TLLnt{t'_{{\mathrm{2}}}}  \TLLsym{)}  \TLLsym{:}  \TLLnt{C_{{\mathrm{2}}}}$ holds.
  
  
\item[] Case $\Box$ Introduction.\\ 
  \begin{center}
    \scriptsize
    \begin{math}
      $$\mprset{flushleft}
      \inferrule* [right=$\TLLdruletyXXboxIName$] {
         \Gamma_{{\mathrm{1}}}  \TLLsym{,}  \mathit{x}  \TLLsym{:}  \TLLnt{A}  \TLLsym{,}  \Gamma_{{\mathrm{2}}}  \vdash  \TLLnt{t'_{{\mathrm{1}}}}  \TLLsym{:}  \Box \, \TLLnt{C_{{\mathrm{1}}}}  \TLLsym{,} \, ... \, \TLLsym{,}  \Gamma_{{\mathrm{1}}}  \TLLsym{,}  \mathit{x}  \TLLsym{:}  \TLLnt{A}  \TLLsym{,}  \Gamma_{{\mathrm{2}}}  \vdash  \TLLnt{t'_{\TLLmv{k}}}  \TLLsym{:}  \Box \, \TLLnt{C_{\TLLmv{k}}}   \quad  \mathit{x_{{\mathrm{1}}}}  \TLLsym{:}  \Box \, \TLLnt{C_{{\mathrm{1}}}}  \TLLsym{,} \, ... \, \TLLsym{,}  \mathit{x_{\TLLmv{k}}}  \TLLsym{:}  \Box \, \TLLnt{C_{\TLLmv{k}}}  \vdash  \TLLnt{t}  \TLLsym{:}  \TLLnt{C}
      }{\Gamma_{{\mathrm{1}}}  \TLLsym{,}  \mathit{x}  \TLLsym{:}  \TLLnt{A}  \TLLsym{,}  \Gamma_{{\mathrm{2}}}  \vdash   \mathsf{let}_\Box\, \mathit{x_{{\mathrm{1}}}}  \TLLsym{:}  \Box \, \TLLnt{C_{{\mathrm{1}}}}  \TLLsym{,} \, ... \, \TLLsym{,}  \mathit{x_{\TLLmv{k}}}  \TLLsym{:}  \Box \, \TLLnt{C_{\TLLmv{k}}} \,\mathsf{be}\, \TLLnt{t'_{{\mathrm{1}}}}  \TLLsym{,} \, ... \, \TLLsym{,}  \TLLnt{t'_{\TLLmv{k}}} \,\mathsf{in}\, \TLLnt{t}   \TLLsym{:}  \Box \, \TLLnt{C}}
    \end{math}
  \end{center}
  In this case $\TLLnt{B} = \Box \, \TLLnt{C}$ and
  $\TLLnt{t_{{\mathrm{2}}}} =  \mathsf{let}_\Box\, \mathit{x_{{\mathrm{1}}}}  \TLLsym{:}  \Box \, \TLLnt{C_{{\mathrm{1}}}}  \TLLsym{,} \, ... \, \TLLsym{,}  \mathit{x_{\TLLmv{k}}}  \TLLsym{:}  \Box \, \TLLnt{C_{\TLLmv{k}}} \,\mathsf{be}\, \TLLnt{t'_{{\mathrm{1}}}}  \TLLsym{,} \, ... \, \TLLsym{,}  \TLLnt{t'_{\TLLmv{k}}} \,\mathsf{in}\, \TLLnt{t} $.  By the induction hypothesis
  we know that 
  \[ \Gamma_{{\mathrm{1}}}  \TLLsym{,}  \Gamma_{{\mathrm{2}}}  \vdash  \TLLsym{[}  \TLLnt{t_{{\mathrm{1}}}}  \TLLsym{/}  \mathit{x}  \TLLsym{]}  \TLLnt{t'_{{\mathrm{1}}}}  \TLLsym{:}  \Box \, \TLLnt{C_{{\mathrm{1}}}} , \ldots , \Gamma_{{\mathrm{1}}}  \TLLsym{,}  \mathit{x}  \TLLsym{:}  \TLLnt{A}  \TLLsym{,}  \Gamma_{{\mathrm{2}}}  \vdash  \TLLsym{[}  \TLLnt{t_{{\mathrm{1}}}}  \TLLsym{/}  \mathit{x}  \TLLsym{]}  \TLLnt{t'_{\TLLmv{k}}}  \TLLsym{:}  \Box \, \TLLnt{C_{\TLLmv{k}}} \] all hold. Then by reapplying
  the rule we know that
  \[ \Gamma_{{\mathrm{1}}}  \TLLsym{,}  \Gamma_{{\mathrm{2}}}  \vdash   \mathsf{let}_\Box\, \mathit{x_{{\mathrm{1}}}}  \TLLsym{:}  \Box \, \TLLnt{C_{{\mathrm{1}}}}  \TLLsym{,} \, ... \, \TLLsym{,}  \mathit{x_{\TLLmv{k}}}  \TLLsym{:}  \Box \, \TLLnt{C_{\TLLmv{k}}} \,\mathsf{be}\, \TLLsym{[}  \TLLnt{t_{{\mathrm{1}}}}  \TLLsym{/}  \mathit{x}  \TLLsym{]}  \TLLnt{t'_{{\mathrm{1}}}}  \TLLsym{,} \, ... \, \TLLsym{,}  \TLLsym{[}  \TLLnt{t_{{\mathrm{1}}}}  \TLLsym{/}  \mathit{x}  \TLLsym{]}  \TLLnt{t'_{\TLLmv{k}}} \,\mathsf{in}\, \TLLnt{t}   \TLLsym{:}  \Box \, \TLLnt{C}, \] but by the definition
  of substitution and the fact that $\TLLsym{[}  \TLLnt{t_{{\mathrm{1}}}}  \TLLsym{/}  \mathit{x}  \TLLsym{]}  \TLLnt{t} = \TLLnt{t}$ because $\TLLnt{t}$ does not depend on $\mathit{x}$ we know that
  \[ \Gamma_{{\mathrm{1}}}  \TLLsym{,}  \Gamma_{{\mathrm{2}}}  \vdash  \TLLsym{[}  \TLLnt{t_{{\mathrm{1}}}}  \TLLsym{/}  \mathit{x}  \TLLsym{]}  \TLLsym{(}   \mathsf{let}_\Box\, \mathit{x_{{\mathrm{1}}}}  \TLLsym{:}  \Box \, \TLLnt{C_{{\mathrm{1}}}}  \TLLsym{,} \, ... \, \TLLsym{,}  \mathit{x_{\TLLmv{k}}}  \TLLsym{:}  \Box \, \TLLnt{C_{\TLLmv{k}}} \,\mathsf{be}\, \TLLnt{t'_{{\mathrm{1}}}}  \TLLsym{,} \, ... \, \TLLsym{,}  \TLLnt{t'_{\TLLmv{k}}} \,\mathsf{in}\, \TLLnt{t}   \TLLsym{)}  \TLLsym{:}  \Box \, \TLLnt{C} \] holds.
\end{itemize}
\end{proof}
% subsubsection proof_of_lemma_substitution_for_typing (end)

\subsubsection{Proof of Weakening}
\label{subsubsec:proof_of_lemma:weakening}

\begin{lemma*}[Weakening]
  If $\Gamma_{{\mathrm{1}}}  \TLLsym{,}  \Gamma_{{\mathrm{2}}}  \vdash  \TLLnt{t}  \TLLsym{:}  \TLLnt{B}$, then $\Gamma_{{\mathrm{1}}}  \TLLsym{,}  \mathit{x}  \TLLsym{:}  \TLLnt{A}  \TLLsym{,}  \Gamma_{{\mathrm{2}}}  \vdash  \TLLnt{t}  \TLLsym{:}  \TLLnt{B}$.
\end{lemma*}

\begin{proof}
  This proof is by induction on the form of $\Gamma_{{\mathrm{1}}}  \TLLsym{,}  \Gamma_{{\mathrm{2}}}  \vdash  \TLLnt{t}  \TLLsym{:}  \TLLnt{B}$.  We
  only show a few cases, because the others are similar.
\begin{itemize}
\item[] Case Identity.\\ 
  \begin{center}
    \begin{math}
      $$\mprset{flushleft}
      \inferrule* [right=$\TLLdruletyXXaxName$] {
        \ 
      }{\Gamma_{{\mathrm{1}}}  \TLLsym{,}  \Gamma_{{\mathrm{2}}}  \TLLsym{,}  \mathit{y}  \TLLsym{:}  \TLLnt{C}  \vdash  \mathit{y}  \TLLsym{:}  \TLLnt{C}}
    \end{math}
  \end{center}
  In this case we have that $\TLLnt{B} = \TLLnt{C}$ and $\TLLnt{t} = \mathit{y}$.  We
  must show that $\Gamma_{{\mathrm{1}}}  \TLLsym{,}  \mathit{x}  \TLLsym{:}  \TLLnt{A}  \TLLsym{,}  \Gamma_{{\mathrm{2}}}  \TLLsym{,}  \mathit{y}  \TLLsym{:}  \TLLnt{C}  \vdash  \mathit{y}  \TLLsym{:}  \TLLnt{C}$ holds, but this
  clearly holds by reapplying the rule.

\item[] Case $\Box$ Introduction.\\ 
  \begin{center}
    \scriptsize
    \begin{math}
      $$\mprset{flushleft}
      \inferrule* [right=$\TLLdruletyXXboxIName$] {
         \Gamma_{{\mathrm{1}}}  \TLLsym{,}  \Gamma_{{\mathrm{2}}}  \vdash  \TLLnt{t_{{\mathrm{1}}}}  \TLLsym{:}  \Box \, \TLLnt{C_{{\mathrm{1}}}}  \TLLsym{,} \, ... \, \TLLsym{,}  \Gamma_{{\mathrm{1}}}  \TLLsym{,}  \Gamma_{{\mathrm{2}}}  \vdash  \TLLnt{t_{\TLLmv{k}}}  \TLLsym{:}  \Box \, \TLLnt{C_{\TLLmv{k}}}   \quad  \mathit{x_{{\mathrm{1}}}}  \TLLsym{:}  \Box \, \TLLnt{C_{{\mathrm{1}}}}  \TLLsym{,} \, ... \, \TLLsym{,}  \mathit{x_{\TLLmv{k}}}  \TLLsym{:}  \Box \, \TLLnt{C_{\TLLmv{k}}}  \vdash  \TLLnt{t'}  \TLLsym{:}  \TLLnt{C}
      }{\Gamma_{{\mathrm{1}}}  \TLLsym{,}  \Gamma_{{\mathrm{2}}}  \vdash   \mathsf{let}_\Box\, \mathit{x_{{\mathrm{1}}}}  \TLLsym{:}  \Box \, \TLLnt{C_{{\mathrm{1}}}}  \TLLsym{,} \, ... \, \TLLsym{,}  \mathit{x_{\TLLmv{k}}}  \TLLsym{:}  \Box \, \TLLnt{C_{\TLLmv{k}}} \,\mathsf{be}\, \TLLnt{t_{{\mathrm{1}}}}  \TLLsym{,} \, ... \, \TLLsym{,}  \TLLnt{t_{\TLLmv{k}}} \,\mathsf{in}\, \TLLnt{t'}   \TLLsym{:}  \Box \, \TLLnt{C}}
    \end{math}
  \end{center}
  This case is similar to the previous case.  First, apply the
  induction hypothesis to the left-most premise, and then reapply
  the rule.
\end{itemize}
\end{proof}
% subsubsection proof_of_lemma:weakening (end)

\subsubsection{Proof of Soundness of $\sf{TCS4}$}
\label{subsec:proof_of_soundness_of_tcs4}

\begin{theorem*}
The type theory $\sf{TCS4}$ has \textit{sound} models provided by the
structures $\cal C$ defined above.  In other words, given a tense
adjoint modal category $\cal C$, using the above interpretation, the
following hold:
\begin{itemize}
\item Assume $\Gamma \vdash t : A$ in $\sf{TCS4}$. Then $\sem{\Gamma
  \vdash t \colon A}$ is a morphism with domain $\sem{\Gamma}$ and
  codomain $\sem{A}$;
\item Assume $\Gamma \vdash t = s \colon A$. Then $\sem{\Gamma
  \vdash t \colon A} = \sem{\Gamma \vdash s \colon A}$.
\end{itemize}
\end{theorem*}
\begin{proof}
  The first part holds by induction on $\Gamma  \vdash  \TLLnt{t}  \TLLsym{:}  \TLLnt{A}$, and the
  second by induction on $ \Gamma  \vdash  \TLLnt{t}  =  \TLLnt{s}  :  \TLLnt{A} $.  We give a few cases of each part, as the others are similar.  Throughout the proof we
  drop semantic brackets on objects, and we assume, without loss of
  generality, that the interpretation of contexts are left associated.
  We begin with the first part.

  \begin{itemize}
  \item[] Case Identity.\\
    \[
    \TLLdruletyXXax{}
    \]
    We need to provide a morphism $\Gamma \times \TLLnt{A} \mto^{f}
    \TLLnt{A}$ and we choose $f = \pi_2$ (the $2$nd projection), as usual.

    \item[] Case Implication Introduction.\\
    \[
    \TLLdruletyXXimpI{}
    \]
    By the induction hypothesis we know that there is a morphism
    $\Gamma \times \TLLnt{A} \mto^f \TLLnt{B}$.  Then we need  to find a
    morphism $\Gamma \mto^{g} (\TLLnt{A}  \to  \TLLnt{B})$.  Choose $g =
    \mathsf{curry}(f)$ where $\mathsf{curry} :
    \mathsf{Hom}_{\cat{C}}( \TLLnt{A}  \times  \TLLnt{B} ,\TLLnt{C}) \mto
    \mathsf{Hom}_{\cat{C}}(\TLLnt{A},\TLLnt{B}  \to  \TLLnt{C})$ is a natural isomorphism
    that exists because $\cat{C}$ is closed.

  \item[] Case $\Box$ Introduction.\\
    \[
    \TLLdruletyXXboxI{}
    \]

    By the induction hypothesis we have the family of morphisms $\Gamma
    \mto^{f_1} \Box \, \TLLnt{A_{{\mathrm{1}}}}, \ldots, \Gamma \mto^{f_k} \Box \, \TLLnt{A_{\TLLmv{k}}}$, and a given morphism
    $\Box \, \TLLnt{A_{{\mathrm{1}}}} \times \cdots \times \Box \, \TLLnt{A_{\TLLmv{k}}} \mto^f \TLLnt{B}$.  We need 
     to find a morphism $\Gamma \mto^{g} \TLLnt{B}$.  As in previous work, we choose $g =
    \langle f_1;\delta_{\TLLnt{A_{{\mathrm{1}}}}},\ldots,f_k;\delta_{\TLLnt{A_{\TLLmv{k}}}} \rangle;\mathsf{m};\Box f$, where $\langle -,-\rangle :
    \mathsf{Hom}_{\cat{C}}(\Gamma,\Box \, \TLLnt{A_{{\mathrm{1}}}}) \times \cdots \times
    \mathsf{Hom}_{\cat{C}}(\Gamma,\Box \, \TLLnt{A_{\TLLmv{k}}}) \mto \mathsf{Hom}_{\cat{C}}(\Gamma,\Box \, \TLLnt{A_{{\mathrm{1}}}}
    \times \cdots \times \Box \, \TLLnt{A_{\TLLmv{k}}})$ exists because $\cat{C}$ is cartesian and we make the simplifying assumption that $\Box$ is an endofunctor.

  \item[] Case $\Box$ Elimination.\\
    \[
    \TLLdruletyXXboxE{}
    \]
    By the induction hypothesis there is a morphism $\Gamma \mto^f
    \Box \, \TLLnt{B}$.  It suffices to find a morphism $\Gamma \mto^{g}
    \TLLnt{B}$.  Choose $g = f;\eta_B$ where $\eta_B : \Box \, \TLLnt{B} \mto \TLLnt{B}$
    is the unit of the adjunction.
  \end{itemize}

  We now turn to the second part:
  \begin{itemize}
  \item[] Case Unboxing $\Box$.\\
    {\scriptsize
      \begin{mathpar}
        \TLLdruleeqXXunbox{}
      \end{mathpar}
    }
    Using the interpretations given above we must show that:
    \[
    \langle f_1;\delta_{\TLLnt{A_{{\mathrm{1}}}}},\ldots,f_k;\delta_{\TLLnt{A_{\TLLmv{k}}}} \rangle;\mathsf{m};\Box f;\eta_B = \langle f_1,\ldots,f_k \rangle;f : \Gamma \mto \TLLnt{B}.
    \]
    This holds by the following equational reasoning:
    \[\small
    \begin{array}{lll}
      \langle f_1;\delta_{\TLLnt{A_{{\mathrm{1}}}}},\ldots,f_k;\delta_{\TLLnt{A_{\TLLmv{k}}}} \rangle;\mathsf{m};\Box f;\eta_B
      & = & \langle f_1;\delta_{\TLLnt{A_{{\mathrm{1}}}}},\ldots,f_k;\delta_{\TLLnt{A_{\TLLmv{k}}}} \rangle;\mathsf{m};\eta;f\\
      & = & \langle f_1;\delta_{\TLLnt{A_{{\mathrm{1}}}}},\ldots,f_k;\delta_{\TLLnt{A_{\TLLmv{k}}}} \rangle;(\eta_{\TLLnt{A_{{\mathrm{1}}}}} \times \cdots \times \eta_{\TLLnt{A_{\TLLmv{k}}}});f\\
      & = & \langle f_1;\delta_{\TLLnt{A_{{\mathrm{1}}}}};\eta_{\TLLnt{A_{{\mathrm{1}}}}},\ldots,f_k;\delta_{\TLLnt{A_{\TLLmv{k}}}};\eta_{\TLLnt{A_{\TLLmv{k}}}} \rangle;f\\
      & = & \langle f_1,\ldots,f_k \rangle;f\\
    \end{array}
    \]
  \end{itemize}
\end{proof}
% subsubsection proof_of_soundness_of_tcs4 (end)

\subsubsection{Proof of Completeness for $\sf{TCS4}$}
\label{subsec:proof_of_completeness_for_tcs4}

\begin{theorem*}
\label{thm:tcs4-completeness}
The adjoint modal models are \textit{complete} in the appropriate
sense for the type theory $\sf{TCS4}$. This is to say, if we have
equality of the interpretations $\sem{\Gamma \vdash t \colon A} =
\sem{\Gamma \vdash s \colon A}$ (where \mbox{$\sem{\ } $} is the
interpretation defined above) in the tense modal category $\cal C$ for
any derived sequents $\Gamma \vdash t \colon A$ and $\Gamma \vdash s
\colon A$ then we can derive the equation in the type theory
$\sf{TCS4}$ $\;$ $\Gamma \vdash t = s \colon A$.
\end{theorem*}

\begin{proof}
  This result can be shown by constructing a cartesian closed category
  with two comonads, one for $\Box$ and one for $\BBox$, internal to the type theory
  $\sf{TCS4}$ where the objects are types and the morphisms are
  $\alpha$-equivalence classes of terms in context $\Gamma  \vdash  \TLLnt{t}  \TLLsym{:}  \TLLnt{A}$.
  This category is called the syntactic category for the $\sf{TCS4}$
  type theory.

  Showing that this syntactic category is cartesian closed is well
  known, but we illustrate the proof by describing the case of the
  $\Box$ comonad.

  We denote a morphism by the $\alpha$-equivalence class:
  \[
  [\vec{x},\TLLnt{t} ]^{\vec{\TLLnt{A}},\TLLnt{B}} = [ \vec{x}:\vec{\TLLnt{A}} \vdash \TLLnt{t} : \TLLnt{B} ]
  \]
  We then have the following definitions:
  \begin{itemize}
  \item (Identity) $\id = [x,x]^{\TLLnt{A},\TLLnt{A}}$
  \item (Composition) Given morphisms $[\vec{x},\TLLnt{t} ]^{\vec{A},\TLLnt{B_{\TLLmv{i}}}}$ and
    $[\vec{y},\TLLnt{t'} ]^{\vec{B},C}$, their composition
    $[\vec{x},\TLLnt{t} ]^{\vec{A},\TLLnt{B_{\TLLmv{i}}}};[\vec{y},\TLLnt{t'} ]^{\vec{B},C} =
    [\vec{x_{i-1}},\vec{y},\vec{x_{i+1}},[ \TLLnt{t} / x_i ] \TLLnt{t'} ]^{\vec{A},C}$.
  \item (Equality) Two parallel morphisms $[\vec{x},\TLLnt{t} ]^{\vec{A},B}$ and $[\vec{x},\TLLnt{t'} ]^{\vec{A},B}$ are
    equal if and only if
    $\vec{x} : \vec{A} \vdash \TLLnt{t} = \TLLnt{t'} : B$.
  \end{itemize}
  Using basic facts about substitution one can show that composition
  preserves identity and is associative.

  We first must show that $\Box$ is an endofunctor on the syntactic
  category.  Suppose we have the morphism
  $[\vec{x},t]^{\vec{A},B}$.  Then we must construct a morphism
  $[\vec{y},t']^{\overrightarrow{\Box \, \TLLnt{A}},\Box \, \TLLnt{B}}$.  The latter morphism can be defined in two steps.
  The first is to change the $\vec{A}$ to $\overrightarrow{\Box \, \TLLnt{A}}$:
  \[
    [ \vec{y},\wedge \mathsf{unbox}_\Box\, \, \mathit{y_{\TLLmv{i}}} ]^{\overrightarrow{\Box \, \TLLnt{A}},\vec{A}};[\vec{x},t]^{\vec{A},\vec{B}}
    = [\vec{y}, [ \mathsf{unbox}_\Box\, \, \mathit{y_{\TLLmv{i}}}/\mathit{x_{\TLLmv{i}}} ] \TLLnt{t} ]^{\overrightarrow{\Box \, \TLLnt{A}},\vec{B}}
    \]
    The second step is to change $B$ into $\Box \, \TLLnt{B}$:
    \[
      [\vec{y},\mathsf{let}_\Box\,\vec{y}:\overrightarrow{\Box \, \TLLnt{A}}\,\mathsf{be}\,\vec{y}\,\mathsf{in}\,[ \mathsf{unbox}_\Box\, \, \mathit{y_{\TLLmv{i}}}/\mathit{x_{\TLLmv{i}}} ]\TLLnt{t})]^{\overrightarrow{\Box \, \TLLnt{A}},\Box \, \TLLnt{B}}
      \]
  Straightforward calculations  show that
  this construction preserves identities and composition.

  The unit of the comonad is defined as $[ y,\mathsf{unbox}_\Box\, \, \mathit{y} ]^{\Box \, \TLLnt{A},A}$.  Next we need to define a morphism between $\Box \, \TLLnt{A}$ and $\Box \, \Box \, \TLLnt{A}$:
  \[
  [y, \mathsf{let}_\Box\, \mathit{y}  \TLLsym{:}  \Box \, \TLLnt{A} \,\mathsf{be}\, \mathit{y} \,\mathsf{in}\, \mathit{y}  ]^{\Box \, \TLLnt{A},\Box \, \Box \, \TLLnt{A}}
  \]
  Finally, using these constructions it is possible to show the usual
  diagrams defining the comonad $\Box$ hold.  The definition for
  $\BBox$ is similar.
\end{proof}
% subsubsection proof_of_completeness_for_tcs4 (end)

% subsection proofs (end)

% section appendix (end)


\end{document}

$\bLozenge$
$\blacksquare $
