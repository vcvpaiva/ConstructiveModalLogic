\section{Monoidal  and Cartesian Categories}
\label{sec:symmetric_monoidal_closed_categories}

\begin{definition}
  \label{def:monoidal-category}
  A \textbf{symmetric monoidal category (SMC)} is a category, $\cat{M}$,
  with the following data:
  \begin{itemize}
  \item An object $I$ of $\cat{M}$,
  \item A bi-functor $\otimes : \cat{M} \times \cat{M} \mto \cat{M}$,
  \item The following natural isomorphisms:
    \[
    \begin{array}{lll}
      \lambda_A : I \otimes A \mto A\\
      \rho_A : A \otimes I \mto A\\      
      \alpha_{A,B,C} : (A \otimes B) \otimes C \mto A \otimes (B \otimes C)\\
    \end{array}
    \]
  \item A symmetry natural transformation:
    \[
    \beta_{A,B} : A \otimes B \mto B \otimes A
    \]
  \item Subject to the following coherence diagrams:
    \begin{mathpar}
      \bfig
      \vSquares|ammmmma|/->`->```->``<-/[
        ((A \otimes B) \otimes C) \otimes D`
        (A \otimes (B \otimes C)) \otimes D`
        (A \otimes B) \otimes (C \otimes D)``
        A \otimes (B \otimes (C \otimes D))`
        A \otimes ((B \otimes C) \otimes D);
        \alpha_{A,B,C} \otimes \id_D`
        \alpha_{A \otimes B,C,D}```
        \alpha_{A,B,C \otimes D}``
        \id_A \otimes \alpha_{B,C,D}]      
      
      \morphism(1433,1000)|m|<0,-1000>[
        (A \otimes (B \otimes C)) \otimes D`
        A \otimes ((B \otimes C) \otimes D);
        \alpha_{A,B \otimes C,D}]
      \efig
      \and
      \bfig
      \hSquares|aammmaa|/->`->`->``->`->`->/[
        (A \otimes B) \otimes C`
        A \otimes (B \otimes C)`
        (B \otimes C) \otimes A`
        (B \otimes A) \otimes C`
        B \otimes (A \otimes C)`
        B \otimes (C \otimes A);
        \alpha_{A,B,C}`
        \beta_{A,B \otimes C}`
        \beta_{A,B} \otimes \id_C``
        \alpha_{B,C,A}`
        \alpha_{B,A,C}`
        \id_B \otimes \beta_{A,C}]
      \efig      
    \end{mathpar}
    \begin{mathpar}
      \bfig
      \Vtriangle[
        (A \otimes I) \otimes B`
        A \otimes (I \otimes B)`
        A \otimes B;
        \alpha_{A,I,B}`
        \rho_{A}`
        \lambda_{B}]
      \efig
      \and
      \bfig
      \btriangle[
        A \otimes B`
        B \otimes A`
        A \otimes B;
        \beta_{A,B}`
        \id_{A \otimes B}`
        \beta_{B,A}]
      \efig
      \and
      \bfig
      \Vtriangle[
        I \otimes A`
        A \otimes I`
        A;
        \beta_{I,A}`
        \lambda_A`
        \rho_A]
      \efig
    \end{mathpar}    
  \end{itemize}
\end{definition}

A \textit{cartesian  category} is a symmetric monoidal  category whose tensor product is a bona fide cartesian product (these are always symmetric) and its unit $I$ is a real terminal object 1.
Simply instantiating the definition above we have  four natural transformations and 5 commuting diagrams.

%\begin{definition}
  \label{def:monoidal-category}
  A \textbf{cartesian category (CC)} is a category, $\cat{M}$,
  with the following data:
  \begin{itemize}
  \item An object $1$ of $\cat{M}$,
  \item A bi-functor $\times : \cat{M} \times \cat{M} \mto \cat{M}$,
  \item The following natural isomorphisms:
    \[
    \begin{array}{lll}
      \lambda_A : 1 \times A \mto A\\
      \rho_A : A \times 1 \mto A\\      
      \alpha_{A,B,C} : (A \times B) \times C \mto A \times (B \times C)\\
    \end{array}
    \]
  \item A symmetry natural transformation:
    \[
    \beta_{A,B} : A \times B \mto B \times A
    \]
  \item Subject to the following coherence diagrams:
    \begin{mathpar}
      \bfig
      \vSquares|ammmmma|/->`->```->``<-/[
        ((A \times B) \times C) \times D`
        (A \times (B \times C)) \times D`
        (A \times B) \times (C \times D)``
        A \times (B \times (C \times D))`
        A \times ((B \times C) \times D);
        \alpha_{A,B,C} \times \id_D`
        \alpha_{A \times B,C,D}```
        \alpha_{A,B,C \times D}``
        \id_A \times \alpha_{B,C,D}]      
      
      \morphism(1433,1000)|m|<0,-1000>[
        (A \times (B \times C)) \times D`
        A \times ((B \times C) \times D);
        \alpha_{A,B \times C,D}]
      \efig
      \and
      \bfig
      \hSquares|aammmaa|/->`->`->``->`->`->/[
        (A \times B) \times C`
        A \times (B \times C)`
        (B \times C) \times A`
        (B \times A) \times C`
        B \times (A \times C)`
        B \times (C \times A);
        \alpha_{A,B,C}`
        \beta_{A,B \times C}`
        \beta_{A,B} \times \id_C``
        \alpha_{B,C,A}`
        \alpha_{B,A,C}`
        \id_B \times \beta_{A,C}]
      \efig      
    \end{mathpar}
    \begin{mathpar}
      \bfig
      \Vtriangle[
        (A \times 1) \times B`
        A \times (1 \times B)`
        A \times B;
        \alpha_{A,1,B}`
        \rho_{A}`
        \lambda_{B}]
      \efig
      \and
      \bfig
      \btriangle[
        A \times B`
        B \times A`
        A \times B;
        \beta_{A,B}`
        \id_{A \times B}`
        \beta_{B,A}]
      \efig
      \and
      \bfig
      \Vtriangle[
        1 \times A`
        A \times 1`
        A;
        \beta_{1,A}`
        \lambda_A`
        \rho_A]
      \efig
    \end{mathpar}    
  \end{itemize}
%\end{definition}


However, saying that a category $\cat{C}$ has  cartesian products  means that given objects $A$ and $B$ of $\cat{C}$, there is an object $A\times B$ and morphisms  $\pi_1\colon A\times B\to A$, and  $\pi_2\colon A\times B\to B$ such that, for any $C$ in $\cat{C}$, if there are maps $p_1\colon C\to A$, and  $p_2\colon C\to B$ then there is a unique $h\colon C\to A\times B$ such that the following diagram commutes.
\texttt{need to add other triangle here}

\begin{mathpar}
\bfig
      \btriangle[
        A \times B`
        C`
        B;
        h`
        \pi_{2}`
        p_{2}]
      \efig
\end{mathpar}      
      Similarly for the terminal object 1: this is such that for any $A$ in $\cat{C}$ we have a single map $t_A\colon A\to 1$. 


\begin{definition}
  \label{def:SMCC}
  A symmetric monoidal \textbf{closed} category (SMCC) is a symmetric
  monoidal category, $(\cat{M},I,\otimes)$, such that, for any object
  $B$ of $\cat{M}$, the functor $- \otimes B : \cat{M} \mto \cat{M}$
  has a specified right adjoint.  Hence, for any objects $A$ and $C$
  of $\cat{M}$ there is an object $A \limp B$ of $\cat{M}$ and a
  natural bijection:
  \[
  \Hom{\cat{M}}{A \otimes B}{C} \cong \Hom{\cat{M}}{A}{B \limp C}
  \]
\end{definition}

Since a \textit{cartesian closed category} is a symmetric monoidal closed category whose tensor product is a  cartesian product (with unit  a  terminal object 1), 
the following lemma summarizes the symmetric monoidal structure we already have in a cartesian closed category. Its proof is well known.

\begin{lemma}[CCCs are SMCs]
  \label{lemma:CCC-is-SMC}
  Suppose $(\cat{C}, 1, \times, \Rightarrow)$ is a cartesian closed
  category.  Then the following defines the structure of a symmetric
  monoidal category:
  \[
  \begin{array}{rll}
    \lambda_A & = & \pi_2 : 1 \times A \mto A\\
    \lambda^{-1}_A & = & \langle \t_A , \id_A  \rangle : A \mto 1 \times A\\
    \\
    \rho_A & = & \pi_1 : A \times 1 \mto A\\
    \rho^{-1}_A & = & \langle \id_A , \t_A  \rangle : A \mto A \times 1\\
    \\
    \alpha_{A,B,C} & = & \langle \pi_1;\pi_1, \pi_2 \times \id_C \rangle : (A \times B) \times C \mto A \times (B \times C)\\
    \alpha^{-1}_{A,B,C} & = & \langle \id_A \times \pi_1, \pi_2;\pi_2 \rangle : A \times (B \times C) \mto (A \times B) \times C\\
    \\
    \beta_{A,B} & = & \langle \pi_2, \pi_1 \rangle : A \times B \mto B \times A\\
  \end{array}
  \]
  Each of the above morphisms satisfy the appropriate diagrams.
\end{lemma}

\begin{definition}
  \label{def:SMCFUN}
  Suppose we are given two symmetric monoidal closed categories $(\cat{M}_1,I_1,\otimes_1,\alpha_1,\lambda_1,\rho_1,\beta_1)$ and
  $(\cat{M}_2,I_2,\otimes_2,\alpha_2,\lambda_2,\rho_2,\beta_2)$.  Then a
  \textbf{symmetric monoidal functor} is a functor $F : \cat{M}_1 \mto
  \cat{M}_2$, a map $m_I : I_2 \mto FI_1$ and a natural transformation
  $m_{A,B} : FA \otimes_2 FB \mto F(A \otimes_1 B)$ subject to the
  following coherence conditions:
  \begin{mathpar}
    \bfig
    \vSquares|ammmmma|/->`->`->``->`->`->/[
      (FA \otimes_2 FB) \otimes_2 FC`
      FA \otimes_2 (FB \otimes_2 FC)`
      F(A \otimes_1 B) \otimes_2 FC`
      FA \otimes_2 F(B \otimes_1 C)`
      F((A \otimes_1 B) \otimes_1 C)`
      F(A \otimes_1 (B \otimes_1 C));
      {\alpha_2}_{FA,FB,FC}`
      m_{A,B} \otimes \id_{FC}`
      \id_{FA} \otimes m_{B,C}``
      m_{A \otimes_1 B,C}`
      m_{A,B \otimes_1 C}`
      F{\alpha_1}_{A,B,C}]
    \efig
    \end{mathpar}
%    \and
\begin{mathpar}
    \bfig
    \square|amma|/->`->`<-`->/<1000,500>[
      I_2 \otimes_2 FA`
      FA`
      FI_1 \otimes_2 FA`
      F(I_1 \otimes_1 A);
      {\lambda_2}_{FA}`
      m_{I} \otimes \id_{FA}`
      F{\lambda_1}_{A}`
      m_{I_1,A}]
    \efig
    \and
    \bfig
    \square|amma|/->`->`<-`->/<1000,500>[
      FA \otimes_2 I_2`
      FA`
      FA \otimes_2 FI_1`
      F(A \otimes_1 I_1);
      {\rho_2}_{FA}`
      \id_{FA} \otimes m_{I}`
      F{\rho_1}_{A}`
      m_{A,I_1}]
    \efig
     \end{mathpar}
     
      \begin{mathpar}
    \bfig
    \square|amma|/->`->`->`->/<1000,500>[
      FA \otimes_2 FB`
      FB \otimes_2 FA`
      F(A \otimes_1 B)`
      F(B \otimes_1 A);
      {\beta_2}_{FA,FB}`
      m_{A,B}`
      m_{B,A}`
      F{\beta_1}_{A,B}]
    \efig
  \end{mathpar}
\end{definition}
A \textit{cartesian  functor} is a symmetric monoidal closed functor between (symmetric) cartesian closed categories. The map $m_1\colon 1\to F1$ and the natural transformations $m_{A,B} : FA \times FB \mto F(A \times B)$ are subject to the instantiated coherence conditions:
  \begin{mathpar}
    \bfig
    \vSquares|ammmmma|/->`->`->``->`->`->/[
      (FA \times FB) \times FC`
      FA \times (FB \times FC)`
      F(A \times B) \times FC`
      FA \times F(B \times C)`
      F((A \times B) \times C)`
      F(A \times (B \times C));
      {\alpha_2}_{FA,FB,FC}`
      m_{A,B} \times \id_{FC}`
      \id_{FA} \times m_{B,C}``
      m_{A \times B,C}`
      m_{A,B \times C}`
      F{\alpha_1}_{A,B,C}]
    \efig
    \end{mathpar}
%    \and
\begin{mathpar}
    \bfig
    \square|amma|/->`->`<-`->/<1000,500>[
      1 \times FA`
      FA`
      F1 \times FA`
      F(1 \times A);
      {\lambda_2}_{FA}`
      m_{1} \times \id_{FA}`
      F{\lambda_1}_{A}`
      m_{1,A}]
    \efig
    \and
    \bfig
    \square|amma|/->`->`<-`->/<1000,500>[
      FA \times 1`
      FA`
      FA \times F1`
      F(A \times 1);
      {\rho_2}_{FA}`
      \id_{FA} \times m_{1}`
      F{\rho_1}_{A}`
      m_{A,1}]
    \efig
     \end{mathpar}
     
      \begin{mathpar}
    \bfig
    \square|amma|/->`->`->`->/<1000,500>[
      FA \times FB`
      FB \times FA`
      F(A \times B)`
      F(B \times A);
      {\beta_2}_{FA,FB}`
      m_{A,B}`
      m_{B,A}`
      F{\beta_1}_{A,B}]
    \efig
  \end{mathpar}
  
Because every cartesian closed category is a symmetric monoidal category (lemma 3) where the tensor is a real product and because   cartesian products are unique up to isomorphism, we  can rewrite these diagrams somewhat. 

\begin{lemma}[CC-functors are SMC-functors]
  \label{lemma:CCf-is-SMCf}
  Given cartesian (closed) categories $(\cat{C}, 1_\cat{C}, \times_\cat{C}, \Rightarrow)$ and $(\cat{D}, 1_\cat{D}, \times_\cat{D}, \Rightarrow)$  and a cartesian (closed) functor  $F\colon \cat{C}\to \cat{D}$ between them,   then we want to  define the structure of a symmetric   monoidal functor, i.e we want to define $m_{A,B}\colon FA \times FB\to F(A\times B)$ and $m_1\colon 1_\cat{D} \to F1_\cat{C}$, satisfying the appropriate commuting diagrams.
 \end{lemma} 
 
  To obtain the map $m_{A,B}\colon FA\times FB\to F(A\times B)$ we use the basic property  of products. 
  Take the product diagram for $A\times B$, with $\pi_1,\pi_2$ and any $C$ in $\cat{D}$. Apply the functor $F$ to the whole diagram. 
  
  Now looking at the image of the diagram in $\cat{D}$, $FA\times FB$ is not in the picture yet, but since it is the product in $\cat{D}$, so it does have projections to $FA$ and $FB$ and given any object that has projections to $FA$ and $FB$ (say for example the object $F(A\times B)$, which has those projections, as they are $F(\pi_1), F(\pi_2)$), then there is a unique map from $F(A\times B) \to FA\times FB$. Similarly for $m_1$. (?)
  
  

{\tt how to get to your product functors from here? how do I merge the first two triangles below so that they look like usual product diagrams?}


\begin{definition}
  \label{def:prod-functor}
  A \textbf{product functor}, $(F,m) : \cat{C}_1 \to \cat{C}_2$,
  between two cartesian categories consists of an ordinary functor $F : \cat{C}_1 \to \cat{C}_2$, a natural transformation $m_{A,B} : FA
  \times FB \mto F(A \times B)$, and a map $m_1 : 1 \to F1$ subject to the
  following coherence diagrams:
  \begin{mathpar}
    \bfig
    \btriangle|mmm|<1000,500>[
      FA \times FB`
      F(A \times B)`
      FA;
      m_{A.B}`
      \pi_1`
      F\pi_1]
    \efig
    \and
    \bfig    
    \btriangle|mmm|<1000,500>[
      FA \times FB`
      F(A \times B)`
      FB;
      m_{A.B}`
      \pi_2`
      F\pi_2]
    \efig
    \and
    \bfig
    \btriangle|mmm|<1000,500>[
      FC`
      FA \times FB`
      F(A \times B);
      \langle Ff, Fg \rangle`
      F(\langle f, g \rangle)`
      m_{A,B}]
    \efig
    \and
    \bfig
    \btriangle|mmm|<1000,500>[
      FA`
      1`
      F1;
      \t_{FA}`
      F\t_{A}`
      m_1]
    \efig
  \end{mathpar} 
\end{definition}

As is expected we can show that product functors are indeed symmetric
monoidal functors. {\texttt{can we?}}

\begin{definition}
  \label{def:SMCNAT}
  Suppose $(\cat{M}_1,I_1,\otimes_1)$ and $(\cat{M}_2,I_2,\otimes_2)$
  are SMCs, and $(F,m)\colon \cat{M}_1\to \cat{M}_2$ and $(G,n)\colon \cat{M}_1\to \cat{M}_2$ are  symmetric monoidal functors between monoidal categories.  Then a symmetric  \textbf{monoidal natural transformation} is a natural transformation,
  $f : F \mto G$, subject to the following coherence diagrams:
  \begin{mathpar}
    \bfig
    \square<1000,500>[
      FA \otimes_2 FB`
      F(A \otimes_1 B)`
      GA \otimes_2 GB`
      G(A \otimes_1 B);
      m_{A,B}`
      f_A \otimes_2 f_B`
      f_{A \otimes_1 B}`
      n_{A,B}]
    \efig
    \and
    \bfig
    \Vtriangle/->`<-`<-/[
      FI_1`
      GI_1`
      I_2;
      f_{I_1}`
      m_{I_1}`
      n_{I_1}]
    \efig
  \end{mathpar}  
\end{definition}
\vspace{.3in}

A \textit{cartesian (or product)} natural transformation does not simplify things much. But when we investigate adjunctions many new properties appear.

 \begin{mathpar}
    \bfig
    \square<1000,500>[
      FA \times FB`
      F(A \times B)`
      GA \times GB`
      G(A \times B);
      m_{A,B}`
      f_A \times f_B`
      f_{A \times B}`
      n_{A,B}]
    \efig
    \and
    \bfig
    \Vtriangle/->`<-`<-/[
      F1`
      G1`
      1;
      f_{1}`
      m_{1}`
      n_{1}]
    \efig
  \end{mathpar}  
  
\begin{definition}
  \label{def:SMCADJ}
  Suppose $(\cat{M}_1,I_1,\otimes_1)$ and $(\cat{M}_2,I_2,\otimes_2)$
  are SMCs, and $(F,m)$ is a symmetric monoidal functor between
  $\cat{M}_1$ and $\cat{M}_2$ and $(G,n)$ is a symmetric monoidal
  functor between $\cat{M}_2$ and $\cat{M}_1$.  Then a
  symmetric \textbf{monoidal adjunction} is an ordinary adjunction
  $\cat{M}_1 : F \dashv G : \cat{M}_2$ such that the unit,
  $\eta : A \to GFA$, and the counit, $\varepsilon_A : FGA \to A$, are
  symmetric monoidal natural transformations.  Thus, the following
  diagrams must commute:
  \begin{mathpar}
    \bfig
    \qtriangle|amm|<1000,500>[
      FGA \otimes_1 FGB`
      FG(A \otimes_1 B)`
      A \otimes_1 B;
      \q{A,B}`
      \varepsilon_A \otimes_1 \varepsilon_B`
      \varepsilon_{A \otimes_1 B}]
    \efig
    \and
    \bfig
    \Vtriangle|amm|/->`<-`=/[
      FGI_1`
      I_1`
      I_1;
      \varepsilon_{I_1}`
      \q{I_1}`]
    \efig
    \and
    \bfig
    \dtriangle|mmb|<1000,500>[
      A \otimes_2 B`
      GFA \otimes_2 GFB`
      GF(A \otimes_2 B);
      \eta_A \otimes_2 \eta_B`
      \eta_{A \otimes_2 B}`
      \p{A,B}]
    \efig
    \and
    \bfig
    \Vtriangle|amm|/->`=`<-/[
      I_2`
      GFI_2`
      I_2;
      \eta_{I_2}``
      p_{I_2}]
    \efig
  \end{mathpar}
  Note that $\p{}$ and $\q{}$ exist because $(FG,\q{})$ and
  $(GF,\p{})$ are symmetric monoidal functors.
\end{definition}



A \textit{product preserving adjunction} is not much simpler. The diagrams become:
\begin{mathpar}
    \bfig
    \qtriangle|amm|<1000,500>[
      FGA \times FGB`
      FG(A \times B)`
      A \times B;
      \q{A,B}`
      \eta_A \times \eta_B`
      \eta_{A \times B}]
    \efig
    \and
    \bfig
    \Vtriangle|amm|/->`<-`=/[
      FG1`
      1`
      1;
      \eta_{1}`
      \q{1}`]
    \efig
    \end{mathpar}
 \begin{mathpar}   
    \bfig
    \dtriangle|mmb|<1000,500>[
      A \times B`
      GFA \times GFB`
      GF(A \times B);
      \varepsilon_A \times \varepsilon_B`
      \varepsilon_{A \times B}`
      \p{A,B}]
    \efig
    \and
    \bfig
    \Vtriangle|amm|/->`=`<-/[
      1`
      GF1`
      1;
      \varepsilon_{1}``
      p_{1}]
    \efig
  \end{mathpar}
  
  However, as Benton shows in \cite{benton1995} (page 14 Proposition 1) for any monoidal adjunction the maps $m_{A,B}$ are the components of a natural \textbf{isomorphism}, with inverses $p_{A,B}$ and furthermore the map $m_1\colon 1\to F1$ is an isomorphism with inverse $p$.
  
  
\begin{definition}
  \label{def:symm-monoidal-monad}
  A \textbf{symmetric monoidal monad} on a symmetric monoidal
  category $\cat{C}$ is a triple $(T,\eta, \mu)$, where
  $(T,\n{})$ is a symmetric monoidal endofunctor on $\cat{C}$,
  $\eta_A : A \mto TA$ and $\mu_A : T^2A \to TA$ are
  symmetric monoidal natural transformations, which make the following
  diagrams commute:
  \begin{mathpar}
    \bfig
    \square|ammb|<600,600>[
      T^3 A`
      T^2A`
      T^2A`
      TA;
      \mu_{TA}`
      T\mu_A`
      \mu_A`
      \mu_A]
    \efig
    \and
    \bfig
    \Atrianglepair/=`<-`=`->`<-/<600,600>[
      TA`
      TA`
      T^2 A`
      TA;`
      \mu_A``
      \eta_{TA}`
      T\eta_A]
    \efig
  \end{mathpar}
  The assumption that $\eta$ and $\mu$ are symmetric
  monoidal natural transformations amount to the following diagrams
  commuting:
  \begin{mathpar}
    \bfig
    \dtriangle|mmb|<1000,600>[
      A \otimes B`
      TA \otimes TB`
      T(A \otimes B);
      \eta_A \otimes \eta_B`
      \eta_A`
      \n{A,B}]    
    \efig
    \and
    \bfig
    \Vtriangle/->`=`<-/<600,600>[
      I`
      TI`
      I;
      \eta_I``
      \n{I}]
    \efig
  \end{mathpar}
  \begin{mathpar}
    \bfig
    \square|ammm|/->`->``/<1050,600>[
      T^2 A \otimes T^2 B`
      T(TA \otimes TB)`
      TA \otimes TB`;
      \n{TA,TB}`
      \mu_A \otimes \mu_B``]

    \square(850,0)|ammm|/->``->`/<1050,600>[
      T(TA \otimes TB)`
      T^2(A \otimes B)``
      T(A \otimes B);
      T\n{A,B}``
      \mu_{A \otimes B}`]
    \morphism(-200,0)<2100,0>[TA \otimes TB`T(A \otimes B);\n{A,B}]
    \efig
    \and
    \bfig
    \square|ammb|/->`->`->`<-/<600,600>[
      I`
      TI`
      TI`
      T^2I;
      \n{I}`
      \n{T}`
      T\n{I}`
      \mu_I]
    \efig
  \end{mathpar}
\end{definition}

Note that the first two diagrams for a \textit{product preserving monad} on a ccc, are exactly the same. The next four diagrams are slightly simplified
\begin{mathpar}
    \bfig
    \dtriangle|mmb|<1000,600>[
      A \times B`
      TA \times TB`
      T(A \times B);
      \eta_A \times \eta_B`
      \eta_A`
      \n{A,B}]    
    \efig
    \and
    \bfig
    \Vtriangle/->`=`<-/<600,600>[
      1`
      T1`
      1;
      \eta_1``
      \n{1}]
    \efig
  \end{mathpar}
  \begin{mathpar}
    \bfig
    \square|ammm|/->`->``/<1050,600>[
      T^2 A \times T^2 B`
      T(TA \times TB)`
      TA \times TB`;
      \n{TA,TB}`
      \mu_A \times \mu_B``]

    \square(850,0)|ammm|/->``->`/<1050,600>[
      T(TA \times TB)`
      T^2(A \times B)``
      T(A \times B);
      T\n{A,B}``
      \mu_{A \times B}`]
    \morphism(-200,0)<2100,0>[TA \times TB`T(A \times B);\n{A,B}]
    \efig
    \and
    \bfig
    \square|ammb|/->`->`->`<-/<600,600>[
      1`
      T1`
      T1`
      T^21;
      \n{1}`
      \n{T}`
      T\n{1}`
      \mu_1]
    \efig
  \end{mathpar}
Finally the dual concept, of a symmetric monoidal comonad, just for completeness.
\begin{definition}
  \label{def:symm-monoidal-comonad}
  A \textbf{symmetric monoidal comonad} on a symmetric monoidal
  category $\cat{C}$ is a triple $(T,\varepsilon, \delta)$, where
  $(T,\m{})$ is a symmetric monoidal endofunctor on $\cat{C}$,
  $\varepsilon_A : TA \mto A$ and $\delta_A : TA \to T^2 A$ are
  symmetric monoidal natural transformations, which make the following
  diagrams commute:
  \begin{mathpar}
    \bfig
    \square|amma|<600,600>[
      TA`
      T^2A`
      T^2A`
      T^3A;
      \delta_A`
      \delta_A`
      T\delta_A`
      \delta_{TA}]
    \efig
    \and
    \bfig
    \Atrianglepair/=`->`=`<-`->/<600,600>[
      TA`
      TA`
      T^2 A`
      TA;`
      \delta_A``
      \varepsilon_{TA}`
      T\varepsilon_A]
    \efig
  \end{mathpar}
  The assumption that $\varepsilon$ and $\delta$ are symmetric
  monoidal natural transformations amount to the following diagrams
  commuting:
  \begin{mathpar}
    \bfig
    \qtriangle|amm|/->`->`->/<1000,600>[
      TA \otimes TB`
      T(A \otimes B)`
      A \otimes B;
      \m{A,B}`
      \varepsilon_A \otimes \varepsilon_B`
    \varepsilon_{A \otimes B}]
    \efig
    \and
    \bfig
    \Vtriangle|amm|/->`<-`=/<600,600>[
      TI`
      I`
      I;
      \m{I}`
      \varepsilon_I`]
    \efig    
  \end{mathpar}
  \begin{mathpar}
    \bfig
    \square|amab|/`->``->/<1050,600>[
      TA \otimes TB``
      T^2A \otimes T^2B`
      T(TA \otimes TB);`
      \delta_A \otimes \delta_B``
      \m{TA,TB}]
    \square(1050,0)|mmmb|/``->`->/<1050,600>[`
      T(A \otimes B)`
      T(TA \otimes TB)`
      T^2(A \otimes B);``
      \delta_{A \otimes B}`
      T\m{A,B}]
    \morphism(0,600)<2100,0>[TA \otimes TB`T(A \otimes B);\m{A,B}]
    \efig
    \and
    \bfig
    \square<600,600>[
      I`
      TI`
      TI`
      T^2I;
      \m{I}`
      \m{I}`
      \delta_I`
      T\m{I}]
    \efig
  \end{mathpar}
\end{definition}

\begin{lemma}[Right-Adjoints are Strong Symmetric Monoidal Functors]
  \label{lemma:right-adjoints_are_strong_symmetric_monoidal_functors}
  Suppose $\cat{C}_1 : F \vdash G : \cat{C}_2$ is an adjunction
  between cartesian closed categories.  Then there is a natural
  isomorphism $n_{A,B} : GA \otimes GB \mto G(A \otimes B)$ and an
  isomorphism $n_1 : 1 \mto G1$ making $G$ a symmetric monoidal
  functor.  
\end{lemma}
\begin{proof}
  Suppose $\cat{C}_1 : F \vdash G : \cat{C}_2$ is an adjunction
  between cartesian closed categories.  Then it is well known that $G$
  preserves products, because it is a right adjoint.  In fact, the
  proof of this fact is quite easy using the Yoneda lemma:
  \begin{center}
    \begin{math}
      \begin{array}{lllllll}
        \Hom{\cat{C}_1}{X}{G(A \times B)}
        & \cong & \Hom{\cat{C}_2}{FX}{A \times B} & \text{(Adjunction)}\\
        & \cong & \Hom{\cat{C}_2}{FX}{A} \times \Hom{\cat{C}_2}{FX}{B} & \\
        & \cong & \Hom{\cat{C}_1}{X}{GA} \times \Hom{\cat{C}_1}{X}{GB} & \text{(Adjunction)}\\
        & \cong & \Hom{\cat{C}_1}{X}{GA \times GB} 
      \end{array}
    \end{math}
  \end{center}
  The previous proof tells us a lot, first, it shows that $G(A \times
  B) \cong GA \times GB$, but it also tells us how to construct the
  natural isomorphism exhibiting this fact.  Enter $G(A \times B)$ for
  $X$ in the above proof, and then trace $\id_{G(A \times B)}$ through
  it from top to bottom.  This will produce the natural transformation
  $n^{-1}_{A,B} = \langle G\pi_1, G\pi_2\rangle : G(A \times B) \mto GA
  \times GB$.  The fact that it is natural is easy to show using the
  cartesian structure.

  Now enter $GA \times GB$ for $X$ in the above proof, and trace
  $\id_{GA \times GB}$ through it from the bottom to the top.  This
  will produce the inverse \\
  $n_{A,B} = \eta_{GA \times GB};G(\langle F(\pi_1);\varepsilon_{A}, F(\pi_2);\varepsilon_{B} \rangle) : GA \times GB \mto G(A \times B)$.
  It is also straightforward to show that $n_{A,B}$ is natural.

  We explicitly prove that $n_{A,B}$ and $n^{-1}_{A,B}$ are indeed
  isomorphisms, but it is implied by the above proof.  First, we show
  that $n_{A,B};n^{-1}_{A,B} = \id_{GA \times GB}$:
  \begin{center}
    \begin{math}
      \begin{array}{lll}
        n_{A,B};n^{-1}_{A,B}
        & = & \eta_{GA \times GB};G(\langle F\pi_1;\varepsilon_A, F\pi_2;\varepsilon_B \rangle);\langle G\pi_1, G\pi_2\rangle\\
        & = & \eta_{GA \times GB};\langle G(F\pi_1;\varepsilon_A), G(F\pi_2;\varepsilon_B)\rangle\\
        & = & \eta_{GA \times GB};\langle GF\pi_1;G\varepsilon_A, GF\pi_2;G\varepsilon_B\rangle\\
        & = & \langle \eta_{GA \times GB};GF\pi_1;G\varepsilon_A, \eta_{GA \times GB};GF\pi_2;G\varepsilon_B\rangle\\
        & = & \langle \pi_1;\eta_{GA};G\varepsilon_A, \pi_2;\eta_{GB};G\varepsilon_B\rangle\\
        & = & \langle \pi_1;\id_{GA}, \pi_2;\id_{GB}\rangle\\
        & = & \langle \pi_1, \pi_2\rangle\\
        & = & \id_{GA \times GB}
      \end{array}
    \end{math}
  \end{center}
  Next we show that $n^{-1}_{A,B};n_{A,B} = \id_{G(A \times B)}$:
  \begin{center}
    \begin{math}
      \begin{array}{lll}
        n^{-1}_{A,B};n_{A,B}
        & = & \langle G\pi_1, G\pi_2\rangle;\eta_{GA \times GB};G(\langle F\pi_1;\varepsilon_A, F\pi_2;\varepsilon_B \rangle)\\
        & = & \eta_{G(A \times B)};GF(\langle G\pi_1, G\pi_2\rangle);G(\langle F\pi_1;\varepsilon_A, F\pi_2;\varepsilon_B \rangle)\\
        & = & \eta_{G(A \times B)};G(F(\langle G\pi_1, G\pi_2\rangle);\langle F\pi_1;\varepsilon_A, F\pi_2;\varepsilon_B \rangle)\\
        & = & \eta_{G(A \times B)};G(F(\langle G\pi_1, G\pi_2\rangle);\langle F\pi_1;\varepsilon_A, F\pi_2;\varepsilon_B \rangle)\\
        & = & \eta_{G(A \times B)};G(\langle FG\pi_1;\varepsilon_A, FG\pi_2;\varepsilon_B \rangle)\\
        & = & \eta_{G(A \times B)};G(\langle \varepsilon_{A \times B};\pi_1, \varepsilon_{A \times B};\pi_2 \rangle)\\
        & = & \eta_{G(A \times B)};G(\varepsilon_{A \times B};\langle \pi_1, \pi_2 \rangle)\\
        & = & \eta_{G(A \times B)};G(\varepsilon_{A \times B};\id_{A \times B})\\
        & = & \eta_{G(A \times B)};G(\varepsilon_{A \times B})\\
        & = & \id_{G(A \times B)}
      \end{array}
    \end{math}
  \end{center}

  The isomorphism $n_1 : 1 \mto G1$ can be constructed using the fact
  that we have the following natural isomorphism as a result of the
  adjunction:
  \[
  \Hom{\cat{C}_1}{1}{G1} \cong \Hom{\cat{C}_2}{F1}{1}
  \]
  Trace $\mathsf{t} : F1 \mto 1$ across the above natural isomorphism
  from left-to-right, and we obtain the morphism $n_1 =
  \eta_1;G\mathsf{t}_{F1} : 1 \mto G1$, but its inverse $n^{-1}_1 =
  \mathsf{t_{G1}} : G1 \mto 1$ is simply the terminal morphism.  It is
  easy to see that $n_1;n^{-1}_1 = \id_1$ by the uniqueness of the
  terminal morphism.  We obtain the fact that they are mutual inverse
  by the following equational reasoning:
  \begin{center}
    \begin{math}
      \begin{array}{lll}
        n^{-1}_1;n_1
        & = & \mathsf{t_{G1}};\eta_1;G\mathsf{t}_{F1}\\
        & = & \eta_{G1};GF(\mathsf{t_{G1}});G\mathsf{t}_{F1}\\
        & = & \eta_{G1};G(F(\mathsf{t_{G1}};\mathsf{t}_{F1}))\\
        & = & \eta_{G1};G(\varepsilon_1)\\
        & = & \id_{G1}\\
      \end{array}
    \end{math}
  \end{center}
  Thus, $n_1 : 1 \mto G1$ is an isomorphism.

  The rest of this proof shows that $n_{A,B}$ and $n_1$ respect the
  coherence diagrams from Definition~\ref{def:SMCFUN} making $G$
  symmetric monoidal.  We know that product functors,
  Definition~\ref{def:prod-functor}, are in bijection with strong
  symmetric monoidal functors, and it is vastly easier to show that
  $n_{A,B}$ respects the diagrams of a product functors, than proving
  it respects the coherence diagrams of a symmetric monoidal functor
  directly.  Actually, the proof that $n_{A,B}$ respects the coherence
  diagrams of a product functor are part of the proof that $n_{A,B}$
  respects the symmetric monoidal diagrams directly.  Thus, we simply
  show that $G$ is a product functor:

  \begin{center}
    \footnotesize
    \begin{math}
      \begin{array}{lll}
        \begin{array}{lll}
          \t_{GA};n_{1}
          & = & \t_{GA};\eta_1;G\mathsf{t}_{F1}\\
          & = & \eta_{GA};GF\t_{GA};G\mathsf{t}_{F1}\\
          & = & \eta_{GA};G(F\t_{GA};\mathsf{t}_{F1})\\
          & = & \eta_{GA};G(FG\t_{A};\mathsf{t}_{FG1})\\
          & = & \eta_{GA};G(FG\t_{A};\varepsilon_{1})\\
          & = & \eta_{GA};G(\varepsilon_{A};\t_{A})\\
          & = & \eta_{GA};G(\varepsilon_{A});G\t_{A}\\
          & = & G\t_{A}\\
        \end{array}
        \\\\
        \begin{array}{lll}
          n_{A,B};G\pi_1
          & = & \eta_{GA \times GB};G(\langle F(\pi_1);\varepsilon_A, F(\pi_2);\varepsilon_B \rangle);G\pi_1\\
          & = & \eta_{GA \times GB};G(\langle F(\pi_1);\varepsilon_A, F(\pi_2);\varepsilon_B \rangle;\pi_1)\\
          & = & \eta_{GA \times GB};G(F(\pi_1);\varepsilon_A)\\
          & = & \eta_{GA \times GB};GF(\pi_1);G(\varepsilon_A)\\
          & = & \pi_1;\eta_{GA};G(\varepsilon_A)\\
          & = & \pi_1
        \end{array}
        \\\\
        \begin{array}{lll}
          n_{A,B};G\pi_2
          & = & \eta_{GA \times GB};G(\langle F(\pi_1);\varepsilon_A, F(\pi_2);\varepsilon_B \rangle);G\pi_2\\
          & = & \eta_{GA \times GB};G(\langle F(\pi_1);\varepsilon_A, F(\pi_2);\varepsilon_B \rangle;\pi_2)\\
          & = & \eta_{GA \times GB};G(F(\pi_2);\varepsilon_B)\\
          & = & \eta_{GA \times GB};GF(\pi_2);G(\varepsilon_B)\\
          & = & \pi_2;\eta_{GB};G(\varepsilon_B)\\
          & = & \pi_2
        \end{array}
        \\\\
        \begin{array}{lll}
          \langle Gf, Gg\rangle;n_{A,B}
          & = & \langle Gf, Gg\rangle;\eta_{GA \times GB};G(\langle F(\pi_1);\varepsilon_A, F(\pi_2);\varepsilon_B \rangle)\\
          & = & \eta_{GC};GF(\langle Gf, Gg\rangle);G(\langle F(\pi_1);\varepsilon_A, F(\pi_2);\varepsilon_B \rangle)\\
          & = & \eta_{GC};G(F(\langle Gf, Gg\rangle);\langle F(\pi_1);\varepsilon_A, F(\pi_2);\varepsilon_B \rangle)\\
          & = & \eta_{GC};G(\langle F(\langle Gf, Gg\rangle;\pi_1);\varepsilon_A, F(\langle Gf, Gg\rangle;\pi_2);\varepsilon_B \rangle)\\
          & = & \eta_{GC};G(\langle FGf;\varepsilon_A, FGg;\varepsilon_B \rangle)\\
          & = & \eta_{GC};G(\langle \varepsilon_{C};f, \varepsilon_{C};g \rangle)\\
          & = & \eta_{GC};G(\varepsilon_{C};\langle f, g \rangle)\\
          & = & \eta_{GC};G(\varepsilon_{C});G(\langle f, g \rangle)\\
          & = & G(\langle f, g \rangle)\\
        \end{array}
      \end{array}
    \end{math}
  \end{center}

  %% \begin{itemize}
  %% \item[] Case 1:
  %%   \begin{center}
  %%     \begin{math}
  %%       \bfig
  %%       \vSquares|ammmmma|/->`->`->``->`->`->/[
  %%         (GA \times GB) \times GC`
  %%         GA \times (GB \times GC)`
  %%         G(A \times B) \times GC`
  %%         GA \times G(B \times C)`
  %%         G((A \times B) \times C)`
  %%         G(A \times (B \times C));
  %%         {\alpha}_{GA,GB,GC}`
  %%         n_{A,B} \times \id_{GC}`
  %%         \id_{GA} \times n_{B,C}``
  %%         n_{A \times B,C}`
  %%         n_{A,B \times C}`
  %%         G{\alpha}_{A,B,C}]
  %%       \efig
  %%     \end{math}
  %%   \end{center}
  %%   The previous diagram commutes by the following equational
  %%   reasoning (due to the size of the morphisms we must break the
  %%   reasoning up into multiple steps):
  %%   \begin{center}
  %%     \footnotesize
  %%     \begin{math}
  %%       \begin{array}{lll}
  %%         \alpha_{GA,GB,GC};(\id_{GA} \times n_{B,C})\\
  %%         \,\,\,\,\,\,\,\,= (\langle \pi_1;\pi_1, \pi_2 \times \id_{GC} \rangle);(\id_{GA} \times (\eta_{GB \times GC};G(\langle F(\pi_1);\varepsilon_{GB}, F(\pi_2);\varepsilon_{GC} \rangle)))\\
  %%         \,\,\,\,\,\,\,\,=\langle \pi_1;\pi_1,  (\pi_2 \times \id_{GC});\eta_{GB \times GC};G(\langle F(\pi_1);\varepsilon_{GB}, F(\pi_2);\varepsilon_{GC} \rangle)\rangle\\
  %%         \,\,\,\,\,\,\,\,=\langle \pi_1;\pi_1,  \eta_{(GA \times GB) \times GC};GF(\pi_2 \times \id_{GC});G(\langle F(\pi_1);\varepsilon_{GB}, F(\pi_2);\varepsilon_{GC} \rangle)\rangle\\
  %%         \,\,\,\,\,\,\,\,=\langle \pi_1;\pi_1,  \eta_{(GA \times GB) \times GC};G(F(\pi_2 \times \id_{GC});\langle F(\pi_1);\varepsilon_{GB}, F(\pi_2);\varepsilon_{GC} \rangle)\rangle\\
  %%         \,\,\,\,\,\,\,\,=\langle \pi_1;\pi_1,  \eta_{(GA \times GB) \times GC};G(\langle F(\pi_2 \times \id_{GC});F(\pi_1);\varepsilon_{GB}, F(\pi_2 \times \id_{GC});F(\pi_2);\varepsilon_{GC} \rangle)\rangle\\
  %%         \,\,\,\,\,\,\,\,=\langle \pi_1;\pi_1,  \eta_{(GA \times GB) \times GC};G(\langle F(\pi_1;\pi_2);\varepsilon_{GB}, F(\pi_2);\varepsilon_{GC} \rangle)\rangle\\
  %%       \end{array}
  %%     \end{math}
  %%   \end{center}

  %%   \begin{center}
  %%     \footnotesize
  %%     \begin{math}
  %%       \begin{array}{lll}
  %%         \alpha_{GA,GB,GC};(\id_{GA} \times n_{B,C});n_{A,B \times C}\\
  %%         \,\,\,\,\,\,\,\,=\langle \pi_1;\pi_1,  \eta_{(GA \times GB) \times GC};G(\langle F(\pi_1;\pi_2);\varepsilon_{GB}, F(\pi_2);\varepsilon_{GC} \rangle)\rangle;\eta_{GA \times G(B \times C)};G(\langle F(\pi_1);\varepsilon_{GA}, F(\pi_2);\varepsilon_{G(B \times C)} \rangle)\\
  %%         \,\,\,\,\,\,\,\,=\eta_{(GA \times GB) \times GC};GF(\langle \pi_1;\pi_1,  \eta_{(GA \times GB) \times GC};G(\langle F(\pi_1;\pi_2);\varepsilon_{GB}, F(\pi_2);\varepsilon_{GC} \rangle)\rangle);G(\langle F(\pi_1);\varepsilon_{GA}, F(\pi_2);\varepsilon_{G(B \times C)} \rangle)\\
  %%         \,\,\,\,\,\,\,\,=\eta_{(GA \times GB) \times GC};G(F(\langle \pi_1;\pi_1,  \eta_{(GA \times GB) \times GC};G(\langle F(\pi_1;\pi_2);\varepsilon_{GB}, F(\pi_2);\varepsilon_{GC} \rangle)\rangle);\langle F(\pi_1);\varepsilon_{GA}, F(\pi_2);\varepsilon_{G(B \times C)} \rangle)\\
  %%         \,\,\,\,\,\,\,\,=\eta_{(GA \times GB) \times GC};G(\langle F(\pi_1;\pi_1);\varepsilon_{GA}, F(\eta_{(GA \times GB) \times GC};G(\langle F(\pi_1;\pi_2);\varepsilon_{GB}, F(\pi_2);\varepsilon_{GC} \rangle));\varepsilon_{G(B \times C)} \rangle)\\
  %%         \,\,\,\,\,\,\,\,=\eta_{(GA \times GB) \times GC};G(\langle F(\pi_1;\pi_1);\varepsilon_{GA}, F(\eta_{(GA \times GB) \times GC});\varepsilon_{F((GA \times GB) \times GC)};\langle F(\pi_1;\pi_2);\varepsilon_{GB}, F(\pi_2);\varepsilon_{GC} \rangle) \rangle\\
  %%         \,\,\,\,\,\,\,\,=\eta_{(GA \times GB) \times GC};G(\langle F(\pi_1;\pi_1);\varepsilon_{GA}, \langle F(\pi_1;\pi_2);\varepsilon_{GB}, F(\pi_2);\varepsilon_{GC} \rangle) \rangle          
  %%       \end{array}
  %%     \end{math}
  %%   \end{center}
  %%   Now we simplify the opposite side:
  %%   \begin{center}
  %%     \footnotesize
  %%     \begin{math}
  %%       \begin{array}{lll}
  %%         (n_{A,B} \times \id_{GC});n_{A \times B,C}\\
  %%         \,\,\,\,\,\,\,\,= (n_{A,B} \times \id_{GC});\eta_{G(A \times B) \times GC};G(\langle F(\pi_1);\varepsilon_{G(A \times B)}, F(\pi_2);\varepsilon_{GC} \rangle)\\
  %%         \,\,\,\,\,\,\,\,= \eta_{(GA \times GB) \times GC};GF(n_{A,B} \times \id_{GC});G(\langle F(\pi_1);\varepsilon_{G(A \times B)}, F(\pi_2);\varepsilon_{GC} \rangle)\\
  %%         \,\,\,\,\,\,\,\,= \eta_{(GA \times GB) \times GC};G(F(n_{A,B} \times \id_{GC});\langle F(\pi_1);\varepsilon_{G(A \times B)}, F(\pi_2);\varepsilon_{GC} \rangle)\\
  %%         \,\,\,\,\,\,\,\,= \eta_{(GA \times GB) \times GC};G(\langle F(\pi_1;n_{A,B});\varepsilon_{G(A \times B)}, F(\pi_2);\varepsilon_{GC} \rangle)\\
  %%       \end{array}
  %%     \end{math}
  %%   \end{center}

  %%   \begin{center}
  %%     \scriptsize
  %%     \begin{math}
  %%       \begin{array}{lll}
  %%         (n_{A,B} \times \id_{GC});n_{A \times B,C};G(\alpha_{A,B,C})\\
  %%         \,\,\,\,\,\,\,\,= \eta_{(GA \times GB) \times GC};G(\langle F(\pi_1;n_{A,B});\varepsilon_{G(A \times B)}, F(\pi_2);\varepsilon_{GC} \rangle);G(\langle \pi_1;\pi_1, \pi_2 \times \id_C \rangle)\\
  %%         \,\,\,\,\,\,\,\,= \eta_{(GA \times GB) \times GC};G(\langle F(\pi_1;n_{A,B});\varepsilon_{G(A \times B)}, F(\pi_2);\varepsilon_{GC} \rangle;\langle \pi_1;\pi_1, \pi_2 \times \id_C \rangle)\\
  %%         \,\,\,\,\,\,\,\,= \eta_{(GA \times GB) \times GC};G(\langle F(\pi_1;n_{A,B});\varepsilon_{G(A \times B)};\pi_1, \langle F(\pi_1;n_{A,B});\varepsilon_{G(A \times B)};\pi_2,  F(\pi_2);\varepsilon_{GC} \rangle \rangle)\\
  %%         \,\,\,\,\,\,\,\,= \eta_{(GA \times GB) \times GC};G(\langle F(\pi_1);F(n_{A,B});FG\pi_1;\varepsilon_{GA},  \langle F(\pi_1;n_{A,B});\varepsilon_{G(A \times B)};\pi_2,  F(\pi_2);\varepsilon_{GC} \rangle \rangle)\\
  %%         \,\,\,\,\,\,\,\,= \eta_{(GA \times GB) \times GC};G(\langle

  %%         F(\pi_1);F(n_{A,B};G\pi_1);\varepsilon_{GA}, \langle F(\pi_1;n_{A,B});\varepsilon_{G(A \times B)};\pi_2,  F(\pi_2);\varepsilon_{GC} \rangle \rangle)\\
  %%         \,\,\,\,\,\,\,\,= \eta_{(GA \times GB) \times GC};G(\langle

  %%         F(\pi_1;n_{A,B};G\pi_1);\varepsilon_{GA},

  %%         \langle F(\pi_1;n_{A,B});FG\pi_2;\varepsilon_{GB},

  %%         F(\pi_2);\varepsilon_{GC} \rangle \rangle)\\
  
  %%         \,\,\,\,\,\,\,\,= \eta_{(GA \times GB) \times GC};G(\langle

  %%         F(\pi_1;n_{A,B};G\pi_1);\varepsilon_{GA},

  %%         \langle F(\pi_1;n_{A,B};G\pi_2);\varepsilon_{GB},

  %%         F(\pi_2);\varepsilon_{GC} \rangle \rangle)\\          
  %%       \end{array}
  %%     \end{math}
  %%   \end{center}
  %%   To finish off the previous bit of reasoning we must show that $n_{A,B};G\pi_1 = \pi_1$ and $n_{A,B};G\pi_2 = \pi_2$:
  %%   \begin{center}
  %%     \begin{math}
  %%       \begin{array}{lll}
  %%         n_{A,B};G\pi_1
  %%         & = & \eta_{GA \times GB};G(\langle F(\pi_1);\varepsilon_A, F(\pi_2);\varepsilon_B \rangle);G\pi_1\\
  %%         & = & \eta_{GA \times GB};G(\langle F(\pi_1);\varepsilon_A, F(\pi_2);\varepsilon_B \rangle;\pi_1)\\
  %%         & = & \eta_{GA \times GB};G(F(\pi_1);\varepsilon_A)\\
  %%         & = & \eta_{GA \times GB};GF(\pi_1);G(\varepsilon_A)\\
  %%         & = & \pi_1;\eta_{GA};G(\varepsilon_A)\\
  %%         & = & \pi_1
  %%       \end{array}
  %%     \end{math}
  %%   \end{center}
  
  %%   \begin{center}
  %%     \begin{math}
  %%       \begin{array}{lll}
  %%         n_{A,B};G\pi_2
  %%         & = & \eta_{GA \times GB};G(\langle F(\pi_1);\varepsilon_A, F(\pi_2);\varepsilon_B \rangle);G\pi_2\\
  %%         & = & \eta_{GA \times GB};G(\langle F(\pi_1);\varepsilon_A, F(\pi_2);\varepsilon_B \rangle;\pi_2)\\
  %%         & = & \eta_{GA \times GB};G(F(\pi_2);\varepsilon_B)\\
  %%         & = & \eta_{GA \times GB};GF(\pi_2);G(\varepsilon_B)\\
  %%         & = & \pi_2;\eta_{GB};G(\varepsilon_B)\\
  %%         & = & \pi_2
  %%       \end{array}
  %%     \end{math}
  %%   \end{center}
  
  %%   Thus, we obtain our result, because both sides of the diagram reduce equivalent morphisms.
  
  %% \item[] Case 2:
  %%   \begin{center}
  %%     \begin{math}
  %%       \bfig
  %%       \square|amma|/->`->`<-`->/<1000,500>[
  %%         1 \times GA`
  %%         GA`
  %%         G1 \times GA`
  %%         G(1 \times A);
  %%         {\lambda}_{GA}`
  %%         n_{1} \times \id_{GA}`
  %%         G{\lambda}_{A}`
  %%         n_{1,A}]
  %%       \efig
  %%     \end{math}
  %%   \end{center}

  %% \item[] Case 3:
  %%   \begin{center}
  %%     \begin{math}
  %%       \begin{array}{lll}
  %%         \bfig
  %%         \square|amma|/->`->`<-`->/<1000,500>[
  %%           GA \times 1`
  %%           GA`
  %%           GA \times G1`
  %%           G(A \times 1);
  %%           {\rho}_{GA}`
  %%           \id_{GA} \times n_{1}`
  %%           G{\rho}_{A}`
  %%           n_{A,1}]
  %%         \efig
  %%       \end{array}
  %%     \end{math}
  %%   \end{center}

  %% \item[] Case 4:
  %%   \begin{center}
  %%     \begin{math}
  %%       \bfig
  %%       \square|amma|/->`->`->`->/<1000,500>[
  %%         GA \times GB`
  %%         GB \times GA`
  %%         G(A \times B)`
  %%         G(B \times A);
  %%         {\beta}_{GA,GB}`
  %%         n_{A,B}`
  %%         n_{B,A}`
  %%         G{\beta}_{A,B}]
  %%       \efig
  %%     \end{math}
  %%   \end{center}
  %% \end{itemize}
\end{proof}
% section symmetric_monoidal_categories (end)
