\section{Monoidal Monads are Strong w.r.t. Any Monoidal Comonad}
\label{sec:monoidal_monads_are_strong_w.r.t._any_monoidal_comonad}

We show the following theorem:

\begin{theorem}[Monoidal Monads are $\Box$-Strong]
  \label{thm:monoidal_monads_are_$\box$-strong}
  Given any monoidal monad, $(\Diamond,\eta,\mu)$, and monoidal
  comonad $(\Box, \varepsilon, \delta)$ on some monoidal category
  $\cat{C}$, then $\Diamond$ is $\Box$-strong.
\end{theorem}

First, we use a slightly different notion of $\Box$-strong.

\begin{definition}[$\Box$-strong monad]
  \label{def:comonad-strong-monad-2}
  Suppose $(\Box, \varepsilon, \delta)$ is a (monoidal) comonad on a
  cartesian closed category $\cat{C}$.  Then a \emph{$\Box$-strong
    monad} is a monad $(\Diamond,\eta,\mu)$ on $\cat{C}$ such that
  there exists a natural transformation:
  \[
  \st{A}{B} : \Box A \pd \Diamond B \mto \Diamond(\Box A \pd B)
  \]
  subject to the following  conditions, described by the diagrams.
  
  First $\Box$ is a monoidal comonad. Hence (as before) we have six
  diagrams. Two for $\Box$ comonad:
  \begin{mathpar}
    \bfig
    \square|amma|<600,600>[
      \Box A`
      \Box ^2A`
      \Box ^2A`
      \Box ^3A;
      \delta_A`
      \delta_A`
      \Box\delta_A`
      \delta_{\Box A}]
    \efig
    \and
    \bfig
    \Atrianglepair/=`->`=`<-`->/<600,600>[
      \Box A`
      \Box A`
      \Box^2 A`
      \Box A;`
      \delta_A``
      \varepsilon_{\Box A}`
      \Box\varepsilon_A]
    \efig
  \end{mathpar}
  And four for $\varepsilon$ and $\delta$ as symmetric monoidal natural transformations:
  \begin{mathpar}
    \bfig
    \qtriangle|amm|/->`->`->/<1000,600>[
      \Box A \times \Box B`
      \Box (A \times B)`
      A \times B;
      \m{A,B}`
      \varepsilon_A \times \varepsilon_B`
    \varepsilon_{A \times B}]
    \efig
    \and
    \bfig
    \Vtriangle|amm|/<-`->`=/<600,600>[
      \Box 1`
      1`
      1;
      \m{1}`
      \varepsilon_1`]
    \efig    
  \end{mathpar}
  \begin{mathpar}
    \bfig
    \square|amab|/`->``->/<1050,600>[
      \Box A \times \Box B``
      \Box^2A \times \Box ^2B`
      \Box (\Box A \times \Box B);`
      \delta_A \times \delta_B``
      \m{\Box A,\Box B}]
    \square(1050,0)|mmmb|/``->`->/<1050,600>[`
      \Box (A \times B)`
      \Box (\Box A \times \Box B)`
      \Box ^2(A \times B);``
      \delta_{A \times B}`
      \Box \m{A,B}]
    \morphism(0,600)<2100,0>[\Box A \times \Box B`\Box (A \times B);\m{A,B}]
    \efig
    \and
    \bfig
    \square<600,600>[
      1`
      \Box 1`
      \Box 1`
      \Box^21;
      \m{1}`
      \m{1}`
      \delta_1`
      \Box \m{1}]
    \efig
  \end{mathpar}
  
 Second $\Diamond$ is a  (monoidal) monad:
   \begin{mathpar}
    \bfig
    \square|ammb|<600,600>[
      \Diamond^3 A`
      \Diamond^2A`
      \Diamond^2A`
      \Diamond A;
      \mu_{\Diamond A}`
      \Diamond \mu_A`
      \mu_A`
      \mu_A]
    \efig
    \and
    \bfig
    \Atrianglepair/=`<-`=`->`<-/<600,600>[
      \Diamond A`
      \Diamond A`
      \Diamond ^2 A`
      \Diamond A;`
      \mu_A``
      \eta_{\Diamond A}`
      \Diamond \eta_A]
    \efig
  \end{mathpar}
  And, as before,  $\mu$ and $\eta$ are monoidal natural transformations:
   \begin{mathpar}
    \bfig
    \dtriangle|mmb|<1000,600>[
      A \times B`
      \Diamond A \times \Diamond B`
      \Diamond (A \times B);
      \eta_A \times \eta_B`
      \eta_{A\times B}`
      \n{A,B}]    
    \efig
    \and
    \bfig
    \Vtriangle/->`=`<-/<600,600>[
      1`
      \Diamond 1`
      1;
      \eta_1``
      \n{1}]
    \efig
  \end{mathpar}
  
  \begin{mathpar}
    \bfig
    \square|ammm|/->`->``/<1050,600>[
      \Diamond ^2 A \times\Diamond ^2 B`
      \Diamond (\Diamond A\times\Diamond B)`
      \Diamond A \times \Diamond B`;
      \n{\Diamond A,\Diamond B}`
      \mu_A \times \mu_B``]

    \square(850,0)|ammm|/->``->`/<1050,600>[
      \Diamond (\Diamond A\times\Diamond B)`
      \Diamond ^2(A \times B)``
      \Diamond (A \times B);
      \Diamond \n{A,B}``
      \mu_{A \times B}`]
    \morphism(-200,0)<2100,0>[\Diamond A \times \Diamond B`\Diamond (A \times B);\n{A,B}]
    \efig
    \and
    \bfig
    \square|ammb|/->`->`->`<-/<600,600>[
      1`
      \Diamond 1`
      \Diamond 1`
      \Diamond^2 1;
      \n{1}`
      \n{1}`
      \Diamond\n{1}`
      \mu_1]
    \efig
  \end{mathpar}

Finally, the interesting ones. The equations that mix the comonad with
the monad structure:

\begin{center}
  \begin{math}
    \begin{array}{cccccc}
      \bfig
      \vSquares|ammmmma|/>``>```>`>/[\Box 1 \times \Diamond A`\Diamond (\Box 1 \times A)``\Diamond(1 \times A)`1 \times \Diamond A`\Diamond A;\st{1}{A}``\Diamond(\varepsilon_1 \times \id_A)```\Diamond\lambda`\lambda]
      \morphism(0,1000)|m|/->/<0,-950>[`;\varepsilon_1 \times \id_{\Diamond A}]      
      \efig
      \\
      \bfig
      \hSquares|aamaaaa|/->`->`->```->`/[
        \Box A \times 1`
        \Box A \times \Diamond 1`
        \Diamond (\Box A \times 1)`
        \Box A`
        A`
        ;
        \id_{\Box A} \times \eta`
        \st{A}{1}`
        \rho```
        \varepsilon`]
      \qtriangle(1944,0)/->``->/<800,500>[\Diamond (\Box A \times 1)`\Diamond\Box A`\Diamond A;\Diamond \rho``\Diamond\varepsilon]
      \morphism(1040,0)/->/<1630,0>[`;\eta_A]
      \efig
    \end{array}
    %% \begin{array}{lll}
    %%   \bfig
    %%   \square|amma|<850,600>[
    %%     \Box A \times \Diamond B`
    %%     \Diamond (\Box A \times B)`
    %%     \Box A`
    %%     \Diamond\Box A;
    %%     \st{A}{B}`
    %%     \pi_1`
    %%     \Diamond\pi_1`
    %%     \eta_A]
    %%   \place(425,300)[(1)]
    %%   \efig
    %%   & \quad &
    %%   \bfig
    %%   \qtriangle<850,600>[
    %%     \Box A \times \Diamond B`
    %%     \Diamond (\Box A \times B)`
    %%     \Diamond B;
    %%     \st{A}{B}`
    %%     \pi_2`
    %%     \Diamond\pi_2]
    %%   \place(600,400)[(2)]
    %%   \efig
    %% \end{array}
  \end{math}
  \\
  \vspace{30px} 
  \begin{math}
    \bfig
    \btriangle<800,500>[\Box A \pd B`\Box A \pd \Diamond B`\Diamond(\Box A \pd B);\id_{\Box A} \pd \eta_{B}`\eta_{\Box A \times B}`\st_{A,B}]
    \place(200,200)[(3)]
    \efig
  \end{math}
    \\
    \vspace{30px}
    %% \begin{math}
    %%   \bfig
    %%   \hSquares|aamaaaa|/->`->`->``=`->`->/[
    %%     \Box A \pd \Diamond 1`
    %%     \Diamond (\Box A \pd 1)`
    %%     \Diamond\Box A`
    %%     \Box A \pd 1`
    %%     \Box A`
    %%     \Diamond\Box A;\st{A}{1}`\Diamond(\rho_{\Box A})`\id_A \pd \varepsilon```\rho`\eta_{\Box A}]
    %%   \efig
    %% \end{math}  
    \vspace{30px}
    \begin{math}
      \bfig
        \vSquares|ammmmmm|/->`->`->```->`/[
          \Box A \times (\Box B \times \Diamond C)`
          \Box A \times \Diamond(\Box B \times C)`
          (\Box A \times \Box B) \times \Diamond C`
          \Diamond(\Box A \times (\Box B \times C))``
          \Diamond((\Box A \times \Box B) \times C);
          \id_{\Box A} \pd \st{B}{C}`
          \alpha^{-1}`
          \st{A}{\Box B \times C}```
          \Diamond\alpha^{-1}`]
        \morphism(1554,0)|m|/->/<0,-500>[`\Diamond(\Box(A \times B) \times C);\Diamond(\m{A,B} \times \id_C)]
        
        \morphism(0,500)|m|/->/<0,-1000>[`\Box(A \times B) \times \Diamond C;\m{A,B} \times \id_{\Diamond C}]

        \morphism(350,-500)|a|/->/<800,0>[`;\st{A \times B}{C}]
        \place(800,350)[(4)]
        \efig
    \end{math}
    \\
    \vspace{30px}
    \begin{math}
      \bfig
      \vSquares|ammmmma|/->`->```->``->/[
        \Box A \times \Diamond\Diamond B`
        \Box A \times \Diamond B`
        \Diamond(\Box A \times \Diamond B)``
        \Diamond\Diamond(\Box A \times B)`
        \Diamond(\Box A \times B);
        \id_{\Box A} \pd \mu_{B}`
        \st{A}{\Diamond B}```
        \Diamond(\st{A}{B})``
        \mu_{\Box A \pd B}]
      \morphism(1150,1000)|m|<0,-920>[`;\st{A}{B}]
      \place(600,500)[(5)]
      \efig
    \end{math}
    \\
    \vspace{30px}
    \begin{math}
      \bfig
      \hSquares|aamamaa|/``->``->`->`->/[
        \Box A \times \Diamond B``
        \Diamond B \times \Box A`
        \Diamond B \times \Box A`
        \Diamond B \times \Diamond \Box A`
        \Diamond (B \times \Box A);``
        \beta``
        \Diamond \beta`
        \id_{\Diamond B} \pd \eta_{\Box A}`
        \mathsf{n}_{B,\Box A}]
      \morphism(200,500)<1883,0>[`;\st{A}{B}]
      \place(1100,250)[(6)]
      \efig
    \end{math}
    \\
    \vspace{30px}
    \begin{math}
      \bfig
      \hSquares|aamamaa|/``->``->`->`->/[
        \Box A \times \Diamond B``
        \Diamond(\Box A \times B)`
        A \times \Diamond B`
        \Diamond A \times \Diamond B`
        \Diamond (A \times B);``
        \varepsilon_A \times \id_{\Diamond B}``
        \Diamond(\varepsilon_A \times \id_{B})`
        \eta_A \times \id_{\Diamond B}`
        \m{A,B}]
      \morphism(200,500)<1605,0>[`;\st{A}{B}]
      \place(1000,250)[(7)]
      \efig
    \end{math}        
  \end{center}
\end{definition}

\begin{proof}
  First, we define the $\Box$-strength map as follows:
  \[
  \st{A}{B} = (\eta_{\Box A} \pd \id_{\Diamond B});\mathsf{n}_{\Box A,B} : \Box A \pd \Diamond B \mto \Diamond(\Box A \pd B)
  \]
  We can see that $\st{A}{B}$ is a natural transformation, because it
  is defined as a composition of natural transformations.
    
  Next we must show that all of the appropriate diagrams given in
  Definition~\ref{def:comonad-strong-monad-2} commute.
  \begin{itemize}
  \item[] \textit{Case 1. the first projection interacts with $\Diamond$:}
    \[
    \bfig
    \square|amma|<850,600>[
      \Box A \times \Diamond B`
      \Diamond (\Box A \times B)`
      \Box A`
      \Diamond\Box A;
      \st{A}{B}`
      \pi_1`
      \Diamond\pi_1`
      \eta_A]
    \efig
    \]
    This diagrams commutes, because the following diagram commutes:
    \[
    \bfig
    \square|ammm|/->`->``/<1500,2000>[
      \Box A \times \Diamond B`
      \Diamond\Box A \times \Diamond B`
      \Box A`;
      \eta_{\Box A} \times \id_{\Diamond B}`
      \pi_1``]

    \square(1500,0)|ammm|/->``->`/<1500,2000>[
      \Diamond\Box A \times \Diamond B`
      \Diamond (\Box A \times B)``
      \Diamond\Box A;
      \p{\Box A,B}``
      \Diamond\pi_1`]

    \morphism<3000,0>[\Box A`\Diamond\Box A;\eta_{\Box A}]

    \morphism(484,1500)<1000,0>[
      \Box A \times 1`
      \Diamond\Box A \times \Diamond 1;
      \eta_{\Box A} \times \eta_1]

    \morphism(1484,1500)<700,-500>[
      \Diamond\Box A \times \Diamond 1`
      \Diamond(\Box A \times 1);
      \p{\Box A,1}]

    \morphism(484,1500)|m|/{@{>}@/_1em/}/<1700,-500>[
      \Box A \times 1`
      \Diamond(\Box A \times 1);
      \eta_{\Box A \times 1}]

    \morphism(484,1500)|m|<700,-800>[
      \Box A \times 1`
      \Box A;
      \pi_1]

    \morphism(0,2000)|m|<484,-500>[
      \Box A \times \Diamond B`
      \Box A \times 1;
      \id_{\Box A} \times \t_{\Diamond B}]

    \morphism(1500,2000)|m|<-17,-500>[
      \Diamond\Box A \times \Diamond B`
      \Diamond\Box A \times \Diamond 1;
      \id_{\Diamond\Box A} \times \Diamond\t_{B}]

    \morphism(3000,2000)|m|<-815,-1000>[
      \Diamond(\Box A \times B)`
      \Diamond(\Box A \times 1);
      \Diamond (id_{\Box A} \times \t_{B})]

    \morphism(2184,1000)|m|<816,-1000>[
      \Diamond(\Box A \times 1)`
      \Diamond\Box A;
      \Diamond\pi_1]

    \morphism(1184,700)|m|<1816,-700>[
      \Box A`
      \Diamond\Box A;
      \eta_{\Box A}]
    
    \place(2700,1000)[1]
    \place(2150,1700)[2]
    \place(800,1750)[3]
    \place(1300,1300)[4]
    \place(1800,800)[5]
    \place(500,500)[6]
    \efig    
    \]
    Diagrams 1 and 6 commute because we are in a cartesian closed
    category, diagram 2 commutes by naturality of $\p{}$, diagram 3
    commutes because $\Diamond$ is a product functor, diagram 4
    commutes because $\eta$ is the unit of a symmetric monoidal
    adjunction, and diagram 5 commutes by naturality of $\eta$.
    
  \item[] \textit{Case 2. the second projection interacts with $\Diamond$:}
    \[
    \bfig
    \qtriangle<850,600>[
      \Box A \times \Diamond B`
      \Diamond (\Box A \times B)`
      \Diamond B;
      \st{A}{B}`
      \pi_2`
      \Diamond\pi_2]
    \efig
    \]
    This case is similar to the previous case.
    
    %% \item[] \textit{Case 1. the object 1 behaves as the unit for products}
  %%   $$
  %%   \bfig
  %%   \vSquares|ammmmma|/>``>```>`>/[\Box 1 \times \Diamond A`\Diamond (\Box 1 \times A)``\Diamond(1 \times A)`1 \times \Diamond A`\Diamond A;\st{1}{A}``\Diamond(\varepsilon_1 \times \id_A)```\Diamond\lambda`\lambda]
  %%   \morphism(0,1000)|m|/->/<0,-950>[`;\varepsilon_1 \times \id_{\Diamond A}]
  %%   \efig
  %%   $$
  %%   This diagram commutes by commutativity of the following diagram:
  %%   %% Equational version:
  %%   %% \begin{center}
  %%   %%   \begin{math}
  %%   %%     \begin{array}{rllllllll}
  %%   %%       & & \st{1}{A};\Diamond (\varepsilon_1 \times \id_A);\Diamond\lambda_A\\
  %%   %%       \text{(Definition of $\mathsf{st}$)}
  %%   %%       & = & (\eta_{\Box 1} \times \id_{\Diamond A});\p{\Box 1,\Diamond A};\Diamond (\varepsilon_1 \times \id_A);\Diamond\lambda_A\\
  %%   %%       \text{(Naturality of $\mathsf{p}$)}
  %%   %%       & = & (\eta_{\Box 1} \times \id_{\Diamond A});(\Diamond \varepsilon_1 \times \Diamond\id_A);\p{1,A};\Diamond\lambda_A\\
  %%   %%       \text{(Functoriality of $\times$)}
  %%   %%       & = & ((\eta_{\Box 1};\Diamond \varepsilon_1) \times (\id_{\Diamond A};\Diamond\id_A));\p{1,A};\Diamond\lambda_A\\
  %%   %%       & = & ((\eta_{\Box 1};\Diamond \varepsilon_1) \times (\id_{\Diamond A};\id_{\Diamond A}));\p{1,A};\Diamond\lambda_A\\
  %%   %%       \text{(Naturality of $\eta$)}
  %%   %%       & = & ((\varepsilon_1;\eta_{1}) \times (\id_{\Diamond A};\id_{\Diamond A}));\p{1,A};\Diamond\lambda_A\\
  %%   %%       \text{(Definition of $\mathsf{p}$)}
  %%   %%       & = & ((\varepsilon_1;\p{1}) \times (\id_{\Diamond A};\id_{\Diamond A}));\p{1,A};\Diamond\lambda_A\\
  %%   %%       \text{(Functoriality of $\times$)}
  %%   %%       & = & (\varepsilon_1 \times \id_{\Diamond A});(\p{1} \times \id_{\Diamond A});\p{1,A};\Diamond\lambda_A\\
  %%   %%       \text{($\Diamond$ is Symmetric Monoidal)}
  %%   %%       & = & (\varepsilon_1 \times \id_{\Diamond A});\lambda_{\Diamond A}\\
  %%   %%     \end{array}
  %%   %%   \end{math}
  %%   %% \end{center}
  %%   $$
  %%   \bfig
  %%   \square|amma|<1000,500>[
  %%     \Diamond\Box 1 \times \Diamond A`
  %%     \Diamond (\Box 1 \times A)`
  %%     \Diamond 1 \times \Diamond A`
  %%     \Diamond (1 \times A);
  %%     \p{\Box 1,A}`
  %%     \Diamond \varepsilon_1 \times \id_{\Diamond A}`
  %%     \Diamond (\varepsilon_1 \times \id_A)`
  %%     \p{1,A}]

  %%   \square(-1000,0)|amma|<1000,500>[
  %%     \Box 1 \times \Diamond A`
  %%     \Diamond\Box 1 \times \Diamond A`
  %%     1 \times \Diamond A`
  %%     \Diamond 1 \times \Diamond A;
  %%     \eta_{\Box 1} \times \id_{\Diamond A}`
  %%     \varepsilon_1 \times \id_{\Diamond A}`
  %%     \Diamond \varepsilon_1 \times \id_{\Diamond A}`
  %%     (\p{1} = \eta_1) \times \id_{\Diamond A}]

  %%   \qtriangle(-1000,-500)|mmm|/`->`->/<2000,500>[
  %%     1 \times \Diamond A`
  %%     \Diamond (1 \times A)`
  %%     \Diamond A;`
  %%     \lambda_{\Diamond A}`
  %%     \Diamond\lambda_A]

  %%   \place(500,250)[1]
  %%   \place(-500,250)[2]
  %%   \place(500,-200)[3]
  %%   \efig
  %%   $$
  %%   \noindent
  %%   Diagram 1 commutes by naturality of $\mathsf{p}$, diagram 2
  %%   commutes by naturality of $\eta$, and diagram 3 commutes because
  %%   $\Diamond$ is a symmetric monoidal functor.

  \item[] \textit{Case 3. unit $\eta$ of the monad and strength interact well, $\Box  A $ is a parameter}
    $$
    \bfig
    \btriangle<800,500>[
      \Box A \pd B`
      \Box A \pd \Diamond B`
      \Diamond(\Box A \pd B);
      \id_{\Box A} \pd \eta_{B}`
      \eta_{\Box A \times B}`
      \st{A}{B}]
    \efig
    $$

    The previous diagram commutes, because the following diagram commutes:
    $$
    \bfig
    \btriangle|ama|/->`->`->/<1500,500>[
      \Box A \pd B`
      \Box A \pd \Diamond B`
      \Diamond\Box A \pd \Diamond B;
      \id_{\Box A} \pd \eta_{B}`
      \eta_{\Box A} \pd \eta_B`
      \eta_{\Box A} \pd \id_{\Diamond B}]

    \qtriangle(0,0)/->``<-/<1500,500>[
      \Box A \pd B`
      \Diamond (\Box A \times B)`
      \Diamond\Box A \pd \Diamond B;
      \eta_{\Box A \times B}``
      \p{\Box A,B}]

    \place(250,200)[1]
    \place(1200,300)[2]
    \efig
    $$
    \noindent
    Diagram 1 clearly commutes, and diagram 2 commutes because $\eta$
    is a symmetric monoidal natural transformation.
    
  %% \item[] \textit{Case 3. co-unit of the comonad $\varepsilon$ and unit of the monad $\eta$ interact well?}
  %%   $$
  %%   \bfig
  %%   \hSquares|aamaaaa|/->`->`->```->`/[
  %%     \Box A \times 1`
  %%     \Box A \times \Diamond 1`
  %%     \Diamond (\Box A \times 1)`
  %%     \Box A`
  %%     A`
  %%     ;
  %%     \id_{\Box A} \times \eta_1`
  %%     \st{A}{1}`
  %%     \rho_{\Box A}```
  %%     \varepsilon_A`]
  %%   \qtriangle(1978,0)/->``->/<800,500>[\Diamond (\Box A \times 1)`\Diamond\Box A`\Diamond A;\Diamond \rho_{\Box A}``\Diamond\varepsilon_A]
  %%   \morphism(1060,0)/->/<1640,0>[`;\eta_A]
  %%   \efig
  %%   $$
  %%   \noindent
  %%   Recall that
  %%   $\st{A}{1} = (\eta_{\Box A} \pd \id_{\Diamond 1});\p{\Box A,1}$.
  %%   Now the previous diagram commutes, because the following diagram commutes:
  %%   $$
  %%   \bfig
  %%   \btriangle|mma|<1444,1000>[\Box A \pd 1`\Diamond\Box A \pd \Diamond 1`\Diamond(\Box A \pd 1);\eta_{\Box A} \pd \eta_1`\eta_{\Box A \pd 1}`\p{\Box A,1}]
  %%   \dtriangle(-1000,0)/->``->/<1000,1000>[\Box A \pd 1`\Box A \pd 1`\Diamond\Box A \pd \Diamond 1;\id_{\Box A} \pd \eta_1``\eta_{\Box A} \pd
  %%     \id_{\Diamond 1}]

  %%   \hSquares(0,0)/->`->```<-``/<1000>[\Box A \pd 1`\Box A`\Diamond\Box A```\Diamond (\Box A \times 1);\rho_{\Box A}`\eta_{\Box A}```\Diamond (\rho_{\Box A})``]

  %%   \square(746,1000)/->`<-`<-`/<698,500>[A`\Diamond A`\Box A`\Diamond\Box A;\eta_A`\varepsilon_A`\Diamond\varepsilon_A`]

  %%   \place(-400,300)[1]
  %%   \place(400,300)[2]
  %%   \place(1100,700)[3]
  %%   \place(1100,1250)[4]
  %%   \efig
  %%   $$
  %%   \noindent
  %%   Diagram 1 commutes by functorality of $\times$, diagram 2 commutes
  %%   because $\eta$ is a monoidal natural transformation, and diagrams
  %%   3 and 4 commute by naturality of $\eta$.

  \item[] \textit{Case 4. associativity $\alpha$ interacts with co-monoidicity of $\Box$}
    $$
    \bfig
    \vSquares|ammmmmm|/->`->`->```->`/[
      \Box A \times (\Box B \times \Diamond C)`
      \Box A \times \Diamond(\Box B \times C)`
      (\Box A \times \Box B) \times \Diamond C`
      \Diamond(\Box A \times (\Box B \times C))``
      \Diamond((\Box A \times \Box B) \times C);
      \id_{\Box A} \pd \st{B}{C}`
      \alpha^{-1}`
      \st{A}{\Box B \times C}```
      \Diamond\alpha^{-1}`]
    \morphism(1554,0)|m|/->/<0,-500>[`\Diamond(\Box(A \times B) \times C);\Diamond(\m{A,B} \times \id_C)]
    
    \morphism(0,500)|m|/->/<0,-1000>[`\Box(A \times B) \times \Diamond C;\m{A,B} \times \id_{\Diamond C}]

    \morphism(350,-500)|a|/->/<800,0>[`;\st{A \times B}{C}]
    \efig
    $$
    \noindent
    Recall that:
    \[
    \begin{array}{rlll}
      \st{B}{C}              & = & (\eta_{\Box B} \pd \id_{\Diamond C});\p{\Box B,C}\\
      \st{A \pd B}{C}        & = & (\eta_{\Box (A \pd B)} \pd \id_{\Diamond C});\p{\Box (A \pd B),C}\\
      \st{A}{\Box B \pd C} & = & (\eta_{\Box A} \pd \id_{\Diamond (\Box B \pd C)});\p{\Box A,(\Box B \pd C)}\\
    \end{array}
    \]
    In addition, we require the following diagram (whose commutativity
    is implied by the fact that $\Diamond$ is a symmetric monoidal
    functor):
    $$
    \bfig
    \vSquares|ammmmma|/->`->`->``->`->`->/[
      \Diamond A \pd (\Diamond B \pd \Diamond C)`
      (\Diamond A \pd \Diamond B) \pd \Diamond C`
      \Diamond A \pd \Diamond (B \pd C)`
      \Diamond (A \pd B) \pd \Diamond C`
      \Diamond (A \pd (B \pd C))`
      \Diamond ((A \pd B) \pd C);
      \alpha^{-1}_{\Diamond A,\Diamond B,\Diamond C}`
      \id_{\Diamond A} \pd \p{B,C}`
      \p{A,B} \pd \id_{\Diamond C}``
      \p{A,B \pd C}`
      \p{A \pd B,C}`
      \Diamond\alpha^{-1}_{A,B,C}]
    \efig
    $$
    \noindent
    Finally, this case follows because the following diagram commutes:
    \begin{center}
      \rotatebox{90}{$
    \bfig
    \btriangle|mmm|<1769,1000>[
      \Box A \pd (\Diamond\Box B \pd \Diamond C)`
      \Diamond\Box A \pd (\Diamond\Box B \pd \Diamond C)`
      \Diamond\Box A \pd (\Diamond\Box B \pd C);
      \eta_{\Box A} \pd \id_{\Diamond \Box B \pd C}`
      \eta_{\Box A} \pd \p{\Box B,C}`
      \id_{\Diamond\Box A} \pd \p{\Box B,C}]

    \qtriangle|mam|/->``->/<1769,1000>[
      \Box A \pd (\Diamond\Box B \pd \Diamond C)`
      \Box A \pd \Diamond (\Box B \pd C)`
      \Diamond\Box A \pd (\Diamond\Box B \pd C);
      \id_{\Box A} \pd \p{\Box B,C}``
      \eta_{\Box A} \pd \id_{\Diamond (\Box B \pd C)}]    

    \qtriangle(-1800,0)|mmm|<1800,1000>[
      \Box A \pd (\Box B \pd \Diamond C)`
      \Box A \pd (\Diamond\Box B \pd \Diamond C)`
      \Diamond\Box A \pd (\Diamond\Box B \pd \Diamond C);
      \id_{\Box A} \pd (\eta_{\Box B} \pd \id_{\Diamond C})`
      \eta_{\Box A} \pd (\eta_{\Box B} \pd \id_{\Diamond C})`
      \eta_{\Box A} \pd \id_{\Diamond \Box B \pd C}]

    \square(0,-500)|mmmm|/`->`->`/<1769,500>[
      \Diamond\Box A \pd (\Diamond\Box B \pd \Diamond C)`
      \Diamond\Box A \pd (\Diamond\Box B \pd C)`
      (\Diamond\Box A \pd \Diamond\Box B) \pd \Diamond C`
      \Diamond (\Box A \pd (\Box B \pd C));`
      \alpha_{\Diamond\Box A,\Diamond\Box B,\Diamond C}`
      \p{\Box A,\Box B \pd C}`]   

    \square(0,-1000)|mmmm|/`->`->`/<1769,500>[
      (\Diamond\Box A \pd \Diamond\Box B) \pd \Diamond C`
      \Diamond (\Box A \pd (\Box B \pd C))`
      \Diamond (\Box A \pd \Box B) \pd \Diamond C`
      \Diamond ((\Box A \pd \Box B) \pd C);`
      \p{\Box A,\Box B} \pd \id_{\Diamond C}`
      \Diamond \alpha_{\Box A,\Box B,C}`]        

    \square(0,-1500)|mmmm|<1769,500>[
      \Diamond (\Box A \pd \Box B) \pd \Diamond C`
      \Diamond ((\Box A \pd \Box B) \pd C)`
      \Diamond\Box(A \pd B) \pd \Diamond C`
      \Diamond (\Box(A \pd B) \pd C);
      \p{\Box A \pd \Box B, C}`
      \Diamond \m{A,B} \pd \id_{\Diamond C}`
      \Diamond (\m{A,B} \pd \id_C)`
      \p{\Box (A \pd B),C}]

    \btriangle(-1800,-500)|mmm|/->``->/<1800,1500>[
      \Box A \pd (\Box B \pd \Diamond C)`
      (\Box A \pd \Box B) \pd \Diamond C`
      (\Diamond\Box A \pd \Diamond\Box B) \pd \Diamond C;
      \alpha_{\Box A,\Box B,\Diamond C}``
      (\eta_{\Box A} \pd \eta_{\Box B}) \pd \id_{\Diamond C}]

    \qtriangle(-1800,-1000)|mmm|/`->`/<1800,500>[
      (\Box A \pd \Box B) \pd \Diamond C`
      (\Diamond\Box A \pd \Diamond\Box B) \pd \Diamond C`
      \Diamond (\Box A \pd \Box B) \pd \Diamond C;`
      \eta_{\Box A \pd \Box B} \pd \id_{\Diamond C}`]

    \btriangle(-1800,-1500)|mmm|/->``->/<1800,1000>[
      (\Box A \pd \Box B) \pd \Diamond C`
      \Box(A \pd B) \pd \Diamond C`
      \Diamond\Box(A \pd B) \pd \Diamond C;
      \m{A,B} \pd \id_{\Diamond C}``
      \eta_{\Box (A \pd B)} \pd \id_{\Diamond C}]

    \place(1200,700)[1]
    \place(500,300)[2]
    \place(900,-500)[3]
    \place(900,-1250)[4]
    \place(-500,700)[5]
    \place(-1000,0)[6]
    \place(-400,-700)[7]
    \place(-1000,-1100)[8]
    \efig
    $}
    \end{center}
    Diagrams 1, 2 and 5 commute by functorality of $\times$, diagram 3
    commutes by the additional diagram from above, diagram 4 commutes
    by naturality of $\mathsf{p}$, diagram 6 commutes by naturality of
    $\alpha$, diagram 7 commutes by the fact that $\eta$ is a monoidal
    natural transformation, and diagram 8 commutes by naturality of
    $\eta$.
    

  \item[] \textit{Case 5.  strength interacts with monoidicity of $\Diamond$}
    $$
    \bfig
    \vSquares|ammmmma|/->`->```->``->/[
      \Box A \times \Diamond\Diamond B`
      \Box A \times \Diamond B`
      \Diamond(\Box A \times \Diamond B)``
      \Diamond\Diamond(\Box A \times B)`
      \Diamond(\Box A \times B);
      \id_{\Box A} \pd \mu_{B}`
      \st{A}{\Diamond B}```
      \Diamond(\st{A}{B})``
      \mu_{\Box A \pd B}]
    \morphism(1150,1000)|m|<0,-920>[`;\st{A}{B}]
    \efig
    $$
    \noindent
    Recall that:
    \[
    \begin{array}{rlll}
      \st{A}{B} & = & (\eta_{\Box A} \pd \id_{\Diamond B});\p{\Box A,B}\\
      \st{A}{\Diamond B} & = & (\eta_{\Box A} \pd \id_{\Diamond \Diamond B});\p{\Box A,\Diamond B}\\
    \end{array}
    \]
    This case follows from the fact that the following diagram
    commutes:
    \begin{center}
      \rotatebox{90}{$\bfig
    \qtriangle|mmm|<1500,1000>[
      \Box A \pd \Diamond\Diamond B`
      \Box A \pd \Diamond B`
      \Diamond\Box A \pd \Diamond B;
      \id_{\Box A} \pd \mu_B`
      \eta_{\Box A} \pd \mu_B`
      \eta_{\Box A} \pd \id_{\Diamond B}]

    \morphism(-1500,1000)|m|/<-/<1500,0>[
      \Diamond\Box A \pd \Diamond\Diamond B`
      \Box A \pd \Diamond\Diamond B;
      \eta_{\Box A} \pd \id_{\Diamond\Diamond B}]

    \btriangle(-1500,0)|mmm|<3000,1000>[
      \Diamond\Box A \pd \Diamond\Diamond B`
      \Diamond\Diamond\Box A \pd \Diamond\Diamond B`
      \Diamond\Box A \pd \Diamond B;
      \Diamond\eta_{\Box A} \pd \id_{\Diamond\Diamond B}`
      \id_{\Diamond\Box A} \pd \mu_B`
      \mu_{\Box A} \pd \mu_B]

    \square(-3000,0)|mmmm|/<-`->``<-/<1500,1000>[
      \Diamond(\Box A \pd \Diamond B)`
      \Diamond\Box A \pd \Diamond\Diamond B`
      \Diamond(\Diamond\Box A \pd \Diamond B)`
      \Diamond\Diamond\Box A \pd \Diamond\Diamond B;
      \p{\Box A,\Diamond B}`
      \Diamond (\eta_{\Box A} \pd \id_{\Diamond B})``
      \p{\Diamond\Box A,\Diamond B}]

    \square(-3000,-500)|mmmm|/`->`->`->/<4500,500>[
      \Diamond (\Diamond\Box A \pd \Diamond B)`
      \Diamond\Box A \pd \Diamond B`
      \Diamond\Diamond (\Box A \pd B)`
      \Diamond (\Box A \pd B);`
      \Diamond (\p{\Box A,B})`
      \p{\Box A,B}`
      \mu_{\Box A \pd B}]

    \place(-2300,500)[1]
    \place(-800,-250)[2]
    \place(-800,400)[3]
    \place(0,700)[4]
    \place(1050,700)[5]
    \efig$}
    \end{center}       
    Diagram commutes by naturality of $\mathsf{p}$, diagram 2 commutes
    because $\mu$ is a monoidal natural transformation, diagram 3
    commutes because $\mu$ is the monadic multiplication and by
    functorality of $\times$, and diagrams 4 and 5 commute by
    functoriality of $\times$.
    
  \item[] \textit{Case 6. commuting $\beta$ interacts with $\Diamond$}
    $$
    \bfig
    \hSquares|aamamaa|/``->``->`->`->/[
      \Diamond B \times \Box A``
      \Box A \times \Diamond B`
      \Diamond B \times \Diamond \Box A`
      \Diamond (B \times \Box A)`
      \Diamond (\Box A \times B);``
      \id_{\Diamond B} \pd \eta_{\Box A}``
      st_{A,B}`
      \p{B,\Box A}`
      \Diamond\beta_{B,\Box A}]
    \morphism(200,500)<1817,0>[`;\beta_{\Diamond B,\Box A}]
    \efig
    $$
    \noindent
    The previous diagram commutes by commutativity of the following
    diagram:
    $$
    \bfig
    \vSquares|ammmmma|[
      \Diamond B \times \Box A`
      \Box A \times \Diamond B`
      \Diamond B \times \Diamond\Box A`
      \Diamond\Box A \times \Diamond B`
      \Diamond (B \times \Box A)`
      \Diamond (\Box A \times B);
      \beta_{\Diamond B,\Box A}`
      \id_{\Diamond B} \times \eta_{\Box A}`
      \eta_{\Box A} \times \id_{\Diamond B}`
      \beta_{\Diamond B,\Diamond\Box A}`
      \p{B,\Box A}`
      \p{\Box A,B}`
      \Diamond\beta_{B,\Box A}]

    \place(600,750)[1]
    \place(600,250)[2]
    \efig
    $$
    \noindent
    Diagram 1 commutes because $\beta$ is a symmetric monoidal
    functor, and diagram 2 commutes by naturality of $\beta$.

  \item[] \textit{Case 7.  $\varepsilon$ interacts with $\Diamond$ and its monoidicity} 
    $$
    \bfig
    \hSquares|aamamaa|/``->``->`->`->/[
      \Box A \times \Diamond B``
      \Diamond(\Box A \times B)`
      A \times \Diamond B`
      \Diamond A \times \Diamond B`
      \Diamond (A \times B);``
      \varepsilon_A \times \id_{\Diamond B}``
      \Diamond(\varepsilon_A \times \id_{B})`
      \eta_A \times \id_{\Diamond B}`
      \p{A,B}]
    \morphism(200,500)<1605,0>[`;\st{A}{B}]
    \efig
    $$
    \noindent
    The previous diagram commutes by commutativity of the following
    diagram:
    $$
    \bfig
    \hSquares|aammmaa|[
      \Box A \times \Diamond B`
      \Diamond\Box A \times \Diamond B`
      \Diamond (\Box A \times B)`
      A \times \Diamond B`
      \Diamond A \times \Diamond B`
      \Diamond (A \times B);
      \eta_{\Box A} \times \id_{\Diamond B}`
      \p{\Box A,B}`
      \varepsilon_A \times \id_{\Diamond B}`
      \Diamond\varepsilon_A \times \id_{\Diamond B}`
      \Diamond (\varepsilon_A \times \id_B)`
      \eta_A \times \id_{\Diamond B}`
      \p{A,B}]
    \place(1750,250)[1]
    \place(600,250)[2]
    \efig
    $$
    \noindent
    Diagram 1 commutes by naturality of $\mathsf{p}$, and diagram 2
    commutes by naturality of $\eta$.
  \end{itemize}

\end{proof}\begin{proof}
  We must show that given the definition of an adjoint CS4 categorical
  model (Definition~\ref{def:CS4-single-adjoint-cat-model}) we can
  define an appropriate monad and comonad on a CCC with coproducts
  where the monad is strong with respect to the comonad.

  Suppose $(H,m)$ and $(J,n)$ are the adjoint monoidal functors given
  in Definition~\ref{def:CS4-single-adjoint-cat-model}, and define
  $\Box = JH$ and $\Diamond = HJ$.  By definition we assumed that
  $(\Box, q)$, where $q_{A,B} : \Box A \times \Box B \to \Box (A
  \times B)$, is monoidal, but we must show that $\Diamond$ is also
  monoidal.  We know that both $(H,n)$ and $(J,m)$ are monoidal
  endofunctors on $\cat{C}$ which implies that their composition
  $\Diamond$ is monoidal where
  \[
  \begin{array}{lll}
    \mathsf{p}_{1} = \eta_{1} : 1 \to \Diamond 1\\
    \mathsf{p}_{A,B} = \m{HA,HB};J(\mathsf{n}_{A,B})
    \colon \Diamond A \pd \Diamond B \mto \Diamond(A \pd B)
  \end{array}
  \]
  and the following diagrams commute (proofs omitted):
  \begin{mathpar}
    \scriptsize
    \bfig
    \vSquares|ammmmma|/->`->`->``->`->`->/[
      (\Diamond A \times \Diamond B) \times \Diamond C`
      \Diamond A \times (\Diamond B \times \Diamond C)`
      \Diamond(A \times B) \times \Diamond C`
      \Diamond A \times \Diamond(B \times C)`
      \Diamond ((A \times B) \times C)`
      \Diamond (A \times (B \times C));
      \alpha`
      \mathsf{p}_{A,B} \times \id_{\Diamond C}`
      \id_{\Diamond A} \times \mathsf{p}_{B,C}``
      \mathsf{p}_{A \times B,C}`
      \mathsf{p}_{A,B \times C}`
      \Diamond \alpha]
    \efig
    \and
    \bfig
    \hSquares|ammmmaa|/->``->`<-``->`/[
      1 \times \Diamond A`
      \Diamond A``
      \Diamond 1 \times \Diamond A`
      \Diamond(1 \times A)`;
      \lambda_{\Diamond A}``
      \mathsf{p}_{1} \times \id_{\Diamond A}`
      \Diamond \lambda_A``
      \mathsf{p}_{1,A}`]
    \efig
    \and
    \bfig
    \hSquares|ammmmaa|/->``->`<-``->`/[
      \Diamond A \times 1`
      \Diamond A``
      \Diamond A \times \Diamond 1`
      \Diamond(A \times 1)`;
      \rho_{\Diamond A}``
      \id_{\Diamond A} \times \mathsf{p}_{1}`
      \Diamond \rho_A``
      \mathsf{p}_{A,1}`]
    \efig
    \and
    \bfig
    \hSquares|ammmmaa|/->``->`->``->`/[
      \Diamond A \times \Diamond B`
      \Diamond B \times \Diamond A``
      \Diamond (A \times B)`
      \Diamond (B \times A)`;
      \beta_{\Diamond A,\Diamond B}``
      \mathsf{p}_{A,B}`
      \mathsf{p}_{B,A}``
      \Diamond\beta_{A,B}`]
    \efig
  \end{mathpar}

  Furthermore, suppose $J \dashv H$, where the unit, $\varepsilon :
  \Box A \to A$, and the counit, $\eta : A \to \Diamond A$, are
  monoidal natural transformations.  This implies that the following
  diagrams commute:
  \begin{mathpar}
    \bfig
    \btriangle<800,500>[A \times B`\Diamond A \times \Diamond B`\Diamond (A \times B);\eta_A \times \eta_B`\eta_{A \times B}`\mathsf{p}_{A,B}]
    \efig
    \and
    \bfig
    \qtriangle<800,500>[\Box A \times \Box B`\Box (A \times B)`A \times B;\mathsf{q}_{A,B}`\varepsilon_A \times \varepsilon_B`\varepsilon_{A \times B}]
    \efig
    \and
    \bfig
    \qtriangle<800,500>[1`J 1`\Diamond 1;n_{1}`\eta_1`J m_1]       
    \efig
    \and
    \bfig
    \hSquares|ammmmaa|/->``=`->``<-`/[
      \Box 1`
      1``
      \Box 1`
      H 1`;
      \varepsilon_1```
      m_1``
      H n_1`]
    \efig
    \and
    \bfig
    \qtriangle<800,500>[H A`H\Box A`H A;\eta_{H A}`\id_{H A}`H\varepsilon_A]       
    \efig
    \and
    \bfig
    \qtriangle<800,500>[J A`\Box J A`J A;J\eta_A`\id_{J A}`\varepsilon_{J A}]
    \efig    
  \end{mathpar}
  It is a well-known fact about adjoints that $(\Box, \varepsilon,
  \delta)$, where $\delta : \Box A \to \Box\Box A$ is a comonad, and
  $(\Diamond, \eta, \mu)$, where $\mu : \Diamond\Diamond A \to
  \Diamond A$ is a monad.  In addition, $\mu$ and $\delta$ are monoidal
  natural transformations where we have the following:
  \[
  \begin{array}{lll}
    \d{1} = \p{1};\Diamond\p{1} : 1 \mto \Diamond^2 1\\
    \d{A,B} =  \p{\Diamond A,\Diamond B};\Diamond\p{A,B} : \Diamond^2 A \times \Diamond^2 B \mto \Diamond^2 (A \times B)\\
    \\
    \b{1} = \q{1};\Box\q{1} : 1 \mto \Box^2 1\\
    \b{A,B} = \q{\Box A,\Box B};\Box\q{A,B} : \Box^2 A \times \Box^2 B \mto \Box^2 (A \times B)\\
  \end{array}
  \]
  Thus, the following diagrams commute:
  \begin{mathpar}
    \bfig
    \hSquares|ammmmaa|/->``->`->``->`/[
      \Diamond^3 A`
      \Diamond^2 A``
      \Diamond^2 A`
      \Diamond A`;
      \Diamond \mu_A``
      \mu_{\Diamond A}`
      \mu_A``
      \mu_A`]
    \efig
    \and
    \bfig
    \qtriangle/->`=`->/<800,500>[\Diamond A`\Diamond^2 A`\Diamond A;\eta_{\Diamond A}``\mu_A]
    \btriangle(0,0)/->`=`->/<800,500>[\Diamond A`\Diamond^2 A`\Diamond A;\Diamond \eta_{A}``\mu_A]
    \efig
    \and
    \bfig
    \hSquares|ammmmaa|/->``->`->``->`/[
      \Box A`
      \Box^2 A``
      \Box^2 A`
      \Box^3 A`;
      \delta_A``
      \delta_A`
      \delta_{\Box A}``
      \Box\delta_A`]
    \efig
    \and
    \bfig
    \qtriangle/->`=`->/<800,500>[\Box A`\Box^2 A`\Box A;\delta_A``\Box \varepsilon]
    \btriangle(0,0)/->`=`->/<800,500>[\Box A`\Box^2 A`\Box A;\delta_A``\varepsilon_{\Box A}]
    \efig
    \and
    %% \bfig   
    %% \vSquares|ammmmma|/->`->```->``->/[
    %%   \Diamond^2 A \times \Diamond^2 B`
    %%   \Diamond A \times \Diamond B`
    %%   \Diamond(\Diamond A \times \Diamond B)``
    %%   \Diamond^2(A \times B)`
    %%   \Diamond(A \times B);
    %%   \mu_A \times \mu_B`
    %%   \p{\Diamond A,\Diamond B}```
    %%   \Diamond\p{A,B}``
    %%   \mu_{A \times B}]
    %% \morphism(1108,0)/<-/<0,1000>[\Diamond(A \times B)`\Diamond A \times \Diamond B;\p{A,B}]
    %% \efig
    \and
    \bfig
    \square<1000,1000>[
      \Diamond^2 A \times \Diamond^2 B`
      \Diamond A \times \Diamond B`
      \Diamond^2(A \times B)`
      \Diamond(A \times B);
      \mu_A \times \mu_B`
      \d{A,B}`
      \p{A,B}`
      \mu_{A \times B}]
    \efig
    \and
    \bfig
    \Vtriangle/->`<-`<-/[
      \Diamond^2 1`
      \Diamond 1`
      1;
      \mu_1`
      \d{1}`
      \p{1}]
    \efig
    \and        
    \bfig
    \square<1000,1000>[
      \Box A \times \Box B`
      \Box^2 A \times \Box^2 B`
      \Box(A \times B)`
      \Box^2(A \times B);
      \delta_A \times \delta_B`
      \q{A,B}`
      \b{A,B}`
      \delta_{A \times B}]
    %% \vSquares|ammmmma|/->``->```->`->/[
    %%   \Box A \times \Box B`
    %%   \Box^2 A \times \Box^2 B``
    %%   \Box(\Box A \times \Box B)`
    %%   \Box(A \times B)`
    %%   \Box^2(A \times B);
    %%   \varepsilon_A \times \varepsilon_B``
    %%   \q{\Box A,\Box B}```
    %%   \Box\q{A,B}`
    %%   \varepsilon_{A \times B}]
    %% \morphism(0,0)/<-/<0,1000>[\Box(A \times B)`\Box A \times \Box B;\q{A,B}]
    \efig
    \and
    \bfig
    \Vtriangle/->`<-`<-/[
      \Box 1`
      \Box^2 1`
      1;
      \delta_1`
      \q{1}`
      \d{1}]
    \efig
  \end{mathpar}

  We can now define the $\Box$-strength map as follows:
  \[
  \st{A}{B} = (\eta_{\Box A} \pd \id_{\Diamond B});\mathsf{p}_{\Box A,B} : \Box A \pd \Diamond B \mto \Diamond(\Box A \pd B)
  \]
  We can see that $\st{A}{B}$ is a natural transformation, because it
  is defined as a composition of natural transformations.
  
  %% To prove that the appropriate diagrams commute we first note that
  %% the triangle  
  Next we must show that all of the appropriate diagrams given in
  Definition~\ref{def:comonad-strong-monad} commute.
  \begin{itemize}
  \item[] \textit{Case 1. the first projection interacts with $\Diamond$:}
    \[
    \bfig
    \square|amma|<850,600>[
      \Box A \times \Diamond B`
      \Diamond (\Box A \times B)`
      \Box A`
      \Diamond\Box A;
      \st{A}{B}`
      \pi_1`
      \Diamond\pi_1`
      \eta_A]
    \efig
    \]
    This diagrams commutes, because the following diagram commutes:
    \[
    \bfig
    \square|ammm|/->`->``/<1500,2000>[
      \Box A \times \Diamond B`
      \Diamond\Box A \times \Diamond B`
      \Box A`;
      \eta_{\Box A} \times \id_{\Diamond B}`
      \pi_1``]

    \square(1500,0)|ammm|/->``->`/<1500,2000>[
      \Diamond\Box A \times \Diamond B`
      \Diamond (\Box A \times B)``
      \Diamond\Box A;
      \p{\Box A,B}``
      \Diamond\pi_1`]

    \morphism<3000,0>[\Box A`\Diamond\Box A;\eta_{\Box A}]

    \morphism(484,1500)<1000,0>[
      \Box A \times 1`
      \Diamond\Box A \times \Diamond 1;
      \eta_{\Box A} \times \eta_1]

    \morphism(1484,1500)<700,-500>[
      \Diamond\Box A \times \Diamond 1`
      \Diamond(\Box A \times 1);
      \p{\Box A,1}]

    \morphism(484,1500)|m|/{@{>}@/_1em/}/<1700,-500>[
      \Box A \times 1`
      \Diamond(\Box A \times 1);
      \eta_{\Box A \times 1}]

    \morphism(484,1500)|m|<700,-800>[
      \Box A \times 1`
      \Box A;
      \pi_1]

    \morphism(0,2000)|m|<484,-500>[
      \Box A \times \Diamond B`
      \Box A \times 1;
      \id_{\Box A} \times \t_{\Diamond B}]

    \morphism(1500,2000)|m|<-17,-500>[
      \Diamond\Box A \times \Diamond B`
      \Diamond\Box A \times \Diamond 1;
      \id_{\Diamond\Box A} \times \Diamond\t_{B}]

    \morphism(3000,2000)|m|<-815,-1000>[
      \Diamond(\Box A \times B)`
      \Diamond(\Box A \times 1);
      \Diamond (id_{\Box A} \times \t_{B})]

    \morphism(2184,1000)|m|<816,-1000>[
      \Diamond(\Box A \times 1)`
      \Diamond\Box A;
      \Diamond\pi_1]

    \morphism(1184,700)|m|<1816,-700>[
      \Box A`
      \Diamond\Box A;
      \eta_{\Box A}]
    
    \place(2700,1000)[1]
    \place(2150,1700)[2]
    \place(800,1750)[3]
    \place(1300,1300)[4]
    \place(1800,800)[5]
    \place(500,500)[6]
    \efig    
    \]
    Diagrams 1 and 6 commute because we are in a cartesian closed
    category, diagram 2 commutes by naturality of $\p{}$, diagram 3
    commutes because $\Diamond$ is a product functor, diagram 4
    commutes because $\eta$ is the unit of a symmetric monoidal
    adjunction, and diagram 5 commutes by naturality of $\eta$.
    
  \item[] \textit{Case 2. the second projection interacts with $\Diamond$:}
    \[
    \bfig
    \qtriangle<850,600>[
      \Box A \times \Diamond B`
      \Diamond (\Box A \times B)`
      \Diamond B;
      \st{A}{B}`
      \pi_2`
      \Diamond\pi_2]
    \efig
    \]
    This case is similar to the previous case.
    
    %% \item[] \textit{Case 1. the object 1 behaves as the unit for products}
  %%   $$
  %%   \bfig
  %%   \vSquares|ammmmma|/>``>```>`>/[\Box 1 \times \Diamond A`\Diamond (\Box 1 \times A)``\Diamond(1 \times A)`1 \times \Diamond A`\Diamond A;\st{1}{A}``\Diamond(\varepsilon_1 \times \id_A)```\Diamond\lambda`\lambda]
  %%   \morphism(0,1000)|m|/->/<0,-950>[`;\varepsilon_1 \times \id_{\Diamond A}]
  %%   \efig
  %%   $$
  %%   This diagram commutes by commutativity of the following diagram:
  %%   %% Equational version:
  %%   %% \begin{center}
  %%   %%   \begin{math}
  %%   %%     \begin{array}{rllllllll}
  %%   %%       & & \st{1}{A};\Diamond (\varepsilon_1 \times \id_A);\Diamond\lambda_A\\
  %%   %%       \text{(Definition of $\mathsf{st}$)}
  %%   %%       & = & (\eta_{\Box 1} \times \id_{\Diamond A});\p{\Box 1,\Diamond A};\Diamond (\varepsilon_1 \times \id_A);\Diamond\lambda_A\\
  %%   %%       \text{(Naturality of $\mathsf{p}$)}
  %%   %%       & = & (\eta_{\Box 1} \times \id_{\Diamond A});(\Diamond \varepsilon_1 \times \Diamond\id_A);\p{1,A};\Diamond\lambda_A\\
  %%   %%       \text{(Functoriality of $\times$)}
  %%   %%       & = & ((\eta_{\Box 1};\Diamond \varepsilon_1) \times (\id_{\Diamond A};\Diamond\id_A));\p{1,A};\Diamond\lambda_A\\
  %%   %%       & = & ((\eta_{\Box 1};\Diamond \varepsilon_1) \times (\id_{\Diamond A};\id_{\Diamond A}));\p{1,A};\Diamond\lambda_A\\
  %%   %%       \text{(Naturality of $\eta$)}
  %%   %%       & = & ((\varepsilon_1;\eta_{1}) \times (\id_{\Diamond A};\id_{\Diamond A}));\p{1,A};\Diamond\lambda_A\\
  %%   %%       \text{(Definition of $\mathsf{p}$)}
  %%   %%       & = & ((\varepsilon_1;\p{1}) \times (\id_{\Diamond A};\id_{\Diamond A}));\p{1,A};\Diamond\lambda_A\\
  %%   %%       \text{(Functoriality of $\times$)}
  %%   %%       & = & (\varepsilon_1 \times \id_{\Diamond A});(\p{1} \times \id_{\Diamond A});\p{1,A};\Diamond\lambda_A\\
  %%   %%       \text{($\Diamond$ is Symmetric Monoidal)}
  %%   %%       & = & (\varepsilon_1 \times \id_{\Diamond A});\lambda_{\Diamond A}\\
  %%   %%     \end{array}
  %%   %%   \end{math}
  %%   %% \end{center}
  %%   $$
  %%   \bfig
  %%   \square|amma|<1000,500>[
  %%     \Diamond\Box 1 \times \Diamond A`
  %%     \Diamond (\Box 1 \times A)`
  %%     \Diamond 1 \times \Diamond A`
  %%     \Diamond (1 \times A);
  %%     \p{\Box 1,A}`
  %%     \Diamond \varepsilon_1 \times \id_{\Diamond A}`
  %%     \Diamond (\varepsilon_1 \times \id_A)`
  %%     \p{1,A}]

  %%   \square(-1000,0)|amma|<1000,500>[
  %%     \Box 1 \times \Diamond A`
  %%     \Diamond\Box 1 \times \Diamond A`
  %%     1 \times \Diamond A`
  %%     \Diamond 1 \times \Diamond A;
  %%     \eta_{\Box 1} \times \id_{\Diamond A}`
  %%     \varepsilon_1 \times \id_{\Diamond A}`
  %%     \Diamond \varepsilon_1 \times \id_{\Diamond A}`
  %%     (\p{1} = \eta_1) \times \id_{\Diamond A}]

  %%   \qtriangle(-1000,-500)|mmm|/`->`->/<2000,500>[
  %%     1 \times \Diamond A`
  %%     \Diamond (1 \times A)`
  %%     \Diamond A;`
  %%     \lambda_{\Diamond A}`
  %%     \Diamond\lambda_A]

  %%   \place(500,250)[1]
  %%   \place(-500,250)[2]
  %%   \place(500,-200)[3]
  %%   \efig
  %%   $$
  %%   \noindent
  %%   Diagram 1 commutes by naturality of $\mathsf{p}$, diagram 2
  %%   commutes by naturality of $\eta$, and diagram 3 commutes because
  %%   $\Diamond$ is a symmetric monoidal functor.

  \item[] \textit{Case 3. unit $\eta$ of the monad and strength interact well, $\Box  A $ is a parameter}
    $$
    \bfig
    \btriangle<800,500>[
      \Box A \pd B`
      \Box A \pd \Diamond B`
      \Diamond(\Box A \pd B);
      \id_{\Box A} \pd \eta_{B}`
      \eta_{\Box A \times B}`
      \st{A}{B}]
    \efig
    $$

    The previous diagram commutes, because the following diagram commutes:
    $$
    \bfig
    \btriangle|ama|/->`->`->/<1500,500>[
      \Box A \pd B`
      \Box A \pd \Diamond B`
      \Diamond\Box A \pd \Diamond B;
      \id_{\Box A} \pd \eta_{B}`
      \eta_{\Box A} \pd \eta_B`
      \eta_{\Box A} \pd \id_{\Diamond B}]

    \qtriangle(0,0)/->``<-/<1500,500>[
      \Box A \pd B`
      \Diamond (\Box A \times B)`
      \Diamond\Box A \pd \Diamond B;
      \eta_{\Box A \times B}``
      \p{\Box A,B}]

    \place(250,200)[1]
    \place(1200,300)[2]
    \efig
    $$
    \noindent
    Diagram 1 clearly commutes, and diagram 2 commutes because $\eta$
    is a symmetric monoidal natural transformation.
    
  %% \item[] \textit{Case 3. co-unit of the comonad $\varepsilon$ and unit of the monad $\eta$ interact well?}
  %%   $$
  %%   \bfig
  %%   \hSquares|aamaaaa|/->`->`->```->`/[
  %%     \Box A \times 1`
  %%     \Box A \times \Diamond 1`
  %%     \Diamond (\Box A \times 1)`
  %%     \Box A`
  %%     A`
  %%     ;
  %%     \id_{\Box A} \times \eta_1`
  %%     \st{A}{1}`
  %%     \rho_{\Box A}```
  %%     \varepsilon_A`]
  %%   \qtriangle(1978,0)/->``->/<800,500>[\Diamond (\Box A \times 1)`\Diamond\Box A`\Diamond A;\Diamond \rho_{\Box A}``\Diamond\varepsilon_A]
  %%   \morphism(1060,0)/->/<1640,0>[`;\eta_A]
  %%   \efig
  %%   $$
  %%   \noindent
  %%   Recall that
  %%   $\st{A}{1} = (\eta_{\Box A} \pd \id_{\Diamond 1});\p{\Box A,1}$.
  %%   Now the previous diagram commutes, because the following diagram commutes:
  %%   $$
  %%   \bfig
  %%   \btriangle|mma|<1444,1000>[\Box A \pd 1`\Diamond\Box A \pd \Diamond 1`\Diamond(\Box A \pd 1);\eta_{\Box A} \pd \eta_1`\eta_{\Box A \pd 1}`\p{\Box A,1}]
  %%   \dtriangle(-1000,0)/->``->/<1000,1000>[\Box A \pd 1`\Box A \pd 1`\Diamond\Box A \pd \Diamond 1;\id_{\Box A} \pd \eta_1``\eta_{\Box A} \pd
  %%     \id_{\Diamond 1}]

  %%   \hSquares(0,0)/->`->```<-``/<1000>[\Box A \pd 1`\Box A`\Diamond\Box A```\Diamond (\Box A \times 1);\rho_{\Box A}`\eta_{\Box A}```\Diamond (\rho_{\Box A})``]

  %%   \square(746,1000)/->`<-`<-`/<698,500>[A`\Diamond A`\Box A`\Diamond\Box A;\eta_A`\varepsilon_A`\Diamond\varepsilon_A`]

  %%   \place(-400,300)[1]
  %%   \place(400,300)[2]
  %%   \place(1100,700)[3]
  %%   \place(1100,1250)[4]
  %%   \efig
  %%   $$
  %%   \noindent
  %%   Diagram 1 commutes by functorality of $\times$, diagram 2 commutes
  %%   because $\eta$ is a monoidal natural transformation, and diagrams
  %%   3 and 4 commute by naturality of $\eta$.

  \item[] \textit{Case 4. associativity $\alpha$ interacts with co-monoidicity of $\Box$}
    $$
    \bfig
    \vSquares|ammmmmm|/->`->`->```->`/[
      \Box A \times (\Box B \times \Diamond C)`
      \Box A \times \Diamond(\Box B \times C)`
      (\Box A \times \Box B) \times \Diamond C`
      \Diamond(\Box A \times (\Box B \times C))``
      \Diamond((\Box A \times \Box B) \times C);
      \id_{\Box A} \pd \st{B}{C}`
      \alpha^{-1}`
      \st{A}{\Box B \times C}```
      \Diamond\alpha^{-1}`]
    \morphism(1554,0)|m|/->/<0,-500>[`\Diamond(\Box(A \times B) \times C);\Diamond(\m{A,B} \times \id_C)]
    
    \morphism(0,500)|m|/->/<0,-1000>[`\Box(A \times B) \times \Diamond C;\m{A,B} \times \id_{\Diamond C}]

    \morphism(350,-500)|a|/->/<800,0>[`;\st{A \times B}{C}]
    \efig
    $$
    \noindent
    Recall that:
    \[
    \begin{array}{rlll}
      \st{B}{C}              & = & (\eta_{\Box B} \pd \id_{\Diamond C});\p{\Box B,C}\\
      \st{A \pd B}{C}        & = & (\eta_{\Box (A \pd B)} \pd \id_{\Diamond C});\p{\Box (A \pd B),C}\\
      \st{A}{\Box B \pd C} & = & (\eta_{\Box A} \pd \id_{\Diamond (\Box B \pd C)});\p{\Box A,(\Box B \pd C)}\\
    \end{array}
    \]
    In addition, we require the following diagram (whose commutativity
    is implied by the fact that $\Diamond$ is a symmetric monoidal
    functor):
    $$
    \bfig
    \vSquares|ammmmma|/->`->`->``->`->`->/[
      \Diamond A \pd (\Diamond B \pd \Diamond C)`
      (\Diamond A \pd \Diamond B) \pd \Diamond C`
      \Diamond A \pd \Diamond (B \pd C)`
      \Diamond (A \pd B) \pd \Diamond C`
      \Diamond (A \pd (B \pd C))`
      \Diamond ((A \pd B) \pd C);
      \alpha^{-1}_{\Diamond A,\Diamond B,\Diamond C}`
      \id_{\Diamond A} \pd \p{B,C}`
      \p{A,B} \pd \id_{\Diamond C}``
      \p{A,B \pd C}`
      \p{A \pd B,C}`
      \Diamond\alpha^{-1}_{A,B,C}]
    \efig
    $$
    \noindent
    Finally, this case follows because the following diagram commutes:
    \begin{center}
      \rotatebox{90}{$
    \bfig
    \btriangle|mmm|<1769,1000>[
      \Box A \pd (\Diamond\Box B \pd \Diamond C)`
      \Diamond\Box A \pd (\Diamond\Box B \pd \Diamond C)`
      \Diamond\Box A \pd (\Diamond\Box B \pd C);
      \eta_{\Box A} \pd \id_{\Diamond \Box B \pd C}`
      \eta_{\Box A} \pd \p{\Box B,C}`
      \id_{\Diamond\Box A} \pd \p{\Box B,C}]

    \qtriangle|mam|/->``->/<1769,1000>[
      \Box A \pd (\Diamond\Box B \pd \Diamond C)`
      \Box A \pd \Diamond (\Box B \pd C)`
      \Diamond\Box A \pd (\Diamond\Box B \pd C);
      \id_{\Box A} \pd \p{\Box B,C}``
      \eta_{\Box A} \pd \id_{\Diamond (\Box B \pd C)}]    

    \qtriangle(-1800,0)|mmm|<1800,1000>[
      \Box A \pd (\Box B \pd \Diamond C)`
      \Box A \pd (\Diamond\Box B \pd \Diamond C)`
      \Diamond\Box A \pd (\Diamond\Box B \pd \Diamond C);
      \id_{\Box A} \pd (\eta_{\Box B} \pd \id_{\Diamond C})`
      \eta_{\Box A} \pd (\eta_{\Box B} \pd \id_{\Diamond C})`
      \eta_{\Box A} \pd \id_{\Diamond \Box B \pd C}]

    \square(0,-500)|mmmm|/`->`->`/<1769,500>[
      \Diamond\Box A \pd (\Diamond\Box B \pd \Diamond C)`
      \Diamond\Box A \pd (\Diamond\Box B \pd C)`
      (\Diamond\Box A \pd \Diamond\Box B) \pd \Diamond C`
      \Diamond (\Box A \pd (\Box B \pd C));`
      \alpha_{\Diamond\Box A,\Diamond\Box B,\Diamond C}`
      \p{\Box A,\Box B \pd C}`]   

    \square(0,-1000)|mmmm|/`->`->`/<1769,500>[
      (\Diamond\Box A \pd \Diamond\Box B) \pd \Diamond C`
      \Diamond (\Box A \pd (\Box B \pd C))`
      \Diamond (\Box A \pd \Box B) \pd \Diamond C`
      \Diamond ((\Box A \pd \Box B) \pd C);`
      \p{\Box A,\Box B} \pd \id_{\Diamond C}`
      \Diamond \alpha_{\Box A,\Box B,C}`]        

    \square(0,-1500)|mmmm|<1769,500>[
      \Diamond (\Box A \pd \Box B) \pd \Diamond C`
      \Diamond ((\Box A \pd \Box B) \pd C)`
      \Diamond\Box(A \pd B) \pd \Diamond C`
      \Diamond (\Box(A \pd B) \pd C);
      \p{\Box A \pd \Box B, C}`
      \Diamond \m{A,B} \pd \id_{\Diamond C}`
      \Diamond (\m{A,B} \pd \id_C)`
      \p{\Box (A \pd B),C}]

    \btriangle(-1800,-500)|mmm|/->``->/<1800,1500>[
      \Box A \pd (\Box B \pd \Diamond C)`
      (\Box A \pd \Box B) \pd \Diamond C`
      (\Diamond\Box A \pd \Diamond\Box B) \pd \Diamond C;
      \alpha_{\Box A,\Box B,\Diamond C}``
      (\eta_{\Box A} \pd \eta_{\Box B}) \pd \id_{\Diamond C}]

    \qtriangle(-1800,-1000)|mmm|/`->`/<1800,500>[
      (\Box A \pd \Box B) \pd \Diamond C`
      (\Diamond\Box A \pd \Diamond\Box B) \pd \Diamond C`
      \Diamond (\Box A \pd \Box B) \pd \Diamond C;`
      \eta_{\Box A \pd \Box B} \pd \id_{\Diamond C}`]

    \btriangle(-1800,-1500)|mmm|/->``->/<1800,1000>[
      (\Box A \pd \Box B) \pd \Diamond C`
      \Box(A \pd B) \pd \Diamond C`
      \Diamond\Box(A \pd B) \pd \Diamond C;
      \m{A,B} \pd \id_{\Diamond C}``
      \eta_{\Box (A \pd B)} \pd \id_{\Diamond C}]

    \place(1200,700)[1]
    \place(500,300)[2]
    \place(900,-500)[3]
    \place(900,-1250)[4]
    \place(-500,700)[5]
    \place(-1000,0)[6]
    \place(-400,-700)[7]
    \place(-1000,-1100)[8]
    \efig
    $}
    \end{center}
    Diagrams 1, 2 and 5 commute by functorality of $\times$, diagram 3
    commutes by the additional diagram from above, diagram 4 commutes
    by naturality of $\mathsf{p}$, diagram 6 commutes by naturality of
    $\alpha$, diagram 7 commutes by the fact that $\eta$ is a monoidal
    natural transformation, and diagram 8 commutes by naturality of
    $\eta$.
    

  \item[] \textit{Case 5.  strength interacts with monoidicity of $\Diamond$}
    $$
    \bfig
    \vSquares|ammmmma|/->`->```->``->/[
      \Box A \times \Diamond\Diamond B`
      \Box A \times \Diamond B`
      \Diamond(\Box A \times \Diamond B)``
      \Diamond\Diamond(\Box A \times B)`
      \Diamond(\Box A \times B);
      \id_{\Box A} \pd \mu_{B}`
      \st{A}{\Diamond B}```
      \Diamond(\st{A}{B})``
      \mu_{\Box A \pd B}]
    \morphism(1150,1000)|m|<0,-920>[`;\st{A}{B}]
    \efig
    $$
    \noindent
    Recall that:
    \[
    \begin{array}{rlll}
      \st{A}{B} & = & (\eta_{\Box A} \pd \id_{\Diamond B});\p{\Box A,B}\\
      \st{A}{\Diamond B} & = & (\eta_{\Box A} \pd \id_{\Diamond \Diamond B});\p{\Box A,\Diamond B}\\
    \end{array}
    \]
    This case follows from the fact that the following diagram
    commutes:
    \begin{center}
      \rotatebox{90}{$\bfig
    \qtriangle|mmm|<1500,1000>[
      \Box A \pd \Diamond\Diamond B`
      \Box A \pd \Diamond B`
      \Diamond\Box A \pd \Diamond B;
      \id_{\Box A} \pd \mu_B`
      \eta_{\Box A} \pd \mu_B`
      \eta_{\Box A} \pd \id_{\Diamond B}]

    \morphism(-1500,1000)|m|/<-/<1500,0>[
      \Diamond\Box A \pd \Diamond\Diamond B`
      \Box A \pd \Diamond\Diamond B;
      \eta_{\Box A} \pd \id_{\Diamond\Diamond B}]

    \btriangle(-1500,0)|mmm|<3000,1000>[
      \Diamond\Box A \pd \Diamond\Diamond B`
      \Diamond\Diamond\Box A \pd \Diamond\Diamond B`
      \Diamond\Box A \pd \Diamond B;
      \Diamond\eta_{\Box A} \pd \id_{\Diamond\Diamond B}`
      \id_{\Diamond\Box A} \pd \mu_B`
      \mu_{\Box A} \pd \mu_B]

    \square(-3000,0)|mmmm|/<-`->``<-/<1500,1000>[
      \Diamond(\Box A \pd \Diamond B)`
      \Diamond\Box A \pd \Diamond\Diamond B`
      \Diamond(\Diamond\Box A \pd \Diamond B)`
      \Diamond\Diamond\Box A \pd \Diamond\Diamond B;
      \p{\Box A,\Diamond B}`
      \Diamond (\eta_{\Box A} \pd \id_{\Diamond B})``
      \p{\Diamond\Box A,\Diamond B}]

    \square(-3000,-500)|mmmm|/`->`->`->/<4500,500>[
      \Diamond (\Diamond\Box A \pd \Diamond B)`
      \Diamond\Box A \pd \Diamond B`
      \Diamond\Diamond (\Box A \pd B)`
      \Diamond (\Box A \pd B);`
      \Diamond (\p{\Box A,B})`
      \p{\Box A,B}`
      \mu_{\Box A \pd B}]

    \place(-2300,500)[1]
    \place(-800,-250)[2]
    \place(-800,400)[3]
    \place(0,700)[4]
    \place(1050,700)[5]
    \efig$}
    \end{center}       
    Diagram commutes by naturality of $\mathsf{p}$, diagram 2 commutes
    because $\mu$ is a monoidal natural transformation, diagram 3
    commutes because $\mu$ is the monadic multiplication and by
    functorality of $\times$, and diagrams 4 and 5 commute by
    functoriality of $\times$.
    
  \item[] \textit{Case 6. commuting $\beta$ interacts with $\Diamond$}
    $$
    \bfig
    \hSquares|aamamaa|/``->``->`->`->/[
      \Diamond B \times \Box A``
      \Box A \times \Diamond B`
      \Diamond B \times \Diamond \Box A`
      \Diamond (B \times \Box A)`
      \Diamond (\Box A \times B);``
      \id_{\Diamond B} \pd \eta_{\Box A}``
      st_{A,B}`
      \p{B,\Box A}`
      \Diamond\beta_{B,\Box A}]
    \morphism(200,500)<1817,0>[`;\beta_{\Diamond B,\Box A}]
    \efig
    $$
    \noindent
    The previous diagram commutes by commutativity of the following
    diagram:
    $$
    \bfig
    \vSquares|ammmmma|[
      \Diamond B \times \Box A`
      \Box A \times \Diamond B`
      \Diamond B \times \Diamond\Box A`
      \Diamond\Box A \times \Diamond B`
      \Diamond (B \times \Box A)`
      \Diamond (\Box A \times B);
      \beta_{\Diamond B,\Box A}`
      \id_{\Diamond B} \times \eta_{\Box A}`
      \eta_{\Box A} \times \id_{\Diamond B}`
      \beta_{\Diamond B,\Diamond\Box A}`
      \p{B,\Box A}`
      \p{\Box A,B}`
      \Diamond\beta_{B,\Box A}]

    \place(600,750)[1]
    \place(600,250)[2]
    \efig
    $$
    \noindent
    Diagram 1 commutes because $\beta$ is a symmetric monoidal
    functor, and diagram 2 commutes by naturality of $\beta$.

  \item[] \textit{Case 7.  $\varepsilon$ interacts with $\Diamond$ and its monoidicity} 
    $$
    \bfig
    \hSquares|aamamaa|/``->``->`->`->/[
      \Box A \times \Diamond B``
      \Diamond(\Box A \times B)`
      A \times \Diamond B`
      \Diamond A \times \Diamond B`
      \Diamond (A \times B);``
      \varepsilon_A \times \id_{\Diamond B}``
      \Diamond(\varepsilon_A \times \id_{B})`
      \eta_A \times \id_{\Diamond B}`
      \p{A,B}]
    \morphism(200,500)<1605,0>[`;\st{A}{B}]
    \efig
    $$
    \noindent
    The previous diagram commutes by commutativity of the following
    diagram:
    $$
    \bfig
    \hSquares|aammmaa|[
      \Box A \times \Diamond B`
      \Diamond\Box A \times \Diamond B`
      \Diamond (\Box A \times B)`
      A \times \Diamond B`
      \Diamond A \times \Diamond B`
      \Diamond (A \times B);
      \eta_{\Box A} \times \id_{\Diamond B}`
      \p{\Box A,B}`
      \varepsilon_A \times \id_{\Diamond B}`
      \Diamond\varepsilon_A \times \id_{\Diamond B}`
      \Diamond (\varepsilon_A \times \id_B)`
      \eta_A \times \id_{\Diamond B}`
      \p{A,B}]
    \place(1750,250)[1]
    \place(600,250)[2]
    \efig
    $$
    \noindent
    Diagram 1 commutes by naturality of $\mathsf{p}$, and diagram 2
    commutes by naturality of $\eta$.
  \end{itemize}

\end{proof}\begin{proof}
  We must show that given the definition of an adjoint CS4 categorical
  model (Definition~\ref{def:CS4-single-adjoint-cat-model}) we can
  define an appropriate monad and comonad on a CCC with coproducts
  where the monad is strong with respect to the comonad.

  Suppose $(H,m)$ and $(J,n)$ are the adjoint monoidal functors given
  in Definition~\ref{def:CS4-single-adjoint-cat-model}, and define
  $\Box = JH$ and $\Diamond = HJ$.  By definition we assumed that
  $(\Box, q)$, where $q_{A,B} : \Box A \times \Box B \to \Box (A
  \times B)$, is monoidal, but we must show that $\Diamond$ is also
  monoidal.  We know that both $(H,n)$ and $(J,m)$ are monoidal
  endofunctors on $\cat{C}$ which implies that their composition
  $\Diamond$ is monoidal where
  \[
  \begin{array}{lll}
    \mathsf{p}_{1} = \eta_{1} : 1 \to \Diamond 1\\
    \mathsf{p}_{A,B} = \m{HA,HB};J(\mathsf{n}_{A,B})
    \colon \Diamond A \pd \Diamond B \mto \Diamond(A \pd B)
  \end{array}
  \]
  and the following diagrams commute (proofs omitted):
  \begin{mathpar}
    \scriptsize
    \bfig
    \vSquares|ammmmma|/->`->`->``->`->`->/[
      (\Diamond A \times \Diamond B) \times \Diamond C`
      \Diamond A \times (\Diamond B \times \Diamond C)`
      \Diamond(A \times B) \times \Diamond C`
      \Diamond A \times \Diamond(B \times C)`
      \Diamond ((A \times B) \times C)`
      \Diamond (A \times (B \times C));
      \alpha`
      \mathsf{p}_{A,B} \times \id_{\Diamond C}`
      \id_{\Diamond A} \times \mathsf{p}_{B,C}``
      \mathsf{p}_{A \times B,C}`
      \mathsf{p}_{A,B \times C}`
      \Diamond \alpha]
    \efig
    \and
    \bfig
    \hSquares|ammmmaa|/->``->`<-``->`/[
      1 \times \Diamond A`
      \Diamond A``
      \Diamond 1 \times \Diamond A`
      \Diamond(1 \times A)`;
      \lambda_{\Diamond A}``
      \mathsf{p}_{1} \times \id_{\Diamond A}`
      \Diamond \lambda_A``
      \mathsf{p}_{1,A}`]
    \efig
    \and
    \bfig
    \hSquares|ammmmaa|/->``->`<-``->`/[
      \Diamond A \times 1`
      \Diamond A``
      \Diamond A \times \Diamond 1`
      \Diamond(A \times 1)`;
      \rho_{\Diamond A}``
      \id_{\Diamond A} \times \mathsf{p}_{1}`
      \Diamond \rho_A``
      \mathsf{p}_{A,1}`]
    \efig
    \and
    \bfig
    \hSquares|ammmmaa|/->``->`->``->`/[
      \Diamond A \times \Diamond B`
      \Diamond B \times \Diamond A``
      \Diamond (A \times B)`
      \Diamond (B \times A)`;
      \beta_{\Diamond A,\Diamond B}``
      \mathsf{p}_{A,B}`
      \mathsf{p}_{B,A}``
      \Diamond\beta_{A,B}`]
    \efig
  \end{mathpar}

  Furthermore, suppose $J \dashv H$, where the unit, $\varepsilon :
  \Box A \to A$, and the counit, $\eta : A \to \Diamond A$, are
  monoidal natural transformations.  This implies that the following
  diagrams commute:
  \begin{mathpar}
    \bfig
    \btriangle<800,500>[A \times B`\Diamond A \times \Diamond B`\Diamond (A \times B);\eta_A \times \eta_B`\eta_{A \times B}`\mathsf{p}_{A,B}]
    \efig
    \and
    \bfig
    \qtriangle<800,500>[\Box A \times \Box B`\Box (A \times B)`A \times B;\mathsf{q}_{A,B}`\varepsilon_A \times \varepsilon_B`\varepsilon_{A \times B}]
    \efig
    \and
    \bfig
    \qtriangle<800,500>[1`J 1`\Diamond 1;n_{1}`\eta_1`J m_1]       
    \efig
    \and
    \bfig
    \hSquares|ammmmaa|/->``=`->``<-`/[
      \Box 1`
      1``
      \Box 1`
      H 1`;
      \varepsilon_1```
      m_1``
      H n_1`]
    \efig
    \and
    \bfig
    \qtriangle<800,500>[H A`H\Box A`H A;\eta_{H A}`\id_{H A}`H\varepsilon_A]       
    \efig
    \and
    \bfig
    \qtriangle<800,500>[J A`\Box J A`J A;J\eta_A`\id_{J A}`\varepsilon_{J A}]
    \efig    
  \end{mathpar}
  It is a well-known fact about adjoints that $(\Box, \varepsilon,
  \delta)$, where $\delta : \Box A \to \Box\Box A$ is a comonad, and
  $(\Diamond, \eta, \mu)$, where $\mu : \Diamond\Diamond A \to
  \Diamond A$ is a monad.  In addition, $\mu$ and $\delta$ are monoidal
  natural transformations where we have the following:
  \[
  \begin{array}{lll}
    \d{1} = \p{1};\Diamond\p{1} : 1 \mto \Diamond^2 1\\
    \d{A,B} =  \p{\Diamond A,\Diamond B};\Diamond\p{A,B} : \Diamond^2 A \times \Diamond^2 B \mto \Diamond^2 (A \times B)\\
    \\
    \b{1} = \q{1};\Box\q{1} : 1 \mto \Box^2 1\\
    \b{A,B} = \q{\Box A,\Box B};\Box\q{A,B} : \Box^2 A \times \Box^2 B \mto \Box^2 (A \times B)\\
  \end{array}
  \]
  Thus, the following diagrams commute:
  \begin{mathpar}
    \bfig
    \hSquares|ammmmaa|/->``->`->``->`/[
      \Diamond^3 A`
      \Diamond^2 A``
      \Diamond^2 A`
      \Diamond A`;
      \Diamond \mu_A``
      \mu_{\Diamond A}`
      \mu_A``
      \mu_A`]
    \efig
    \and
    \bfig
    \qtriangle/->`=`->/<800,500>[\Diamond A`\Diamond^2 A`\Diamond A;\eta_{\Diamond A}``\mu_A]
    \btriangle(0,0)/->`=`->/<800,500>[\Diamond A`\Diamond^2 A`\Diamond A;\Diamond \eta_{A}``\mu_A]
    \efig
    \and
    \bfig
    \hSquares|ammmmaa|/->``->`->``->`/[
      \Box A`
      \Box^2 A``
      \Box^2 A`
      \Box^3 A`;
      \delta_A``
      \delta_A`
      \delta_{\Box A}``
      \Box\delta_A`]
    \efig
    \and
    \bfig
    \qtriangle/->`=`->/<800,500>[\Box A`\Box^2 A`\Box A;\delta_A``\Box \varepsilon]
    \btriangle(0,0)/->`=`->/<800,500>[\Box A`\Box^2 A`\Box A;\delta_A``\varepsilon_{\Box A}]
    \efig
    \and
    %% \bfig   
    %% \vSquares|ammmmma|/->`->```->``->/[
    %%   \Diamond^2 A \times \Diamond^2 B`
    %%   \Diamond A \times \Diamond B`
    %%   \Diamond(\Diamond A \times \Diamond B)``
    %%   \Diamond^2(A \times B)`
    %%   \Diamond(A \times B);
    %%   \mu_A \times \mu_B`
    %%   \p{\Diamond A,\Diamond B}```
    %%   \Diamond\p{A,B}``
    %%   \mu_{A \times B}]
    %% \morphism(1108,0)/<-/<0,1000>[\Diamond(A \times B)`\Diamond A \times \Diamond B;\p{A,B}]
    %% \efig
    \and
    \bfig
    \square<1000,1000>[
      \Diamond^2 A \times \Diamond^2 B`
      \Diamond A \times \Diamond B`
      \Diamond^2(A \times B)`
      \Diamond(A \times B);
      \mu_A \times \mu_B`
      \d{A,B}`
      \p{A,B}`
      \mu_{A \times B}]
    \efig
    \and
    \bfig
    \Vtriangle/->`<-`<-/[
      \Diamond^2 1`
      \Diamond 1`
      1;
      \mu_1`
      \d{1}`
      \p{1}]
    \efig
    \and        
    \bfig
    \square<1000,1000>[
      \Box A \times \Box B`
      \Box^2 A \times \Box^2 B`
      \Box(A \times B)`
      \Box^2(A \times B);
      \delta_A \times \delta_B`
      \q{A,B}`
      \b{A,B}`
      \delta_{A \times B}]
    %% \vSquares|ammmmma|/->``->```->`->/[
    %%   \Box A \times \Box B`
    %%   \Box^2 A \times \Box^2 B``
    %%   \Box(\Box A \times \Box B)`
    %%   \Box(A \times B)`
    %%   \Box^2(A \times B);
    %%   \varepsilon_A \times \varepsilon_B``
    %%   \q{\Box A,\Box B}```
    %%   \Box\q{A,B}`
    %%   \varepsilon_{A \times B}]
    %% \morphism(0,0)/<-/<0,1000>[\Box(A \times B)`\Box A \times \Box B;\q{A,B}]
    \efig
    \and
    \bfig
    \Vtriangle/->`<-`<-/[
      \Box 1`
      \Box^2 1`
      1;
      \delta_1`
      \q{1}`
      \d{1}]
    \efig
  \end{mathpar}

  We can now define the $\Box$-strength map as follows:
  \[
  \st{A}{B} = (\eta_{\Box A} \pd \id_{\Diamond B});\mathsf{p}_{\Box A,B} : \Box A \pd \Diamond B \mto \Diamond(\Box A \pd B)
  \]
  We can see that $\st{A}{B}$ is a natural transformation, because it
  is defined as a composition of natural transformations.
  
  %% To prove that the appropriate diagrams commute we first note that
  %% the triangle  
  Next we must show that all of the appropriate diagrams given in
  Definition~\ref{def:comonad-strong-monad} commute.
  \begin{itemize}
  \item[] \textit{Case 1. the first projection interacts with $\Diamond$:}
    \[
    \bfig
    \square|amma|<850,600>[
      \Box A \times \Diamond B`
      \Diamond (\Box A \times B)`
      \Box A`
      \Diamond\Box A;
      \st{A}{B}`
      \pi_1`
      \Diamond\pi_1`
      \eta_A]
    \efig
    \]
    This diagrams commutes, because the following diagram commutes:
    \[
    \bfig
    \square|ammm|/->`->``/<1500,2000>[
      \Box A \times \Diamond B`
      \Diamond\Box A \times \Diamond B`
      \Box A`;
      \eta_{\Box A} \times \id_{\Diamond B}`
      \pi_1``]

    \square(1500,0)|ammm|/->``->`/<1500,2000>[
      \Diamond\Box A \times \Diamond B`
      \Diamond (\Box A \times B)``
      \Diamond\Box A;
      \p{\Box A,B}``
      \Diamond\pi_1`]

    \morphism<3000,0>[\Box A`\Diamond\Box A;\eta_{\Box A}]

    \morphism(484,1500)<1000,0>[
      \Box A \times 1`
      \Diamond\Box A \times \Diamond 1;
      \eta_{\Box A} \times \eta_1]

    \morphism(1484,1500)<700,-500>[
      \Diamond\Box A \times \Diamond 1`
      \Diamond(\Box A \times 1);
      \p{\Box A,1}]

    \morphism(484,1500)|m|/{@{>}@/_1em/}/<1700,-500>[
      \Box A \times 1`
      \Diamond(\Box A \times 1);
      \eta_{\Box A \times 1}]

    \morphism(484,1500)|m|<700,-800>[
      \Box A \times 1`
      \Box A;
      \pi_1]

    \morphism(0,2000)|m|<484,-500>[
      \Box A \times \Diamond B`
      \Box A \times 1;
      \id_{\Box A} \times \t_{\Diamond B}]

    \morphism(1500,2000)|m|<-17,-500>[
      \Diamond\Box A \times \Diamond B`
      \Diamond\Box A \times \Diamond 1;
      \id_{\Diamond\Box A} \times \Diamond\t_{B}]

    \morphism(3000,2000)|m|<-815,-1000>[
      \Diamond(\Box A \times B)`
      \Diamond(\Box A \times 1);
      \Diamond (id_{\Box A} \times \t_{B})]

    \morphism(2184,1000)|m|<816,-1000>[
      \Diamond(\Box A \times 1)`
      \Diamond\Box A;
      \Diamond\pi_1]

    \morphism(1184,700)|m|<1816,-700>[
      \Box A`
      \Diamond\Box A;
      \eta_{\Box A}]
    
    \place(2700,1000)[1]
    \place(2150,1700)[2]
    \place(800,1750)[3]
    \place(1300,1300)[4]
    \place(1800,800)[5]
    \place(500,500)[6]
    \efig    
    \]
    Diagrams 1 and 6 commute because we are in a cartesian closed
    category, diagram 2 commutes by naturality of $\p{}$, diagram 3
    commutes because $\Diamond$ is a product functor, diagram 4
    commutes because $\eta$ is the unit of a symmetric monoidal
    adjunction, and diagram 5 commutes by naturality of $\eta$.
    
  \item[] \textit{Case 2. the second projection interacts with $\Diamond$:}
    \[
    \bfig
    \qtriangle<850,600>[
      \Box A \times \Diamond B`
      \Diamond (\Box A \times B)`
      \Diamond B;
      \st{A}{B}`
      \pi_2`
      \Diamond\pi_2]
    \efig
    \]
    This case is similar to the previous case.
    
    %% \item[] \textit{Case 1. the object 1 behaves as the unit for products}
  %%   $$
  %%   \bfig
  %%   \vSquares|ammmmma|/>``>```>`>/[\Box 1 \times \Diamond A`\Diamond (\Box 1 \times A)``\Diamond(1 \times A)`1 \times \Diamond A`\Diamond A;\st{1}{A}``\Diamond(\varepsilon_1 \times \id_A)```\Diamond\lambda`\lambda]
  %%   \morphism(0,1000)|m|/->/<0,-950>[`;\varepsilon_1 \times \id_{\Diamond A}]
  %%   \efig
  %%   $$
  %%   This diagram commutes by commutativity of the following diagram:
  %%   %% Equational version:
  %%   %% \begin{center}
  %%   %%   \begin{math}
  %%   %%     \begin{array}{rllllllll}
  %%   %%       & & \st{1}{A};\Diamond (\varepsilon_1 \times \id_A);\Diamond\lambda_A\\
  %%   %%       \text{(Definition of $\mathsf{st}$)}
  %%   %%       & = & (\eta_{\Box 1} \times \id_{\Diamond A});\p{\Box 1,\Diamond A};\Diamond (\varepsilon_1 \times \id_A);\Diamond\lambda_A\\
  %%   %%       \text{(Naturality of $\mathsf{p}$)}
  %%   %%       & = & (\eta_{\Box 1} \times \id_{\Diamond A});(\Diamond \varepsilon_1 \times \Diamond\id_A);\p{1,A};\Diamond\lambda_A\\
  %%   %%       \text{(Functoriality of $\times$)}
  %%   %%       & = & ((\eta_{\Box 1};\Diamond \varepsilon_1) \times (\id_{\Diamond A};\Diamond\id_A));\p{1,A};\Diamond\lambda_A\\
  %%   %%       & = & ((\eta_{\Box 1};\Diamond \varepsilon_1) \times (\id_{\Diamond A};\id_{\Diamond A}));\p{1,A};\Diamond\lambda_A\\
  %%   %%       \text{(Naturality of $\eta$)}
  %%   %%       & = & ((\varepsilon_1;\eta_{1}) \times (\id_{\Diamond A};\id_{\Diamond A}));\p{1,A};\Diamond\lambda_A\\
  %%   %%       \text{(Definition of $\mathsf{p}$)}
  %%   %%       & = & ((\varepsilon_1;\p{1}) \times (\id_{\Diamond A};\id_{\Diamond A}));\p{1,A};\Diamond\lambda_A\\
  %%   %%       \text{(Functoriality of $\times$)}
  %%   %%       & = & (\varepsilon_1 \times \id_{\Diamond A});(\p{1} \times \id_{\Diamond A});\p{1,A};\Diamond\lambda_A\\
  %%   %%       \text{($\Diamond$ is Symmetric Monoidal)}
  %%   %%       & = & (\varepsilon_1 \times \id_{\Diamond A});\lambda_{\Diamond A}\\
  %%   %%     \end{array}
  %%   %%   \end{math}
  %%   %% \end{center}
  %%   $$
  %%   \bfig
  %%   \square|amma|<1000,500>[
  %%     \Diamond\Box 1 \times \Diamond A`
  %%     \Diamond (\Box 1 \times A)`
  %%     \Diamond 1 \times \Diamond A`
  %%     \Diamond (1 \times A);
  %%     \p{\Box 1,A}`
  %%     \Diamond \varepsilon_1 \times \id_{\Diamond A}`
  %%     \Diamond (\varepsilon_1 \times \id_A)`
  %%     \p{1,A}]

  %%   \square(-1000,0)|amma|<1000,500>[
  %%     \Box 1 \times \Diamond A`
  %%     \Diamond\Box 1 \times \Diamond A`
  %%     1 \times \Diamond A`
  %%     \Diamond 1 \times \Diamond A;
  %%     \eta_{\Box 1} \times \id_{\Diamond A}`
  %%     \varepsilon_1 \times \id_{\Diamond A}`
  %%     \Diamond \varepsilon_1 \times \id_{\Diamond A}`
  %%     (\p{1} = \eta_1) \times \id_{\Diamond A}]

  %%   \qtriangle(-1000,-500)|mmm|/`->`->/<2000,500>[
  %%     1 \times \Diamond A`
  %%     \Diamond (1 \times A)`
  %%     \Diamond A;`
  %%     \lambda_{\Diamond A}`
  %%     \Diamond\lambda_A]

  %%   \place(500,250)[1]
  %%   \place(-500,250)[2]
  %%   \place(500,-200)[3]
  %%   \efig
  %%   $$
  %%   \noindent
  %%   Diagram 1 commutes by naturality of $\mathsf{p}$, diagram 2
  %%   commutes by naturality of $\eta$, and diagram 3 commutes because
  %%   $\Diamond$ is a symmetric monoidal functor.

  \item[] \textit{Case 3. unit $\eta$ of the monad and strength interact well, $\Box  A $ is a parameter}
    $$
    \bfig
    \btriangle<800,500>[
      \Box A \pd B`
      \Box A \pd \Diamond B`
      \Diamond(\Box A \pd B);
      \id_{\Box A} \pd \eta_{B}`
      \eta_{\Box A \times B}`
      \st{A}{B}]
    \efig
    $$

    The previous diagram commutes, because the following diagram commutes:
    $$
    \bfig
    \btriangle|ama|/->`->`->/<1500,500>[
      \Box A \pd B`
      \Box A \pd \Diamond B`
      \Diamond\Box A \pd \Diamond B;
      \id_{\Box A} \pd \eta_{B}`
      \eta_{\Box A} \pd \eta_B`
      \eta_{\Box A} \pd \id_{\Diamond B}]

    \qtriangle(0,0)/->``<-/<1500,500>[
      \Box A \pd B`
      \Diamond (\Box A \times B)`
      \Diamond\Box A \pd \Diamond B;
      \eta_{\Box A \times B}``
      \p{\Box A,B}]

    \place(250,200)[1]
    \place(1200,300)[2]
    \efig
    $$
    \noindent
    Diagram 1 clearly commutes, and diagram 2 commutes because $\eta$
    is a symmetric monoidal natural transformation.
    
  %% \item[] \textit{Case 3. co-unit of the comonad $\varepsilon$ and unit of the monad $\eta$ interact well?}
  %%   $$
  %%   \bfig
  %%   \hSquares|aamaaaa|/->`->`->```->`/[
  %%     \Box A \times 1`
  %%     \Box A \times \Diamond 1`
  %%     \Diamond (\Box A \times 1)`
  %%     \Box A`
  %%     A`
  %%     ;
  %%     \id_{\Box A} \times \eta_1`
  %%     \st{A}{1}`
  %%     \rho_{\Box A}```
  %%     \varepsilon_A`]
  %%   \qtriangle(1978,0)/->``->/<800,500>[\Diamond (\Box A \times 1)`\Diamond\Box A`\Diamond A;\Diamond \rho_{\Box A}``\Diamond\varepsilon_A]
  %%   \morphism(1060,0)/->/<1640,0>[`;\eta_A]
  %%   \efig
  %%   $$
  %%   \noindent
  %%   Recall that
  %%   $\st{A}{1} = (\eta_{\Box A} \pd \id_{\Diamond 1});\p{\Box A,1}$.
  %%   Now the previous diagram commutes, because the following diagram commutes:
  %%   $$
  %%   \bfig
  %%   \btriangle|mma|<1444,1000>[\Box A \pd 1`\Diamond\Box A \pd \Diamond 1`\Diamond(\Box A \pd 1);\eta_{\Box A} \pd \eta_1`\eta_{\Box A \pd 1}`\p{\Box A,1}]
  %%   \dtriangle(-1000,0)/->``->/<1000,1000>[\Box A \pd 1`\Box A \pd 1`\Diamond\Box A \pd \Diamond 1;\id_{\Box A} \pd \eta_1``\eta_{\Box A} \pd
  %%     \id_{\Diamond 1}]

  %%   \hSquares(0,0)/->`->```<-``/<1000>[\Box A \pd 1`\Box A`\Diamond\Box A```\Diamond (\Box A \times 1);\rho_{\Box A}`\eta_{\Box A}```\Diamond (\rho_{\Box A})``]

  %%   \square(746,1000)/->`<-`<-`/<698,500>[A`\Diamond A`\Box A`\Diamond\Box A;\eta_A`\varepsilon_A`\Diamond\varepsilon_A`]

  %%   \place(-400,300)[1]
  %%   \place(400,300)[2]
  %%   \place(1100,700)[3]
  %%   \place(1100,1250)[4]
  %%   \efig
  %%   $$
  %%   \noindent
  %%   Diagram 1 commutes by functorality of $\times$, diagram 2 commutes
  %%   because $\eta$ is a monoidal natural transformation, and diagrams
  %%   3 and 4 commute by naturality of $\eta$.

  \item[] \textit{Case 4. associativity $\alpha$ interacts with co-monoidicity of $\Box$}
    $$
    \bfig
    \vSquares|ammmmmm|/->`->`->```->`/[
      \Box A \times (\Box B \times \Diamond C)`
      \Box A \times \Diamond(\Box B \times C)`
      (\Box A \times \Box B) \times \Diamond C`
      \Diamond(\Box A \times (\Box B \times C))``
      \Diamond((\Box A \times \Box B) \times C);
      \id_{\Box A} \pd \st{B}{C}`
      \alpha^{-1}`
      \st{A}{\Box B \times C}```
      \Diamond\alpha^{-1}`]
    \morphism(1554,0)|m|/->/<0,-500>[`\Diamond(\Box(A \times B) \times C);\Diamond(\m{A,B} \times \id_C)]
    
    \morphism(0,500)|m|/->/<0,-1000>[`\Box(A \times B) \times \Diamond C;\m{A,B} \times \id_{\Diamond C}]

    \morphism(350,-500)|a|/->/<800,0>[`;\st{A \times B}{C}]
    \efig
    $$
    \noindent
    Recall that:
    \[
    \begin{array}{rlll}
      \st{B}{C}              & = & (\eta_{\Box B} \pd \id_{\Diamond C});\p{\Box B,C}\\
      \st{A \pd B}{C}        & = & (\eta_{\Box (A \pd B)} \pd \id_{\Diamond C});\p{\Box (A \pd B),C}\\
      \st{A}{\Box B \pd C} & = & (\eta_{\Box A} \pd \id_{\Diamond (\Box B \pd C)});\p{\Box A,(\Box B \pd C)}\\
    \end{array}
    \]
    In addition, we require the following diagram (whose commutativity
    is implied by the fact that $\Diamond$ is a symmetric monoidal
    functor):
    $$
    \bfig
    \vSquares|ammmmma|/->`->`->``->`->`->/[
      \Diamond A \pd (\Diamond B \pd \Diamond C)`
      (\Diamond A \pd \Diamond B) \pd \Diamond C`
      \Diamond A \pd \Diamond (B \pd C)`
      \Diamond (A \pd B) \pd \Diamond C`
      \Diamond (A \pd (B \pd C))`
      \Diamond ((A \pd B) \pd C);
      \alpha^{-1}_{\Diamond A,\Diamond B,\Diamond C}`
      \id_{\Diamond A} \pd \p{B,C}`
      \p{A,B} \pd \id_{\Diamond C}``
      \p{A,B \pd C}`
      \p{A \pd B,C}`
      \Diamond\alpha^{-1}_{A,B,C}]
    \efig
    $$
    \noindent
    Finally, this case follows because the following diagram commutes:
    \begin{center}
      \rotatebox{90}{$
    \bfig
    \btriangle|mmm|<1769,1000>[
      \Box A \pd (\Diamond\Box B \pd \Diamond C)`
      \Diamond\Box A \pd (\Diamond\Box B \pd \Diamond C)`
      \Diamond\Box A \pd (\Diamond\Box B \pd C);
      \eta_{\Box A} \pd \id_{\Diamond \Box B \pd C}`
      \eta_{\Box A} \pd \p{\Box B,C}`
      \id_{\Diamond\Box A} \pd \p{\Box B,C}]

    \qtriangle|mam|/->``->/<1769,1000>[
      \Box A \pd (\Diamond\Box B \pd \Diamond C)`
      \Box A \pd \Diamond (\Box B \pd C)`
      \Diamond\Box A \pd (\Diamond\Box B \pd C);
      \id_{\Box A} \pd \p{\Box B,C}``
      \eta_{\Box A} \pd \id_{\Diamond (\Box B \pd C)}]    

    \qtriangle(-1800,0)|mmm|<1800,1000>[
      \Box A \pd (\Box B \pd \Diamond C)`
      \Box A \pd (\Diamond\Box B \pd \Diamond C)`
      \Diamond\Box A \pd (\Diamond\Box B \pd \Diamond C);
      \id_{\Box A} \pd (\eta_{\Box B} \pd \id_{\Diamond C})`
      \eta_{\Box A} \pd (\eta_{\Box B} \pd \id_{\Diamond C})`
      \eta_{\Box A} \pd \id_{\Diamond \Box B \pd C}]

    \square(0,-500)|mmmm|/`->`->`/<1769,500>[
      \Diamond\Box A \pd (\Diamond\Box B \pd \Diamond C)`
      \Diamond\Box A \pd (\Diamond\Box B \pd C)`
      (\Diamond\Box A \pd \Diamond\Box B) \pd \Diamond C`
      \Diamond (\Box A \pd (\Box B \pd C));`
      \alpha_{\Diamond\Box A,\Diamond\Box B,\Diamond C}`
      \p{\Box A,\Box B \pd C}`]   

    \square(0,-1000)|mmmm|/`->`->`/<1769,500>[
      (\Diamond\Box A \pd \Diamond\Box B) \pd \Diamond C`
      \Diamond (\Box A \pd (\Box B \pd C))`
      \Diamond (\Box A \pd \Box B) \pd \Diamond C`
      \Diamond ((\Box A \pd \Box B) \pd C);`
      \p{\Box A,\Box B} \pd \id_{\Diamond C}`
      \Diamond \alpha_{\Box A,\Box B,C}`]        

    \square(0,-1500)|mmmm|<1769,500>[
      \Diamond (\Box A \pd \Box B) \pd \Diamond C`
      \Diamond ((\Box A \pd \Box B) \pd C)`
      \Diamond\Box(A \pd B) \pd \Diamond C`
      \Diamond (\Box(A \pd B) \pd C);
      \p{\Box A \pd \Box B, C}`
      \Diamond \m{A,B} \pd \id_{\Diamond C}`
      \Diamond (\m{A,B} \pd \id_C)`
      \p{\Box (A \pd B),C}]

    \btriangle(-1800,-500)|mmm|/->``->/<1800,1500>[
      \Box A \pd (\Box B \pd \Diamond C)`
      (\Box A \pd \Box B) \pd \Diamond C`
      (\Diamond\Box A \pd \Diamond\Box B) \pd \Diamond C;
      \alpha_{\Box A,\Box B,\Diamond C}``
      (\eta_{\Box A} \pd \eta_{\Box B}) \pd \id_{\Diamond C}]

    \qtriangle(-1800,-1000)|mmm|/`->`/<1800,500>[
      (\Box A \pd \Box B) \pd \Diamond C`
      (\Diamond\Box A \pd \Diamond\Box B) \pd \Diamond C`
      \Diamond (\Box A \pd \Box B) \pd \Diamond C;`
      \eta_{\Box A \pd \Box B} \pd \id_{\Diamond C}`]

    \btriangle(-1800,-1500)|mmm|/->``->/<1800,1000>[
      (\Box A \pd \Box B) \pd \Diamond C`
      \Box(A \pd B) \pd \Diamond C`
      \Diamond\Box(A \pd B) \pd \Diamond C;
      \m{A,B} \pd \id_{\Diamond C}``
      \eta_{\Box (A \pd B)} \pd \id_{\Diamond C}]

    \place(1200,700)[1]
    \place(500,300)[2]
    \place(900,-500)[3]
    \place(900,-1250)[4]
    \place(-500,700)[5]
    \place(-1000,0)[6]
    \place(-400,-700)[7]
    \place(-1000,-1100)[8]
    \efig
    $}
    \end{center}
    Diagrams 1, 2 and 5 commute by functorality of $\times$, diagram 3
    commutes by the additional diagram from above, diagram 4 commutes
    by naturality of $\mathsf{p}$, diagram 6 commutes by naturality of
    $\alpha$, diagram 7 commutes by the fact that $\eta$ is a monoidal
    natural transformation, and diagram 8 commutes by naturality of
    $\eta$.
    

  \item[] \textit{Case 5.  strength interacts with monoidicity of $\Diamond$}
    $$
    \bfig
    \vSquares|ammmmma|/->`->```->``->/[
      \Box A \times \Diamond\Diamond B`
      \Box A \times \Diamond B`
      \Diamond(\Box A \times \Diamond B)``
      \Diamond\Diamond(\Box A \times B)`
      \Diamond(\Box A \times B);
      \id_{\Box A} \pd \mu_{B}`
      \st{A}{\Diamond B}```
      \Diamond(\st{A}{B})``
      \mu_{\Box A \pd B}]
    \morphism(1150,1000)|m|<0,-920>[`;\st{A}{B}]
    \efig
    $$
    \noindent
    Recall that:
    \[
    \begin{array}{rlll}
      \st{A}{B} & = & (\eta_{\Box A} \pd \id_{\Diamond B});\p{\Box A,B}\\
      \st{A}{\Diamond B} & = & (\eta_{\Box A} \pd \id_{\Diamond \Diamond B});\p{\Box A,\Diamond B}\\
    \end{array}
    \]
    This case follows from the fact that the following diagram
    commutes:
    \begin{center}
      \rotatebox{90}{$\bfig
    \qtriangle|mmm|<1500,1000>[
      \Box A \pd \Diamond\Diamond B`
      \Box A \pd \Diamond B`
      \Diamond\Box A \pd \Diamond B;
      \id_{\Box A} \pd \mu_B`
      \eta_{\Box A} \pd \mu_B`
      \eta_{\Box A} \pd \id_{\Diamond B}]

    \morphism(-1500,1000)|m|/<-/<1500,0>[
      \Diamond\Box A \pd \Diamond\Diamond B`
      \Box A \pd \Diamond\Diamond B;
      \eta_{\Box A} \pd \id_{\Diamond\Diamond B}]

    \btriangle(-1500,0)|mmm|<3000,1000>[
      \Diamond\Box A \pd \Diamond\Diamond B`
      \Diamond\Diamond\Box A \pd \Diamond\Diamond B`
      \Diamond\Box A \pd \Diamond B;
      \Diamond\eta_{\Box A} \pd \id_{\Diamond\Diamond B}`
      \id_{\Diamond\Box A} \pd \mu_B`
      \mu_{\Box A} \pd \mu_B]

    \square(-3000,0)|mmmm|/<-`->``<-/<1500,1000>[
      \Diamond(\Box A \pd \Diamond B)`
      \Diamond\Box A \pd \Diamond\Diamond B`
      \Diamond(\Diamond\Box A \pd \Diamond B)`
      \Diamond\Diamond\Box A \pd \Diamond\Diamond B;
      \p{\Box A,\Diamond B}`
      \Diamond (\eta_{\Box A} \pd \id_{\Diamond B})``
      \p{\Diamond\Box A,\Diamond B}]

    \square(-3000,-500)|mmmm|/`->`->`->/<4500,500>[
      \Diamond (\Diamond\Box A \pd \Diamond B)`
      \Diamond\Box A \pd \Diamond B`
      \Diamond\Diamond (\Box A \pd B)`
      \Diamond (\Box A \pd B);`
      \Diamond (\p{\Box A,B})`
      \p{\Box A,B}`
      \mu_{\Box A \pd B}]

    \place(-2300,500)[1]
    \place(-800,-250)[2]
    \place(-800,400)[3]
    \place(0,700)[4]
    \place(1050,700)[5]
    \efig$}
    \end{center}       
    Diagram commutes by naturality of $\mathsf{p}$, diagram 2 commutes
    because $\mu$ is a monoidal natural transformation, diagram 3
    commutes because $\mu$ is the monadic multiplication and by
    functorality of $\times$, and diagrams 4 and 5 commute by
    functoriality of $\times$.
    
  \item[] \textit{Case 6. commuting $\beta$ interacts with $\Diamond$}
    $$
    \bfig
    \hSquares|aamamaa|/``->``->`->`->/[
      \Diamond B \times \Box A``
      \Box A \times \Diamond B`
      \Diamond B \times \Diamond \Box A`
      \Diamond (B \times \Box A)`
      \Diamond (\Box A \times B);``
      \id_{\Diamond B} \pd \eta_{\Box A}``
      st_{A,B}`
      \p{B,\Box A}`
      \Diamond\beta_{B,\Box A}]
    \morphism(200,500)<1817,0>[`;\beta_{\Diamond B,\Box A}]
    \efig
    $$
    \noindent
    The previous diagram commutes by commutativity of the following
    diagram:
    $$
    \bfig
    \vSquares|ammmmma|[
      \Diamond B \times \Box A`
      \Box A \times \Diamond B`
      \Diamond B \times \Diamond\Box A`
      \Diamond\Box A \times \Diamond B`
      \Diamond (B \times \Box A)`
      \Diamond (\Box A \times B);
      \beta_{\Diamond B,\Box A}`
      \id_{\Diamond B} \times \eta_{\Box A}`
      \eta_{\Box A} \times \id_{\Diamond B}`
      \beta_{\Diamond B,\Diamond\Box A}`
      \p{B,\Box A}`
      \p{\Box A,B}`
      \Diamond\beta_{B,\Box A}]

    \place(600,750)[1]
    \place(600,250)[2]
    \efig
    $$
    \noindent
    Diagram 1 commutes because $\beta$ is a symmetric monoidal
    functor, and diagram 2 commutes by naturality of $\beta$.

  \item[] \textit{Case 7.  $\varepsilon$ interacts with $\Diamond$ and its monoidicity} 
    $$
    \bfig
    \hSquares|aamamaa|/``->``->`->`->/[
      \Box A \times \Diamond B``
      \Diamond(\Box A \times B)`
      A \times \Diamond B`
      \Diamond A \times \Diamond B`
      \Diamond (A \times B);``
      \varepsilon_A \times \id_{\Diamond B}``
      \Diamond(\varepsilon_A \times \id_{B})`
      \eta_A \times \id_{\Diamond B}`
      \p{A,B}]
    \morphism(200,500)<1605,0>[`;\st{A}{B}]
    \efig
    $$
    \noindent
    The previous diagram commutes by commutativity of the following
    diagram:
    $$
    \bfig
    \hSquares|aammmaa|[
      \Box A \times \Diamond B`
      \Diamond\Box A \times \Diamond B`
      \Diamond (\Box A \times B)`
      A \times \Diamond B`
      \Diamond A \times \Diamond B`
      \Diamond (A \times B);
      \eta_{\Box A} \times \id_{\Diamond B}`
      \p{\Box A,B}`
      \varepsilon_A \times \id_{\Diamond B}`
      \Diamond\varepsilon_A \times \id_{\Diamond B}`
      \Diamond (\varepsilon_A \times \id_B)`
      \eta_A \times \id_{\Diamond B}`
      \p{A,B}]
    \place(1750,250)[1]
    \place(600,250)[2]
    \efig
    $$
    \noindent
    Diagram 1 commutes by naturality of $\mathsf{p}$, and diagram 2
    commutes by naturality of $\eta$.
  \end{itemize}

\end{proof}\begin{proof}
  We must show that given the definition of an adjoint CS4 categorical
  model (Definition~\ref{def:CS4-single-adjoint-cat-model}) we can
  define an appropriate monad and comonad on a CCC with coproducts
  where the monad is strong with respect to the comonad.

  Suppose $(H,m)$ and $(J,n)$ are the adjoint monoidal functors given
  in Definition~\ref{def:CS4-single-adjoint-cat-model}, and define
  $\Box = JH$ and $\Diamond = HJ$.  By definition we assumed that
  $(\Box, q)$, where $q_{A,B} : \Box A \times \Box B \to \Box (A
  \times B)$, is monoidal, but we must show that $\Diamond$ is also
  monoidal.  We know that both $(H,n)$ and $(J,m)$ are monoidal
  endofunctors on $\cat{C}$ which implies that their composition
  $\Diamond$ is monoidal where
  \[
  \begin{array}{lll}
    \mathsf{p}_{1} = \eta_{1} : 1 \to \Diamond 1\\
    \mathsf{p}_{A,B} = \m{HA,HB};J(\mathsf{n}_{A,B})
    \colon \Diamond A \pd \Diamond B \mto \Diamond(A \pd B)
  \end{array}
  \]
  and the following diagrams commute (proofs omitted):
  \begin{mathpar}
    \scriptsize
    \bfig
    \vSquares|ammmmma|/->`->`->``->`->`->/[
      (\Diamond A \times \Diamond B) \times \Diamond C`
      \Diamond A \times (\Diamond B \times \Diamond C)`
      \Diamond(A \times B) \times \Diamond C`
      \Diamond A \times \Diamond(B \times C)`
      \Diamond ((A \times B) \times C)`
      \Diamond (A \times (B \times C));
      \alpha`
      \mathsf{p}_{A,B} \times \id_{\Diamond C}`
      \id_{\Diamond A} \times \mathsf{p}_{B,C}``
      \mathsf{p}_{A \times B,C}`
      \mathsf{p}_{A,B \times C}`
      \Diamond \alpha]
    \efig
    \and
    \bfig
    \hSquares|ammmmaa|/->``->`<-``->`/[
      1 \times \Diamond A`
      \Diamond A``
      \Diamond 1 \times \Diamond A`
      \Diamond(1 \times A)`;
      \lambda_{\Diamond A}``
      \mathsf{p}_{1} \times \id_{\Diamond A}`
      \Diamond \lambda_A``
      \mathsf{p}_{1,A}`]
    \efig
    \and
    \bfig
    \hSquares|ammmmaa|/->``->`<-``->`/[
      \Diamond A \times 1`
      \Diamond A``
      \Diamond A \times \Diamond 1`
      \Diamond(A \times 1)`;
      \rho_{\Diamond A}``
      \id_{\Diamond A} \times \mathsf{p}_{1}`
      \Diamond \rho_A``
      \mathsf{p}_{A,1}`]
    \efig
    \and
    \bfig
    \hSquares|ammmmaa|/->``->`->``->`/[
      \Diamond A \times \Diamond B`
      \Diamond B \times \Diamond A``
      \Diamond (A \times B)`
      \Diamond (B \times A)`;
      \beta_{\Diamond A,\Diamond B}``
      \mathsf{p}_{A,B}`
      \mathsf{p}_{B,A}``
      \Diamond\beta_{A,B}`]
    \efig
  \end{mathpar}

  Furthermore, suppose $J \dashv H$, where the unit, $\varepsilon :
  \Box A \to A$, and the counit, $\eta : A \to \Diamond A$, are
  monoidal natural transformations.  This implies that the following
  diagrams commute:
  \begin{mathpar}
    \bfig
    \btriangle<800,500>[A \times B`\Diamond A \times \Diamond B`\Diamond (A \times B);\eta_A \times \eta_B`\eta_{A \times B}`\mathsf{p}_{A,B}]
    \efig
    \and
    \bfig
    \qtriangle<800,500>[\Box A \times \Box B`\Box (A \times B)`A \times B;\mathsf{q}_{A,B}`\varepsilon_A \times \varepsilon_B`\varepsilon_{A \times B}]
    \efig
    \and
    \bfig
    \qtriangle<800,500>[1`J 1`\Diamond 1;n_{1}`\eta_1`J m_1]       
    \efig
    \and
    \bfig
    \hSquares|ammmmaa|/->``=`->``<-`/[
      \Box 1`
      1``
      \Box 1`
      H 1`;
      \varepsilon_1```
      m_1``
      H n_1`]
    \efig
    \and
    \bfig
    \qtriangle<800,500>[H A`H\Box A`H A;\eta_{H A}`\id_{H A}`H\varepsilon_A]       
    \efig
    \and
    \bfig
    \qtriangle<800,500>[J A`\Box J A`J A;J\eta_A`\id_{J A}`\varepsilon_{J A}]
    \efig    
  \end{mathpar}
  It is a well-known fact about adjoints that $(\Box, \varepsilon,
  \delta)$, where $\delta : \Box A \to \Box\Box A$ is a comonad, and
  $(\Diamond, \eta, \mu)$, where $\mu : \Diamond\Diamond A \to
  \Diamond A$ is a monad.  In addition, $\mu$ and $\delta$ are monoidal
  natural transformations where we have the following:
  \[
  \begin{array}{lll}
    \d{1} = \p{1};\Diamond\p{1} : 1 \mto \Diamond^2 1\\
    \d{A,B} =  \p{\Diamond A,\Diamond B};\Diamond\p{A,B} : \Diamond^2 A \times \Diamond^2 B \mto \Diamond^2 (A \times B)\\
    \\
    \b{1} = \q{1};\Box\q{1} : 1 \mto \Box^2 1\\
    \b{A,B} = \q{\Box A,\Box B};\Box\q{A,B} : \Box^2 A \times \Box^2 B \mto \Box^2 (A \times B)\\
  \end{array}
  \]
  Thus, the following diagrams commute:
  \begin{mathpar}
    \bfig
    \hSquares|ammmmaa|/->``->`->``->`/[
      \Diamond^3 A`
      \Diamond^2 A``
      \Diamond^2 A`
      \Diamond A`;
      \Diamond \mu_A``
      \mu_{\Diamond A}`
      \mu_A``
      \mu_A`]
    \efig
    \and
    \bfig
    \qtriangle/->`=`->/<800,500>[\Diamond A`\Diamond^2 A`\Diamond A;\eta_{\Diamond A}``\mu_A]
    \btriangle(0,0)/->`=`->/<800,500>[\Diamond A`\Diamond^2 A`\Diamond A;\Diamond \eta_{A}``\mu_A]
    \efig
    \and
    \bfig
    \hSquares|ammmmaa|/->``->`->``->`/[
      \Box A`
      \Box^2 A``
      \Box^2 A`
      \Box^3 A`;
      \delta_A``
      \delta_A`
      \delta_{\Box A}``
      \Box\delta_A`]
    \efig
    \and
    \bfig
    \qtriangle/->`=`->/<800,500>[\Box A`\Box^2 A`\Box A;\delta_A``\Box \varepsilon]
    \btriangle(0,0)/->`=`->/<800,500>[\Box A`\Box^2 A`\Box A;\delta_A``\varepsilon_{\Box A}]
    \efig
    \and
    %% \bfig   
    %% \vSquares|ammmmma|/->`->```->``->/[
    %%   \Diamond^2 A \times \Diamond^2 B`
    %%   \Diamond A \times \Diamond B`
    %%   \Diamond(\Diamond A \times \Diamond B)``
    %%   \Diamond^2(A \times B)`
    %%   \Diamond(A \times B);
    %%   \mu_A \times \mu_B`
    %%   \p{\Diamond A,\Diamond B}```
    %%   \Diamond\p{A,B}``
    %%   \mu_{A \times B}]
    %% \morphism(1108,0)/<-/<0,1000>[\Diamond(A \times B)`\Diamond A \times \Diamond B;\p{A,B}]
    %% \efig
    \and
    \bfig
    \square<1000,1000>[
      \Diamond^2 A \times \Diamond^2 B`
      \Diamond A \times \Diamond B`
      \Diamond^2(A \times B)`
      \Diamond(A \times B);
      \mu_A \times \mu_B`
      \d{A,B}`
      \p{A,B}`
      \mu_{A \times B}]
    \efig
    \and
    \bfig
    \Vtriangle/->`<-`<-/[
      \Diamond^2 1`
      \Diamond 1`
      1;
      \mu_1`
      \d{1}`
      \p{1}]
    \efig
    \and        
    \bfig
    \square<1000,1000>[
      \Box A \times \Box B`
      \Box^2 A \times \Box^2 B`
      \Box(A \times B)`
      \Box^2(A \times B);
      \delta_A \times \delta_B`
      \q{A,B}`
      \b{A,B}`
      \delta_{A \times B}]
    %% \vSquares|ammmmma|/->``->```->`->/[
    %%   \Box A \times \Box B`
    %%   \Box^2 A \times \Box^2 B``
    %%   \Box(\Box A \times \Box B)`
    %%   \Box(A \times B)`
    %%   \Box^2(A \times B);
    %%   \varepsilon_A \times \varepsilon_B``
    %%   \q{\Box A,\Box B}```
    %%   \Box\q{A,B}`
    %%   \varepsilon_{A \times B}]
    %% \morphism(0,0)/<-/<0,1000>[\Box(A \times B)`\Box A \times \Box B;\q{A,B}]
    \efig
    \and
    \bfig
    \Vtriangle/->`<-`<-/[
      \Box 1`
      \Box^2 1`
      1;
      \delta_1`
      \q{1}`
      \d{1}]
    \efig
  \end{mathpar}

  We can now define the $\Box$-strength map as follows:
  \[
  \st{A}{B} = (\eta_{\Box A} \pd \id_{\Diamond B});\mathsf{p}_{\Box A,B} : \Box A \pd \Diamond B \mto \Diamond(\Box A \pd B)
  \]
  We can see that $\st{A}{B}$ is a natural transformation, because it
  is defined as a composition of natural transformations.
  
  %% To prove that the appropriate diagrams commute we first note that
  %% the triangle  
  Next we must show that all of the appropriate diagrams given in
  Definition~\ref{def:comonad-strong-monad} commute.
  \begin{itemize}
  \item[] \textit{Case 1. the first projection interacts with $\Diamond$:}
    \[
    \bfig
    \square|amma|<850,600>[
      \Box A \times \Diamond B`
      \Diamond (\Box A \times B)`
      \Box A`
      \Diamond\Box A;
      \st{A}{B}`
      \pi_1`
      \Diamond\pi_1`
      \eta_A]
    \efig
    \]
    This diagrams commutes, because the following diagram commutes:
    \[
    \bfig
    \square|ammm|/->`->``/<1500,2000>[
      \Box A \times \Diamond B`
      \Diamond\Box A \times \Diamond B`
      \Box A`;
      \eta_{\Box A} \times \id_{\Diamond B}`
      \pi_1``]

    \square(1500,0)|ammm|/->``->`/<1500,2000>[
      \Diamond\Box A \times \Diamond B`
      \Diamond (\Box A \times B)``
      \Diamond\Box A;
      \p{\Box A,B}``
      \Diamond\pi_1`]

    \morphism<3000,0>[\Box A`\Diamond\Box A;\eta_{\Box A}]

    \morphism(484,1500)<1000,0>[
      \Box A \times 1`
      \Diamond\Box A \times \Diamond 1;
      \eta_{\Box A} \times \eta_1]

    \morphism(1484,1500)<700,-500>[
      \Diamond\Box A \times \Diamond 1`
      \Diamond(\Box A \times 1);
      \p{\Box A,1}]

    \morphism(484,1500)|m|/{@{>}@/_1em/}/<1700,-500>[
      \Box A \times 1`
      \Diamond(\Box A \times 1);
      \eta_{\Box A \times 1}]

    \morphism(484,1500)|m|<700,-800>[
      \Box A \times 1`
      \Box A;
      \pi_1]

    \morphism(0,2000)|m|<484,-500>[
      \Box A \times \Diamond B`
      \Box A \times 1;
      \id_{\Box A} \times \t_{\Diamond B}]

    \morphism(1500,2000)|m|<-17,-500>[
      \Diamond\Box A \times \Diamond B`
      \Diamond\Box A \times \Diamond 1;
      \id_{\Diamond\Box A} \times \Diamond\t_{B}]

    \morphism(3000,2000)|m|<-815,-1000>[
      \Diamond(\Box A \times B)`
      \Diamond(\Box A \times 1);
      \Diamond (id_{\Box A} \times \t_{B})]

    \morphism(2184,1000)|m|<816,-1000>[
      \Diamond(\Box A \times 1)`
      \Diamond\Box A;
      \Diamond\pi_1]

    \morphism(1184,700)|m|<1816,-700>[
      \Box A`
      \Diamond\Box A;
      \eta_{\Box A}]
    
    \place(2700,1000)[1]
    \place(2150,1700)[2]
    \place(800,1750)[3]
    \place(1300,1300)[4]
    \place(1800,800)[5]
    \place(500,500)[6]
    \efig    
    \]
    Diagrams 1 and 6 commute because we are in a cartesian closed
    category, diagram 2 commutes by naturality of $\p{}$, diagram 3
    commutes because $\Diamond$ is a product functor, diagram 4
    commutes because $\eta$ is the unit of a symmetric monoidal
    adjunction, and diagram 5 commutes by naturality of $\eta$.
    
  \item[] \textit{Case 2. the second projection interacts with $\Diamond$:}
    \[
    \bfig
    \qtriangle<850,600>[
      \Box A \times \Diamond B`
      \Diamond (\Box A \times B)`
      \Diamond B;
      \st{A}{B}`
      \pi_2`
      \Diamond\pi_2]
    \efig
    \]
    This case is similar to the previous case.
    
    %% \item[] \textit{Case 1. the object 1 behaves as the unit for products}
  %%   $$
  %%   \bfig
  %%   \vSquares|ammmmma|/>``>```>`>/[\Box 1 \times \Diamond A`\Diamond (\Box 1 \times A)``\Diamond(1 \times A)`1 \times \Diamond A`\Diamond A;\st{1}{A}``\Diamond(\varepsilon_1 \times \id_A)```\Diamond\lambda`\lambda]
  %%   \morphism(0,1000)|m|/->/<0,-950>[`;\varepsilon_1 \times \id_{\Diamond A}]
  %%   \efig
  %%   $$
  %%   This diagram commutes by commutativity of the following diagram:
  %%   %% Equational version:
  %%   %% \begin{center}
  %%   %%   \begin{math}
  %%   %%     \begin{array}{rllllllll}
  %%   %%       & & \st{1}{A};\Diamond (\varepsilon_1 \times \id_A);\Diamond\lambda_A\\
  %%   %%       \text{(Definition of $\mathsf{st}$)}
  %%   %%       & = & (\eta_{\Box 1} \times \id_{\Diamond A});\p{\Box 1,\Diamond A};\Diamond (\varepsilon_1 \times \id_A);\Diamond\lambda_A\\
  %%   %%       \text{(Naturality of $\mathsf{p}$)}
  %%   %%       & = & (\eta_{\Box 1} \times \id_{\Diamond A});(\Diamond \varepsilon_1 \times \Diamond\id_A);\p{1,A};\Diamond\lambda_A\\
  %%   %%       \text{(Functoriality of $\times$)}
  %%   %%       & = & ((\eta_{\Box 1};\Diamond \varepsilon_1) \times (\id_{\Diamond A};\Diamond\id_A));\p{1,A};\Diamond\lambda_A\\
  %%   %%       & = & ((\eta_{\Box 1};\Diamond \varepsilon_1) \times (\id_{\Diamond A};\id_{\Diamond A}));\p{1,A};\Diamond\lambda_A\\
  %%   %%       \text{(Naturality of $\eta$)}
  %%   %%       & = & ((\varepsilon_1;\eta_{1}) \times (\id_{\Diamond A};\id_{\Diamond A}));\p{1,A};\Diamond\lambda_A\\
  %%   %%       \text{(Definition of $\mathsf{p}$)}
  %%   %%       & = & ((\varepsilon_1;\p{1}) \times (\id_{\Diamond A};\id_{\Diamond A}));\p{1,A};\Diamond\lambda_A\\
  %%   %%       \text{(Functoriality of $\times$)}
  %%   %%       & = & (\varepsilon_1 \times \id_{\Diamond A});(\p{1} \times \id_{\Diamond A});\p{1,A};\Diamond\lambda_A\\
  %%   %%       \text{($\Diamond$ is Symmetric Monoidal)}
  %%   %%       & = & (\varepsilon_1 \times \id_{\Diamond A});\lambda_{\Diamond A}\\
  %%   %%     \end{array}
  %%   %%   \end{math}
  %%   %% \end{center}
  %%   $$
  %%   \bfig
  %%   \square|amma|<1000,500>[
  %%     \Diamond\Box 1 \times \Diamond A`
  %%     \Diamond (\Box 1 \times A)`
  %%     \Diamond 1 \times \Diamond A`
  %%     \Diamond (1 \times A);
  %%     \p{\Box 1,A}`
  %%     \Diamond \varepsilon_1 \times \id_{\Diamond A}`
  %%     \Diamond (\varepsilon_1 \times \id_A)`
  %%     \p{1,A}]

  %%   \square(-1000,0)|amma|<1000,500>[
  %%     \Box 1 \times \Diamond A`
  %%     \Diamond\Box 1 \times \Diamond A`
  %%     1 \times \Diamond A`
  %%     \Diamond 1 \times \Diamond A;
  %%     \eta_{\Box 1} \times \id_{\Diamond A}`
  %%     \varepsilon_1 \times \id_{\Diamond A}`
  %%     \Diamond \varepsilon_1 \times \id_{\Diamond A}`
  %%     (\p{1} = \eta_1) \times \id_{\Diamond A}]

  %%   \qtriangle(-1000,-500)|mmm|/`->`->/<2000,500>[
  %%     1 \times \Diamond A`
  %%     \Diamond (1 \times A)`
  %%     \Diamond A;`
  %%     \lambda_{\Diamond A}`
  %%     \Diamond\lambda_A]

  %%   \place(500,250)[1]
  %%   \place(-500,250)[2]
  %%   \place(500,-200)[3]
  %%   \efig
  %%   $$
  %%   \noindent
  %%   Diagram 1 commutes by naturality of $\mathsf{p}$, diagram 2
  %%   commutes by naturality of $\eta$, and diagram 3 commutes because
  %%   $\Diamond$ is a symmetric monoidal functor.

  \item[] \textit{Case 3. unit $\eta$ of the monad and strength interact well, $\Box  A $ is a parameter}
    $$
    \bfig
    \btriangle<800,500>[
      \Box A \pd B`
      \Box A \pd \Diamond B`
      \Diamond(\Box A \pd B);
      \id_{\Box A} \pd \eta_{B}`
      \eta_{\Box A \times B}`
      \st{A}{B}]
    \efig
    $$

    The previous diagram commutes, because the following diagram commutes:
    $$
    \bfig
    \btriangle|ama|/->`->`->/<1500,500>[
      \Box A \pd B`
      \Box A \pd \Diamond B`
      \Diamond\Box A \pd \Diamond B;
      \id_{\Box A} \pd \eta_{B}`
      \eta_{\Box A} \pd \eta_B`
      \eta_{\Box A} \pd \id_{\Diamond B}]

    \qtriangle(0,0)/->``<-/<1500,500>[
      \Box A \pd B`
      \Diamond (\Box A \times B)`
      \Diamond\Box A \pd \Diamond B;
      \eta_{\Box A \times B}``
      \p{\Box A,B}]

    \place(250,200)[1]
    \place(1200,300)[2]
    \efig
    $$
    \noindent
    Diagram 1 clearly commutes, and diagram 2 commutes because $\eta$
    is a symmetric monoidal natural transformation.
    
  %% \item[] \textit{Case 3. co-unit of the comonad $\varepsilon$ and unit of the monad $\eta$ interact well?}
  %%   $$
  %%   \bfig
  %%   \hSquares|aamaaaa|/->`->`->```->`/[
  %%     \Box A \times 1`
  %%     \Box A \times \Diamond 1`
  %%     \Diamond (\Box A \times 1)`
  %%     \Box A`
  %%     A`
  %%     ;
  %%     \id_{\Box A} \times \eta_1`
  %%     \st{A}{1}`
  %%     \rho_{\Box A}```
  %%     \varepsilon_A`]
  %%   \qtriangle(1978,0)/->``->/<800,500>[\Diamond (\Box A \times 1)`\Diamond\Box A`\Diamond A;\Diamond \rho_{\Box A}``\Diamond\varepsilon_A]
  %%   \morphism(1060,0)/->/<1640,0>[`;\eta_A]
  %%   \efig
  %%   $$
  %%   \noindent
  %%   Recall that
  %%   $\st{A}{1} = (\eta_{\Box A} \pd \id_{\Diamond 1});\p{\Box A,1}$.
  %%   Now the previous diagram commutes, because the following diagram commutes:
  %%   $$
  %%   \bfig
  %%   \btriangle|mma|<1444,1000>[\Box A \pd 1`\Diamond\Box A \pd \Diamond 1`\Diamond(\Box A \pd 1);\eta_{\Box A} \pd \eta_1`\eta_{\Box A \pd 1}`\p{\Box A,1}]
  %%   \dtriangle(-1000,0)/->``->/<1000,1000>[\Box A \pd 1`\Box A \pd 1`\Diamond\Box A \pd \Diamond 1;\id_{\Box A} \pd \eta_1``\eta_{\Box A} \pd
  %%     \id_{\Diamond 1}]

  %%   \hSquares(0,0)/->`->```<-``/<1000>[\Box A \pd 1`\Box A`\Diamond\Box A```\Diamond (\Box A \times 1);\rho_{\Box A}`\eta_{\Box A}```\Diamond (\rho_{\Box A})``]

  %%   \square(746,1000)/->`<-`<-`/<698,500>[A`\Diamond A`\Box A`\Diamond\Box A;\eta_A`\varepsilon_A`\Diamond\varepsilon_A`]

  %%   \place(-400,300)[1]
  %%   \place(400,300)[2]
  %%   \place(1100,700)[3]
  %%   \place(1100,1250)[4]
  %%   \efig
  %%   $$
  %%   \noindent
  %%   Diagram 1 commutes by functorality of $\times$, diagram 2 commutes
  %%   because $\eta$ is a monoidal natural transformation, and diagrams
  %%   3 and 4 commute by naturality of $\eta$.

  \item[] \textit{Case 4. associativity $\alpha$ interacts with co-monoidicity of $\Box$}
    $$
    \bfig
    \vSquares|ammmmmm|/->`->`->```->`/[
      \Box A \times (\Box B \times \Diamond C)`
      \Box A \times \Diamond(\Box B \times C)`
      (\Box A \times \Box B) \times \Diamond C`
      \Diamond(\Box A \times (\Box B \times C))``
      \Diamond((\Box A \times \Box B) \times C);
      \id_{\Box A} \pd \st{B}{C}`
      \alpha^{-1}`
      \st{A}{\Box B \times C}```
      \Diamond\alpha^{-1}`]
    \morphism(1554,0)|m|/->/<0,-500>[`\Diamond(\Box(A \times B) \times C);\Diamond(\m{A,B} \times \id_C)]
    
    \morphism(0,500)|m|/->/<0,-1000>[`\Box(A \times B) \times \Diamond C;\m{A,B} \times \id_{\Diamond C}]

    \morphism(350,-500)|a|/->/<800,0>[`;\st{A \times B}{C}]
    \efig
    $$
    \noindent
    Recall that:
    \[
    \begin{array}{rlll}
      \st{B}{C}              & = & (\eta_{\Box B} \pd \id_{\Diamond C});\p{\Box B,C}\\
      \st{A \pd B}{C}        & = & (\eta_{\Box (A \pd B)} \pd \id_{\Diamond C});\p{\Box (A \pd B),C}\\
      \st{A}{\Box B \pd C} & = & (\eta_{\Box A} \pd \id_{\Diamond (\Box B \pd C)});\p{\Box A,(\Box B \pd C)}\\
    \end{array}
    \]
    In addition, we require the following diagram (whose commutativity
    is implied by the fact that $\Diamond$ is a symmetric monoidal
    functor):
    $$
    \bfig
    \vSquares|ammmmma|/->`->`->``->`->`->/[
      \Diamond A \pd (\Diamond B \pd \Diamond C)`
      (\Diamond A \pd \Diamond B) \pd \Diamond C`
      \Diamond A \pd \Diamond (B \pd C)`
      \Diamond (A \pd B) \pd \Diamond C`
      \Diamond (A \pd (B \pd C))`
      \Diamond ((A \pd B) \pd C);
      \alpha^{-1}_{\Diamond A,\Diamond B,\Diamond C}`
      \id_{\Diamond A} \pd \p{B,C}`
      \p{A,B} \pd \id_{\Diamond C}``
      \p{A,B \pd C}`
      \p{A \pd B,C}`
      \Diamond\alpha^{-1}_{A,B,C}]
    \efig
    $$
    \noindent
    Finally, this case follows because the following diagram commutes:
    \begin{center}
      \rotatebox{90}{$
    \bfig
    \btriangle|mmm|<1769,1000>[
      \Box A \pd (\Diamond\Box B \pd \Diamond C)`
      \Diamond\Box A \pd (\Diamond\Box B \pd \Diamond C)`
      \Diamond\Box A \pd (\Diamond\Box B \pd C);
      \eta_{\Box A} \pd \id_{\Diamond \Box B \pd C}`
      \eta_{\Box A} \pd \p{\Box B,C}`
      \id_{\Diamond\Box A} \pd \p{\Box B,C}]

    \qtriangle|mam|/->``->/<1769,1000>[
      \Box A \pd (\Diamond\Box B \pd \Diamond C)`
      \Box A \pd \Diamond (\Box B \pd C)`
      \Diamond\Box A \pd (\Diamond\Box B \pd C);
      \id_{\Box A} \pd \p{\Box B,C}``
      \eta_{\Box A} \pd \id_{\Diamond (\Box B \pd C)}]    

    \qtriangle(-1800,0)|mmm|<1800,1000>[
      \Box A \pd (\Box B \pd \Diamond C)`
      \Box A \pd (\Diamond\Box B \pd \Diamond C)`
      \Diamond\Box A \pd (\Diamond\Box B \pd \Diamond C);
      \id_{\Box A} \pd (\eta_{\Box B} \pd \id_{\Diamond C})`
      \eta_{\Box A} \pd (\eta_{\Box B} \pd \id_{\Diamond C})`
      \eta_{\Box A} \pd \id_{\Diamond \Box B \pd C}]

    \square(0,-500)|mmmm|/`->`->`/<1769,500>[
      \Diamond\Box A \pd (\Diamond\Box B \pd \Diamond C)`
      \Diamond\Box A \pd (\Diamond\Box B \pd C)`
      (\Diamond\Box A \pd \Diamond\Box B) \pd \Diamond C`
      \Diamond (\Box A \pd (\Box B \pd C));`
      \alpha_{\Diamond\Box A,\Diamond\Box B,\Diamond C}`
      \p{\Box A,\Box B \pd C}`]   

    \square(0,-1000)|mmmm|/`->`->`/<1769,500>[
      (\Diamond\Box A \pd \Diamond\Box B) \pd \Diamond C`
      \Diamond (\Box A \pd (\Box B \pd C))`
      \Diamond (\Box A \pd \Box B) \pd \Diamond C`
      \Diamond ((\Box A \pd \Box B) \pd C);`
      \p{\Box A,\Box B} \pd \id_{\Diamond C}`
      \Diamond \alpha_{\Box A,\Box B,C}`]        

    \square(0,-1500)|mmmm|<1769,500>[
      \Diamond (\Box A \pd \Box B) \pd \Diamond C`
      \Diamond ((\Box A \pd \Box B) \pd C)`
      \Diamond\Box(A \pd B) \pd \Diamond C`
      \Diamond (\Box(A \pd B) \pd C);
      \p{\Box A \pd \Box B, C}`
      \Diamond \m{A,B} \pd \id_{\Diamond C}`
      \Diamond (\m{A,B} \pd \id_C)`
      \p{\Box (A \pd B),C}]

    \btriangle(-1800,-500)|mmm|/->``->/<1800,1500>[
      \Box A \pd (\Box B \pd \Diamond C)`
      (\Box A \pd \Box B) \pd \Diamond C`
      (\Diamond\Box A \pd \Diamond\Box B) \pd \Diamond C;
      \alpha_{\Box A,\Box B,\Diamond C}``
      (\eta_{\Box A} \pd \eta_{\Box B}) \pd \id_{\Diamond C}]

    \qtriangle(-1800,-1000)|mmm|/`->`/<1800,500>[
      (\Box A \pd \Box B) \pd \Diamond C`
      (\Diamond\Box A \pd \Diamond\Box B) \pd \Diamond C`
      \Diamond (\Box A \pd \Box B) \pd \Diamond C;`
      \eta_{\Box A \pd \Box B} \pd \id_{\Diamond C}`]

    \btriangle(-1800,-1500)|mmm|/->``->/<1800,1000>[
      (\Box A \pd \Box B) \pd \Diamond C`
      \Box(A \pd B) \pd \Diamond C`
      \Diamond\Box(A \pd B) \pd \Diamond C;
      \m{A,B} \pd \id_{\Diamond C}``
      \eta_{\Box (A \pd B)} \pd \id_{\Diamond C}]

    \place(1200,700)[1]
    \place(500,300)[2]
    \place(900,-500)[3]
    \place(900,-1250)[4]
    \place(-500,700)[5]
    \place(-1000,0)[6]
    \place(-400,-700)[7]
    \place(-1000,-1100)[8]
    \efig
    $}
    \end{center}
    Diagrams 1, 2 and 5 commute by functorality of $\times$, diagram 3
    commutes by the additional diagram from above, diagram 4 commutes
    by naturality of $\mathsf{p}$, diagram 6 commutes by naturality of
    $\alpha$, diagram 7 commutes by the fact that $\eta$ is a monoidal
    natural transformation, and diagram 8 commutes by naturality of
    $\eta$.
    

  \item[] \textit{Case 5.  strength interacts with monoidicity of $\Diamond$}
    $$
    \bfig
    \vSquares|ammmmma|/->`->```->``->/[
      \Box A \times \Diamond\Diamond B`
      \Box A \times \Diamond B`
      \Diamond(\Box A \times \Diamond B)``
      \Diamond\Diamond(\Box A \times B)`
      \Diamond(\Box A \times B);
      \id_{\Box A} \pd \mu_{B}`
      \st{A}{\Diamond B}```
      \Diamond(\st{A}{B})``
      \mu_{\Box A \pd B}]
    \morphism(1150,1000)|m|<0,-920>[`;\st{A}{B}]
    \efig
    $$
    \noindent
    Recall that:
    \[
    \begin{array}{rlll}
      \st{A}{B} & = & (\eta_{\Box A} \pd \id_{\Diamond B});\p{\Box A,B}\\
      \st{A}{\Diamond B} & = & (\eta_{\Box A} \pd \id_{\Diamond \Diamond B});\p{\Box A,\Diamond B}\\
    \end{array}
    \]
    This case follows from the fact that the following diagram
    commutes:
    \begin{center}
      \rotatebox{90}{$\bfig
    \qtriangle|mmm|<1500,1000>[
      \Box A \pd \Diamond\Diamond B`
      \Box A \pd \Diamond B`
      \Diamond\Box A \pd \Diamond B;
      \id_{\Box A} \pd \mu_B`
      \eta_{\Box A} \pd \mu_B`
      \eta_{\Box A} \pd \id_{\Diamond B}]

    \morphism(-1500,1000)|m|/<-/<1500,0>[
      \Diamond\Box A \pd \Diamond\Diamond B`
      \Box A \pd \Diamond\Diamond B;
      \eta_{\Box A} \pd \id_{\Diamond\Diamond B}]

    \btriangle(-1500,0)|mmm|<3000,1000>[
      \Diamond\Box A \pd \Diamond\Diamond B`
      \Diamond\Diamond\Box A \pd \Diamond\Diamond B`
      \Diamond\Box A \pd \Diamond B;
      \Diamond\eta_{\Box A} \pd \id_{\Diamond\Diamond B}`
      \id_{\Diamond\Box A} \pd \mu_B`
      \mu_{\Box A} \pd \mu_B]

    \square(-3000,0)|mmmm|/<-`->``<-/<1500,1000>[
      \Diamond(\Box A \pd \Diamond B)`
      \Diamond\Box A \pd \Diamond\Diamond B`
      \Diamond(\Diamond\Box A \pd \Diamond B)`
      \Diamond\Diamond\Box A \pd \Diamond\Diamond B;
      \p{\Box A,\Diamond B}`
      \Diamond (\eta_{\Box A} \pd \id_{\Diamond B})``
      \p{\Diamond\Box A,\Diamond B}]

    \square(-3000,-500)|mmmm|/`->`->`->/<4500,500>[
      \Diamond (\Diamond\Box A \pd \Diamond B)`
      \Diamond\Box A \pd \Diamond B`
      \Diamond\Diamond (\Box A \pd B)`
      \Diamond (\Box A \pd B);`
      \Diamond (\p{\Box A,B})`
      \p{\Box A,B}`
      \mu_{\Box A \pd B}]

    \place(-2300,500)[1]
    \place(-800,-250)[2]
    \place(-800,400)[3]
    \place(0,700)[4]
    \place(1050,700)[5]
    \efig$}
    \end{center}       
    Diagram commutes by naturality of $\mathsf{p}$, diagram 2 commutes
    because $\mu$ is a monoidal natural transformation, diagram 3
    commutes because $\mu$ is the monadic multiplication and by
    functorality of $\times$, and diagrams 4 and 5 commute by
    functoriality of $\times$.
    
  \item[] \textit{Case 6. commuting $\beta$ interacts with $\Diamond$}
    $$
    \bfig
    \hSquares|aamamaa|/``->``->`->`->/[
      \Diamond B \times \Box A``
      \Box A \times \Diamond B`
      \Diamond B \times \Diamond \Box A`
      \Diamond (B \times \Box A)`
      \Diamond (\Box A \times B);``
      \id_{\Diamond B} \pd \eta_{\Box A}``
      st_{A,B}`
      \p{B,\Box A}`
      \Diamond\beta_{B,\Box A}]
    \morphism(200,500)<1817,0>[`;\beta_{\Diamond B,\Box A}]
    \efig
    $$
    \noindent
    The previous diagram commutes by commutativity of the following
    diagram:
    $$
    \bfig
    \vSquares|ammmmma|[
      \Diamond B \times \Box A`
      \Box A \times \Diamond B`
      \Diamond B \times \Diamond\Box A`
      \Diamond\Box A \times \Diamond B`
      \Diamond (B \times \Box A)`
      \Diamond (\Box A \times B);
      \beta_{\Diamond B,\Box A}`
      \id_{\Diamond B} \times \eta_{\Box A}`
      \eta_{\Box A} \times \id_{\Diamond B}`
      \beta_{\Diamond B,\Diamond\Box A}`
      \p{B,\Box A}`
      \p{\Box A,B}`
      \Diamond\beta_{B,\Box A}]

    \place(600,750)[1]
    \place(600,250)[2]
    \efig
    $$
    \noindent
    Diagram 1 commutes because $\beta$ is a symmetric monoidal
    functor, and diagram 2 commutes by naturality of $\beta$.

  \item[] \textit{Case 7.  $\varepsilon$ interacts with $\Diamond$ and its monoidicity} 
    $$
    \bfig
    \hSquares|aamamaa|/``->``->`->`->/[
      \Box A \times \Diamond B``
      \Diamond(\Box A \times B)`
      A \times \Diamond B`
      \Diamond A \times \Diamond B`
      \Diamond (A \times B);``
      \varepsilon_A \times \id_{\Diamond B}``
      \Diamond(\varepsilon_A \times \id_{B})`
      \eta_A \times \id_{\Diamond B}`
      \p{A,B}]
    \morphism(200,500)<1605,0>[`;\st{A}{B}]
    \efig
    $$
    \noindent
    The previous diagram commutes by commutativity of the following
    diagram:
    $$
    \bfig
    \hSquares|aammmaa|[
      \Box A \times \Diamond B`
      \Diamond\Box A \times \Diamond B`
      \Diamond (\Box A \times B)`
      A \times \Diamond B`
      \Diamond A \times \Diamond B`
      \Diamond (A \times B);
      \eta_{\Box A} \times \id_{\Diamond B}`
      \p{\Box A,B}`
      \varepsilon_A \times \id_{\Diamond B}`
      \Diamond\varepsilon_A \times \id_{\Diamond B}`
      \Diamond (\varepsilon_A \times \id_B)`
      \eta_A \times \id_{\Diamond B}`
      \p{A,B}]
    \place(1750,250)[1]
    \place(600,250)[2]
    \efig
    $$
    \noindent
    Diagram 1 commutes by naturality of $\mathsf{p}$, and diagram 2
    commutes by naturality of $\eta$.
  \end{itemize}

\end{proof}\begin{proof}
  We must show that given the definition of an adjoint CS4 categorical
  model (Definition~\ref{def:CS4-single-adjoint-cat-model}) we can
  define an appropriate monad and comonad on a CCC with coproducts
  where the monad is strong with respect to the comonad.

  Suppose $(H,m)$ and $(J,n)$ are the adjoint monoidal functors given
  in Definition~\ref{def:CS4-single-adjoint-cat-model}, and define
  $\Box = JH$ and $\Diamond = HJ$.  By definition we assumed that
  $(\Box, q)$, where $q_{A,B} : \Box A \times \Box B \to \Box (A
  \times B)$, is monoidal, but we must show that $\Diamond$ is also
  monoidal.  We know that both $(H,n)$ and $(J,m)$ are monoidal
  endofunctors on $\cat{C}$ which implies that their composition
  $\Diamond$ is monoidal where
  \[
  \begin{array}{lll}
    \mathsf{p}_{1} = \eta_{1} : 1 \to \Diamond 1\\
    \mathsf{p}_{A,B} = \m{HA,HB};J(\mathsf{n}_{A,B})
    \colon \Diamond A \pd \Diamond B \mto \Diamond(A \pd B)
  \end{array}
  \]
  and the following diagrams commute (proofs omitted):
  \begin{mathpar}
    \scriptsize
    \bfig
    \vSquares|ammmmma|/->`->`->``->`->`->/[
      (\Diamond A \times \Diamond B) \times \Diamond C`
      \Diamond A \times (\Diamond B \times \Diamond C)`
      \Diamond(A \times B) \times \Diamond C`
      \Diamond A \times \Diamond(B \times C)`
      \Diamond ((A \times B) \times C)`
      \Diamond (A \times (B \times C));
      \alpha`
      \mathsf{p}_{A,B} \times \id_{\Diamond C}`
      \id_{\Diamond A} \times \mathsf{p}_{B,C}``
      \mathsf{p}_{A \times B,C}`
      \mathsf{p}_{A,B \times C}`
      \Diamond \alpha]
    \efig
    \and
    \bfig
    \hSquares|ammmmaa|/->``->`<-``->`/[
      1 \times \Diamond A`
      \Diamond A``
      \Diamond 1 \times \Diamond A`
      \Diamond(1 \times A)`;
      \lambda_{\Diamond A}``
      \mathsf{p}_{1} \times \id_{\Diamond A}`
      \Diamond \lambda_A``
      \mathsf{p}_{1,A}`]
    \efig
    \and
    \bfig
    \hSquares|ammmmaa|/->``->`<-``->`/[
      \Diamond A \times 1`
      \Diamond A``
      \Diamond A \times \Diamond 1`
      \Diamond(A \times 1)`;
      \rho_{\Diamond A}``
      \id_{\Diamond A} \times \mathsf{p}_{1}`
      \Diamond \rho_A``
      \mathsf{p}_{A,1}`]
    \efig
    \and
    \bfig
    \hSquares|ammmmaa|/->``->`->``->`/[
      \Diamond A \times \Diamond B`
      \Diamond B \times \Diamond A``
      \Diamond (A \times B)`
      \Diamond (B \times A)`;
      \beta_{\Diamond A,\Diamond B}``
      \mathsf{p}_{A,B}`
      \mathsf{p}_{B,A}``
      \Diamond\beta_{A,B}`]
    \efig
  \end{mathpar}

  Furthermore, suppose $J \dashv H$, where the unit, $\varepsilon :
  \Box A \to A$, and the counit, $\eta : A \to \Diamond A$, are
  monoidal natural transformations.  This implies that the following
  diagrams commute:
  \begin{mathpar}
    \bfig
    \btriangle<800,500>[A \times B`\Diamond A \times \Diamond B`\Diamond (A \times B);\eta_A \times \eta_B`\eta_{A \times B}`\mathsf{p}_{A,B}]
    \efig
    \and
    \bfig
    \qtriangle<800,500>[\Box A \times \Box B`\Box (A \times B)`A \times B;\mathsf{q}_{A,B}`\varepsilon_A \times \varepsilon_B`\varepsilon_{A \times B}]
    \efig
    \and
    \bfig
    \qtriangle<800,500>[1`J 1`\Diamond 1;n_{1}`\eta_1`J m_1]       
    \efig
    \and
    \bfig
    \hSquares|ammmmaa|/->``=`->``<-`/[
      \Box 1`
      1``
      \Box 1`
      H 1`;
      \varepsilon_1```
      m_1``
      H n_1`]
    \efig
    \and
    \bfig
    \qtriangle<800,500>[H A`H\Box A`H A;\eta_{H A}`\id_{H A}`H\varepsilon_A]       
    \efig
    \and
    \bfig
    \qtriangle<800,500>[J A`\Box J A`J A;J\eta_A`\id_{J A}`\varepsilon_{J A}]
    \efig    
  \end{mathpar}
  It is a well-known fact about adjoints that $(\Box, \varepsilon,
  \delta)$, where $\delta : \Box A \to \Box\Box A$ is a comonad, and
  $(\Diamond, \eta, \mu)$, where $\mu : \Diamond\Diamond A \to
  \Diamond A$ is a monad.  In addition, $\mu$ and $\delta$ are monoidal
  natural transformations where we have the following:
  \[
  \begin{array}{lll}
    \d{1} = \p{1};\Diamond\p{1} : 1 \mto \Diamond^2 1\\
    \d{A,B} =  \p{\Diamond A,\Diamond B};\Diamond\p{A,B} : \Diamond^2 A \times \Diamond^2 B \mto \Diamond^2 (A \times B)\\
    \\
    \b{1} = \q{1};\Box\q{1} : 1 \mto \Box^2 1\\
    \b{A,B} = \q{\Box A,\Box B};\Box\q{A,B} : \Box^2 A \times \Box^2 B \mto \Box^2 (A \times B)\\
  \end{array}
  \]
  Thus, the following diagrams commute:
  \begin{mathpar}
    \bfig
    \hSquares|ammmmaa|/->``->`->``->`/[
      \Diamond^3 A`
      \Diamond^2 A``
      \Diamond^2 A`
      \Diamond A`;
      \Diamond \mu_A``
      \mu_{\Diamond A}`
      \mu_A``
      \mu_A`]
    \efig
    \and
    \bfig
    \qtriangle/->`=`->/<800,500>[\Diamond A`\Diamond^2 A`\Diamond A;\eta_{\Diamond A}``\mu_A]
    \btriangle(0,0)/->`=`->/<800,500>[\Diamond A`\Diamond^2 A`\Diamond A;\Diamond \eta_{A}``\mu_A]
    \efig
    \and
    \bfig
    \hSquares|ammmmaa|/->``->`->``->`/[
      \Box A`
      \Box^2 A``
      \Box^2 A`
      \Box^3 A`;
      \delta_A``
      \delta_A`
      \delta_{\Box A}``
      \Box\delta_A`]
    \efig
    \and
    \bfig
    \qtriangle/->`=`->/<800,500>[\Box A`\Box^2 A`\Box A;\delta_A``\Box \varepsilon]
    \btriangle(0,0)/->`=`->/<800,500>[\Box A`\Box^2 A`\Box A;\delta_A``\varepsilon_{\Box A}]
    \efig
    \and
    %% \bfig   
    %% \vSquares|ammmmma|/->`->```->``->/[
    %%   \Diamond^2 A \times \Diamond^2 B`
    %%   \Diamond A \times \Diamond B`
    %%   \Diamond(\Diamond A \times \Diamond B)``
    %%   \Diamond^2(A \times B)`
    %%   \Diamond(A \times B);
    %%   \mu_A \times \mu_B`
    %%   \p{\Diamond A,\Diamond B}```
    %%   \Diamond\p{A,B}``
    %%   \mu_{A \times B}]
    %% \morphism(1108,0)/<-/<0,1000>[\Diamond(A \times B)`\Diamond A \times \Diamond B;\p{A,B}]
    %% \efig
    \and
    \bfig
    \square<1000,1000>[
      \Diamond^2 A \times \Diamond^2 B`
      \Diamond A \times \Diamond B`
      \Diamond^2(A \times B)`
      \Diamond(A \times B);
      \mu_A \times \mu_B`
      \d{A,B}`
      \p{A,B}`
      \mu_{A \times B}]
    \efig
    \and
    \bfig
    \Vtriangle/->`<-`<-/[
      \Diamond^2 1`
      \Diamond 1`
      1;
      \mu_1`
      \d{1}`
      \p{1}]
    \efig
    \and        
    \bfig
    \square<1000,1000>[
      \Box A \times \Box B`
      \Box^2 A \times \Box^2 B`
      \Box(A \times B)`
      \Box^2(A \times B);
      \delta_A \times \delta_B`
      \q{A,B}`
      \b{A,B}`
      \delta_{A \times B}]
    %% \vSquares|ammmmma|/->``->```->`->/[
    %%   \Box A \times \Box B`
    %%   \Box^2 A \times \Box^2 B``
    %%   \Box(\Box A \times \Box B)`
    %%   \Box(A \times B)`
    %%   \Box^2(A \times B);
    %%   \varepsilon_A \times \varepsilon_B``
    %%   \q{\Box A,\Box B}```
    %%   \Box\q{A,B}`
    %%   \varepsilon_{A \times B}]
    %% \morphism(0,0)/<-/<0,1000>[\Box(A \times B)`\Box A \times \Box B;\q{A,B}]
    \efig
    \and
    \bfig
    \Vtriangle/->`<-`<-/[
      \Box 1`
      \Box^2 1`
      1;
      \delta_1`
      \q{1}`
      \d{1}]
    \efig
  \end{mathpar}

  We can now define the $\Box$-strength map as follows:
  \[
  \st{A}{B} = (\eta_{\Box A} \pd \id_{\Diamond B});\mathsf{p}_{\Box A,B} : \Box A \pd \Diamond B \mto \Diamond(\Box A \pd B)
  \]
  We can see that $\st{A}{B}$ is a natural transformation, because it
  is defined as a composition of natural transformations.
  
  %% To prove that the appropriate diagrams commute we first note that
  %% the triangle  
  Next we must show that all of the appropriate diagrams given in
  Definition~\ref{def:comonad-strong-monad} commute.
  \begin{itemize}
  \item[] \textit{Case 1. the first projection interacts with $\Diamond$:}
    \[
    \bfig
    \square|amma|<850,600>[
      \Box A \times \Diamond B`
      \Diamond (\Box A \times B)`
      \Box A`
      \Diamond\Box A;
      \st{A}{B}`
      \pi_1`
      \Diamond\pi_1`
      \eta_A]
    \efig
    \]
    This diagrams commutes, because the following diagram commutes:
    \[
    \bfig
    \square|ammm|/->`->``/<1500,2000>[
      \Box A \times \Diamond B`
      \Diamond\Box A \times \Diamond B`
      \Box A`;
      \eta_{\Box A} \times \id_{\Diamond B}`
      \pi_1``]

    \square(1500,0)|ammm|/->``->`/<1500,2000>[
      \Diamond\Box A \times \Diamond B`
      \Diamond (\Box A \times B)``
      \Diamond\Box A;
      \p{\Box A,B}``
      \Diamond\pi_1`]

    \morphism<3000,0>[\Box A`\Diamond\Box A;\eta_{\Box A}]

    \morphism(484,1500)<1000,0>[
      \Box A \times 1`
      \Diamond\Box A \times \Diamond 1;
      \eta_{\Box A} \times \eta_1]

    \morphism(1484,1500)<700,-500>[
      \Diamond\Box A \times \Diamond 1`
      \Diamond(\Box A \times 1);
      \p{\Box A,1}]

    \morphism(484,1500)|m|/{@{>}@/_1em/}/<1700,-500>[
      \Box A \times 1`
      \Diamond(\Box A \times 1);
      \eta_{\Box A \times 1}]

    \morphism(484,1500)|m|<700,-800>[
      \Box A \times 1`
      \Box A;
      \pi_1]

    \morphism(0,2000)|m|<484,-500>[
      \Box A \times \Diamond B`
      \Box A \times 1;
      \id_{\Box A} \times \t_{\Diamond B}]

    \morphism(1500,2000)|m|<-17,-500>[
      \Diamond\Box A \times \Diamond B`
      \Diamond\Box A \times \Diamond 1;
      \id_{\Diamond\Box A} \times \Diamond\t_{B}]

    \morphism(3000,2000)|m|<-815,-1000>[
      \Diamond(\Box A \times B)`
      \Diamond(\Box A \times 1);
      \Diamond (id_{\Box A} \times \t_{B})]

    \morphism(2184,1000)|m|<816,-1000>[
      \Diamond(\Box A \times 1)`
      \Diamond\Box A;
      \Diamond\pi_1]

    \morphism(1184,700)|m|<1816,-700>[
      \Box A`
      \Diamond\Box A;
      \eta_{\Box A}]
    
    \place(2700,1000)[1]
    \place(2150,1700)[2]
    \place(800,1750)[3]
    \place(1300,1300)[4]
    \place(1800,800)[5]
    \place(500,500)[6]
    \efig    
    \]
    Diagrams 1 and 6 commute because we are in a cartesian closed
    category, diagram 2 commutes by naturality of $\p{}$, diagram 3
    commutes because $\Diamond$ is a product functor, diagram 4
    commutes because $\eta$ is the unit of a symmetric monoidal
    adjunction, and diagram 5 commutes by naturality of $\eta$.
    
  \item[] \textit{Case 2. the second projection interacts with $\Diamond$:}
    \[
    \bfig
    \qtriangle<850,600>[
      \Box A \times \Diamond B`
      \Diamond (\Box A \times B)`
      \Diamond B;
      \st{A}{B}`
      \pi_2`
      \Diamond\pi_2]
    \efig
    \]
    This case is similar to the previous case.
    
    %% \item[] \textit{Case 1. the object 1 behaves as the unit for products}
  %%   $$
  %%   \bfig
  %%   \vSquares|ammmmma|/>``>```>`>/[\Box 1 \times \Diamond A`\Diamond (\Box 1 \times A)``\Diamond(1 \times A)`1 \times \Diamond A`\Diamond A;\st{1}{A}``\Diamond(\varepsilon_1 \times \id_A)```\Diamond\lambda`\lambda]
  %%   \morphism(0,1000)|m|/->/<0,-950>[`;\varepsilon_1 \times \id_{\Diamond A}]
  %%   \efig
  %%   $$
  %%   This diagram commutes by commutativity of the following diagram:
  %%   %% Equational version:
  %%   %% \begin{center}
  %%   %%   \begin{math}
  %%   %%     \begin{array}{rllllllll}
  %%   %%       & & \st{1}{A};\Diamond (\varepsilon_1 \times \id_A);\Diamond\lambda_A\\
  %%   %%       \text{(Definition of $\mathsf{st}$)}
  %%   %%       & = & (\eta_{\Box 1} \times \id_{\Diamond A});\p{\Box 1,\Diamond A};\Diamond (\varepsilon_1 \times \id_A);\Diamond\lambda_A\\
  %%   %%       \text{(Naturality of $\mathsf{p}$)}
  %%   %%       & = & (\eta_{\Box 1} \times \id_{\Diamond A});(\Diamond \varepsilon_1 \times \Diamond\id_A);\p{1,A};\Diamond\lambda_A\\
  %%   %%       \text{(Functoriality of $\times$)}
  %%   %%       & = & ((\eta_{\Box 1};\Diamond \varepsilon_1) \times (\id_{\Diamond A};\Diamond\id_A));\p{1,A};\Diamond\lambda_A\\
  %%   %%       & = & ((\eta_{\Box 1};\Diamond \varepsilon_1) \times (\id_{\Diamond A};\id_{\Diamond A}));\p{1,A};\Diamond\lambda_A\\
  %%   %%       \text{(Naturality of $\eta$)}
  %%   %%       & = & ((\varepsilon_1;\eta_{1}) \times (\id_{\Diamond A};\id_{\Diamond A}));\p{1,A};\Diamond\lambda_A\\
  %%   %%       \text{(Definition of $\mathsf{p}$)}
  %%   %%       & = & ((\varepsilon_1;\p{1}) \times (\id_{\Diamond A};\id_{\Diamond A}));\p{1,A};\Diamond\lambda_A\\
  %%   %%       \text{(Functoriality of $\times$)}
  %%   %%       & = & (\varepsilon_1 \times \id_{\Diamond A});(\p{1} \times \id_{\Diamond A});\p{1,A};\Diamond\lambda_A\\
  %%   %%       \text{($\Diamond$ is Symmetric Monoidal)}
  %%   %%       & = & (\varepsilon_1 \times \id_{\Diamond A});\lambda_{\Diamond A}\\
  %%   %%     \end{array}
  %%   %%   \end{math}
  %%   %% \end{center}
  %%   $$
  %%   \bfig
  %%   \square|amma|<1000,500>[
  %%     \Diamond\Box 1 \times \Diamond A`
  %%     \Diamond (\Box 1 \times A)`
  %%     \Diamond 1 \times \Diamond A`
  %%     \Diamond (1 \times A);
  %%     \p{\Box 1,A}`
  %%     \Diamond \varepsilon_1 \times \id_{\Diamond A}`
  %%     \Diamond (\varepsilon_1 \times \id_A)`
  %%     \p{1,A}]

  %%   \square(-1000,0)|amma|<1000,500>[
  %%     \Box 1 \times \Diamond A`
  %%     \Diamond\Box 1 \times \Diamond A`
  %%     1 \times \Diamond A`
  %%     \Diamond 1 \times \Diamond A;
  %%     \eta_{\Box 1} \times \id_{\Diamond A}`
  %%     \varepsilon_1 \times \id_{\Diamond A}`
  %%     \Diamond \varepsilon_1 \times \id_{\Diamond A}`
  %%     (\p{1} = \eta_1) \times \id_{\Diamond A}]

  %%   \qtriangle(-1000,-500)|mmm|/`->`->/<2000,500>[
  %%     1 \times \Diamond A`
  %%     \Diamond (1 \times A)`
  %%     \Diamond A;`
  %%     \lambda_{\Diamond A}`
  %%     \Diamond\lambda_A]

  %%   \place(500,250)[1]
  %%   \place(-500,250)[2]
  %%   \place(500,-200)[3]
  %%   \efig
  %%   $$
  %%   \noindent
  %%   Diagram 1 commutes by naturality of $\mathsf{p}$, diagram 2
  %%   commutes by naturality of $\eta$, and diagram 3 commutes because
  %%   $\Diamond$ is a symmetric monoidal functor.

  \item[] \textit{Case 3. unit $\eta$ of the monad and strength interact well, $\Box  A $ is a parameter}
    $$
    \bfig
    \btriangle<800,500>[
      \Box A \pd B`
      \Box A \pd \Diamond B`
      \Diamond(\Box A \pd B);
      \id_{\Box A} \pd \eta_{B}`
      \eta_{\Box A \times B}`
      \st{A}{B}]
    \efig
    $$

    The previous diagram commutes, because the following diagram commutes:
    $$
    \bfig
    \btriangle|ama|/->`->`->/<1500,500>[
      \Box A \pd B`
      \Box A \pd \Diamond B`
      \Diamond\Box A \pd \Diamond B;
      \id_{\Box A} \pd \eta_{B}`
      \eta_{\Box A} \pd \eta_B`
      \eta_{\Box A} \pd \id_{\Diamond B}]

    \qtriangle(0,0)/->``<-/<1500,500>[
      \Box A \pd B`
      \Diamond (\Box A \times B)`
      \Diamond\Box A \pd \Diamond B;
      \eta_{\Box A \times B}``
      \p{\Box A,B}]

    \place(250,200)[1]
    \place(1200,300)[2]
    \efig
    $$
    \noindent
    Diagram 1 clearly commutes, and diagram 2 commutes because $\eta$
    is a symmetric monoidal natural transformation.
    
  %% \item[] \textit{Case 3. co-unit of the comonad $\varepsilon$ and unit of the monad $\eta$ interact well?}
  %%   $$
  %%   \bfig
  %%   \hSquares|aamaaaa|/->`->`->```->`/[
  %%     \Box A \times 1`
  %%     \Box A \times \Diamond 1`
  %%     \Diamond (\Box A \times 1)`
  %%     \Box A`
  %%     A`
  %%     ;
  %%     \id_{\Box A} \times \eta_1`
  %%     \st{A}{1}`
  %%     \rho_{\Box A}```
  %%     \varepsilon_A`]
  %%   \qtriangle(1978,0)/->``->/<800,500>[\Diamond (\Box A \times 1)`\Diamond\Box A`\Diamond A;\Diamond \rho_{\Box A}``\Diamond\varepsilon_A]
  %%   \morphism(1060,0)/->/<1640,0>[`;\eta_A]
  %%   \efig
  %%   $$
  %%   \noindent
  %%   Recall that
  %%   $\st{A}{1} = (\eta_{\Box A} \pd \id_{\Diamond 1});\p{\Box A,1}$.
  %%   Now the previous diagram commutes, because the following diagram commutes:
  %%   $$
  %%   \bfig
  %%   \btriangle|mma|<1444,1000>[\Box A \pd 1`\Diamond\Box A \pd \Diamond 1`\Diamond(\Box A \pd 1);\eta_{\Box A} \pd \eta_1`\eta_{\Box A \pd 1}`\p{\Box A,1}]
  %%   \dtriangle(-1000,0)/->``->/<1000,1000>[\Box A \pd 1`\Box A \pd 1`\Diamond\Box A \pd \Diamond 1;\id_{\Box A} \pd \eta_1``\eta_{\Box A} \pd
  %%     \id_{\Diamond 1}]

  %%   \hSquares(0,0)/->`->```<-``/<1000>[\Box A \pd 1`\Box A`\Diamond\Box A```\Diamond (\Box A \times 1);\rho_{\Box A}`\eta_{\Box A}```\Diamond (\rho_{\Box A})``]

  %%   \square(746,1000)/->`<-`<-`/<698,500>[A`\Diamond A`\Box A`\Diamond\Box A;\eta_A`\varepsilon_A`\Diamond\varepsilon_A`]

  %%   \place(-400,300)[1]
  %%   \place(400,300)[2]
  %%   \place(1100,700)[3]
  %%   \place(1100,1250)[4]
  %%   \efig
  %%   $$
  %%   \noindent
  %%   Diagram 1 commutes by functorality of $\times$, diagram 2 commutes
  %%   because $\eta$ is a monoidal natural transformation, and diagrams
  %%   3 and 4 commute by naturality of $\eta$.

  \item[] \textit{Case 4. associativity $\alpha$ interacts with co-monoidicity of $\Box$}
    $$
    \bfig
    \vSquares|ammmmmm|/->`->`->```->`/[
      \Box A \times (\Box B \times \Diamond C)`
      \Box A \times \Diamond(\Box B \times C)`
      (\Box A \times \Box B) \times \Diamond C`
      \Diamond(\Box A \times (\Box B \times C))``
      \Diamond((\Box A \times \Box B) \times C);
      \id_{\Box A} \pd \st{B}{C}`
      \alpha^{-1}`
      \st{A}{\Box B \times C}```
      \Diamond\alpha^{-1}`]
    \morphism(1554,0)|m|/->/<0,-500>[`\Diamond(\Box(A \times B) \times C);\Diamond(\m{A,B} \times \id_C)]
    
    \morphism(0,500)|m|/->/<0,-1000>[`\Box(A \times B) \times \Diamond C;\m{A,B} \times \id_{\Diamond C}]

    \morphism(350,-500)|a|/->/<800,0>[`;\st{A \times B}{C}]
    \efig
    $$
    \noindent
    Recall that:
    \[
    \begin{array}{rlll}
      \st{B}{C}              & = & (\eta_{\Box B} \pd \id_{\Diamond C});\p{\Box B,C}\\
      \st{A \pd B}{C}        & = & (\eta_{\Box (A \pd B)} \pd \id_{\Diamond C});\p{\Box (A \pd B),C}\\
      \st{A}{\Box B \pd C} & = & (\eta_{\Box A} \pd \id_{\Diamond (\Box B \pd C)});\p{\Box A,(\Box B \pd C)}\\
    \end{array}
    \]
    In addition, we require the following diagram (whose commutativity
    is implied by the fact that $\Diamond$ is a symmetric monoidal
    functor):
    $$
    \bfig
    \vSquares|ammmmma|/->`->`->``->`->`->/[
      \Diamond A \pd (\Diamond B \pd \Diamond C)`
      (\Diamond A \pd \Diamond B) \pd \Diamond C`
      \Diamond A \pd \Diamond (B \pd C)`
      \Diamond (A \pd B) \pd \Diamond C`
      \Diamond (A \pd (B \pd C))`
      \Diamond ((A \pd B) \pd C);
      \alpha^{-1}_{\Diamond A,\Diamond B,\Diamond C}`
      \id_{\Diamond A} \pd \p{B,C}`
      \p{A,B} \pd \id_{\Diamond C}``
      \p{A,B \pd C}`
      \p{A \pd B,C}`
      \Diamond\alpha^{-1}_{A,B,C}]
    \efig
    $$
    \noindent
    Finally, this case follows because the following diagram commutes:
    \begin{center}
      \rotatebox{90}{$
    \bfig
    \btriangle|mmm|<1769,1000>[
      \Box A \pd (\Diamond\Box B \pd \Diamond C)`
      \Diamond\Box A \pd (\Diamond\Box B \pd \Diamond C)`
      \Diamond\Box A \pd (\Diamond\Box B \pd C);
      \eta_{\Box A} \pd \id_{\Diamond \Box B \pd C}`
      \eta_{\Box A} \pd \p{\Box B,C}`
      \id_{\Diamond\Box A} \pd \p{\Box B,C}]

    \qtriangle|mam|/->``->/<1769,1000>[
      \Box A \pd (\Diamond\Box B \pd \Diamond C)`
      \Box A \pd \Diamond (\Box B \pd C)`
      \Diamond\Box A \pd (\Diamond\Box B \pd C);
      \id_{\Box A} \pd \p{\Box B,C}``
      \eta_{\Box A} \pd \id_{\Diamond (\Box B \pd C)}]    

    \qtriangle(-1800,0)|mmm|<1800,1000>[
      \Box A \pd (\Box B \pd \Diamond C)`
      \Box A \pd (\Diamond\Box B \pd \Diamond C)`
      \Diamond\Box A \pd (\Diamond\Box B \pd \Diamond C);
      \id_{\Box A} \pd (\eta_{\Box B} \pd \id_{\Diamond C})`
      \eta_{\Box A} \pd (\eta_{\Box B} \pd \id_{\Diamond C})`
      \eta_{\Box A} \pd \id_{\Diamond \Box B \pd C}]

    \square(0,-500)|mmmm|/`->`->`/<1769,500>[
      \Diamond\Box A \pd (\Diamond\Box B \pd \Diamond C)`
      \Diamond\Box A \pd (\Diamond\Box B \pd C)`
      (\Diamond\Box A \pd \Diamond\Box B) \pd \Diamond C`
      \Diamond (\Box A \pd (\Box B \pd C));`
      \alpha_{\Diamond\Box A,\Diamond\Box B,\Diamond C}`
      \p{\Box A,\Box B \pd C}`]   

    \square(0,-1000)|mmmm|/`->`->`/<1769,500>[
      (\Diamond\Box A \pd \Diamond\Box B) \pd \Diamond C`
      \Diamond (\Box A \pd (\Box B \pd C))`
      \Diamond (\Box A \pd \Box B) \pd \Diamond C`
      \Diamond ((\Box A \pd \Box B) \pd C);`
      \p{\Box A,\Box B} \pd \id_{\Diamond C}`
      \Diamond \alpha_{\Box A,\Box B,C}`]        

    \square(0,-1500)|mmmm|<1769,500>[
      \Diamond (\Box A \pd \Box B) \pd \Diamond C`
      \Diamond ((\Box A \pd \Box B) \pd C)`
      \Diamond\Box(A \pd B) \pd \Diamond C`
      \Diamond (\Box(A \pd B) \pd C);
      \p{\Box A \pd \Box B, C}`
      \Diamond \m{A,B} \pd \id_{\Diamond C}`
      \Diamond (\m{A,B} \pd \id_C)`
      \p{\Box (A \pd B),C}]

    \btriangle(-1800,-500)|mmm|/->``->/<1800,1500>[
      \Box A \pd (\Box B \pd \Diamond C)`
      (\Box A \pd \Box B) \pd \Diamond C`
      (\Diamond\Box A \pd \Diamond\Box B) \pd \Diamond C;
      \alpha_{\Box A,\Box B,\Diamond C}``
      (\eta_{\Box A} \pd \eta_{\Box B}) \pd \id_{\Diamond C}]

    \qtriangle(-1800,-1000)|mmm|/`->`/<1800,500>[
      (\Box A \pd \Box B) \pd \Diamond C`
      (\Diamond\Box A \pd \Diamond\Box B) \pd \Diamond C`
      \Diamond (\Box A \pd \Box B) \pd \Diamond C;`
      \eta_{\Box A \pd \Box B} \pd \id_{\Diamond C}`]

    \btriangle(-1800,-1500)|mmm|/->``->/<1800,1000>[
      (\Box A \pd \Box B) \pd \Diamond C`
      \Box(A \pd B) \pd \Diamond C`
      \Diamond\Box(A \pd B) \pd \Diamond C;
      \m{A,B} \pd \id_{\Diamond C}``
      \eta_{\Box (A \pd B)} \pd \id_{\Diamond C}]

    \place(1200,700)[1]
    \place(500,300)[2]
    \place(900,-500)[3]
    \place(900,-1250)[4]
    \place(-500,700)[5]
    \place(-1000,0)[6]
    \place(-400,-700)[7]
    \place(-1000,-1100)[8]
    \efig
    $}
    \end{center}
    Diagrams 1, 2 and 5 commute by functorality of $\times$, diagram 3
    commutes by the additional diagram from above, diagram 4 commutes
    by naturality of $\mathsf{p}$, diagram 6 commutes by naturality of
    $\alpha$, diagram 7 commutes by the fact that $\eta$ is a monoidal
    natural transformation, and diagram 8 commutes by naturality of
    $\eta$.
    

  \item[] \textit{Case 5.  strength interacts with monoidicity of $\Diamond$}
    $$
    \bfig
    \vSquares|ammmmma|/->`->```->``->/[
      \Box A \times \Diamond\Diamond B`
      \Box A \times \Diamond B`
      \Diamond(\Box A \times \Diamond B)``
      \Diamond\Diamond(\Box A \times B)`
      \Diamond(\Box A \times B);
      \id_{\Box A} \pd \mu_{B}`
      \st{A}{\Diamond B}```
      \Diamond(\st{A}{B})``
      \mu_{\Box A \pd B}]
    \morphism(1150,1000)|m|<0,-920>[`;\st{A}{B}]
    \efig
    $$
    \noindent
    Recall that:
    \[
    \begin{array}{rlll}
      \st{A}{B} & = & (\eta_{\Box A} \pd \id_{\Diamond B});\p{\Box A,B}\\
      \st{A}{\Diamond B} & = & (\eta_{\Box A} \pd \id_{\Diamond \Diamond B});\p{\Box A,\Diamond B}\\
    \end{array}
    \]
    This case follows from the fact that the following diagram
    commutes:
    \begin{center}
      \rotatebox{90}{$\bfig
    \qtriangle|mmm|<1500,1000>[
      \Box A \pd \Diamond\Diamond B`
      \Box A \pd \Diamond B`
      \Diamond\Box A \pd \Diamond B;
      \id_{\Box A} \pd \mu_B`
      \eta_{\Box A} \pd \mu_B`
      \eta_{\Box A} \pd \id_{\Diamond B}]

    \morphism(-1500,1000)|m|/<-/<1500,0>[
      \Diamond\Box A \pd \Diamond\Diamond B`
      \Box A \pd \Diamond\Diamond B;
      \eta_{\Box A} \pd \id_{\Diamond\Diamond B}]

    \btriangle(-1500,0)|mmm|<3000,1000>[
      \Diamond\Box A \pd \Diamond\Diamond B`
      \Diamond\Diamond\Box A \pd \Diamond\Diamond B`
      \Diamond\Box A \pd \Diamond B;
      \Diamond\eta_{\Box A} \pd \id_{\Diamond\Diamond B}`
      \id_{\Diamond\Box A} \pd \mu_B`
      \mu_{\Box A} \pd \mu_B]

    \square(-3000,0)|mmmm|/<-`->``<-/<1500,1000>[
      \Diamond(\Box A \pd \Diamond B)`
      \Diamond\Box A \pd \Diamond\Diamond B`
      \Diamond(\Diamond\Box A \pd \Diamond B)`
      \Diamond\Diamond\Box A \pd \Diamond\Diamond B;
      \p{\Box A,\Diamond B}`
      \Diamond (\eta_{\Box A} \pd \id_{\Diamond B})``
      \p{\Diamond\Box A,\Diamond B}]

    \square(-3000,-500)|mmmm|/`->`->`->/<4500,500>[
      \Diamond (\Diamond\Box A \pd \Diamond B)`
      \Diamond\Box A \pd \Diamond B`
      \Diamond\Diamond (\Box A \pd B)`
      \Diamond (\Box A \pd B);`
      \Diamond (\p{\Box A,B})`
      \p{\Box A,B}`
      \mu_{\Box A \pd B}]

    \place(-2300,500)[1]
    \place(-800,-250)[2]
    \place(-800,400)[3]
    \place(0,700)[4]
    \place(1050,700)[5]
    \efig$}
    \end{center}       
    Diagram commutes by naturality of $\mathsf{p}$, diagram 2 commutes
    because $\mu$ is a monoidal natural transformation, diagram 3
    commutes because $\mu$ is the monadic multiplication and by
    functorality of $\times$, and diagrams 4 and 5 commute by
    functoriality of $\times$.
    
  \item[] \textit{Case 6. commuting $\beta$ interacts with $\Diamond$}
    $$
    \bfig
    \hSquares|aamamaa|/``->``->`->`->/[
      \Diamond B \times \Box A``
      \Box A \times \Diamond B`
      \Diamond B \times \Diamond \Box A`
      \Diamond (B \times \Box A)`
      \Diamond (\Box A \times B);``
      \id_{\Diamond B} \pd \eta_{\Box A}``
      st_{A,B}`
      \p{B,\Box A}`
      \Diamond\beta_{B,\Box A}]
    \morphism(200,500)<1817,0>[`;\beta_{\Diamond B,\Box A}]
    \efig
    $$
    \noindent
    The previous diagram commutes by commutativity of the following
    diagram:
    $$
    \bfig
    \vSquares|ammmmma|[
      \Diamond B \times \Box A`
      \Box A \times \Diamond B`
      \Diamond B \times \Diamond\Box A`
      \Diamond\Box A \times \Diamond B`
      \Diamond (B \times \Box A)`
      \Diamond (\Box A \times B);
      \beta_{\Diamond B,\Box A}`
      \id_{\Diamond B} \times \eta_{\Box A}`
      \eta_{\Box A} \times \id_{\Diamond B}`
      \beta_{\Diamond B,\Diamond\Box A}`
      \p{B,\Box A}`
      \p{\Box A,B}`
      \Diamond\beta_{B,\Box A}]

    \place(600,750)[1]
    \place(600,250)[2]
    \efig
    $$
    \noindent
    Diagram 1 commutes because $\beta$ is a symmetric monoidal
    functor, and diagram 2 commutes by naturality of $\beta$.

  \item[] \textit{Case 7.  $\varepsilon$ interacts with $\Diamond$ and its monoidicity} 
    $$
    \bfig
    \hSquares|aamamaa|/``->``->`->`->/[
      \Box A \times \Diamond B``
      \Diamond(\Box A \times B)`
      A \times \Diamond B`
      \Diamond A \times \Diamond B`
      \Diamond (A \times B);``
      \varepsilon_A \times \id_{\Diamond B}``
      \Diamond(\varepsilon_A \times \id_{B})`
      \eta_A \times \id_{\Diamond B}`
      \p{A,B}]
    \morphism(200,500)<1605,0>[`;\st{A}{B}]
    \efig
    $$
    \noindent
    The previous diagram commutes by commutativity of the following
    diagram:
    $$
    \bfig
    \hSquares|aammmaa|[
      \Box A \times \Diamond B`
      \Diamond\Box A \times \Diamond B`
      \Diamond (\Box A \times B)`
      A \times \Diamond B`
      \Diamond A \times \Diamond B`
      \Diamond (A \times B);
      \eta_{\Box A} \times \id_{\Diamond B}`
      \p{\Box A,B}`
      \varepsilon_A \times \id_{\Diamond B}`
      \Diamond\varepsilon_A \times \id_{\Diamond B}`
      \Diamond (\varepsilon_A \times \id_B)`
      \eta_A \times \id_{\Diamond B}`
      \p{A,B}]
    \place(1750,250)[1]
    \place(600,250)[2]
    \efig
    $$
    \noindent
    Diagram 1 commutes by naturality of $\mathsf{p}$, and diagram 2
    commutes by naturality of $\eta$.
  \end{itemize}

\end{proof}\begin{proof}
  We must show that given the definition of an adjoint CS4 categorical
  model (Definition~\ref{def:CS4-single-adjoint-cat-model}) we can
  define an appropriate monad and comonad on a CCC with coproducts
  where the monad is strong with respect to the comonad.

  Suppose $(H,m)$ and $(J,n)$ are the adjoint monoidal functors given
  in Definition~\ref{def:CS4-single-adjoint-cat-model}, and define
  $\Box = JH$ and $\Diamond = HJ$.  By definition we assumed that
  $(\Box, q)$, where $q_{A,B} : \Box A \times \Box B \to \Box (A
  \times B)$, is monoidal, but we must show that $\Diamond$ is also
  monoidal.  We know that both $(H,n)$ and $(J,m)$ are monoidal
  endofunctors on $\cat{C}$ which implies that their composition
  $\Diamond$ is monoidal where
  \[
  \begin{array}{lll}
    \mathsf{p}_{1} = \eta_{1} : 1 \to \Diamond 1\\
    \mathsf{p}_{A,B} = \m{HA,HB};J(\mathsf{n}_{A,B})
    \colon \Diamond A \pd \Diamond B \mto \Diamond(A \pd B)
  \end{array}
  \]
  and the following diagrams commute (proofs omitted):
  \begin{mathpar}
    \scriptsize
    \bfig
    \vSquares|ammmmma|/->`->`->``->`->`->/[
      (\Diamond A \times \Diamond B) \times \Diamond C`
      \Diamond A \times (\Diamond B \times \Diamond C)`
      \Diamond(A \times B) \times \Diamond C`
      \Diamond A \times \Diamond(B \times C)`
      \Diamond ((A \times B) \times C)`
      \Diamond (A \times (B \times C));
      \alpha`
      \mathsf{p}_{A,B} \times \id_{\Diamond C}`
      \id_{\Diamond A} \times \mathsf{p}_{B,C}``
      \mathsf{p}_{A \times B,C}`
      \mathsf{p}_{A,B \times C}`
      \Diamond \alpha]
    \efig
    \and
    \bfig
    \hSquares|ammmmaa|/->``->`<-``->`/[
      1 \times \Diamond A`
      \Diamond A``
      \Diamond 1 \times \Diamond A`
      \Diamond(1 \times A)`;
      \lambda_{\Diamond A}``
      \mathsf{p}_{1} \times \id_{\Diamond A}`
      \Diamond \lambda_A``
      \mathsf{p}_{1,A}`]
    \efig
    \and
    \bfig
    \hSquares|ammmmaa|/->``->`<-``->`/[
      \Diamond A \times 1`
      \Diamond A``
      \Diamond A \times \Diamond 1`
      \Diamond(A \times 1)`;
      \rho_{\Diamond A}``
      \id_{\Diamond A} \times \mathsf{p}_{1}`
      \Diamond \rho_A``
      \mathsf{p}_{A,1}`]
    \efig
    \and
    \bfig
    \hSquares|ammmmaa|/->``->`->``->`/[
      \Diamond A \times \Diamond B`
      \Diamond B \times \Diamond A``
      \Diamond (A \times B)`
      \Diamond (B \times A)`;
      \beta_{\Diamond A,\Diamond B}``
      \mathsf{p}_{A,B}`
      \mathsf{p}_{B,A}``
      \Diamond\beta_{A,B}`]
    \efig
  \end{mathpar}

  Furthermore, suppose $J \dashv H$, where the unit, $\varepsilon :
  \Box A \to A$, and the counit, $\eta : A \to \Diamond A$, are
  monoidal natural transformations.  This implies that the following
  diagrams commute:
  \begin{mathpar}
    \bfig
    \btriangle<800,500>[A \times B`\Diamond A \times \Diamond B`\Diamond (A \times B);\eta_A \times \eta_B`\eta_{A \times B}`\mathsf{p}_{A,B}]
    \efig
    \and
    \bfig
    \qtriangle<800,500>[\Box A \times \Box B`\Box (A \times B)`A \times B;\mathsf{q}_{A,B}`\varepsilon_A \times \varepsilon_B`\varepsilon_{A \times B}]
    \efig
    \and
    \bfig
    \qtriangle<800,500>[1`J 1`\Diamond 1;n_{1}`\eta_1`J m_1]       
    \efig
    \and
    \bfig
    \hSquares|ammmmaa|/->``=`->``<-`/[
      \Box 1`
      1``
      \Box 1`
      H 1`;
      \varepsilon_1```
      m_1``
      H n_1`]
    \efig
    \and
    \bfig
    \qtriangle<800,500>[H A`H\Box A`H A;\eta_{H A}`\id_{H A}`H\varepsilon_A]       
    \efig
    \and
    \bfig
    \qtriangle<800,500>[J A`\Box J A`J A;J\eta_A`\id_{J A}`\varepsilon_{J A}]
    \efig    
  \end{mathpar}
  It is a well-known fact about adjoints that $(\Box, \varepsilon,
  \delta)$, where $\delta : \Box A \to \Box\Box A$ is a comonad, and
  $(\Diamond, \eta, \mu)$, where $\mu : \Diamond\Diamond A \to
  \Diamond A$ is a monad.  In addition, $\mu$ and $\delta$ are monoidal
  natural transformations where we have the following:
  \[
  \begin{array}{lll}
    \d{1} = \p{1};\Diamond\p{1} : 1 \mto \Diamond^2 1\\
    \d{A,B} =  \p{\Diamond A,\Diamond B};\Diamond\p{A,B} : \Diamond^2 A \times \Diamond^2 B \mto \Diamond^2 (A \times B)\\
    \\
    \b{1} = \q{1};\Box\q{1} : 1 \mto \Box^2 1\\
    \b{A,B} = \q{\Box A,\Box B};\Box\q{A,B} : \Box^2 A \times \Box^2 B \mto \Box^2 (A \times B)\\
  \end{array}
  \]
  Thus, the following diagrams commute:
  \begin{mathpar}
    \bfig
    \hSquares|ammmmaa|/->``->`->``->`/[
      \Diamond^3 A`
      \Diamond^2 A``
      \Diamond^2 A`
      \Diamond A`;
      \Diamond \mu_A``
      \mu_{\Diamond A}`
      \mu_A``
      \mu_A`]
    \efig
    \and
    \bfig
    \qtriangle/->`=`->/<800,500>[\Diamond A`\Diamond^2 A`\Diamond A;\eta_{\Diamond A}``\mu_A]
    \btriangle(0,0)/->`=`->/<800,500>[\Diamond A`\Diamond^2 A`\Diamond A;\Diamond \eta_{A}``\mu_A]
    \efig
    \and
    \bfig
    \hSquares|ammmmaa|/->``->`->``->`/[
      \Box A`
      \Box^2 A``
      \Box^2 A`
      \Box^3 A`;
      \delta_A``
      \delta_A`
      \delta_{\Box A}``
      \Box\delta_A`]
    \efig
    \and
    \bfig
    \qtriangle/->`=`->/<800,500>[\Box A`\Box^2 A`\Box A;\delta_A``\Box \varepsilon]
    \btriangle(0,0)/->`=`->/<800,500>[\Box A`\Box^2 A`\Box A;\delta_A``\varepsilon_{\Box A}]
    \efig
    \and
    %% \bfig   
    %% \vSquares|ammmmma|/->`->```->``->/[
    %%   \Diamond^2 A \times \Diamond^2 B`
    %%   \Diamond A \times \Diamond B`
    %%   \Diamond(\Diamond A \times \Diamond B)``
    %%   \Diamond^2(A \times B)`
    %%   \Diamond(A \times B);
    %%   \mu_A \times \mu_B`
    %%   \p{\Diamond A,\Diamond B}```
    %%   \Diamond\p{A,B}``
    %%   \mu_{A \times B}]
    %% \morphism(1108,0)/<-/<0,1000>[\Diamond(A \times B)`\Diamond A \times \Diamond B;\p{A,B}]
    %% \efig
    \and
    \bfig
    \square<1000,1000>[
      \Diamond^2 A \times \Diamond^2 B`
      \Diamond A \times \Diamond B`
      \Diamond^2(A \times B)`
      \Diamond(A \times B);
      \mu_A \times \mu_B`
      \d{A,B}`
      \p{A,B}`
      \mu_{A \times B}]
    \efig
    \and
    \bfig
    \Vtriangle/->`<-`<-/[
      \Diamond^2 1`
      \Diamond 1`
      1;
      \mu_1`
      \d{1}`
      \p{1}]
    \efig
    \and        
    \bfig
    \square<1000,1000>[
      \Box A \times \Box B`
      \Box^2 A \times \Box^2 B`
      \Box(A \times B)`
      \Box^2(A \times B);
      \delta_A \times \delta_B`
      \q{A,B}`
      \b{A,B}`
      \delta_{A \times B}]
    %% \vSquares|ammmmma|/->``->```->`->/[
    %%   \Box A \times \Box B`
    %%   \Box^2 A \times \Box^2 B``
    %%   \Box(\Box A \times \Box B)`
    %%   \Box(A \times B)`
    %%   \Box^2(A \times B);
    %%   \varepsilon_A \times \varepsilon_B``
    %%   \q{\Box A,\Box B}```
    %%   \Box\q{A,B}`
    %%   \varepsilon_{A \times B}]
    %% \morphism(0,0)/<-/<0,1000>[\Box(A \times B)`\Box A \times \Box B;\q{A,B}]
    \efig
    \and
    \bfig
    \Vtriangle/->`<-`<-/[
      \Box 1`
      \Box^2 1`
      1;
      \delta_1`
      \q{1}`
      \d{1}]
    \efig
  \end{mathpar}

  We can now define the $\Box$-strength map as follows:
  \[
  \st{A}{B} = (\eta_{\Box A} \pd \id_{\Diamond B});\mathsf{p}_{\Box A,B} : \Box A \pd \Diamond B \mto \Diamond(\Box A \pd B)
  \]
  We can see that $\st{A}{B}$ is a natural transformation, because it
  is defined as a composition of natural transformations.
  
  %% To prove that the appropriate diagrams commute we first note that
  %% the triangle  
  Next we must show that all of the appropriate diagrams given in
  Definition~\ref{def:comonad-strong-monad} commute.
  \begin{itemize}
  \item[] \textit{Case 1. the first projection interacts with $\Diamond$:}
    \[
    \bfig
    \square|amma|<850,600>[
      \Box A \times \Diamond B`
      \Diamond (\Box A \times B)`
      \Box A`
      \Diamond\Box A;
      \st{A}{B}`
      \pi_1`
      \Diamond\pi_1`
      \eta_A]
    \efig
    \]
    This diagrams commutes, because the following diagram commutes:
    \[
    \bfig
    \square|ammm|/->`->``/<1500,2000>[
      \Box A \times \Diamond B`
      \Diamond\Box A \times \Diamond B`
      \Box A`;
      \eta_{\Box A} \times \id_{\Diamond B}`
      \pi_1``]

    \square(1500,0)|ammm|/->``->`/<1500,2000>[
      \Diamond\Box A \times \Diamond B`
      \Diamond (\Box A \times B)``
      \Diamond\Box A;
      \p{\Box A,B}``
      \Diamond\pi_1`]

    \morphism<3000,0>[\Box A`\Diamond\Box A;\eta_{\Box A}]

    \morphism(484,1500)<1000,0>[
      \Box A \times 1`
      \Diamond\Box A \times \Diamond 1;
      \eta_{\Box A} \times \eta_1]

    \morphism(1484,1500)<700,-500>[
      \Diamond\Box A \times \Diamond 1`
      \Diamond(\Box A \times 1);
      \p{\Box A,1}]

    \morphism(484,1500)|m|/{@{>}@/_1em/}/<1700,-500>[
      \Box A \times 1`
      \Diamond(\Box A \times 1);
      \eta_{\Box A \times 1}]

    \morphism(484,1500)|m|<700,-800>[
      \Box A \times 1`
      \Box A;
      \pi_1]

    \morphism(0,2000)|m|<484,-500>[
      \Box A \times \Diamond B`
      \Box A \times 1;
      \id_{\Box A} \times \t_{\Diamond B}]

    \morphism(1500,2000)|m|<-17,-500>[
      \Diamond\Box A \times \Diamond B`
      \Diamond\Box A \times \Diamond 1;
      \id_{\Diamond\Box A} \times \Diamond\t_{B}]

    \morphism(3000,2000)|m|<-815,-1000>[
      \Diamond(\Box A \times B)`
      \Diamond(\Box A \times 1);
      \Diamond (id_{\Box A} \times \t_{B})]

    \morphism(2184,1000)|m|<816,-1000>[
      \Diamond(\Box A \times 1)`
      \Diamond\Box A;
      \Diamond\pi_1]

    \morphism(1184,700)|m|<1816,-700>[
      \Box A`
      \Diamond\Box A;
      \eta_{\Box A}]
    
    \place(2700,1000)[1]
    \place(2150,1700)[2]
    \place(800,1750)[3]
    \place(1300,1300)[4]
    \place(1800,800)[5]
    \place(500,500)[6]
    \efig    
    \]
    Diagrams 1 and 6 commute because we are in a cartesian closed
    category, diagram 2 commutes by naturality of $\p{}$, diagram 3
    commutes because $\Diamond$ is a product functor, diagram 4
    commutes because $\eta$ is the unit of a symmetric monoidal
    adjunction, and diagram 5 commutes by naturality of $\eta$.
    
  \item[] \textit{Case 2. the second projection interacts with $\Diamond$:}
    \[
    \bfig
    \qtriangle<850,600>[
      \Box A \times \Diamond B`
      \Diamond (\Box A \times B)`
      \Diamond B;
      \st{A}{B}`
      \pi_2`
      \Diamond\pi_2]
    \efig
    \]
    This case is similar to the previous case.
    
    %% \item[] \textit{Case 1. the object 1 behaves as the unit for products}
  %%   $$
  %%   \bfig
  %%   \vSquares|ammmmma|/>``>```>`>/[\Box 1 \times \Diamond A`\Diamond (\Box 1 \times A)``\Diamond(1 \times A)`1 \times \Diamond A`\Diamond A;\st{1}{A}``\Diamond(\varepsilon_1 \times \id_A)```\Diamond\lambda`\lambda]
  %%   \morphism(0,1000)|m|/->/<0,-950>[`;\varepsilon_1 \times \id_{\Diamond A}]
  %%   \efig
  %%   $$
  %%   This diagram commutes by commutativity of the following diagram:
  %%   %% Equational version:
  %%   %% \begin{center}
  %%   %%   \begin{math}
  %%   %%     \begin{array}{rllllllll}
  %%   %%       & & \st{1}{A};\Diamond (\varepsilon_1 \times \id_A);\Diamond\lambda_A\\
  %%   %%       \text{(Definition of $\mathsf{st}$)}
  %%   %%       & = & (\eta_{\Box 1} \times \id_{\Diamond A});\p{\Box 1,\Diamond A};\Diamond (\varepsilon_1 \times \id_A);\Diamond\lambda_A\\
  %%   %%       \text{(Naturality of $\mathsf{p}$)}
  %%   %%       & = & (\eta_{\Box 1} \times \id_{\Diamond A});(\Diamond \varepsilon_1 \times \Diamond\id_A);\p{1,A};\Diamond\lambda_A\\
  %%   %%       \text{(Functoriality of $\times$)}
  %%   %%       & = & ((\eta_{\Box 1};\Diamond \varepsilon_1) \times (\id_{\Diamond A};\Diamond\id_A));\p{1,A};\Diamond\lambda_A\\
  %%   %%       & = & ((\eta_{\Box 1};\Diamond \varepsilon_1) \times (\id_{\Diamond A};\id_{\Diamond A}));\p{1,A};\Diamond\lambda_A\\
  %%   %%       \text{(Naturality of $\eta$)}
  %%   %%       & = & ((\varepsilon_1;\eta_{1}) \times (\id_{\Diamond A};\id_{\Diamond A}));\p{1,A};\Diamond\lambda_A\\
  %%   %%       \text{(Definition of $\mathsf{p}$)}
  %%   %%       & = & ((\varepsilon_1;\p{1}) \times (\id_{\Diamond A};\id_{\Diamond A}));\p{1,A};\Diamond\lambda_A\\
  %%   %%       \text{(Functoriality of $\times$)}
  %%   %%       & = & (\varepsilon_1 \times \id_{\Diamond A});(\p{1} \times \id_{\Diamond A});\p{1,A};\Diamond\lambda_A\\
  %%   %%       \text{($\Diamond$ is Symmetric Monoidal)}
  %%   %%       & = & (\varepsilon_1 \times \id_{\Diamond A});\lambda_{\Diamond A}\\
  %%   %%     \end{array}
  %%   %%   \end{math}
  %%   %% \end{center}
  %%   $$
  %%   \bfig
  %%   \square|amma|<1000,500>[
  %%     \Diamond\Box 1 \times \Diamond A`
  %%     \Diamond (\Box 1 \times A)`
  %%     \Diamond 1 \times \Diamond A`
  %%     \Diamond (1 \times A);
  %%     \p{\Box 1,A}`
  %%     \Diamond \varepsilon_1 \times \id_{\Diamond A}`
  %%     \Diamond (\varepsilon_1 \times \id_A)`
  %%     \p{1,A}]

  %%   \square(-1000,0)|amma|<1000,500>[
  %%     \Box 1 \times \Diamond A`
  %%     \Diamond\Box 1 \times \Diamond A`
  %%     1 \times \Diamond A`
  %%     \Diamond 1 \times \Diamond A;
  %%     \eta_{\Box 1} \times \id_{\Diamond A}`
  %%     \varepsilon_1 \times \id_{\Diamond A}`
  %%     \Diamond \varepsilon_1 \times \id_{\Diamond A}`
  %%     (\p{1} = \eta_1) \times \id_{\Diamond A}]

  %%   \qtriangle(-1000,-500)|mmm|/`->`->/<2000,500>[
  %%     1 \times \Diamond A`
  %%     \Diamond (1 \times A)`
  %%     \Diamond A;`
  %%     \lambda_{\Diamond A}`
  %%     \Diamond\lambda_A]

  %%   \place(500,250)[1]
  %%   \place(-500,250)[2]
  %%   \place(500,-200)[3]
  %%   \efig
  %%   $$
  %%   \noindent
  %%   Diagram 1 commutes by naturality of $\mathsf{p}$, diagram 2
  %%   commutes by naturality of $\eta$, and diagram 3 commutes because
  %%   $\Diamond$ is a symmetric monoidal functor.

  \item[] \textit{Case 3. unit $\eta$ of the monad and strength interact well, $\Box  A $ is a parameter}
    $$
    \bfig
    \btriangle<800,500>[
      \Box A \pd B`
      \Box A \pd \Diamond B`
      \Diamond(\Box A \pd B);
      \id_{\Box A} \pd \eta_{B}`
      \eta_{\Box A \times B}`
      \st{A}{B}]
    \efig
    $$

    The previous diagram commutes, because the following diagram commutes:
    $$
    \bfig
    \btriangle|ama|/->`->`->/<1500,500>[
      \Box A \pd B`
      \Box A \pd \Diamond B`
      \Diamond\Box A \pd \Diamond B;
      \id_{\Box A} \pd \eta_{B}`
      \eta_{\Box A} \pd \eta_B`
      \eta_{\Box A} \pd \id_{\Diamond B}]

    \qtriangle(0,0)/->``<-/<1500,500>[
      \Box A \pd B`
      \Diamond (\Box A \times B)`
      \Diamond\Box A \pd \Diamond B;
      \eta_{\Box A \times B}``
      \p{\Box A,B}]

    \place(250,200)[1]
    \place(1200,300)[2]
    \efig
    $$
    \noindent
    Diagram 1 clearly commutes, and diagram 2 commutes because $\eta$
    is a symmetric monoidal natural transformation.
    
  %% \item[] \textit{Case 3. co-unit of the comonad $\varepsilon$ and unit of the monad $\eta$ interact well?}
  %%   $$
  %%   \bfig
  %%   \hSquares|aamaaaa|/->`->`->```->`/[
  %%     \Box A \times 1`
  %%     \Box A \times \Diamond 1`
  %%     \Diamond (\Box A \times 1)`
  %%     \Box A`
  %%     A`
  %%     ;
  %%     \id_{\Box A} \times \eta_1`
  %%     \st{A}{1}`
  %%     \rho_{\Box A}```
  %%     \varepsilon_A`]
  %%   \qtriangle(1978,0)/->``->/<800,500>[\Diamond (\Box A \times 1)`\Diamond\Box A`\Diamond A;\Diamond \rho_{\Box A}``\Diamond\varepsilon_A]
  %%   \morphism(1060,0)/->/<1640,0>[`;\eta_A]
  %%   \efig
  %%   $$
  %%   \noindent
  %%   Recall that
  %%   $\st{A}{1} = (\eta_{\Box A} \pd \id_{\Diamond 1});\p{\Box A,1}$.
  %%   Now the previous diagram commutes, because the following diagram commutes:
  %%   $$
  %%   \bfig
  %%   \btriangle|mma|<1444,1000>[\Box A \pd 1`\Diamond\Box A \pd \Diamond 1`\Diamond(\Box A \pd 1);\eta_{\Box A} \pd \eta_1`\eta_{\Box A \pd 1}`\p{\Box A,1}]
  %%   \dtriangle(-1000,0)/->``->/<1000,1000>[\Box A \pd 1`\Box A \pd 1`\Diamond\Box A \pd \Diamond 1;\id_{\Box A} \pd \eta_1``\eta_{\Box A} \pd
  %%     \id_{\Diamond 1}]

  %%   \hSquares(0,0)/->`->```<-``/<1000>[\Box A \pd 1`\Box A`\Diamond\Box A```\Diamond (\Box A \times 1);\rho_{\Box A}`\eta_{\Box A}```\Diamond (\rho_{\Box A})``]

  %%   \square(746,1000)/->`<-`<-`/<698,500>[A`\Diamond A`\Box A`\Diamond\Box A;\eta_A`\varepsilon_A`\Diamond\varepsilon_A`]

  %%   \place(-400,300)[1]
  %%   \place(400,300)[2]
  %%   \place(1100,700)[3]
  %%   \place(1100,1250)[4]
  %%   \efig
  %%   $$
  %%   \noindent
  %%   Diagram 1 commutes by functorality of $\times$, diagram 2 commutes
  %%   because $\eta$ is a monoidal natural transformation, and diagrams
  %%   3 and 4 commute by naturality of $\eta$.

  \item[] \textit{Case 4. associativity $\alpha$ interacts with co-monoidicity of $\Box$}
    $$
    \bfig
    \vSquares|ammmmmm|/->`->`->```->`/[
      \Box A \times (\Box B \times \Diamond C)`
      \Box A \times \Diamond(\Box B \times C)`
      (\Box A \times \Box B) \times \Diamond C`
      \Diamond(\Box A \times (\Box B \times C))``
      \Diamond((\Box A \times \Box B) \times C);
      \id_{\Box A} \pd \st{B}{C}`
      \alpha^{-1}`
      \st{A}{\Box B \times C}```
      \Diamond\alpha^{-1}`]
    \morphism(1554,0)|m|/->/<0,-500>[`\Diamond(\Box(A \times B) \times C);\Diamond(\m{A,B} \times \id_C)]
    
    \morphism(0,500)|m|/->/<0,-1000>[`\Box(A \times B) \times \Diamond C;\m{A,B} \times \id_{\Diamond C}]

    \morphism(350,-500)|a|/->/<800,0>[`;\st{A \times B}{C}]
    \efig
    $$
    \noindent
    Recall that:
    \[
    \begin{array}{rlll}
      \st{B}{C}              & = & (\eta_{\Box B} \pd \id_{\Diamond C});\p{\Box B,C}\\
      \st{A \pd B}{C}        & = & (\eta_{\Box (A \pd B)} \pd \id_{\Diamond C});\p{\Box (A \pd B),C}\\
      \st{A}{\Box B \pd C} & = & (\eta_{\Box A} \pd \id_{\Diamond (\Box B \pd C)});\p{\Box A,(\Box B \pd C)}\\
    \end{array}
    \]
    In addition, we require the following diagram (whose commutativity
    is implied by the fact that $\Diamond$ is a symmetric monoidal
    functor):
    $$
    \bfig
    \vSquares|ammmmma|/->`->`->``->`->`->/[
      \Diamond A \pd (\Diamond B \pd \Diamond C)`
      (\Diamond A \pd \Diamond B) \pd \Diamond C`
      \Diamond A \pd \Diamond (B \pd C)`
      \Diamond (A \pd B) \pd \Diamond C`
      \Diamond (A \pd (B \pd C))`
      \Diamond ((A \pd B) \pd C);
      \alpha^{-1}_{\Diamond A,\Diamond B,\Diamond C}`
      \id_{\Diamond A} \pd \p{B,C}`
      \p{A,B} \pd \id_{\Diamond C}``
      \p{A,B \pd C}`
      \p{A \pd B,C}`
      \Diamond\alpha^{-1}_{A,B,C}]
    \efig
    $$
    \noindent
    Finally, this case follows because the following diagram commutes:
    \begin{center}
      \rotatebox{90}{$
    \bfig
    \btriangle|mmm|<1769,1000>[
      \Box A \pd (\Diamond\Box B \pd \Diamond C)`
      \Diamond\Box A \pd (\Diamond\Box B \pd \Diamond C)`
      \Diamond\Box A \pd (\Diamond\Box B \pd C);
      \eta_{\Box A} \pd \id_{\Diamond \Box B \pd C}`
      \eta_{\Box A} \pd \p{\Box B,C}`
      \id_{\Diamond\Box A} \pd \p{\Box B,C}]

    \qtriangle|mam|/->``->/<1769,1000>[
      \Box A \pd (\Diamond\Box B \pd \Diamond C)`
      \Box A \pd \Diamond (\Box B \pd C)`
      \Diamond\Box A \pd (\Diamond\Box B \pd C);
      \id_{\Box A} \pd \p{\Box B,C}``
      \eta_{\Box A} \pd \id_{\Diamond (\Box B \pd C)}]    

    \qtriangle(-1800,0)|mmm|<1800,1000>[
      \Box A \pd (\Box B \pd \Diamond C)`
      \Box A \pd (\Diamond\Box B \pd \Diamond C)`
      \Diamond\Box A \pd (\Diamond\Box B \pd \Diamond C);
      \id_{\Box A} \pd (\eta_{\Box B} \pd \id_{\Diamond C})`
      \eta_{\Box A} \pd (\eta_{\Box B} \pd \id_{\Diamond C})`
      \eta_{\Box A} \pd \id_{\Diamond \Box B \pd C}]

    \square(0,-500)|mmmm|/`->`->`/<1769,500>[
      \Diamond\Box A \pd (\Diamond\Box B \pd \Diamond C)`
      \Diamond\Box A \pd (\Diamond\Box B \pd C)`
      (\Diamond\Box A \pd \Diamond\Box B) \pd \Diamond C`
      \Diamond (\Box A \pd (\Box B \pd C));`
      \alpha_{\Diamond\Box A,\Diamond\Box B,\Diamond C}`
      \p{\Box A,\Box B \pd C}`]   

    \square(0,-1000)|mmmm|/`->`->`/<1769,500>[
      (\Diamond\Box A \pd \Diamond\Box B) \pd \Diamond C`
      \Diamond (\Box A \pd (\Box B \pd C))`
      \Diamond (\Box A \pd \Box B) \pd \Diamond C`
      \Diamond ((\Box A \pd \Box B) \pd C);`
      \p{\Box A,\Box B} \pd \id_{\Diamond C}`
      \Diamond \alpha_{\Box A,\Box B,C}`]        

    \square(0,-1500)|mmmm|<1769,500>[
      \Diamond (\Box A \pd \Box B) \pd \Diamond C`
      \Diamond ((\Box A \pd \Box B) \pd C)`
      \Diamond\Box(A \pd B) \pd \Diamond C`
      \Diamond (\Box(A \pd B) \pd C);
      \p{\Box A \pd \Box B, C}`
      \Diamond \m{A,B} \pd \id_{\Diamond C}`
      \Diamond (\m{A,B} \pd \id_C)`
      \p{\Box (A \pd B),C}]

    \btriangle(-1800,-500)|mmm|/->``->/<1800,1500>[
      \Box A \pd (\Box B \pd \Diamond C)`
      (\Box A \pd \Box B) \pd \Diamond C`
      (\Diamond\Box A \pd \Diamond\Box B) \pd \Diamond C;
      \alpha_{\Box A,\Box B,\Diamond C}``
      (\eta_{\Box A} \pd \eta_{\Box B}) \pd \id_{\Diamond C}]

    \qtriangle(-1800,-1000)|mmm|/`->`/<1800,500>[
      (\Box A \pd \Box B) \pd \Diamond C`
      (\Diamond\Box A \pd \Diamond\Box B) \pd \Diamond C`
      \Diamond (\Box A \pd \Box B) \pd \Diamond C;`
      \eta_{\Box A \pd \Box B} \pd \id_{\Diamond C}`]

    \btriangle(-1800,-1500)|mmm|/->``->/<1800,1000>[
      (\Box A \pd \Box B) \pd \Diamond C`
      \Box(A \pd B) \pd \Diamond C`
      \Diamond\Box(A \pd B) \pd \Diamond C;
      \m{A,B} \pd \id_{\Diamond C}``
      \eta_{\Box (A \pd B)} \pd \id_{\Diamond C}]

    \place(1200,700)[1]
    \place(500,300)[2]
    \place(900,-500)[3]
    \place(900,-1250)[4]
    \place(-500,700)[5]
    \place(-1000,0)[6]
    \place(-400,-700)[7]
    \place(-1000,-1100)[8]
    \efig
    $}
    \end{center}
    Diagrams 1, 2 and 5 commute by functorality of $\times$, diagram 3
    commutes by the additional diagram from above, diagram 4 commutes
    by naturality of $\mathsf{p}$, diagram 6 commutes by naturality of
    $\alpha$, diagram 7 commutes by the fact that $\eta$ is a monoidal
    natural transformation, and diagram 8 commutes by naturality of
    $\eta$.
    

  \item[] \textit{Case 5.  strength interacts with monoidicity of $\Diamond$}
    $$
    \bfig
    \vSquares|ammmmma|/->`->```->``->/[
      \Box A \times \Diamond\Diamond B`
      \Box A \times \Diamond B`
      \Diamond(\Box A \times \Diamond B)``
      \Diamond\Diamond(\Box A \times B)`
      \Diamond(\Box A \times B);
      \id_{\Box A} \pd \mu_{B}`
      \st{A}{\Diamond B}```
      \Diamond(\st{A}{B})``
      \mu_{\Box A \pd B}]
    \morphism(1150,1000)|m|<0,-920>[`;\st{A}{B}]
    \efig
    $$
    \noindent
    Recall that:
    \[
    \begin{array}{rlll}
      \st{A}{B} & = & (\eta_{\Box A} \pd \id_{\Diamond B});\p{\Box A,B}\\
      \st{A}{\Diamond B} & = & (\eta_{\Box A} \pd \id_{\Diamond \Diamond B});\p{\Box A,\Diamond B}\\
    \end{array}
    \]
    This case follows from the fact that the following diagram
    commutes:
    \begin{center}
      \rotatebox{90}{$\bfig
    \qtriangle|mmm|<1500,1000>[
      \Box A \pd \Diamond\Diamond B`
      \Box A \pd \Diamond B`
      \Diamond\Box A \pd \Diamond B;
      \id_{\Box A} \pd \mu_B`
      \eta_{\Box A} \pd \mu_B`
      \eta_{\Box A} \pd \id_{\Diamond B}]

    \morphism(-1500,1000)|m|/<-/<1500,0>[
      \Diamond\Box A \pd \Diamond\Diamond B`
      \Box A \pd \Diamond\Diamond B;
      \eta_{\Box A} \pd \id_{\Diamond\Diamond B}]

    \btriangle(-1500,0)|mmm|<3000,1000>[
      \Diamond\Box A \pd \Diamond\Diamond B`
      \Diamond\Diamond\Box A \pd \Diamond\Diamond B`
      \Diamond\Box A \pd \Diamond B;
      \Diamond\eta_{\Box A} \pd \id_{\Diamond\Diamond B}`
      \id_{\Diamond\Box A} \pd \mu_B`
      \mu_{\Box A} \pd \mu_B]

    \square(-3000,0)|mmmm|/<-`->``<-/<1500,1000>[
      \Diamond(\Box A \pd \Diamond B)`
      \Diamond\Box A \pd \Diamond\Diamond B`
      \Diamond(\Diamond\Box A \pd \Diamond B)`
      \Diamond\Diamond\Box A \pd \Diamond\Diamond B;
      \p{\Box A,\Diamond B}`
      \Diamond (\eta_{\Box A} \pd \id_{\Diamond B})``
      \p{\Diamond\Box A,\Diamond B}]

    \square(-3000,-500)|mmmm|/`->`->`->/<4500,500>[
      \Diamond (\Diamond\Box A \pd \Diamond B)`
      \Diamond\Box A \pd \Diamond B`
      \Diamond\Diamond (\Box A \pd B)`
      \Diamond (\Box A \pd B);`
      \Diamond (\p{\Box A,B})`
      \p{\Box A,B}`
      \mu_{\Box A \pd B}]

    \place(-2300,500)[1]
    \place(-800,-250)[2]
    \place(-800,400)[3]
    \place(0,700)[4]
    \place(1050,700)[5]
    \efig$}
    \end{center}       
    Diagram commutes by naturality of $\mathsf{p}$, diagram 2 commutes
    because $\mu$ is a monoidal natural transformation, diagram 3
    commutes because $\mu$ is the monadic multiplication and by
    functorality of $\times$, and diagrams 4 and 5 commute by
    functoriality of $\times$.
    
  \item[] \textit{Case 6. commuting $\beta$ interacts with $\Diamond$}
    $$
    \bfig
    \hSquares|aamamaa|/``->``->`->`->/[
      \Diamond B \times \Box A``
      \Box A \times \Diamond B`
      \Diamond B \times \Diamond \Box A`
      \Diamond (B \times \Box A)`
      \Diamond (\Box A \times B);``
      \id_{\Diamond B} \pd \eta_{\Box A}``
      st_{A,B}`
      \p{B,\Box A}`
      \Diamond\beta_{B,\Box A}]
    \morphism(200,500)<1817,0>[`;\beta_{\Diamond B,\Box A}]
    \efig
    $$
    \noindent
    The previous diagram commutes by commutativity of the following
    diagram:
    $$
    \bfig
    \vSquares|ammmmma|[
      \Diamond B \times \Box A`
      \Box A \times \Diamond B`
      \Diamond B \times \Diamond\Box A`
      \Diamond\Box A \times \Diamond B`
      \Diamond (B \times \Box A)`
      \Diamond (\Box A \times B);
      \beta_{\Diamond B,\Box A}`
      \id_{\Diamond B} \times \eta_{\Box A}`
      \eta_{\Box A} \times \id_{\Diamond B}`
      \beta_{\Diamond B,\Diamond\Box A}`
      \p{B,\Box A}`
      \p{\Box A,B}`
      \Diamond\beta_{B,\Box A}]

    \place(600,750)[1]
    \place(600,250)[2]
    \efig
    $$
    \noindent
    Diagram 1 commutes because $\beta$ is a symmetric monoidal
    functor, and diagram 2 commutes by naturality of $\beta$.

  \item[] \textit{Case 7.  $\varepsilon$ interacts with $\Diamond$ and its monoidicity} 
    $$
    \bfig
    \hSquares|aamamaa|/``->``->`->`->/[
      \Box A \times \Diamond B``
      \Diamond(\Box A \times B)`
      A \times \Diamond B`
      \Diamond A \times \Diamond B`
      \Diamond (A \times B);``
      \varepsilon_A \times \id_{\Diamond B}``
      \Diamond(\varepsilon_A \times \id_{B})`
      \eta_A \times \id_{\Diamond B}`
      \p{A,B}]
    \morphism(200,500)<1605,0>[`;\st{A}{B}]
    \efig
    $$
    \noindent
    The previous diagram commutes by commutativity of the following
    diagram:
    $$
    \bfig
    \hSquares|aammmaa|[
      \Box A \times \Diamond B`
      \Diamond\Box A \times \Diamond B`
      \Diamond (\Box A \times B)`
      A \times \Diamond B`
      \Diamond A \times \Diamond B`
      \Diamond (A \times B);
      \eta_{\Box A} \times \id_{\Diamond B}`
      \p{\Box A,B}`
      \varepsilon_A \times \id_{\Diamond B}`
      \Diamond\varepsilon_A \times \id_{\Diamond B}`
      \Diamond (\varepsilon_A \times \id_B)`
      \eta_A \times \id_{\Diamond B}`
      \p{A,B}]
    \place(1750,250)[1]
    \place(600,250)[2]
    \efig
    $$
    \noindent
    Diagram 1 commutes by naturality of $\mathsf{p}$, and diagram 2
    commutes by naturality of $\eta$.
  \end{itemize}

\end{proof}\begin{proof}
  We must show that given the definition of an adjoint CS4 categorical
  model (Definition~\ref{def:CS4-single-adjoint-cat-model}) we can
  define an appropriate monad and comonad on a CCC with coproducts
  where the monad is strong with respect to the comonad.

  Suppose $(H,m)$ and $(J,n)$ are the adjoint monoidal functors given
  in Definition~\ref{def:CS4-single-adjoint-cat-model}, and define
  $\Box = JH$ and $\Diamond = HJ$.  By definition we assumed that
  $(\Box, q)$, where $q_{A,B} : \Box A \times \Box B \to \Box (A
  \times B)$, is monoidal, but we must show that $\Diamond$ is also
  monoidal.  We know that both $(H,n)$ and $(J,m)$ are monoidal
  endofunctors on $\cat{C}$ which implies that their composition
  $\Diamond$ is monoidal where
  \[
  \begin{array}{lll}
    \mathsf{p}_{1} = \eta_{1} : 1 \to \Diamond 1\\
    \mathsf{p}_{A,B} = \m{HA,HB};J(\mathsf{n}_{A,B})
    \colon \Diamond A \pd \Diamond B \mto \Diamond(A \pd B)
  \end{array}
  \]
  and the following diagrams commute (proofs omitted):
  \begin{mathpar}
    \scriptsize
    \bfig
    \vSquares|ammmmma|/->`->`->``->`->`->/[
      (\Diamond A \times \Diamond B) \times \Diamond C`
      \Diamond A \times (\Diamond B \times \Diamond C)`
      \Diamond(A \times B) \times \Diamond C`
      \Diamond A \times \Diamond(B \times C)`
      \Diamond ((A \times B) \times C)`
      \Diamond (A \times (B \times C));
      \alpha`
      \mathsf{p}_{A,B} \times \id_{\Diamond C}`
      \id_{\Diamond A} \times \mathsf{p}_{B,C}``
      \mathsf{p}_{A \times B,C}`
      \mathsf{p}_{A,B \times C}`
      \Diamond \alpha]
    \efig
    \and
    \bfig
    \hSquares|ammmmaa|/->``->`<-``->`/[
      1 \times \Diamond A`
      \Diamond A``
      \Diamond 1 \times \Diamond A`
      \Diamond(1 \times A)`;
      \lambda_{\Diamond A}``
      \mathsf{p}_{1} \times \id_{\Diamond A}`
      \Diamond \lambda_A``
      \mathsf{p}_{1,A}`]
    \efig
    \and
    \bfig
    \hSquares|ammmmaa|/->``->`<-``->`/[
      \Diamond A \times 1`
      \Diamond A``
      \Diamond A \times \Diamond 1`
      \Diamond(A \times 1)`;
      \rho_{\Diamond A}``
      \id_{\Diamond A} \times \mathsf{p}_{1}`
      \Diamond \rho_A``
      \mathsf{p}_{A,1}`]
    \efig
    \and
    \bfig
    \hSquares|ammmmaa|/->``->`->``->`/[
      \Diamond A \times \Diamond B`
      \Diamond B \times \Diamond A``
      \Diamond (A \times B)`
      \Diamond (B \times A)`;
      \beta_{\Diamond A,\Diamond B}``
      \mathsf{p}_{A,B}`
      \mathsf{p}_{B,A}``
      \Diamond\beta_{A,B}`]
    \efig
  \end{mathpar}

  Furthermore, suppose $J \dashv H$, where the unit, $\varepsilon :
  \Box A \to A$, and the counit, $\eta : A \to \Diamond A$, are
  monoidal natural transformations.  This implies that the following
  diagrams commute:
  \begin{mathpar}
    \bfig
    \btriangle<800,500>[A \times B`\Diamond A \times \Diamond B`\Diamond (A \times B);\eta_A \times \eta_B`\eta_{A \times B}`\mathsf{p}_{A,B}]
    \efig
    \and
    \bfig
    \qtriangle<800,500>[\Box A \times \Box B`\Box (A \times B)`A \times B;\mathsf{q}_{A,B}`\varepsilon_A \times \varepsilon_B`\varepsilon_{A \times B}]
    \efig
    \and
    \bfig
    \qtriangle<800,500>[1`J 1`\Diamond 1;n_{1}`\eta_1`J m_1]       
    \efig
    \and
    \bfig
    \hSquares|ammmmaa|/->``=`->``<-`/[
      \Box 1`
      1``
      \Box 1`
      H 1`;
      \varepsilon_1```
      m_1``
      H n_1`]
    \efig
    \and
    \bfig
    \qtriangle<800,500>[H A`H\Box A`H A;\eta_{H A}`\id_{H A}`H\varepsilon_A]       
    \efig
    \and
    \bfig
    \qtriangle<800,500>[J A`\Box J A`J A;J\eta_A`\id_{J A}`\varepsilon_{J A}]
    \efig    
  \end{mathpar}
  It is a well-known fact about adjoints that $(\Box, \varepsilon,
  \delta)$, where $\delta : \Box A \to \Box\Box A$ is a comonad, and
  $(\Diamond, \eta, \mu)$, where $\mu : \Diamond\Diamond A \to
  \Diamond A$ is a monad.  In addition, $\mu$ and $\delta$ are monoidal
  natural transformations where we have the following:
  \[
  \begin{array}{lll}
    \d{1} = \p{1};\Diamond\p{1} : 1 \mto \Diamond^2 1\\
    \d{A,B} =  \p{\Diamond A,\Diamond B};\Diamond\p{A,B} : \Diamond^2 A \times \Diamond^2 B \mto \Diamond^2 (A \times B)\\
    \\
    \b{1} = \q{1};\Box\q{1} : 1 \mto \Box^2 1\\
    \b{A,B} = \q{\Box A,\Box B};\Box\q{A,B} : \Box^2 A \times \Box^2 B \mto \Box^2 (A \times B)\\
  \end{array}
  \]
  Thus, the following diagrams commute:
  \begin{mathpar}
    \bfig
    \hSquares|ammmmaa|/->``->`->``->`/[
      \Diamond^3 A`
      \Diamond^2 A``
      \Diamond^2 A`
      \Diamond A`;
      \Diamond \mu_A``
      \mu_{\Diamond A}`
      \mu_A``
      \mu_A`]
    \efig
    \and
    \bfig
    \qtriangle/->`=`->/<800,500>[\Diamond A`\Diamond^2 A`\Diamond A;\eta_{\Diamond A}``\mu_A]
    \btriangle(0,0)/->`=`->/<800,500>[\Diamond A`\Diamond^2 A`\Diamond A;\Diamond \eta_{A}``\mu_A]
    \efig
    \and
    \bfig
    \hSquares|ammmmaa|/->``->`->``->`/[
      \Box A`
      \Box^2 A``
      \Box^2 A`
      \Box^3 A`;
      \delta_A``
      \delta_A`
      \delta_{\Box A}``
      \Box\delta_A`]
    \efig
    \and
    \bfig
    \qtriangle/->`=`->/<800,500>[\Box A`\Box^2 A`\Box A;\delta_A``\Box \varepsilon]
    \btriangle(0,0)/->`=`->/<800,500>[\Box A`\Box^2 A`\Box A;\delta_A``\varepsilon_{\Box A}]
    \efig
    \and
    %% \bfig   
    %% \vSquares|ammmmma|/->`->```->``->/[
    %%   \Diamond^2 A \times \Diamond^2 B`
    %%   \Diamond A \times \Diamond B`
    %%   \Diamond(\Diamond A \times \Diamond B)``
    %%   \Diamond^2(A \times B)`
    %%   \Diamond(A \times B);
    %%   \mu_A \times \mu_B`
    %%   \p{\Diamond A,\Diamond B}```
    %%   \Diamond\p{A,B}``
    %%   \mu_{A \times B}]
    %% \morphism(1108,0)/<-/<0,1000>[\Diamond(A \times B)`\Diamond A \times \Diamond B;\p{A,B}]
    %% \efig
    \and
    \bfig
    \square<1000,1000>[
      \Diamond^2 A \times \Diamond^2 B`
      \Diamond A \times \Diamond B`
      \Diamond^2(A \times B)`
      \Diamond(A \times B);
      \mu_A \times \mu_B`
      \d{A,B}`
      \p{A,B}`
      \mu_{A \times B}]
    \efig
    \and
    \bfig
    \Vtriangle/->`<-`<-/[
      \Diamond^2 1`
      \Diamond 1`
      1;
      \mu_1`
      \d{1}`
      \p{1}]
    \efig
    \and        
    \bfig
    \square<1000,1000>[
      \Box A \times \Box B`
      \Box^2 A \times \Box^2 B`
      \Box(A \times B)`
      \Box^2(A \times B);
      \delta_A \times \delta_B`
      \q{A,B}`
      \b{A,B}`
      \delta_{A \times B}]
    %% \vSquares|ammmmma|/->``->```->`->/[
    %%   \Box A \times \Box B`
    %%   \Box^2 A \times \Box^2 B``
    %%   \Box(\Box A \times \Box B)`
    %%   \Box(A \times B)`
    %%   \Box^2(A \times B);
    %%   \varepsilon_A \times \varepsilon_B``
    %%   \q{\Box A,\Box B}```
    %%   \Box\q{A,B}`
    %%   \varepsilon_{A \times B}]
    %% \morphism(0,0)/<-/<0,1000>[\Box(A \times B)`\Box A \times \Box B;\q{A,B}]
    \efig
    \and
    \bfig
    \Vtriangle/->`<-`<-/[
      \Box 1`
      \Box^2 1`
      1;
      \delta_1`
      \q{1}`
      \d{1}]
    \efig
  \end{mathpar}

  We can now define the $\Box$-strength map as follows:
  \[
  \st{A}{B} = (\eta_{\Box A} \pd \id_{\Diamond B});\mathsf{p}_{\Box A,B} : \Box A \pd \Diamond B \mto \Diamond(\Box A \pd B)
  \]
  We can see that $\st{A}{B}$ is a natural transformation, because it
  is defined as a composition of natural transformations.
  
  %% To prove that the appropriate diagrams commute we first note that
  %% the triangle  
  Next we must show that all of the appropriate diagrams given in
  Definition~\ref{def:comonad-strong-monad} commute.
  \begin{itemize}
  \item[] \textit{Case 1. the first projection interacts with $\Diamond$:}
    \[
    \bfig
    \square|amma|<850,600>[
      \Box A \times \Diamond B`
      \Diamond (\Box A \times B)`
      \Box A`
      \Diamond\Box A;
      \st{A}{B}`
      \pi_1`
      \Diamond\pi_1`
      \eta_A]
    \efig
    \]
    This diagrams commutes, because the following diagram commutes:
    \[
    \bfig
    \square|ammm|/->`->``/<1500,2000>[
      \Box A \times \Diamond B`
      \Diamond\Box A \times \Diamond B`
      \Box A`;
      \eta_{\Box A} \times \id_{\Diamond B}`
      \pi_1``]

    \square(1500,0)|ammm|/->``->`/<1500,2000>[
      \Diamond\Box A \times \Diamond B`
      \Diamond (\Box A \times B)``
      \Diamond\Box A;
      \p{\Box A,B}``
      \Diamond\pi_1`]

    \morphism<3000,0>[\Box A`\Diamond\Box A;\eta_{\Box A}]

    \morphism(484,1500)<1000,0>[
      \Box A \times 1`
      \Diamond\Box A \times \Diamond 1;
      \eta_{\Box A} \times \eta_1]

    \morphism(1484,1500)<700,-500>[
      \Diamond\Box A \times \Diamond 1`
      \Diamond(\Box A \times 1);
      \p{\Box A,1}]

    \morphism(484,1500)|m|/{@{>}@/_1em/}/<1700,-500>[
      \Box A \times 1`
      \Diamond(\Box A \times 1);
      \eta_{\Box A \times 1}]

    \morphism(484,1500)|m|<700,-800>[
      \Box A \times 1`
      \Box A;
      \pi_1]

    \morphism(0,2000)|m|<484,-500>[
      \Box A \times \Diamond B`
      \Box A \times 1;
      \id_{\Box A} \times \t_{\Diamond B}]

    \morphism(1500,2000)|m|<-17,-500>[
      \Diamond\Box A \times \Diamond B`
      \Diamond\Box A \times \Diamond 1;
      \id_{\Diamond\Box A} \times \Diamond\t_{B}]

    \morphism(3000,2000)|m|<-815,-1000>[
      \Diamond(\Box A \times B)`
      \Diamond(\Box A \times 1);
      \Diamond (id_{\Box A} \times \t_{B})]

    \morphism(2184,1000)|m|<816,-1000>[
      \Diamond(\Box A \times 1)`
      \Diamond\Box A;
      \Diamond\pi_1]

    \morphism(1184,700)|m|<1816,-700>[
      \Box A`
      \Diamond\Box A;
      \eta_{\Box A}]
    
    \place(2700,1000)[1]
    \place(2150,1700)[2]
    \place(800,1750)[3]
    \place(1300,1300)[4]
    \place(1800,800)[5]
    \place(500,500)[6]
    \efig    
    \]
    Diagrams 1 and 6 commute because we are in a cartesian closed
    category, diagram 2 commutes by naturality of $\p{}$, diagram 3
    commutes because $\Diamond$ is a product functor, diagram 4
    commutes because $\eta$ is the unit of a symmetric monoidal
    adjunction, and diagram 5 commutes by naturality of $\eta$.
    
  \item[] \textit{Case 2. the second projection interacts with $\Diamond$:}
    \[
    \bfig
    \qtriangle<850,600>[
      \Box A \times \Diamond B`
      \Diamond (\Box A \times B)`
      \Diamond B;
      \st{A}{B}`
      \pi_2`
      \Diamond\pi_2]
    \efig
    \]
    This case is similar to the previous case.
    
    %% \item[] \textit{Case 1. the object 1 behaves as the unit for products}
  %%   $$
  %%   \bfig
  %%   \vSquares|ammmmma|/>``>```>`>/[\Box 1 \times \Diamond A`\Diamond (\Box 1 \times A)``\Diamond(1 \times A)`1 \times \Diamond A`\Diamond A;\st{1}{A}``\Diamond(\varepsilon_1 \times \id_A)```\Diamond\lambda`\lambda]
  %%   \morphism(0,1000)|m|/->/<0,-950>[`;\varepsilon_1 \times \id_{\Diamond A}]
  %%   \efig
  %%   $$
  %%   This diagram commutes by commutativity of the following diagram:
  %%   %% Equational version:
  %%   %% \begin{center}
  %%   %%   \begin{math}
  %%   %%     \begin{array}{rllllllll}
  %%   %%       & & \st{1}{A};\Diamond (\varepsilon_1 \times \id_A);\Diamond\lambda_A\\
  %%   %%       \text{(Definition of $\mathsf{st}$)}
  %%   %%       & = & (\eta_{\Box 1} \times \id_{\Diamond A});\p{\Box 1,\Diamond A};\Diamond (\varepsilon_1 \times \id_A);\Diamond\lambda_A\\
  %%   %%       \text{(Naturality of $\mathsf{p}$)}
  %%   %%       & = & (\eta_{\Box 1} \times \id_{\Diamond A});(\Diamond \varepsilon_1 \times \Diamond\id_A);\p{1,A};\Diamond\lambda_A\\
  %%   %%       \text{(Functoriality of $\times$)}
  %%   %%       & = & ((\eta_{\Box 1};\Diamond \varepsilon_1) \times (\id_{\Diamond A};\Diamond\id_A));\p{1,A};\Diamond\lambda_A\\
  %%   %%       & = & ((\eta_{\Box 1};\Diamond \varepsilon_1) \times (\id_{\Diamond A};\id_{\Diamond A}));\p{1,A};\Diamond\lambda_A\\
  %%   %%       \text{(Naturality of $\eta$)}
  %%   %%       & = & ((\varepsilon_1;\eta_{1}) \times (\id_{\Diamond A};\id_{\Diamond A}));\p{1,A};\Diamond\lambda_A\\
  %%   %%       \text{(Definition of $\mathsf{p}$)}
  %%   %%       & = & ((\varepsilon_1;\p{1}) \times (\id_{\Diamond A};\id_{\Diamond A}));\p{1,A};\Diamond\lambda_A\\
  %%   %%       \text{(Functoriality of $\times$)}
  %%   %%       & = & (\varepsilon_1 \times \id_{\Diamond A});(\p{1} \times \id_{\Diamond A});\p{1,A};\Diamond\lambda_A\\
  %%   %%       \text{($\Diamond$ is Symmetric Monoidal)}
  %%   %%       & = & (\varepsilon_1 \times \id_{\Diamond A});\lambda_{\Diamond A}\\
  %%   %%     \end{array}
  %%   %%   \end{math}
  %%   %% \end{center}
  %%   $$
  %%   \bfig
  %%   \square|amma|<1000,500>[
  %%     \Diamond\Box 1 \times \Diamond A`
  %%     \Diamond (\Box 1 \times A)`
  %%     \Diamond 1 \times \Diamond A`
  %%     \Diamond (1 \times A);
  %%     \p{\Box 1,A}`
  %%     \Diamond \varepsilon_1 \times \id_{\Diamond A}`
  %%     \Diamond (\varepsilon_1 \times \id_A)`
  %%     \p{1,A}]

  %%   \square(-1000,0)|amma|<1000,500>[
  %%     \Box 1 \times \Diamond A`
  %%     \Diamond\Box 1 \times \Diamond A`
  %%     1 \times \Diamond A`
  %%     \Diamond 1 \times \Diamond A;
  %%     \eta_{\Box 1} \times \id_{\Diamond A}`
  %%     \varepsilon_1 \times \id_{\Diamond A}`
  %%     \Diamond \varepsilon_1 \times \id_{\Diamond A}`
  %%     (\p{1} = \eta_1) \times \id_{\Diamond A}]

  %%   \qtriangle(-1000,-500)|mmm|/`->`->/<2000,500>[
  %%     1 \times \Diamond A`
  %%     \Diamond (1 \times A)`
  %%     \Diamond A;`
  %%     \lambda_{\Diamond A}`
  %%     \Diamond\lambda_A]

  %%   \place(500,250)[1]
  %%   \place(-500,250)[2]
  %%   \place(500,-200)[3]
  %%   \efig
  %%   $$
  %%   \noindent
  %%   Diagram 1 commutes by naturality of $\mathsf{p}$, diagram 2
  %%   commutes by naturality of $\eta$, and diagram 3 commutes because
  %%   $\Diamond$ is a symmetric monoidal functor.

  \item[] \textit{Case 3. unit $\eta$ of the monad and strength interact well, $\Box  A $ is a parameter}
    $$
    \bfig
    \btriangle<800,500>[
      \Box A \pd B`
      \Box A \pd \Diamond B`
      \Diamond(\Box A \pd B);
      \id_{\Box A} \pd \eta_{B}`
      \eta_{\Box A \times B}`
      \st{A}{B}]
    \efig
    $$

    The previous diagram commutes, because the following diagram commutes:
    $$
    \bfig
    \btriangle|ama|/->`->`->/<1500,500>[
      \Box A \pd B`
      \Box A \pd \Diamond B`
      \Diamond\Box A \pd \Diamond B;
      \id_{\Box A} \pd \eta_{B}`
      \eta_{\Box A} \pd \eta_B`
      \eta_{\Box A} \pd \id_{\Diamond B}]

    \qtriangle(0,0)/->``<-/<1500,500>[
      \Box A \pd B`
      \Diamond (\Box A \times B)`
      \Diamond\Box A \pd \Diamond B;
      \eta_{\Box A \times B}``
      \p{\Box A,B}]

    \place(250,200)[1]
    \place(1200,300)[2]
    \efig
    $$
    \noindent
    Diagram 1 clearly commutes, and diagram 2 commutes because $\eta$
    is a symmetric monoidal natural transformation.
    
  %% \item[] \textit{Case 3. co-unit of the comonad $\varepsilon$ and unit of the monad $\eta$ interact well?}
  %%   $$
  %%   \bfig
  %%   \hSquares|aamaaaa|/->`->`->```->`/[
  %%     \Box A \times 1`
  %%     \Box A \times \Diamond 1`
  %%     \Diamond (\Box A \times 1)`
  %%     \Box A`
  %%     A`
  %%     ;
  %%     \id_{\Box A} \times \eta_1`
  %%     \st{A}{1}`
  %%     \rho_{\Box A}```
  %%     \varepsilon_A`]
  %%   \qtriangle(1978,0)/->``->/<800,500>[\Diamond (\Box A \times 1)`\Diamond\Box A`\Diamond A;\Diamond \rho_{\Box A}``\Diamond\varepsilon_A]
  %%   \morphism(1060,0)/->/<1640,0>[`;\eta_A]
  %%   \efig
  %%   $$
  %%   \noindent
  %%   Recall that
  %%   $\st{A}{1} = (\eta_{\Box A} \pd \id_{\Diamond 1});\p{\Box A,1}$.
  %%   Now the previous diagram commutes, because the following diagram commutes:
  %%   $$
  %%   \bfig
  %%   \btriangle|mma|<1444,1000>[\Box A \pd 1`\Diamond\Box A \pd \Diamond 1`\Diamond(\Box A \pd 1);\eta_{\Box A} \pd \eta_1`\eta_{\Box A \pd 1}`\p{\Box A,1}]
  %%   \dtriangle(-1000,0)/->``->/<1000,1000>[\Box A \pd 1`\Box A \pd 1`\Diamond\Box A \pd \Diamond 1;\id_{\Box A} \pd \eta_1``\eta_{\Box A} \pd
  %%     \id_{\Diamond 1}]

  %%   \hSquares(0,0)/->`->```<-``/<1000>[\Box A \pd 1`\Box A`\Diamond\Box A```\Diamond (\Box A \times 1);\rho_{\Box A}`\eta_{\Box A}```\Diamond (\rho_{\Box A})``]

  %%   \square(746,1000)/->`<-`<-`/<698,500>[A`\Diamond A`\Box A`\Diamond\Box A;\eta_A`\varepsilon_A`\Diamond\varepsilon_A`]

  %%   \place(-400,300)[1]
  %%   \place(400,300)[2]
  %%   \place(1100,700)[3]
  %%   \place(1100,1250)[4]
  %%   \efig
  %%   $$
  %%   \noindent
  %%   Diagram 1 commutes by functorality of $\times$, diagram 2 commutes
  %%   because $\eta$ is a monoidal natural transformation, and diagrams
  %%   3 and 4 commute by naturality of $\eta$.

  \item[] \textit{Case 4. associativity $\alpha$ interacts with co-monoidicity of $\Box$}
    $$
    \bfig
    \vSquares|ammmmmm|/->`->`->```->`/[
      \Box A \times (\Box B \times \Diamond C)`
      \Box A \times \Diamond(\Box B \times C)`
      (\Box A \times \Box B) \times \Diamond C`
      \Diamond(\Box A \times (\Box B \times C))``
      \Diamond((\Box A \times \Box B) \times C);
      \id_{\Box A} \pd \st{B}{C}`
      \alpha^{-1}`
      \st{A}{\Box B \times C}```
      \Diamond\alpha^{-1}`]
    \morphism(1554,0)|m|/->/<0,-500>[`\Diamond(\Box(A \times B) \times C);\Diamond(\m{A,B} \times \id_C)]
    
    \morphism(0,500)|m|/->/<0,-1000>[`\Box(A \times B) \times \Diamond C;\m{A,B} \times \id_{\Diamond C}]

    \morphism(350,-500)|a|/->/<800,0>[`;\st{A \times B}{C}]
    \efig
    $$
    \noindent
    Recall that:
    \[
    \begin{array}{rlll}
      \st{B}{C}              & = & (\eta_{\Box B} \pd \id_{\Diamond C});\p{\Box B,C}\\
      \st{A \pd B}{C}        & = & (\eta_{\Box (A \pd B)} \pd \id_{\Diamond C});\p{\Box (A \pd B),C}\\
      \st{A}{\Box B \pd C} & = & (\eta_{\Box A} \pd \id_{\Diamond (\Box B \pd C)});\p{\Box A,(\Box B \pd C)}\\
    \end{array}
    \]
    In addition, we require the following diagram (whose commutativity
    is implied by the fact that $\Diamond$ is a symmetric monoidal
    functor):
    $$
    \bfig
    \vSquares|ammmmma|/->`->`->``->`->`->/[
      \Diamond A \pd (\Diamond B \pd \Diamond C)`
      (\Diamond A \pd \Diamond B) \pd \Diamond C`
      \Diamond A \pd \Diamond (B \pd C)`
      \Diamond (A \pd B) \pd \Diamond C`
      \Diamond (A \pd (B \pd C))`
      \Diamond ((A \pd B) \pd C);
      \alpha^{-1}_{\Diamond A,\Diamond B,\Diamond C}`
      \id_{\Diamond A} \pd \p{B,C}`
      \p{A,B} \pd \id_{\Diamond C}``
      \p{A,B \pd C}`
      \p{A \pd B,C}`
      \Diamond\alpha^{-1}_{A,B,C}]
    \efig
    $$
    \noindent
    Finally, this case follows because the following diagram commutes:
    \begin{center}
      \rotatebox{90}{$
    \bfig
    \btriangle|mmm|<1769,1000>[
      \Box A \pd (\Diamond\Box B \pd \Diamond C)`
      \Diamond\Box A \pd (\Diamond\Box B \pd \Diamond C)`
      \Diamond\Box A \pd (\Diamond\Box B \pd C);
      \eta_{\Box A} \pd \id_{\Diamond \Box B \pd C}`
      \eta_{\Box A} \pd \p{\Box B,C}`
      \id_{\Diamond\Box A} \pd \p{\Box B,C}]

    \qtriangle|mam|/->``->/<1769,1000>[
      \Box A \pd (\Diamond\Box B \pd \Diamond C)`
      \Box A \pd \Diamond (\Box B \pd C)`
      \Diamond\Box A \pd (\Diamond\Box B \pd C);
      \id_{\Box A} \pd \p{\Box B,C}``
      \eta_{\Box A} \pd \id_{\Diamond (\Box B \pd C)}]    

    \qtriangle(-1800,0)|mmm|<1800,1000>[
      \Box A \pd (\Box B \pd \Diamond C)`
      \Box A \pd (\Diamond\Box B \pd \Diamond C)`
      \Diamond\Box A \pd (\Diamond\Box B \pd \Diamond C);
      \id_{\Box A} \pd (\eta_{\Box B} \pd \id_{\Diamond C})`
      \eta_{\Box A} \pd (\eta_{\Box B} \pd \id_{\Diamond C})`
      \eta_{\Box A} \pd \id_{\Diamond \Box B \pd C}]

    \square(0,-500)|mmmm|/`->`->`/<1769,500>[
      \Diamond\Box A \pd (\Diamond\Box B \pd \Diamond C)`
      \Diamond\Box A \pd (\Diamond\Box B \pd C)`
      (\Diamond\Box A \pd \Diamond\Box B) \pd \Diamond C`
      \Diamond (\Box A \pd (\Box B \pd C));`
      \alpha_{\Diamond\Box A,\Diamond\Box B,\Diamond C}`
      \p{\Box A,\Box B \pd C}`]   

    \square(0,-1000)|mmmm|/`->`->`/<1769,500>[
      (\Diamond\Box A \pd \Diamond\Box B) \pd \Diamond C`
      \Diamond (\Box A \pd (\Box B \pd C))`
      \Diamond (\Box A \pd \Box B) \pd \Diamond C`
      \Diamond ((\Box A \pd \Box B) \pd C);`
      \p{\Box A,\Box B} \pd \id_{\Diamond C}`
      \Diamond \alpha_{\Box A,\Box B,C}`]        

    \square(0,-1500)|mmmm|<1769,500>[
      \Diamond (\Box A \pd \Box B) \pd \Diamond C`
      \Diamond ((\Box A \pd \Box B) \pd C)`
      \Diamond\Box(A \pd B) \pd \Diamond C`
      \Diamond (\Box(A \pd B) \pd C);
      \p{\Box A \pd \Box B, C}`
      \Diamond \m{A,B} \pd \id_{\Diamond C}`
      \Diamond (\m{A,B} \pd \id_C)`
      \p{\Box (A \pd B),C}]

    \btriangle(-1800,-500)|mmm|/->``->/<1800,1500>[
      \Box A \pd (\Box B \pd \Diamond C)`
      (\Box A \pd \Box B) \pd \Diamond C`
      (\Diamond\Box A \pd \Diamond\Box B) \pd \Diamond C;
      \alpha_{\Box A,\Box B,\Diamond C}``
      (\eta_{\Box A} \pd \eta_{\Box B}) \pd \id_{\Diamond C}]

    \qtriangle(-1800,-1000)|mmm|/`->`/<1800,500>[
      (\Box A \pd \Box B) \pd \Diamond C`
      (\Diamond\Box A \pd \Diamond\Box B) \pd \Diamond C`
      \Diamond (\Box A \pd \Box B) \pd \Diamond C;`
      \eta_{\Box A \pd \Box B} \pd \id_{\Diamond C}`]

    \btriangle(-1800,-1500)|mmm|/->``->/<1800,1000>[
      (\Box A \pd \Box B) \pd \Diamond C`
      \Box(A \pd B) \pd \Diamond C`
      \Diamond\Box(A \pd B) \pd \Diamond C;
      \m{A,B} \pd \id_{\Diamond C}``
      \eta_{\Box (A \pd B)} \pd \id_{\Diamond C}]

    \place(1200,700)[1]
    \place(500,300)[2]
    \place(900,-500)[3]
    \place(900,-1250)[4]
    \place(-500,700)[5]
    \place(-1000,0)[6]
    \place(-400,-700)[7]
    \place(-1000,-1100)[8]
    \efig
    $}
    \end{center}
    Diagrams 1, 2 and 5 commute by functorality of $\times$, diagram 3
    commutes by the additional diagram from above, diagram 4 commutes
    by naturality of $\mathsf{p}$, diagram 6 commutes by naturality of
    $\alpha$, diagram 7 commutes by the fact that $\eta$ is a monoidal
    natural transformation, and diagram 8 commutes by naturality of
    $\eta$.
    

  \item[] \textit{Case 5.  strength interacts with monoidicity of $\Diamond$}
    $$
    \bfig
    \vSquares|ammmmma|/->`->```->``->/[
      \Box A \times \Diamond\Diamond B`
      \Box A \times \Diamond B`
      \Diamond(\Box A \times \Diamond B)``
      \Diamond\Diamond(\Box A \times B)`
      \Diamond(\Box A \times B);
      \id_{\Box A} \pd \mu_{B}`
      \st{A}{\Diamond B}```
      \Diamond(\st{A}{B})``
      \mu_{\Box A \pd B}]
    \morphism(1150,1000)|m|<0,-920>[`;\st{A}{B}]
    \efig
    $$
    \noindent
    Recall that:
    \[
    \begin{array}{rlll}
      \st{A}{B} & = & (\eta_{\Box A} \pd \id_{\Diamond B});\p{\Box A,B}\\
      \st{A}{\Diamond B} & = & (\eta_{\Box A} \pd \id_{\Diamond \Diamond B});\p{\Box A,\Diamond B}\\
    \end{array}
    \]
    This case follows from the fact that the following diagram
    commutes:
    \begin{center}
      \rotatebox{90}{$\bfig
    \qtriangle|mmm|<1500,1000>[
      \Box A \pd \Diamond\Diamond B`
      \Box A \pd \Diamond B`
      \Diamond\Box A \pd \Diamond B;
      \id_{\Box A} \pd \mu_B`
      \eta_{\Box A} \pd \mu_B`
      \eta_{\Box A} \pd \id_{\Diamond B}]

    \morphism(-1500,1000)|m|/<-/<1500,0>[
      \Diamond\Box A \pd \Diamond\Diamond B`
      \Box A \pd \Diamond\Diamond B;
      \eta_{\Box A} \pd \id_{\Diamond\Diamond B}]

    \btriangle(-1500,0)|mmm|<3000,1000>[
      \Diamond\Box A \pd \Diamond\Diamond B`
      \Diamond\Diamond\Box A \pd \Diamond\Diamond B`
      \Diamond\Box A \pd \Diamond B;
      \Diamond\eta_{\Box A} \pd \id_{\Diamond\Diamond B}`
      \id_{\Diamond\Box A} \pd \mu_B`
      \mu_{\Box A} \pd \mu_B]

    \square(-3000,0)|mmmm|/<-`->``<-/<1500,1000>[
      \Diamond(\Box A \pd \Diamond B)`
      \Diamond\Box A \pd \Diamond\Diamond B`
      \Diamond(\Diamond\Box A \pd \Diamond B)`
      \Diamond\Diamond\Box A \pd \Diamond\Diamond B;
      \p{\Box A,\Diamond B}`
      \Diamond (\eta_{\Box A} \pd \id_{\Diamond B})``
      \p{\Diamond\Box A,\Diamond B}]

    \square(-3000,-500)|mmmm|/`->`->`->/<4500,500>[
      \Diamond (\Diamond\Box A \pd \Diamond B)`
      \Diamond\Box A \pd \Diamond B`
      \Diamond\Diamond (\Box A \pd B)`
      \Diamond (\Box A \pd B);`
      \Diamond (\p{\Box A,B})`
      \p{\Box A,B}`
      \mu_{\Box A \pd B}]

    \place(-2300,500)[1]
    \place(-800,-250)[2]
    \place(-800,400)[3]
    \place(0,700)[4]
    \place(1050,700)[5]
    \efig$}
    \end{center}       
    Diagram commutes by naturality of $\mathsf{p}$, diagram 2 commutes
    because $\mu$ is a monoidal natural transformation, diagram 3
    commutes because $\mu$ is the monadic multiplication and by
    functorality of $\times$, and diagrams 4 and 5 commute by
    functoriality of $\times$.
    
  \item[] \textit{Case 6. commuting $\beta$ interacts with $\Diamond$}
    $$
    \bfig
    \hSquares|aamamaa|/``->``->`->`->/[
      \Diamond B \times \Box A``
      \Box A \times \Diamond B`
      \Diamond B \times \Diamond \Box A`
      \Diamond (B \times \Box A)`
      \Diamond (\Box A \times B);``
      \id_{\Diamond B} \pd \eta_{\Box A}``
      st_{A,B}`
      \p{B,\Box A}`
      \Diamond\beta_{B,\Box A}]
    \morphism(200,500)<1817,0>[`;\beta_{\Diamond B,\Box A}]
    \efig
    $$
    \noindent
    The previous diagram commutes by commutativity of the following
    diagram:
    $$
    \bfig
    \vSquares|ammmmma|[
      \Diamond B \times \Box A`
      \Box A \times \Diamond B`
      \Diamond B \times \Diamond\Box A`
      \Diamond\Box A \times \Diamond B`
      \Diamond (B \times \Box A)`
      \Diamond (\Box A \times B);
      \beta_{\Diamond B,\Box A}`
      \id_{\Diamond B} \times \eta_{\Box A}`
      \eta_{\Box A} \times \id_{\Diamond B}`
      \beta_{\Diamond B,\Diamond\Box A}`
      \p{B,\Box A}`
      \p{\Box A,B}`
      \Diamond\beta_{B,\Box A}]

    \place(600,750)[1]
    \place(600,250)[2]
    \efig
    $$
    \noindent
    Diagram 1 commutes because $\beta$ is a symmetric monoidal
    functor, and diagram 2 commutes by naturality of $\beta$.

  \item[] \textit{Case 7.  $\varepsilon$ interacts with $\Diamond$ and its monoidicity} 
    $$
    \bfig
    \hSquares|aamamaa|/``->``->`->`->/[
      \Box A \times \Diamond B``
      \Diamond(\Box A \times B)`
      A \times \Diamond B`
      \Diamond A \times \Diamond B`
      \Diamond (A \times B);``
      \varepsilon_A \times \id_{\Diamond B}``
      \Diamond(\varepsilon_A \times \id_{B})`
      \eta_A \times \id_{\Diamond B}`
      \p{A,B}]
    \morphism(200,500)<1605,0>[`;\st{A}{B}]
    \efig
    $$
    \noindent
    The previous diagram commutes by commutativity of the following
    diagram:
    $$
    \bfig
    \hSquares|aammmaa|[
      \Box A \times \Diamond B`
      \Diamond\Box A \times \Diamond B`
      \Diamond (\Box A \times B)`
      A \times \Diamond B`
      \Diamond A \times \Diamond B`
      \Diamond (A \times B);
      \eta_{\Box A} \times \id_{\Diamond B}`
      \p{\Box A,B}`
      \varepsilon_A \times \id_{\Diamond B}`
      \Diamond\varepsilon_A \times \id_{\Diamond B}`
      \Diamond (\varepsilon_A \times \id_B)`
      \eta_A \times \id_{\Diamond B}`
      \p{A,B}]
    \place(1750,250)[1]
    \place(600,250)[2]
    \efig
    $$
    \noindent
    Diagram 1 commutes by naturality of $\mathsf{p}$, and diagram 2
    commutes by naturality of $\eta$.
  \end{itemize}

\end{proof}\begin{proof}
  We must show that given the definition of an adjoint CS4 categorical
  model (Definition~\ref{def:CS4-single-adjoint-cat-model}) we can
  define an appropriate monad and comonad on a CCC with coproducts
  where the monad is strong with respect to the comonad.

  Suppose $(H,m)$ and $(J,n)$ are the adjoint monoidal functors given
  in Definition~\ref{def:CS4-single-adjoint-cat-model}, and define
  $\Box = JH$ and $\Diamond = HJ$.  By definition we assumed that
  $(\Box, q)$, where $q_{A,B} : \Box A \times \Box B \to \Box (A
  \times B)$, is monoidal, but we must show that $\Diamond$ is also
  monoidal.  We know that both $(H,n)$ and $(J,m)$ are monoidal
  endofunctors on $\cat{C}$ which implies that their composition
  $\Diamond$ is monoidal where
  \[
  \begin{array}{lll}
    \mathsf{p}_{1} = \eta_{1} : 1 \to \Diamond 1\\
    \mathsf{p}_{A,B} = \m{HA,HB};J(\mathsf{n}_{A,B})
    \colon \Diamond A \pd \Diamond B \mto \Diamond(A \pd B)
  \end{array}
  \]
  and the following diagrams commute (proofs omitted):
  \begin{mathpar}
    \scriptsize
    \bfig
    \vSquares|ammmmma|/->`->`->``->`->`->/[
      (\Diamond A \times \Diamond B) \times \Diamond C`
      \Diamond A \times (\Diamond B \times \Diamond C)`
      \Diamond(A \times B) \times \Diamond C`
      \Diamond A \times \Diamond(B \times C)`
      \Diamond ((A \times B) \times C)`
      \Diamond (A \times (B \times C));
      \alpha`
      \mathsf{p}_{A,B} \times \id_{\Diamond C}`
      \id_{\Diamond A} \times \mathsf{p}_{B,C}``
      \mathsf{p}_{A \times B,C}`
      \mathsf{p}_{A,B \times C}`
      \Diamond \alpha]
    \efig
    \and
    \bfig
    \hSquares|ammmmaa|/->``->`<-``->`/[
      1 \times \Diamond A`
      \Diamond A``
      \Diamond 1 \times \Diamond A`
      \Diamond(1 \times A)`;
      \lambda_{\Diamond A}``
      \mathsf{p}_{1} \times \id_{\Diamond A}`
      \Diamond \lambda_A``
      \mathsf{p}_{1,A}`]
    \efig
    \and
    \bfig
    \hSquares|ammmmaa|/->``->`<-``->`/[
      \Diamond A \times 1`
      \Diamond A``
      \Diamond A \times \Diamond 1`
      \Diamond(A \times 1)`;
      \rho_{\Diamond A}``
      \id_{\Diamond A} \times \mathsf{p}_{1}`
      \Diamond \rho_A``
      \mathsf{p}_{A,1}`]
    \efig
    \and
    \bfig
    \hSquares|ammmmaa|/->``->`->``->`/[
      \Diamond A \times \Diamond B`
      \Diamond B \times \Diamond A``
      \Diamond (A \times B)`
      \Diamond (B \times A)`;
      \beta_{\Diamond A,\Diamond B}``
      \mathsf{p}_{A,B}`
      \mathsf{p}_{B,A}``
      \Diamond\beta_{A,B}`]
    \efig
  \end{mathpar}

  Furthermore, suppose $J \dashv H$, where the unit, $\varepsilon :
  \Box A \to A$, and the counit, $\eta : A \to \Diamond A$, are
  monoidal natural transformations.  This implies that the following
  diagrams commute:
  \begin{mathpar}
    \bfig
    \btriangle<800,500>[A \times B`\Diamond A \times \Diamond B`\Diamond (A \times B);\eta_A \times \eta_B`\eta_{A \times B}`\mathsf{p}_{A,B}]
    \efig
    \and
    \bfig
    \qtriangle<800,500>[\Box A \times \Box B`\Box (A \times B)`A \times B;\mathsf{q}_{A,B}`\varepsilon_A \times \varepsilon_B`\varepsilon_{A \times B}]
    \efig
    \and
    \bfig
    \qtriangle<800,500>[1`J 1`\Diamond 1;n_{1}`\eta_1`J m_1]       
    \efig
    \and
    \bfig
    \hSquares|ammmmaa|/->``=`->``<-`/[
      \Box 1`
      1``
      \Box 1`
      H 1`;
      \varepsilon_1```
      m_1``
      H n_1`]
    \efig
    \and
    \bfig
    \qtriangle<800,500>[H A`H\Box A`H A;\eta_{H A}`\id_{H A}`H\varepsilon_A]       
    \efig
    \and
    \bfig
    \qtriangle<800,500>[J A`\Box J A`J A;J\eta_A`\id_{J A}`\varepsilon_{J A}]
    \efig    
  \end{mathpar}
  It is a well-known fact about adjoints that $(\Box, \varepsilon,
  \delta)$, where $\delta : \Box A \to \Box\Box A$ is a comonad, and
  $(\Diamond, \eta, \mu)$, where $\mu : \Diamond\Diamond A \to
  \Diamond A$ is a monad.  In addition, $\mu$ and $\delta$ are monoidal
  natural transformations where we have the following:
  \[
  \begin{array}{lll}
    \d{1} = \p{1};\Diamond\p{1} : 1 \mto \Diamond^2 1\\
    \d{A,B} =  \p{\Diamond A,\Diamond B};\Diamond\p{A,B} : \Diamond^2 A \times \Diamond^2 B \mto \Diamond^2 (A \times B)\\
    \\
    \b{1} = \q{1};\Box\q{1} : 1 \mto \Box^2 1\\
    \b{A,B} = \q{\Box A,\Box B};\Box\q{A,B} : \Box^2 A \times \Box^2 B \mto \Box^2 (A \times B)\\
  \end{array}
  \]
  Thus, the following diagrams commute:
  \begin{mathpar}
    \bfig
    \hSquares|ammmmaa|/->``->`->``->`/[
      \Diamond^3 A`
      \Diamond^2 A``
      \Diamond^2 A`
      \Diamond A`;
      \Diamond \mu_A``
      \mu_{\Diamond A}`
      \mu_A``
      \mu_A`]
    \efig
    \and
    \bfig
    \qtriangle/->`=`->/<800,500>[\Diamond A`\Diamond^2 A`\Diamond A;\eta_{\Diamond A}``\mu_A]
    \btriangle(0,0)/->`=`->/<800,500>[\Diamond A`\Diamond^2 A`\Diamond A;\Diamond \eta_{A}``\mu_A]
    \efig
    \and
    \bfig
    \hSquares|ammmmaa|/->``->`->``->`/[
      \Box A`
      \Box^2 A``
      \Box^2 A`
      \Box^3 A`;
      \delta_A``
      \delta_A`
      \delta_{\Box A}``
      \Box\delta_A`]
    \efig
    \and
    \bfig
    \qtriangle/->`=`->/<800,500>[\Box A`\Box^2 A`\Box A;\delta_A``\Box \varepsilon]
    \btriangle(0,0)/->`=`->/<800,500>[\Box A`\Box^2 A`\Box A;\delta_A``\varepsilon_{\Box A}]
    \efig
    \and
    %% \bfig   
    %% \vSquares|ammmmma|/->`->```->``->/[
    %%   \Diamond^2 A \times \Diamond^2 B`
    %%   \Diamond A \times \Diamond B`
    %%   \Diamond(\Diamond A \times \Diamond B)``
    %%   \Diamond^2(A \times B)`
    %%   \Diamond(A \times B);
    %%   \mu_A \times \mu_B`
    %%   \p{\Diamond A,\Diamond B}```
    %%   \Diamond\p{A,B}``
    %%   \mu_{A \times B}]
    %% \morphism(1108,0)/<-/<0,1000>[\Diamond(A \times B)`\Diamond A \times \Diamond B;\p{A,B}]
    %% \efig
    \and
    \bfig
    \square<1000,1000>[
      \Diamond^2 A \times \Diamond^2 B`
      \Diamond A \times \Diamond B`
      \Diamond^2(A \times B)`
      \Diamond(A \times B);
      \mu_A \times \mu_B`
      \d{A,B}`
      \p{A,B}`
      \mu_{A \times B}]
    \efig
    \and
    \bfig
    \Vtriangle/->`<-`<-/[
      \Diamond^2 1`
      \Diamond 1`
      1;
      \mu_1`
      \d{1}`
      \p{1}]
    \efig
    \and        
    \bfig
    \square<1000,1000>[
      \Box A \times \Box B`
      \Box^2 A \times \Box^2 B`
      \Box(A \times B)`
      \Box^2(A \times B);
      \delta_A \times \delta_B`
      \q{A,B}`
      \b{A,B}`
      \delta_{A \times B}]
    %% \vSquares|ammmmma|/->``->```->`->/[
    %%   \Box A \times \Box B`
    %%   \Box^2 A \times \Box^2 B``
    %%   \Box(\Box A \times \Box B)`
    %%   \Box(A \times B)`
    %%   \Box^2(A \times B);
    %%   \varepsilon_A \times \varepsilon_B``
    %%   \q{\Box A,\Box B}```
    %%   \Box\q{A,B}`
    %%   \varepsilon_{A \times B}]
    %% \morphism(0,0)/<-/<0,1000>[\Box(A \times B)`\Box A \times \Box B;\q{A,B}]
    \efig
    \and
    \bfig
    \Vtriangle/->`<-`<-/[
      \Box 1`
      \Box^2 1`
      1;
      \delta_1`
      \q{1}`
      \d{1}]
    \efig
  \end{mathpar}

  We can now define the $\Box$-strength map as follows:
  \[
  \st{A}{B} = (\eta_{\Box A} \pd \id_{\Diamond B});\mathsf{p}_{\Box A,B} : \Box A \pd \Diamond B \mto \Diamond(\Box A \pd B)
  \]
  We can see that $\st{A}{B}$ is a natural transformation, because it
  is defined as a composition of natural transformations.
  
  %% To prove that the appropriate diagrams commute we first note that
  %% the triangle  
  Next we must show that all of the appropriate diagrams given in
  Definition~\ref{def:comonad-strong-monad} commute.
  \begin{itemize}
  \item[] \textit{Case 1. the first projection interacts with $\Diamond$:}
    \[
    \bfig
    \square|amma|<850,600>[
      \Box A \times \Diamond B`
      \Diamond (\Box A \times B)`
      \Box A`
      \Diamond\Box A;
      \st{A}{B}`
      \pi_1`
      \Diamond\pi_1`
      \eta_A]
    \efig
    \]
    This diagrams commutes, because the following diagram commutes:
    \[
    \bfig
    \square|ammm|/->`->``/<1500,2000>[
      \Box A \times \Diamond B`
      \Diamond\Box A \times \Diamond B`
      \Box A`;
      \eta_{\Box A} \times \id_{\Diamond B}`
      \pi_1``]

    \square(1500,0)|ammm|/->``->`/<1500,2000>[
      \Diamond\Box A \times \Diamond B`
      \Diamond (\Box A \times B)``
      \Diamond\Box A;
      \p{\Box A,B}``
      \Diamond\pi_1`]

    \morphism<3000,0>[\Box A`\Diamond\Box A;\eta_{\Box A}]

    \morphism(484,1500)<1000,0>[
      \Box A \times 1`
      \Diamond\Box A \times \Diamond 1;
      \eta_{\Box A} \times \eta_1]

    \morphism(1484,1500)<700,-500>[
      \Diamond\Box A \times \Diamond 1`
      \Diamond(\Box A \times 1);
      \p{\Box A,1}]

    \morphism(484,1500)|m|/{@{>}@/_1em/}/<1700,-500>[
      \Box A \times 1`
      \Diamond(\Box A \times 1);
      \eta_{\Box A \times 1}]

    \morphism(484,1500)|m|<700,-800>[
      \Box A \times 1`
      \Box A;
      \pi_1]

    \morphism(0,2000)|m|<484,-500>[
      \Box A \times \Diamond B`
      \Box A \times 1;
      \id_{\Box A} \times \t_{\Diamond B}]

    \morphism(1500,2000)|m|<-17,-500>[
      \Diamond\Box A \times \Diamond B`
      \Diamond\Box A \times \Diamond 1;
      \id_{\Diamond\Box A} \times \Diamond\t_{B}]

    \morphism(3000,2000)|m|<-815,-1000>[
      \Diamond(\Box A \times B)`
      \Diamond(\Box A \times 1);
      \Diamond (id_{\Box A} \times \t_{B})]

    \morphism(2184,1000)|m|<816,-1000>[
      \Diamond(\Box A \times 1)`
      \Diamond\Box A;
      \Diamond\pi_1]

    \morphism(1184,700)|m|<1816,-700>[
      \Box A`
      \Diamond\Box A;
      \eta_{\Box A}]
    
    \place(2700,1000)[1]
    \place(2150,1700)[2]
    \place(800,1750)[3]
    \place(1300,1300)[4]
    \place(1800,800)[5]
    \place(500,500)[6]
    \efig    
    \]
    Diagrams 1 and 6 commute because we are in a cartesian closed
    category, diagram 2 commutes by naturality of $\p{}$, diagram 3
    commutes because $\Diamond$ is a product functor, diagram 4
    commutes because $\eta$ is the unit of a symmetric monoidal
    adjunction, and diagram 5 commutes by naturality of $\eta$.
    
  \item[] \textit{Case 2. the second projection interacts with $\Diamond$:}
    \[
    \bfig
    \qtriangle<850,600>[
      \Box A \times \Diamond B`
      \Diamond (\Box A \times B)`
      \Diamond B;
      \st{A}{B}`
      \pi_2`
      \Diamond\pi_2]
    \efig
    \]
    This case is similar to the previous case.
    
    %% \item[] \textit{Case 1. the object 1 behaves as the unit for products}
  %%   $$
  %%   \bfig
  %%   \vSquares|ammmmma|/>``>```>`>/[\Box 1 \times \Diamond A`\Diamond (\Box 1 \times A)``\Diamond(1 \times A)`1 \times \Diamond A`\Diamond A;\st{1}{A}``\Diamond(\varepsilon_1 \times \id_A)```\Diamond\lambda`\lambda]
  %%   \morphism(0,1000)|m|/->/<0,-950>[`;\varepsilon_1 \times \id_{\Diamond A}]
  %%   \efig
  %%   $$
  %%   This diagram commutes by commutativity of the following diagram:
  %%   %% Equational version:
  %%   %% \begin{center}
  %%   %%   \begin{math}
  %%   %%     \begin{array}{rllllllll}
  %%   %%       & & \st{1}{A};\Diamond (\varepsilon_1 \times \id_A);\Diamond\lambda_A\\
  %%   %%       \text{(Definition of $\mathsf{st}$)}
  %%   %%       & = & (\eta_{\Box 1} \times \id_{\Diamond A});\p{\Box 1,\Diamond A};\Diamond (\varepsilon_1 \times \id_A);\Diamond\lambda_A\\
  %%   %%       \text{(Naturality of $\mathsf{p}$)}
  %%   %%       & = & (\eta_{\Box 1} \times \id_{\Diamond A});(\Diamond \varepsilon_1 \times \Diamond\id_A);\p{1,A};\Diamond\lambda_A\\
  %%   %%       \text{(Functoriality of $\times$)}
  %%   %%       & = & ((\eta_{\Box 1};\Diamond \varepsilon_1) \times (\id_{\Diamond A};\Diamond\id_A));\p{1,A};\Diamond\lambda_A\\
  %%   %%       & = & ((\eta_{\Box 1};\Diamond \varepsilon_1) \times (\id_{\Diamond A};\id_{\Diamond A}));\p{1,A};\Diamond\lambda_A\\
  %%   %%       \text{(Naturality of $\eta$)}
  %%   %%       & = & ((\varepsilon_1;\eta_{1}) \times (\id_{\Diamond A};\id_{\Diamond A}));\p{1,A};\Diamond\lambda_A\\
  %%   %%       \text{(Definition of $\mathsf{p}$)}
  %%   %%       & = & ((\varepsilon_1;\p{1}) \times (\id_{\Diamond A};\id_{\Diamond A}));\p{1,A};\Diamond\lambda_A\\
  %%   %%       \text{(Functoriality of $\times$)}
  %%   %%       & = & (\varepsilon_1 \times \id_{\Diamond A});(\p{1} \times \id_{\Diamond A});\p{1,A};\Diamond\lambda_A\\
  %%   %%       \text{($\Diamond$ is Symmetric Monoidal)}
  %%   %%       & = & (\varepsilon_1 \times \id_{\Diamond A});\lambda_{\Diamond A}\\
  %%   %%     \end{array}
  %%   %%   \end{math}
  %%   %% \end{center}
  %%   $$
  %%   \bfig
  %%   \square|amma|<1000,500>[
  %%     \Diamond\Box 1 \times \Diamond A`
  %%     \Diamond (\Box 1 \times A)`
  %%     \Diamond 1 \times \Diamond A`
  %%     \Diamond (1 \times A);
  %%     \p{\Box 1,A}`
  %%     \Diamond \varepsilon_1 \times \id_{\Diamond A}`
  %%     \Diamond (\varepsilon_1 \times \id_A)`
  %%     \p{1,A}]

  %%   \square(-1000,0)|amma|<1000,500>[
  %%     \Box 1 \times \Diamond A`
  %%     \Diamond\Box 1 \times \Diamond A`
  %%     1 \times \Diamond A`
  %%     \Diamond 1 \times \Diamond A;
  %%     \eta_{\Box 1} \times \id_{\Diamond A}`
  %%     \varepsilon_1 \times \id_{\Diamond A}`
  %%     \Diamond \varepsilon_1 \times \id_{\Diamond A}`
  %%     (\p{1} = \eta_1) \times \id_{\Diamond A}]

  %%   \qtriangle(-1000,-500)|mmm|/`->`->/<2000,500>[
  %%     1 \times \Diamond A`
  %%     \Diamond (1 \times A)`
  %%     \Diamond A;`
  %%     \lambda_{\Diamond A}`
  %%     \Diamond\lambda_A]

  %%   \place(500,250)[1]
  %%   \place(-500,250)[2]
  %%   \place(500,-200)[3]
  %%   \efig
  %%   $$
  %%   \noindent
  %%   Diagram 1 commutes by naturality of $\mathsf{p}$, diagram 2
  %%   commutes by naturality of $\eta$, and diagram 3 commutes because
  %%   $\Diamond$ is a symmetric monoidal functor.

  \item[] \textit{Case 3. unit $\eta$ of the monad and strength interact well, $\Box  A $ is a parameter}
    $$
    \bfig
    \btriangle<800,500>[
      \Box A \pd B`
      \Box A \pd \Diamond B`
      \Diamond(\Box A \pd B);
      \id_{\Box A} \pd \eta_{B}`
      \eta_{\Box A \times B}`
      \st{A}{B}]
    \efig
    $$

    The previous diagram commutes, because the following diagram commutes:
    $$
    \bfig
    \btriangle|ama|/->`->`->/<1500,500>[
      \Box A \pd B`
      \Box A \pd \Diamond B`
      \Diamond\Box A \pd \Diamond B;
      \id_{\Box A} \pd \eta_{B}`
      \eta_{\Box A} \pd \eta_B`
      \eta_{\Box A} \pd \id_{\Diamond B}]

    \qtriangle(0,0)/->``<-/<1500,500>[
      \Box A \pd B`
      \Diamond (\Box A \times B)`
      \Diamond\Box A \pd \Diamond B;
      \eta_{\Box A \times B}``
      \p{\Box A,B}]

    \place(250,200)[1]
    \place(1200,300)[2]
    \efig
    $$
    \noindent
    Diagram 1 clearly commutes, and diagram 2 commutes because $\eta$
    is a symmetric monoidal natural transformation.
    
  %% \item[] \textit{Case 3. co-unit of the comonad $\varepsilon$ and unit of the monad $\eta$ interact well?}
  %%   $$
  %%   \bfig
  %%   \hSquares|aamaaaa|/->`->`->```->`/[
  %%     \Box A \times 1`
  %%     \Box A \times \Diamond 1`
  %%     \Diamond (\Box A \times 1)`
  %%     \Box A`
  %%     A`
  %%     ;
  %%     \id_{\Box A} \times \eta_1`
  %%     \st{A}{1}`
  %%     \rho_{\Box A}```
  %%     \varepsilon_A`]
  %%   \qtriangle(1978,0)/->``->/<800,500>[\Diamond (\Box A \times 1)`\Diamond\Box A`\Diamond A;\Diamond \rho_{\Box A}``\Diamond\varepsilon_A]
  %%   \morphism(1060,0)/->/<1640,0>[`;\eta_A]
  %%   \efig
  %%   $$
  %%   \noindent
  %%   Recall that
  %%   $\st{A}{1} = (\eta_{\Box A} \pd \id_{\Diamond 1});\p{\Box A,1}$.
  %%   Now the previous diagram commutes, because the following diagram commutes:
  %%   $$
  %%   \bfig
  %%   \btriangle|mma|<1444,1000>[\Box A \pd 1`\Diamond\Box A \pd \Diamond 1`\Diamond(\Box A \pd 1);\eta_{\Box A} \pd \eta_1`\eta_{\Box A \pd 1}`\p{\Box A,1}]
  %%   \dtriangle(-1000,0)/->``->/<1000,1000>[\Box A \pd 1`\Box A \pd 1`\Diamond\Box A \pd \Diamond 1;\id_{\Box A} \pd \eta_1``\eta_{\Box A} \pd
  %%     \id_{\Diamond 1}]

  %%   \hSquares(0,0)/->`->```<-``/<1000>[\Box A \pd 1`\Box A`\Diamond\Box A```\Diamond (\Box A \times 1);\rho_{\Box A}`\eta_{\Box A}```\Diamond (\rho_{\Box A})``]

  %%   \square(746,1000)/->`<-`<-`/<698,500>[A`\Diamond A`\Box A`\Diamond\Box A;\eta_A`\varepsilon_A`\Diamond\varepsilon_A`]

  %%   \place(-400,300)[1]
  %%   \place(400,300)[2]
  %%   \place(1100,700)[3]
  %%   \place(1100,1250)[4]
  %%   \efig
  %%   $$
  %%   \noindent
  %%   Diagram 1 commutes by functorality of $\times$, diagram 2 commutes
  %%   because $\eta$ is a monoidal natural transformation, and diagrams
  %%   3 and 4 commute by naturality of $\eta$.

  \item[] \textit{Case 4. associativity $\alpha$ interacts with co-monoidicity of $\Box$}
    $$
    \bfig
    \vSquares|ammmmmm|/->`->`->```->`/[
      \Box A \times (\Box B \times \Diamond C)`
      \Box A \times \Diamond(\Box B \times C)`
      (\Box A \times \Box B) \times \Diamond C`
      \Diamond(\Box A \times (\Box B \times C))``
      \Diamond((\Box A \times \Box B) \times C);
      \id_{\Box A} \pd \st{B}{C}`
      \alpha^{-1}`
      \st{A}{\Box B \times C}```
      \Diamond\alpha^{-1}`]
    \morphism(1554,0)|m|/->/<0,-500>[`\Diamond(\Box(A \times B) \times C);\Diamond(\m{A,B} \times \id_C)]
    
    \morphism(0,500)|m|/->/<0,-1000>[`\Box(A \times B) \times \Diamond C;\m{A,B} \times \id_{\Diamond C}]

    \morphism(350,-500)|a|/->/<800,0>[`;\st{A \times B}{C}]
    \efig
    $$
    \noindent
    Recall that:
    \[
    \begin{array}{rlll}
      \st{B}{C}              & = & (\eta_{\Box B} \pd \id_{\Diamond C});\p{\Box B,C}\\
      \st{A \pd B}{C}        & = & (\eta_{\Box (A \pd B)} \pd \id_{\Diamond C});\p{\Box (A \pd B),C}\\
      \st{A}{\Box B \pd C} & = & (\eta_{\Box A} \pd \id_{\Diamond (\Box B \pd C)});\p{\Box A,(\Box B \pd C)}\\
    \end{array}
    \]
    In addition, we require the following diagram (whose commutativity
    is implied by the fact that $\Diamond$ is a symmetric monoidal
    functor):
    $$
    \bfig
    \vSquares|ammmmma|/->`->`->``->`->`->/[
      \Diamond A \pd (\Diamond B \pd \Diamond C)`
      (\Diamond A \pd \Diamond B) \pd \Diamond C`
      \Diamond A \pd \Diamond (B \pd C)`
      \Diamond (A \pd B) \pd \Diamond C`
      \Diamond (A \pd (B \pd C))`
      \Diamond ((A \pd B) \pd C);
      \alpha^{-1}_{\Diamond A,\Diamond B,\Diamond C}`
      \id_{\Diamond A} \pd \p{B,C}`
      \p{A,B} \pd \id_{\Diamond C}``
      \p{A,B \pd C}`
      \p{A \pd B,C}`
      \Diamond\alpha^{-1}_{A,B,C}]
    \efig
    $$
    \noindent
    Finally, this case follows because the following diagram commutes:
    \begin{center}
      \rotatebox{90}{$
    \bfig
    \btriangle|mmm|<1769,1000>[
      \Box A \pd (\Diamond\Box B \pd \Diamond C)`
      \Diamond\Box A \pd (\Diamond\Box B \pd \Diamond C)`
      \Diamond\Box A \pd (\Diamond\Box B \pd C);
      \eta_{\Box A} \pd \id_{\Diamond \Box B \pd C}`
      \eta_{\Box A} \pd \p{\Box B,C}`
      \id_{\Diamond\Box A} \pd \p{\Box B,C}]

    \qtriangle|mam|/->``->/<1769,1000>[
      \Box A \pd (\Diamond\Box B \pd \Diamond C)`
      \Box A \pd \Diamond (\Box B \pd C)`
      \Diamond\Box A \pd (\Diamond\Box B \pd C);
      \id_{\Box A} \pd \p{\Box B,C}``
      \eta_{\Box A} \pd \id_{\Diamond (\Box B \pd C)}]    

    \qtriangle(-1800,0)|mmm|<1800,1000>[
      \Box A \pd (\Box B \pd \Diamond C)`
      \Box A \pd (\Diamond\Box B \pd \Diamond C)`
      \Diamond\Box A \pd (\Diamond\Box B \pd \Diamond C);
      \id_{\Box A} \pd (\eta_{\Box B} \pd \id_{\Diamond C})`
      \eta_{\Box A} \pd (\eta_{\Box B} \pd \id_{\Diamond C})`
      \eta_{\Box A} \pd \id_{\Diamond \Box B \pd C}]

    \square(0,-500)|mmmm|/`->`->`/<1769,500>[
      \Diamond\Box A \pd (\Diamond\Box B \pd \Diamond C)`
      \Diamond\Box A \pd (\Diamond\Box B \pd C)`
      (\Diamond\Box A \pd \Diamond\Box B) \pd \Diamond C`
      \Diamond (\Box A \pd (\Box B \pd C));`
      \alpha_{\Diamond\Box A,\Diamond\Box B,\Diamond C}`
      \p{\Box A,\Box B \pd C}`]   

    \square(0,-1000)|mmmm|/`->`->`/<1769,500>[
      (\Diamond\Box A \pd \Diamond\Box B) \pd \Diamond C`
      \Diamond (\Box A \pd (\Box B \pd C))`
      \Diamond (\Box A \pd \Box B) \pd \Diamond C`
      \Diamond ((\Box A \pd \Box B) \pd C);`
      \p{\Box A,\Box B} \pd \id_{\Diamond C}`
      \Diamond \alpha_{\Box A,\Box B,C}`]        

    \square(0,-1500)|mmmm|<1769,500>[
      \Diamond (\Box A \pd \Box B) \pd \Diamond C`
      \Diamond ((\Box A \pd \Box B) \pd C)`
      \Diamond\Box(A \pd B) \pd \Diamond C`
      \Diamond (\Box(A \pd B) \pd C);
      \p{\Box A \pd \Box B, C}`
      \Diamond \m{A,B} \pd \id_{\Diamond C}`
      \Diamond (\m{A,B} \pd \id_C)`
      \p{\Box (A \pd B),C}]

    \btriangle(-1800,-500)|mmm|/->``->/<1800,1500>[
      \Box A \pd (\Box B \pd \Diamond C)`
      (\Box A \pd \Box B) \pd \Diamond C`
      (\Diamond\Box A \pd \Diamond\Box B) \pd \Diamond C;
      \alpha_{\Box A,\Box B,\Diamond C}``
      (\eta_{\Box A} \pd \eta_{\Box B}) \pd \id_{\Diamond C}]

    \qtriangle(-1800,-1000)|mmm|/`->`/<1800,500>[
      (\Box A \pd \Box B) \pd \Diamond C`
      (\Diamond\Box A \pd \Diamond\Box B) \pd \Diamond C`
      \Diamond (\Box A \pd \Box B) \pd \Diamond C;`
      \eta_{\Box A \pd \Box B} \pd \id_{\Diamond C}`]

    \btriangle(-1800,-1500)|mmm|/->``->/<1800,1000>[
      (\Box A \pd \Box B) \pd \Diamond C`
      \Box(A \pd B) \pd \Diamond C`
      \Diamond\Box(A \pd B) \pd \Diamond C;
      \m{A,B} \pd \id_{\Diamond C}``
      \eta_{\Box (A \pd B)} \pd \id_{\Diamond C}]

    \place(1200,700)[1]
    \place(500,300)[2]
    \place(900,-500)[3]
    \place(900,-1250)[4]
    \place(-500,700)[5]
    \place(-1000,0)[6]
    \place(-400,-700)[7]
    \place(-1000,-1100)[8]
    \efig
    $}
    \end{center}
    Diagrams 1, 2 and 5 commute by functorality of $\times$, diagram 3
    commutes by the additional diagram from above, diagram 4 commutes
    by naturality of $\mathsf{p}$, diagram 6 commutes by naturality of
    $\alpha$, diagram 7 commutes by the fact that $\eta$ is a monoidal
    natural transformation, and diagram 8 commutes by naturality of
    $\eta$.
    

  \item[] \textit{Case 5.  strength interacts with monoidicity of $\Diamond$}
    $$
    \bfig
    \vSquares|ammmmma|/->`->```->``->/[
      \Box A \times \Diamond\Diamond B`
      \Box A \times \Diamond B`
      \Diamond(\Box A \times \Diamond B)``
      \Diamond\Diamond(\Box A \times B)`
      \Diamond(\Box A \times B);
      \id_{\Box A} \pd \mu_{B}`
      \st{A}{\Diamond B}```
      \Diamond(\st{A}{B})``
      \mu_{\Box A \pd B}]
    \morphism(1150,1000)|m|<0,-920>[`;\st{A}{B}]
    \efig
    $$
    \noindent
    Recall that:
    \[
    \begin{array}{rlll}
      \st{A}{B} & = & (\eta_{\Box A} \pd \id_{\Diamond B});\p{\Box A,B}\\
      \st{A}{\Diamond B} & = & (\eta_{\Box A} \pd \id_{\Diamond \Diamond B});\p{\Box A,\Diamond B}\\
    \end{array}
    \]
    This case follows from the fact that the following diagram
    commutes:
    \begin{center}
      \rotatebox{90}{$\bfig
    \qtriangle|mmm|<1500,1000>[
      \Box A \pd \Diamond\Diamond B`
      \Box A \pd \Diamond B`
      \Diamond\Box A \pd \Diamond B;
      \id_{\Box A} \pd \mu_B`
      \eta_{\Box A} \pd \mu_B`
      \eta_{\Box A} \pd \id_{\Diamond B}]

    \morphism(-1500,1000)|m|/<-/<1500,0>[
      \Diamond\Box A \pd \Diamond\Diamond B`
      \Box A \pd \Diamond\Diamond B;
      \eta_{\Box A} \pd \id_{\Diamond\Diamond B}]

    \btriangle(-1500,0)|mmm|<3000,1000>[
      \Diamond\Box A \pd \Diamond\Diamond B`
      \Diamond\Diamond\Box A \pd \Diamond\Diamond B`
      \Diamond\Box A \pd \Diamond B;
      \Diamond\eta_{\Box A} \pd \id_{\Diamond\Diamond B}`
      \id_{\Diamond\Box A} \pd \mu_B`
      \mu_{\Box A} \pd \mu_B]

    \square(-3000,0)|mmmm|/<-`->``<-/<1500,1000>[
      \Diamond(\Box A \pd \Diamond B)`
      \Diamond\Box A \pd \Diamond\Diamond B`
      \Diamond(\Diamond\Box A \pd \Diamond B)`
      \Diamond\Diamond\Box A \pd \Diamond\Diamond B;
      \p{\Box A,\Diamond B}`
      \Diamond (\eta_{\Box A} \pd \id_{\Diamond B})``
      \p{\Diamond\Box A,\Diamond B}]

    \square(-3000,-500)|mmmm|/`->`->`->/<4500,500>[
      \Diamond (\Diamond\Box A \pd \Diamond B)`
      \Diamond\Box A \pd \Diamond B`
      \Diamond\Diamond (\Box A \pd B)`
      \Diamond (\Box A \pd B);`
      \Diamond (\p{\Box A,B})`
      \p{\Box A,B}`
      \mu_{\Box A \pd B}]

    \place(-2300,500)[1]
    \place(-800,-250)[2]
    \place(-800,400)[3]
    \place(0,700)[4]
    \place(1050,700)[5]
    \efig$}
    \end{center}       
    Diagram commutes by naturality of $\mathsf{p}$, diagram 2 commutes
    because $\mu$ is a monoidal natural transformation, diagram 3
    commutes because $\mu$ is the monadic multiplication and by
    functorality of $\times$, and diagrams 4 and 5 commute by
    functoriality of $\times$.
    
  \item[] \textit{Case 6. commuting $\beta$ interacts with $\Diamond$}
    $$
    \bfig
    \hSquares|aamamaa|/``->``->`->`->/[
      \Diamond B \times \Box A``
      \Box A \times \Diamond B`
      \Diamond B \times \Diamond \Box A`
      \Diamond (B \times \Box A)`
      \Diamond (\Box A \times B);``
      \id_{\Diamond B} \pd \eta_{\Box A}``
      st_{A,B}`
      \p{B,\Box A}`
      \Diamond\beta_{B,\Box A}]
    \morphism(200,500)<1817,0>[`;\beta_{\Diamond B,\Box A}]
    \efig
    $$
    \noindent
    The previous diagram commutes by commutativity of the following
    diagram:
    $$
    \bfig
    \vSquares|ammmmma|[
      \Diamond B \times \Box A`
      \Box A \times \Diamond B`
      \Diamond B \times \Diamond\Box A`
      \Diamond\Box A \times \Diamond B`
      \Diamond (B \times \Box A)`
      \Diamond (\Box A \times B);
      \beta_{\Diamond B,\Box A}`
      \id_{\Diamond B} \times \eta_{\Box A}`
      \eta_{\Box A} \times \id_{\Diamond B}`
      \beta_{\Diamond B,\Diamond\Box A}`
      \p{B,\Box A}`
      \p{\Box A,B}`
      \Diamond\beta_{B,\Box A}]

    \place(600,750)[1]
    \place(600,250)[2]
    \efig
    $$
    \noindent
    Diagram 1 commutes because $\beta$ is a symmetric monoidal
    functor, and diagram 2 commutes by naturality of $\beta$.

  \item[] \textit{Case 7.  $\varepsilon$ interacts with $\Diamond$ and its monoidicity} 
    $$
    \bfig
    \hSquares|aamamaa|/``->``->`->`->/[
      \Box A \times \Diamond B``
      \Diamond(\Box A \times B)`
      A \times \Diamond B`
      \Diamond A \times \Diamond B`
      \Diamond (A \times B);``
      \varepsilon_A \times \id_{\Diamond B}``
      \Diamond(\varepsilon_A \times \id_{B})`
      \eta_A \times \id_{\Diamond B}`
      \p{A,B}]
    \morphism(200,500)<1605,0>[`;\st{A}{B}]
    \efig
    $$
    \noindent
    The previous diagram commutes by commutativity of the following
    diagram:
    $$
    \bfig
    \hSquares|aammmaa|[
      \Box A \times \Diamond B`
      \Diamond\Box A \times \Diamond B`
      \Diamond (\Box A \times B)`
      A \times \Diamond B`
      \Diamond A \times \Diamond B`
      \Diamond (A \times B);
      \eta_{\Box A} \times \id_{\Diamond B}`
      \p{\Box A,B}`
      \varepsilon_A \times \id_{\Diamond B}`
      \Diamond\varepsilon_A \times \id_{\Diamond B}`
      \Diamond (\varepsilon_A \times \id_B)`
      \eta_A \times \id_{\Diamond B}`
      \p{A,B}]
    \place(1750,250)[1]
    \place(600,250)[2]
    \efig
    $$
    \noindent
    Diagram 1 commutes by naturality of $\mathsf{p}$, and diagram 2
    commutes by naturality of $\eta$.
  \end{itemize}

\end{proof}

% section monoidal_monads_are_strong_w.r.t._any_monoidal_comonad (end)
