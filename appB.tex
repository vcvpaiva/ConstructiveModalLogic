\section{Cartesian Closed Categories and their Adjunctions}
\label{sec:cartesian_closed_categories_and_symmetric_monoidal_adjunctions}
In this appendix we consider simplifications that arise when the categories
in a symmetric monoidal adjunction are cartesian closed.  The first
big simplification is that when a SMCC is cartesian we no longer have
to specify all of the SMC structure, because some of the structure becomes provable.
The following lemma explains this.

\begin{lemma}[cccs are smccs]
  \label{lemma:CCC-is-SMC}
  Suppose $(\cat{C}, 1, \times, \Rightarrow)$ is a CCC.  Then the
  following defines the structure of an SMC:
  \[
  \begin{array}{rll}
    \lambda_A & = & \pi_2 : 1 \times A \mto A\\
    \lambda^{-1}_A & = & \langle \t_A , \id_A  \rangle : A \mto 1 \times A\\
    \\
    \rho_A & = & \pi_1 : A \times 1 \mto A\\
    \rho^{-1}_A & = & \langle \id_A , \t_A  \rangle : A \mto A \times 1\\
    \\
    \alpha_{A,B,C} & = & \langle \pi_1;\pi_1, \pi_2 \times \id_C \rangle : (A \times B) \times C \mto A \times (B \times C)\\
    \alpha^{-1}_{A,B,C} & = & \langle \id_A \times \pi_1, \pi_2;\pi_2 \rangle : A \times (B \times C) \mto (A \times B) \times C\\
    \\
    \beta_{A,B} & = & \langle \pi_2, \pi_1 \rangle : A \times B \mto B \times A\\
  \end{array}
  \]
  Each of the above morphisms satisfy the appropriate diagrams.
\end{lemma}

It turns out that we can also simplify the definition of a symmetric
monoidal functor between cartesian categories.

\begin{definition}
  \label{def:prod-functor}
  A \textbf{product functor}, $(F,m) : \cat{C}_1 \to \cat{C}_2$,
  between two cartesian categories consists of an ordinary functor $F
  : \cat{C}_1 \to \cat{C}_2$, a natural transformation $m_{A,B} : FA
  \times FB \mto F(A \times B)$, and a map $m_1 : 1 \to F1$ subject to the
  following coherence diagrams:
  \begin{mathpar}
    \bfig
    \btriangle|mmm|<1000,500>[
      FA \times FB`
      F(A \times B)`
      FA;
      m_{A.B}`
      \pi_1`
      F\pi_1]
    \efig
    \and
    \bfig    
    \btriangle|mmm|<1000,500>[
      FA \times FB`
      F(A \times B)`
      FB;
      m_{A.B}`
      \pi_2`
      F\pi_2]
    \efig
    \and
    \bfig
    \btriangle|mmm|<1000,500>[
      FC`
      FA \times FB`
      F(A \times B);
      \langle Ff, Fg \rangle`
      F(\langle f, g \rangle)`
      m_{A,B}]
    \efig
    \and
    \bfig
    \btriangle|mmm|<1000,500>[
      FA`
      1`
      F1;
      \t_{FA}`
      F\t_{A}`
      m_1]
    \efig
  \end{mathpar} 
\end{definition}

\begin{lemma}[Product Functors are Symmetric Monoidal]
  \label{lemma:product_functors_are_symmetric_monoidal}
  If $(F,m) : \cat{C}_1 \to \cat{C}_2$ is a product functor betwen cartesian categories, then
  $(F,m)$ is a symmetric monoidal functor.
\end{lemma}

{\tt Harley, surely the idea is that any symmetric monoidal functor between cccs becomes a product functor? I think it's clear that a product functor is a symm monoidal functor, no? the same way it's clear that a ccc is an smcc, lemma 12 only shows how to see it.}

\iffalse
\begin{proof}
  We simply show that each diagram in the definition of symmetric
  monoidal functor commutes.  This proof requires the following facts
  about cartesian categories:
  \[
  \begin{array}{lll}
    f;\langle g , h \rangle = \langle f;g, f;g \rangle & \text{(P1)}\\
    \\
    f \times g = \langle \pi_1;f , \pi_2;g \rangle & \text{(P2)}\\
    \\
    \langle f , g \rangle;(h \times i) = \langle f;h, g;i \rangle & \text{(P3)}\\
    \\
    (f \times g);\pi_1 = \pi_1;f & \text{(P4)}\\
    \\
    (f \times g);\pi_2 = \pi_2;g & \text{(P5)}\\
  \end{array}
  \]
  \begin{itemize}
  \item[] Associativity:\ \\
    \[
    \bfig
    \vSquares|ammmmma|/->`->`->``->`->`->/[
      (FA \times FB) \times FC`
      FA \times (FB \times FC)`
      F(A \times B) \times FC`
      FA \times F(B \times C)`
      F((A \times B) \times C)`
      F(A \times (B \times C));
      {\alpha}_{FA,FB,FC}`
      m_{A,B} \times \id_{FC}`
      \id_{FA} \times m_{B,C}``
      m_{A \times B,C}`
      m_{A,B \times C}`
      F{\alpha}_{A,B,C}]
    \efig
    \]
    In this case it is easiest to show that each side of the diagram
    reduces to the same morphism.

    First, the right side reduces in the following way:
    \begin{center}
      \begin{math}
        \begin{array}{lll}
          \alpha_{FA,FB,FC};(\id_{FA} \times m_{B,C});m_{A,B \times C} \\
          \,\,\,\,\,\,\,\,= \langle \pi_1;\pi_1, \pi_2 \times \id_{FC} \rangle;(\id_{FA} \times m_{B,C});m_{A,B \times C} & \text{(Definition)}\\
          \,\,\,\,\,\,\,\,= \langle \pi_1;\pi_1, (\pi_2 \times \id_{FC});m_{B,C} \rangle;m_{A,B \times C} & \text{(P3)}\\
        \end{array}
      \end{math}
    \end{center}

    Now the left side reduces to the same morphism:
    \[
    \scriptsize
    \begin{array}{lll}
      (m_{A,B} \times \id_{FC});m_{A \times B,C};\underline{F\alpha_{A,B,C}} \\
    \,\,\,\,\,\,\,\,= (m_{A,B} \times \id_{FC});m_{A \times B,C};\underline{F(\langle \pi_1;\pi_1, \pi_2 \times \id_C \rangle)} & \text{(Definition)}\\
   \,\,\,\,\,\,\,\,= (m_{A,B} \times \id_{FC});\underline{m_{A \times B,C};\langle F\pi_1;F\pi_1, F(\pi_2 \times \id_C) \rangle;m_{A,B \times C}} & \text{(Product Functor)}\\
   \,\,\,\,\,\,\,\,= (m_{A,B} \times \id_{FC});\langle \underline{m_{A \times B,C};F\pi_1};F\pi_1, m_{A \times B,C};F(\pi_2 \times \id_C) \rangle;m_{A,B \times C} & \text{(P1)}\\
   \,\,\,\,\,\,\,\,= (m_{A,B} \times \id_{FC});\langle \pi_1;F\pi_1, m_{A \times B,C};\underline{F(\pi_2 \times \id_C)} \rangle;m_{A,B \times C} & \text{(Product Functor)}\\
   \,\,\,\,\,\,\,\,= (m_{A,B} \times \id_{FC});\langle \pi_1;F\pi_1, m_{A \times B,C};\underline{F(\langle \pi_1;\pi_2, \pi_2 \rangle)} \rangle;m_{A,B \times C} & \text{(P2)}\\
   \,\,\,\,\,\,\,\,= (m_{A,B} \times \id_{FC});\langle \pi_1;F\pi_1, \underline{m_{A \times B,C};\langle F\pi_1;F\pi_2, F\pi_2 \rangle};m_{B,C} \rangle;m_{A,B \times C} & \text{(Product Functor)}\\
   \,\,\,\,\,\,\,\,= (m_{A,B} \times \id_{FC});\langle \pi_1;F\pi_1, \langle \underline{m_{A \times B,C};F\pi_1};F\pi_2, \underline{m_{A \times B,C};F\pi_2} \rangle;m_{B,C} \rangle;m_{A,B \times C} & \text{(P1)}\\
   \,\,\,\,\,\,\,\,= (m_{A,B} \times \id_{FC});\langle \pi_1;F\pi_1, \underline{\langle \pi_1;F\pi_2, \pi_2 \rangle};m_{B,C} \rangle;m_{A,B \times C} & \text{(Product Functor)}\\
   \,\,\,\,\,\,\,\,= \underline{(m_{A,B} \times \id_{FC});\langle \pi_1;F\pi_1, (F\pi_2 \times \id_{FC});m_{B,C} \rangle};m_{A,B \times C} & \text{(P2)}\\
   \,\,\,\,\,\,\,\,= \langle \underline{(m_{A,B} \times \id_{FC});\pi_1};F\pi_1, (m_{A,B} \times \id_{FC});(F\pi_2 \times \id_{FC});m_{B,C} \rangle;m_{A,B \times C} & \text{(P1)}\\
   \,\,\,\,\,\,\,\,= \langle \pi_1;\underline{m_{A,B};F\pi_1}, (m_{A,B} \times \id_{FC});(F\pi_2 \times \id_{FC});m_{B,C} \rangle;m_{A,B \times C} & \text{(P4)}\\
   \,\,\,\,\,\,\,\,= \langle \pi_1;\pi_1, \underline{(m_{A,B} \times \id_{FC});(F\pi_2 \times \id_{FC})};m_{B,C} \rangle;m_{A,B \times C} & \text{(Product Functor)}\\
   \,\,\,\,\,\,\,\,= \langle \pi_1;\pi_1, (\underline{(m_{A,B};F\pi_2)} \times \id_{FC});m_{B,C} \rangle;m_{A,B \times C} & \text{(Functor)}\\
   \,\,\,\,\,\,\,\,= \langle \pi_1;\pi_1, (\pi_2 \times \id_{FC});m_{B,C} \rangle;m_{A,B \times C} & \text{(Product Functor)}\\
    \end{array}
    \]

  \item[] Left Unitor:\\
    \[
    \bfig
    \square|amma|/->`->`<-`->/<1000,500>[
      1 \times FA`
      FA`
      F1 \times FA`
      F(1 \times A);
      {\lambda}_{FA}`
      m_{1} \times \id_{FA}`
      F{\lambda}_{A}`
      m_{1,A}]
    \efig
    \]
  
  First, notice that $\lambda_{FA} = \pi_2$. We can show the following:
  \[
  \begin{array}{lll}
    (m_1 \times \id_{FA});m_{1,A};F\lambda_A\\
    \,\,\,\,\,\,\,\,= (m_1 \times \id_{FA});m_{1,A};F\pi_2 & \text{(Definition)}\\
    \,\,\,\,\,\,\,\,= (m_1 \times \id_{FA});\pi_2 & \text{(Product Functor)}\\
    \,\,\,\,\,\,\,\,= \pi_2 & \text{(P5)}\\
  \end{array}
  \]

\item[] Right Unitor:\\
  \[
  \bfig
  \square|amma|/->`->`<-`->/<1000,500>[
    FA \times 1`
    FA`
    FA \times F1`
    F(A \times 1);
    {\rho}_{FA}`
    \id_{FA} \times m_{1}`
    F{\rho}_{A}`
    m_{A,1}]
  \efig
  \]
  \end{itemize}
  This case is similar to the previous one:
  \[
  \begin{array}{lll}
    (\id_{FA} \times m_1);m_{A,1};F\rho_A\\
    \,\,\,\,\,\,\,\,=  (\id_{FA} \times m_1);m_{A,1};F\pi_1 & \text{(Definition)}\\
    \,\,\,\,\,\,\,\,=  (\id_{FA} \times m_1);\pi_1 & \text{(Product Functor)}\\
    \,\,\,\,\,\,\,\,=  \pi_1 & \text{(P4)}\\
    \,\,\,\,\,\,\,\,=  \rho_{FA} & \text{(Definition)}\\
  \end{array}
  \]

\item[] Symmetry:\\
  \[
  \bfig
  \square|amma|/->`->`->`->/<1000,500>[
    FA \times FB`
    FB \times FA`
    F(A \times B)`
    F(B \times A);
    {\beta}_{FA,FB}`
    m_{A,B}`
    m_{B,A}`
    F{\beta}_{A,B}]
  \efig
  \]
  First, notice that $\beta_{A,B} = \langle \pi_2, \pi_1 \rangle$.  Thus, we have the following:
  \[
  \begin{array}{lll}
    \beta_{FA,FB};m_{B,A}\\
    \,\,\,\,\,\,\,\,=  \langle \pi_2, \pi_1 \rangle;m_{B,A} & \text{(Definition)}\\
    \,\,\,\,\,\,\,\,=  \langle m_{A,B};F\pi_2, m_{A,B};F\pi_1 \rangle;m_{B,A} & \text{(Product Functor)}\\
    \,\,\,\,\,\,\,\,=  m_{A,B};\langle F\pi_2, F\pi_1 \rangle;m_{B,A} & \text{(P1)}\\
    \,\,\,\,\,\,\,\,=  m_{A,B};F(\langle \pi_2, \pi_1 \rangle) & \text{(Product Functor)}\\
    \,\,\,\,\,\,\,\,=  m_{A,B};F\beta_{A,B} & \text{(Definition)}\\
  \end{array}
  \]
\end{proof}
\fi 

No other simplifications arise.  The definition of a natural
transformation between product functors is the same as the definition
of a symmetric monoidal natural transformation. And the same goes for
symmetric monoidal adjunctions.

\begin{lemma}[Product Functors Isomorphisms]
  \label{lemma:product_functors_iso}
  If $(F,m) : \cat{C}_1 \mto \cat{C}_2$ is a product functor between cccs, then
  $m_{A,B} : FA \times FB \mto F(A \times B)$ and $m_1 : 1 \mto F1$ are
  isomorphisms.
\end{lemma}
\begin{proof}
  Define $m^{-1}_1 = \t_{F 1} : F1 \mto 1$.  Then we have the following:
  \[
  \begin{array}{lll}
    m_1;t_{F1} = \id_{1} & \text{(Uniqueness)}\\    
  \end{array}
  \]
  and
  \[
  \begin{array}{llll}
    t_{F1};m_1
    & = & F\t_t & \text{(Product Functor)}\\
    & = & F\id_t & \text{(Uniqueness)}\\
    & = & \id_{Ft}     
  \end{array}
  \]

  Now define $m^{-1}_{A,B} = \langle F\pi_1, F\pi_2 \rangle$.  Then we have the following:
  \[
  \begin{array}{llll}
    m_{A,B};m^{-1}_{A,B}
    & = & m_{A,B};\langle F\pi_1, F\pi_2 \rangle & \text{(Definition)}\\
    & = & \langle m_{A,B};F\pi_1, m_{A,B};F\pi_2 \rangle & \text{(P1)}\\
    & = & \langle \pi_1, \pi_2 \rangle & \text{(Product Functor)}\\
    & = & \id_{FA \times FB} & \text{(Cartesian)}\\
  \end{array}
  \]
  and
  \[
  \begin{array}{llll}
    m^{-1}_{A,B};m_{A,B}
    & = & \langle F\pi_1, F\pi_2 \rangle;m_{A,B} & \text{(Definition)}\\
    & = & F(\langle \pi_1, \pi_2 \rangle) & \text{(Product Functor)}\\
    & = & F(\id_{A \times B}) & \text{(Cartesian)}\\
    & = & \id_{F(A \times B)}\\    
  \end{array}
  \]
\end{proof}
% section cartesian_closed_categories_and_symmetric_monoidal_adjunctions (end)

\begin{lemma}[Symmetric Monoidal Adjunctions have Isomorphisms as Left-adjoints]
  \label{lemma:product_functors_iso}
  If $(F,m)\dashv (G,n) : \cat{C}_1 \mto \cat{C}_2$ is any symmetric monoidal adjunction, then
  $F$  is an 
  isomorphism. 
\end{lemma}
Proof in Benton's LNL-tech report.

Now if we have a symmetric monoidal adjunction between product functors, connecting cccs $\cat{C}_1, \cat{C}_2$, can we say anything else about the unit and co-unit? what about the other monoidal natural transformation $n$? what happens if $\cat{C}_1$ is isomorphic to $\cat{C}_2$?